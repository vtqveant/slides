\documentclass{beamer}

\usepackage[T2A]{fontenc}
\usepackage[utf8]{inputenc}
\usepackage[english,russian]{babel}
\usepackage{amssymb,amsfonts,amsmath,mathtext}
\usepackage{cite,enumerate,float,indentfirst}

\usepackage{graphicx}
\usepackage{booktabs}
\usepackage{tabularx}

%% Gentzen style natural deduction proof trees
\usepackage{bussproofs}
\usepackage{latexsym}

\graphicspath{{images/}}

\usetheme{Rochester}
\usecolortheme{seagull}

\setbeamertemplate{footline}{\scriptsize{\hspace*{0.4cm}\insertframenumber}\vspace*{0.3cm}}
\beamertemplatenavigationsymbolsempty

\errorcontextlines 10000

\begin{document}
\title{\Large{Когнитивная лингвистика, ч. II}}
\subtitle{\textit{когнитивные модели}}
\author{Константин Соколов}
\institute[]
{Mathlingvo, СПбГУ, Eventflow API\\ \bigskip  \url{http://nlu-rg.ru}}
\date{Санкт-Петербург, 2014} 
% Создание заглавной страницы
\begin{frame}
    \thispagestyle{empty}
    \titlepage
\end{frame}

\begin{frame}{Текст}
\setcounter{framenumber}{1}
\begin{itemize}
	\item Дж. Лакофф. Женщины, огонь и опасные вещи. 1985 г.
	    \medskip
	    \begin{itemize}
	        \item книга 1, часть 1. Категории и когнитивные модели.
	    \end{itemize}
\end{itemize}
\end{frame}

\begin{frame}{План}
    \begin{itemize}
        \item 
        \medskip
        \item 
        \medskip
        \item 
        \medskip
    \end{itemize}
\end{frame}

\begin{frame}{}
\begin{center}
	\textbf{Часть первая}
\end{center}
\end{frame}

\begin{frame}{Часть первая}
\begin{itemize}
	\item 
	\medskip
	\item 
	\medskip
	\item  
\end{itemize}
\end{frame}




\begin{frame}{}
    \thispagestyle{empty}
    \begin{center}
        {\large Спасибо!}
    \end{center}
\end{frame}


\end{document}
