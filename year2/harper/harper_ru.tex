\documentclass[a4paper,12pt]{article}

%%% Поля и разметка страницы %%%
\usepackage{lscape} % Для включения альбомных страниц
\usepackage{afterpage}
\usepackage{geometry} % Для последующего задания полей

%%% Кодировки и шрифты %%%
\usepackage[T2A]{fontenc}
\usepackage[utf8]{inputenc}
\usepackage[russian,english]{babel}
\usepackage{amssymb,amsfonts,amsmath,mathtext}
\usepackage{cite,enumerate,float,indentfirst}

%% Gentzen style natural deduction proof trees
\usepackage{bussproofs}
\usepackage{latexsym}

%%% Оформление абзацев %%%
\usepackage{indentfirst} % Красная строка

%%% Общее форматирование
\usepackage[singlelinecheck=off,center]{caption} % Многострочные подписи
\usepackage{soul} % Поддержка переносоустойчивых подчёркиваний и зачёркиваний

%%% Гиперссылки %%%
\usepackage[plainpages=false,pdfpagelabels=false]{hyperref}

%%% Макет страницы %%%
\geometry{a4paper,top=3cm,bottom=3cm,left=3.5cm,right=3cm}

%%% Выравнивание и переносы %%%
\sloppy
\clubpenalty=10000
\widowpenalty=10000

\title{Святая Троица}
\author{Роберт Харпер}
\date{}

\begin{document}

\maketitle

Христианское учение о Троице -- это представление о едином Боге, выступающем в ипостасях Бога-Отца, Бога-Сына и Святого Духа. Доктрина \textit{вычислительного тринитаризма} состоит в том, что вычисление представляется в трех формах: доказательствах высказываний, программах некоторого типа и отображений между структурами. Каждый из этих аспектов есть объект поклонения одной из трех деноминаций: логики, утверждающей главенство доказательств и высказываний, теории языков программирования, обращающейся в первую очередь к программам и типам, и теории категорий, отдающей первенство отображениям и структурам. Основной догмат вычислительного тринитаризма гласит, что логики, языки программирования и категории -- всего лишь манифестации единого божественного понятия вычисления. Не существует наилучшего пути к просветлению, каждый из аспектов даёт возможность понимания того, что в нашей жизни представляет собой опыт вычисления.

Вычислительный тринитаризм имеет следствием, что любая концепция, возникшая в рамках одного из трех подходов, должна также иметь значение с точки зрения двух других. Если же получен результат, существенный и для логики, и для теории типов, и для теории категорий, то можно быть уверенным, что удалось прояснить некоторую существенную сторону понятия вычисления, а значит, совершено подлинное научное открытие. Успехи в нашем понимании вычисления могут происходить из результатов, полученных различными путями (любые данные полезны и уместны), а истинность результатов не зависит от их известности.

Логика говорит о том, какие высказывания существуют (какие виды мыслей мы хотим уметь выразить) и что представляет собой доказательство (как мы можем сообщать наши мысли другим). Языки (в том смысле слова, какой принят в информатике) сообщают, какие существуют типы (какие вычислительные явления вы хотим уметь выразить) и что представляет собой программа (каким образом эти явления возможно осуществить). Категории говорят о том, какие существуют структуры (с какими математическими моделями нам придётся работать) и что представляет собой отображение между ними (как они относятся друг к другу). В этом смысле все три подхода предписывают, чему быть, а не описывают то, что дано заранее. И в том же смысле они представляют собой основания. Если предположить, что они обладают лишь описательной силой, вопрос о происхождении заранее данных концепций останется без ответа, что вновь потребует обращения к проблеме оснований. Здесь я хочу обсудить именно основания, поскольку это поможет, как я надеюсь, исправить некоторые распространенные заблуждения относительно понятий высказывания, типа и структуры. В особенности интересно здесь то, что в рамках рассматриваемой концепции \textit{система типов} не представляет собой произвольный набор ограничений, налагаемых на данное заранее понятие программы (независимо от того, используются ли при её записи горизонтальные линии или нет). Скорее, система типов -- это способ говорить о том, что в первую очередь представляют собой программы и каким образом их можно интерпретировать как доказательства или отображения. 

Далее я опишу основные соответствия между логикой, языками программирования и категориями на основе их структурных свойств (и пока этим ограничусь).

Центральное понятие логики -- понятие \textit{следования}, которое записывается $P_1, \dots, P_n \vdash P$ и выражает выводимость $P$ из $P_1,\dots, P_n$. Такая запись говорит о том, что $P$ выводимо по правилам логики, если $P_i$ даны в качестве аксиом. В отличие от структурного следования (которое здесь обсуждаться не будет), эта форма следования не выражает импликации! В частности, следование из пустого набора аксиом невозможно. Следование обладает по меньшей мере двумя ключевыми структурными свойствами, позволяющими рассматривать его в качестве отношения предпорядка:

\begin{prooftree}
  \AxiomC{}
  \UnaryInfC{$P \vdash P$}
\end{prooftree}

\begin{prooftree}
  \AxiomC{$P \vdash Q$}
  \AxiomC{$Q \vdash R$}
  \BinaryInfC{$P \vdash R$}
\end{prooftree}

Кроме того, часто вводятся следующие дополнительные структурные свойства:

\begin{prooftree}
  \AxiomC{$P_1, \dots, P_n \vdash Q$}
  \UnaryInfC{$P_1, \dots, P_n, P_{n+1} \vdash Q$}
\end{prooftree}

\begin{prooftree}
  \AxiomC{$P_1,\dots,P_i,P_{i+1},\dots,P_n\vdash Q$}
  \UnaryInfC{$P_1,\dots,P_{i+1},P_{i},\dots,P_n\vdash Q$}
\end{prooftree}

\begin{prooftree}
  \AxiomC{$P_1,\dots,P_i,P_i,\dots,P_n\vdash Q$}
  \UnaryInfC{$P_1,\dots,P_i,\dots,P_n\vdash Q$}
\end{prooftree}

Они говорят о том, что ``дополнительные'' аксиомы не влияют на выводимость, ``переупорядочивание'' аксиом не играет роли, ``удвоение'' аксиом не играет роли. Эти условия кажутся неизбежными, но в т.\,н. субструктурных логиках любые из этих аксиом могут не иметь места.

\clearpage

В теории языков программирования базовая концепция -- \textit{суждение о типизации}, записыващееся как $x_1{:}A_1, \dots, x_n{:} A_n \vdash M{:}A$ и утверждающее, что $M$ есть выражение типа $A$, содержащее переменные $x_i$ типа $A_i$. Суждение о типизации должно удовлетворять следующим основным структурным свойствам:

\begin{prooftree}
  \AxiomC{}
  \UnaryInfC{$x:A\vdash x:A$}
\end{prooftree}

\begin{prooftree}
  \AxiomC{$y:B\vdash N:C \quad x:A\vdash M:B$}
  \UnaryInfC{$x:A\vdash [M/y]N:C$}
\end{prooftree}

Можно думать о переменных как об именах ``библиотек'', при этом первое свойство говорит о том, что использоваться может любая библиотека, а второе -- о замкнутости относительно компоновки (как в программе \textit{ld} в Unix и аналогичных), где $[M/x]N$ есть результат компоновки $x$ с библиотекой $M$ в выражении $N$. Обычно мы ожидаем, что аналоги аксиом ``дополнения'', ``переупорядочивания'' и ``удвоения'' будут выполняться и здесь, хотя это не обязательно. Их формулировка оставляется читателю в качестве упражнения.

В теории категорий основная концепция -- концепция \textit{отображения} $f:X \rightarrow Y$ между структурами $X$ и $Y$. Простейшими примерами являются, вероятно, множества и функции между ними, однако чаще рассматриваются, например, топологические пространства и непрерывные отображения между ними. Отображение удовлетворяет аналогичным структурным свойствам:

\begin{prooftree}
  \AxiomC{}
  \UnaryInfC{$\textit{id}_X : X \rightarrow X$}
\end{prooftree}

\begin{prooftree}
  \AxiomC{$f:X \rightarrow Y$}
  \AxiomC{$g:Y \rightarrow Z$}
  \BinaryInfC{$g \circ f : X \rightarrow Z$}
\end{prooftree}

Они выражают существование идентичного отображения и замкнутость отображений относительно композиции. Они соответствуют рефлексивности и транзитивности следования в логике и правилам ``подключения библиотеки'' и ``компоновки'' в теории языков программирования. Как и прежде, можно ожидать существования дополнительных условий замкнутости, соответствующих аксиомам ``дополнения'', ``переупорядочивания'' и ``удвоения'', которые можно охарактеризовать с помощью ряда дополнительных требований. Здесь я не буду углубляться в эту тему, так как эти условия подробно рассматриваются во многих стандартных источниках по теории категорий.

Вычислительный тринитаризм пленит меня своей красотой! Представьте себе мир, в котором логика, программирование и математика едины, в котором любому доказательству отвечает программа, любой программе -- отображение, любому отображению -- доказательство! Представьте себе мир, в котором код -- это математика, где нет разделения на рассуждение и осуществление, нет различия между языком математики и языком вычислений. Тринитаризм -- это центральный организующий принцип теории вычислений, интегрирующий, объединяющий и обогащающий язык логики, программирования и математики. Он дает средства открытия и анализа вычислительных явлений. Новшество в одной области обязано иметь последствия для остальных. Хорошая идея хороша независимо от того, в какой форме она была впервые  сформулирована. Если же какая-то идея не имеет смысла одновременно с логической, теоретико-категорной и теоретико-типовой точки зрения -- она не может быть проявлением божественного.

\end{document}