\documentclass{beamer}

\usepackage[T2A]{fontenc}
\usepackage[utf8]{inputenc}
\usepackage[english,russian]{babel}
\usepackage{amssymb,amsfonts,amsmath,mathtext}
\usepackage{cite,enumerate,float,indentfirst}

\usepackage{graphicx}
\usepackage{booktabs}
\usepackage{tabularx}

%% Gentzen style natural deduction proof trees
\usepackage{bussproofs}
\usepackage{latexsym}

\graphicspath{{images/}}

\usetheme{Rochester}
\usecolortheme{seagull}

\setbeamertemplate{footline}{\scriptsize{\hspace*{0.4cm}\insertframenumber}\vspace*{0.3cm}}
\beamertemplatenavigationsymbolsempty

\errorcontextlines 10000

\begin{document}
\title{\Large{Когнитивные архитектуры}}
\author{Константин Соколов}
\institute[]
{Mathlingvo, СПбГУ, Eventflow API\\ \bigskip  \url{http://nlu-rg.ru}}
\date{Санкт-Петербург, 2014} 
% Создание заглавной страницы
\begin{frame}
    \thispagestyle{empty}
    \titlepage
\end{frame}

\begin{frame}{Тексты}
\setcounter{framenumber}{1}
\begin{itemize}
	\item R. Brooks. Intelligence without representation. 1987.
    \medskip
    \item A. Newell. Unified theories of cognition. 1990.
    \medskip
    \item P. Langley et al. Cognitive architectures: Research issues and challenges. 2008.
\end{itemize}
\end{frame}

\begin{frame}{План}
    \begin{itemize}
        \item Когнитивные архитектуры
        \medskip
        \item Soar 9
        \medskip
        \item Проблема репрезентации
        \medskip
    \end{itemize}
\end{frame}


%
% 1. Обзор когнитивных архитектур
%

\begin{frame}{}
\begin{center}
	\textbf{Когнитивные архитектуры}
\end{center}
\end{frame}

% определение
\begin{frame}{Определение}
\begin{itemize}
	\item Когнитивная архитектура -- это ...
	\medskip
	\item 
	\medskip
	\item  
\end{itemize}
\end{frame}

% мотивация создания, назначение, задачи и функции: использование в задачах моделирования и объяснения когнитивных способностей человека (психология) -- экспериментальный аспект, использование для решения задач (искусственный интеллект, робототехника) -- практический аспект (промышленные применения)
\begin{frame}{}
\begin{itemize}
	\item 
	\medskip
\end{itemize}
\end{frame}

% виды архитектур: символьные, коннекционистские (эмерджентные), гибридные
\begin{frame}{}
\begin{itemize}
	\item 
	\medskip
\end{itemize}
\end{frame}

% основные компоненты архитектуры: долговременная и кратковременная память, восприятие, поиск решений, осуществление действий, обучение
\begin{frame}{}
\begin{itemize}
	\item 
	\medskip
\end{itemize}
\end{frame}

% понятие пространства задачи (problem space) и понятие оператора
\begin{frame}{}
\begin{itemize}
	\item 
	\medskip
\end{itemize}
\end{frame}

% формы репрезентации знаний: продукционные правила, фреймы, семантические сети, графовые структуры и пр.; коннекционистские и распределенные представления знаний; виды памяти (долговременная, кратковременная, эпизодическая, семантическая и пр.) и потребность в различных видах памяти, проблема унифицированной репрезентации
\begin{frame}{}
\begin{itemize}
	\item 
	\medskip
\end{itemize}
\end{frame}

% локализм и глобализм: какие параметры используются при решении задач (часть или все имеющиеся)
\begin{frame}{}
\begin{itemize}
	\item 
	\medskip
\end{itemize}
\end{frame}

% обзор систем: Soar, ACT-R, etc.
\begin{frame}{}
\begin{itemize}
	\item 
	\medskip
\end{itemize}
\end{frame}


%
% 2. Пример когнитивной архитектуры. Soar 9.
%

\begin{frame}{}
\begin{center}
	\textbf{Soar 9}
\end{center}
\end{frame}

% Диаграмма компонентов
\begin{frame}{}
\begin{itemize}
	\item 
	\medskip
\end{itemize}
\end{frame}

% Формализм представления знаний: продукционные правила; синтаксис
\begin{frame}{}
\begin{itemize}
	\item 
	\medskip
\end{itemize}
\end{frame}

% Chunking (см. Newell, p. 7, the power law of practice) и его эффекты (экспоненциальный рост в начале, полиномиальный после)
\begin{frame}{}
\begin{itemize}
	\item 
	\medskip
\end{itemize}
\end{frame}

% Виды памяти, семантическая память, эпизодическая память
\begin{frame}{}
\begin{itemize}
	\item 
	\medskip
\end{itemize}
\end{frame}

% Планирование и поиск в пространстве задачи; impasse и его разрешение как подзадача со своим пространством задачи
\begin{frame}{}
\begin{itemize}
	\item 
	\medskip
\end{itemize}
\end{frame}

% Теория аналогии по Holyoak, решение задач с помощью воображения (imagination)
\begin{frame}{}
\begin{itemize}
	\item 
	\medskip
\end{itemize}
\end{frame}

% Демо
\begin{frame}{}
\begin{itemize}
	\item 
	\medskip
\end{itemize}
\end{frame}

% сf. Content Planner in Tarot
\begin{frame}{}
\begin{itemize}
	\item 
	\medskip
\end{itemize}
\end{frame}


%
% 3. Проблема репрезентации
%

\begin{frame}{}
\begin{center}
	\textbf{Проблема репрезентации}
\end{center}
\end{frame}

% Брукс. "Кембрийский интеллект" (книга 1999 г. -- сборник статей и эссе Брукса). Subsumption architecture. Creature hypothesis.
\begin{frame}{}
\begin{itemize}
	\item Родни Брукс, "Кембрийский интеллект" (сборник статей и эссе, 1999 г.)
	\medskip
	\item Subsumption architecture
	\medskip
	\item Creature hypothesis
\end{itemize}
\end{frame}

% Мир как своя наилучшая модель
\begin{frame}{}
\begin{itemize}
	\item 
	\medskip
\end{itemize}
\end{frame}

% Уровневая структура поведения (avoid, wander, explore, seek); преимущества (real-time, no representation, no reasoning, robustness, real world)
\begin{frame}{}
\begin{itemize}
	\item 
	\medskip
\end{itemize}
\end{frame}

% Эмерджентность интеллектуального поведения
\begin{frame}{}
\begin{itemize}
	\item 
	\medskip
\end{itemize}
\end{frame}

% Способность создания репрезентаций как ключевой аспект интеллектуального поведения; создание систем на основе продукционных правил демонстрирует только интеллектуальное поведение их создателя, задающего репрезентацию -- сама система осуществляет только механический поиск;
\begin{frame}{}
\begin{itemize}
	\item 
	\medskip
\end{itemize}
\end{frame}

% Загадка репрезентации -- содержится ли она в самой системе или мы её "вчитываем"; можно ли надеяться на то, что при движении сверху вниз (от концепций к репрезентациям) и снизу вверх (от сигнала и восприятия к репрезентациям) эти движения вообще способны где-то встретиться? Может быть, они вообще в "параллельных пространствах" и гипотеза о том, что анализ "от наблюдаемого интеллектуального поведения" и анализ "от когнитивного субстрата" должны где-то совпасть -- это вообще-то очень сильная гипотеза. (Ещё ср. с тезисом Чёрча-Тьюринга). Тестелец и "черные брызги".
\begin{frame}{}
\begin{itemize}
	\item 
	\medskip
\end{itemize}
\end{frame}




\begin{frame}{}
    \thispagestyle{empty}
    \begin{center}
        {\large Спасибо!}
    \end{center}
\end{frame}


\end{document}
