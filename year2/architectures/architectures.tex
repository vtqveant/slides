\documentclass{beamer}

\usepackage[T2A]{fontenc}
\usepackage[utf8]{inputenc}
\usepackage[english,russian]{babel}
\usepackage{amssymb,amsfonts,amsmath,mathtext}
\usepackage{cite,enumerate,float,indentfirst}

\usepackage{graphicx}
\usepackage{booktabs}
\usepackage{tabularx}

%% Gentzen style natural deduction proof trees
\usepackage{bussproofs}
\usepackage{latexsym}

\graphicspath{{images/}}

\usetheme{Rochester}
\usecolortheme{seagull}

\setbeamertemplate{footline}{\scriptsize{\hspace*{0.4cm}\insertframenumber}\vspace*{0.3cm}}
\beamertemplatenavigationsymbolsempty

\errorcontextlines 10000

\begin{document}
\title{\Large{Когнитивные архитектуры}}
\author{Константин Соколов}
\institute[]
{Mathlingvo, СПбГУ, Eventflow API\\ \bigskip  \url{http://nlu-rg.ru}}
\date{Санкт-Петербург, 2014} 
% Создание заглавной страницы
\begin{frame}
    \thispagestyle{empty}
    \titlepage
\end{frame}

\begin{frame}{Тексты}
\setcounter{framenumber}{1}
\begin{itemize}
	\item R. Brooks. Intelligence without representation. 1987.
    \medskip
    \item A. Newell. Unified theories of cognition. 1990.
    \medskip
    \item P. Langley et al. Cognitive architectures: Research issues and challenges. 2008.
\end{itemize}
\end{frame}

\begin{frame}{План}
    \begin{itemize}
        \item Когнитивные архитектуры
        \medskip
        \item Soar 9
        \medskip
        \item Проблема репрезентации
        \medskip
    \end{itemize}
\end{frame}


%
% 1. Обзор когнитивных архитектур
%

\begin{frame}{}
\begin{center}
	\textbf{Когнитивные архитектуры}
\end{center}
\end{frame}



% определение
\begin{frame}{Определение (I)}
\begin{itemize}
    \item[] \textit{Когнитивная архитектура} -- это общая вычислительная когнитивная модель, отражающая основные структуры и процессы когнитивного аппарата человека, применяемая для многоуровневого анализа поведения в разнообразных предметных областях
    \item[] \hfill {\small(Newell, A. Unified Theories of Congition. 1990) и др.}
\end{itemize}
\end{frame}

\begin{frame}{Определение (II)}
\begin{itemize}
    \item[] \textit{Когнитивная архитектура} -- это спецификация тех аспектов когнитивного аппарата, которые остаются неизменными в течение жизни агента
    \item[] \hfill {\small CLARION Documentation}
    \bigskip
    \item[]
        \begin{itemize}
            \item организация памяти
            \item представление знаний
            \item механизмы процедурной обработки и принятия решений
            \item механизмы обучения
            \item и пр.
        \end{itemize}
\end{itemize}
\end{frame}

\begin{frame}{Определение (III)}
\begin{center}
\textit{Интеллектуальный агент}:\\фиксированная архитектура + динамические знания
\end{center}
\end{frame}

\begin{frame}{Определение (IV)}
Теория множественного интеллекта (Говард Гарднер, 1993)\\
\bigskip
\begin{itemize}
    \item интеллект как ``способность к решению задач или созданию продуктов, обусловленную конкретными культурными особенностями или социальной средой''
    \medskip
    % \item независимые друг от друга виды интеллекта
    \item виды интеллекта
        \medskip
        \begin{itemize}
           \item логико-математический
           \item языковой
           \item пространственный
           \item музыкальный
           \item телесно-кинестетический
           \item внутриличностный и межличностный
           \item эмоциональный
        \end{itemize}
\end{itemize}
\end{frame}

% мотивация создания, назначение, задачи и функции: использование в задачах моделирования и объяснения когнитивных способностей человека (психология) -- экспериментальный аспект, использование для решения задач (искусственный интеллект, робототехника) -- практический аспект (промышленные применения)
\begin{frame}{Общие слова (I)}
Области применения когнитивных архитектур:\\
\medskip
\begin{itemize}
	\item психология (моделирование и объяснение когнитивных способностей человека)
	\medskip
	\item искусственный интеллект (робототехника, человеко-машинные интерфейсы)
	\medskip
	\item промышленность (создание тренажеров, в т.ч. военного назначения)
	\medskip 
	\item игровая индустрия (боты, правдоподобные агенты)
	\medskip
	\item прикладная лингвистика (диалоговые системы, когнитивная лингвистика)
\end{itemize}
\end{frame}

\begin{frame}{Общие слова (II)}
Исследовательские задачи разработки КА:\\
\medskip
\begin{itemize}
	\item биологическая правдоподобность
	\medskip
	\item психологическая правдоподобность
	\medskip
	\item функциональность агентов
\end{itemize}
\end{frame}

% виды архитектур: символьные, коннекционистские (эмерджентные), гибридные
\begin{frame}{Общие слова (III)}
Основные виды архитектур:\\
\medskip
\begin{itemize}
	\item символьные
	\medskip
	\item коннекционистские (эмерджентные)
	\medskip
	\item гибридные
\end{itemize}
\end{frame}


% /*
\iffalse

% виды архитектур: символьные, коннекционистские (эмерджентные), гибридные
\begin{frame}{Общие слова (IV)}
Критерии оценки и сравнения архитектур (Newell 1990):\\
\medskip
\begin{itemize}
	\item поведение
	\item адаптивность
	\item динамичность
	\item гибкость
	\item развитие
	\item эволюция
	\item обучение
	\item интеграция знаний
	\item обширная база знаний
	\item естественный язык
	\item работа в реальном времени
	\item адекватность строению мозга
\end{itemize}
\end{frame}

\fi
% */


\begin{frame}{Базовая схема когнитивной архитектуры}
\begin{center}
	\begin{figure}[H]
		\includegraphics[scale=0.335]{prototypical_ca.png} 
	\end{figure}
\end{center}
\end{frame}

% основные компоненты архитектуры: долговременная и кратковременная память, восприятие, поиск решений, осуществление действий, обучение
\begin{frame}{Основные компоненты архитектуры (I)}
\begin{itemize}
	\item Восприятие
	    \medskip
	    \begin{itemize}
	        \item сенсоры
	        \item интроспекция
	    \end{itemize}
	\medskip
	\item Память
	    \medskip
	    \begin{itemize}
	        \item кратковременная (STM)
	        \item долговременная (LTM)
	        \item удержание внимания (attentional memory)
	        \item удержание намерения (intentional memory)
	        \item эпизодическая (припоминание и воображение)
	        \item сенсорная
	        \item моторная
	        \item процедурная 
	        \item декларативная 
	    \end{itemize}
\end{itemize}
\end{frame}

\begin{frame}{Основные компоненты архитектуры (II)}
\begin{itemize}
	\item Принятие решений
	    \medskip
	    \begin{itemize}
	        \item поиск в пространстве задачи
	        \item логический и вероятностный вывод
	        \item вывод по аналогии
	        \item немонотонный вывод (пересмотр установок)
	        \item разрешение конфликтов
	        \item обход тупиков (impasse)
	    \end{itemize}	
	\medskip
	\item Осуществление действий
	    \medskip
	    \begin{itemize}
	        \item исполнительные устройства
	        \item ``ментальные акты''
	    \end{itemize}	
	\medskip
	\item Обучение
		\medskip
	    \begin{itemize}
	        \item пополнение базы знаний и семантической памяти
	        \item расширение процедурных навыков
	        \item фиксация успешных стратегий поведения и обхода тупиков
	    \end{itemize}	
\end{itemize}
\end{frame}


% понятие пространства задачи (problem space) и понятие оператора
\begin{frame}{Пространство задачи}
Основа формализма (символьных) когнитивных архитектур -- \textit{принцип пространства задачи} (Newell 1990):\\
\bigskip
Разумная деятельность человека, направленная на решение задач, может быть описана с помощью:
\begin{itemize}
    \item[(1)] множества состояний знания (states of knowledge),
	\item[(2)] операторов, преобразующих одни состояния в другие,
	\item[(3)] ограничений на применение операторов, 
	\item[(4)] знаний управляющего характера, на основе которых осуществляется выбор оператора для применения.
\end{itemize}
\end{frame}

% формы репрезентации знаний: продукционные правила, фреймы, семантические сети, графовые структуры и пр.; коннекционистские и распределенные представления знаний; виды памяти (долговременная, кратковременная, эпизодическая, семантическая и пр.) и потребность в различных видах памяти, проблема унифицированной репрезентации
\begin{frame}{Представление знаний}
Символьные подходы\\
\medskip
\begin{itemize}
	\item продукционные правила
	\item фреймы
	\item семантические сети и графовые структуры
	\item логика первого порядка
	\item планы
	\medskip
\end{itemize}
\end{frame}

% обзор систем: Soar, ACT-R, etc.
\begin{frame}{Символьные архитектуры}
\begin{itemize}
    \item GPS (General Problem Solver) -- 1969
	\item Soar (State, Operator and Result) -- 1983
	\item OSCAR -- 1990
	\item ICARUS -- 1991
	\item EPIC (Executive Process Interactive Control) -- 1997
	\item NARS (Non-Axiomatic Reasoning System)
	\item SNePS (Semantic Network Processing System)
\end{itemize}
\end{frame}

\begin{frame}{Эмерджентные архитектуры}
\begin{itemize}
	\item IBCA (Integrated Biologically-based Cognitive Architecture)
	\item NOMAD (Naturally Organized Mobile Adaptive Device)
	\item Cortronics
	\item LEABRA (Local, Error-driven, Associative, Biologically Realistic Algorithm) -- 1996
\end{itemize}
\end{frame}

\begin{frame}{Гибридные архитектуры}
\begin{itemize}
	\item ACT-R (Adaptive Component of Thought -- Rational) -- 1976
	\item CLARION -- 1997
	\item LIDA (Learning Intelligent Distribution Agent) -- 1998
	\item Polyscheme -- 2002
	\item CogPrime -- 2004
	\item 4CAPS
	\item DUAL
	\item Shruti
\end{itemize}
\end{frame}


%
% 2. Пример когнитивной архитектуры. Soar 9.
%

\begin{frame}{}
\begin{center}
	\textbf{Soar 9}
\end{center}
\end{frame}

% Диаграмма компонентов
\begin{frame}{Диаграмма компонентов}
\begin{center}
	\begin{figure}[H]
		\includegraphics[scale=0.41]{soar9.png} 
	\end{figure}
\end{center}
\end{frame}

\begin{frame}{Память}
\begin{center}
	\begin{figure}[H]
		\includegraphics[scale=0.35]{soar-memory.png} 
	\end{figure}
\end{center}
\end{frame}

\begin{frame}{Основной цикл}
\begin{center}
	\begin{figure}[H]
		\includegraphics[scale=0.36]{soar-cycle.png} 
	\end{figure}
\end{center}
\end{frame}


% Формализм представления знаний: продукционные правила; синтаксис
\begin{frame}[fragile]
\frametitle{Формализм (I)}
{\footnotesize \begin{verbatim}
sp {hello-world
  (state <s> ^type state)
-->
  (write |Hello World|)
  (halt)
}
\end{verbatim}}
\end{frame}

\begin{frame}[fragile]
\frametitle{Формализм (II)}
{\footnotesize \begin{verbatim}
sp {propose*hello-world
  (state <s> ^type state)
-->
  (<s> ^operator <o> +)
  (<o> ^name hello-world)
}

sp {apply*hello-world
  (state <s> ^operator.name hello-world)
-->
  (write |Hello World|)
  (halt)
}
\end{verbatim}}
\end{frame}

% Теория аналогии по Holyoak, решение задач с помощью воображения (imagination)
\begin{frame}{Разное (I)}
\begin{itemize}
	\item решение задач с помощью воображения
	\medskip
	\item теория аналогии по (Holyoak et al., 1997)
\end{itemize}
\end{frame}

\begin{frame}[fragile]
\frametitle{Разное (II)}
Продукционные правила в Tarot (Content Planner):\\
\bigskip
{\footnotesize \begin{verbatim}
:dvp ^ <Relation> accept ^ <Content> ( #c1:marker )
->
#c1 = (:ascription ^ <Actor>(#i1:person ^ I)
                   ^ <Patient>(:entity ^ what)
                   ^ <Subject>( #i1: )).
\end{verbatim}}
\end{frame}

%
% 3. Проблема репрезентации
%

\begin{frame}{}
\begin{center}
	\textbf{Проблема репрезентации}
\end{center}
\end{frame}

% Брукс. "Кембрийский интеллект" (книга 1999 г. -- сборник статей и эссе Брукса). Subsumption architecture. Creature hypothesis.
\begin{frame}{Кембрийский интеллект}
\begin{itemize}
	\item Родни Брукс, ``Intelligence without representation'' (1987)
	\bigskip
	\item Subsumption architecture
	    \medskip
    	\begin{itemize}
	        \item \textit{Creature Hypothesis}
	        \medskip
    	    \item функциональная декомпозиция	        
	        \medskip
	        \item уровневая организация поведения
    	    \medskip
    	    \item наилучшая модель мира -- это сам мир
	        \medskip
	        \item отсутствие централизованной репрезентации
	        \medskip
	        \item эмерджентность интеллектуального поведения
    	\end{itemize}	
\end{itemize}
\end{frame}

% Уровневая структура поведения (avoid, wander, explore, seek); преимущества (real-time, no representation, no reasoning, robustness, real world)
\begin{frame}{Структура поведения}
\begin{itemize}
	\item Поведение робота-пылесоса
	\bigskip
	\begin{itemize}
	    \item избегать столкновений
	    \medskip
	    \item перемещаться
	    \medskip
	    \item исследовать
	    \medskip
	    \item искать
	\end{itemize}
\end{itemize}
\end{frame}

\begin{frame}{Проблема репрезентации}
\begin{itemize}
	\item способность создавать репрезентации -- ключевой аспект интеллектуального поведения
	\medskip
	\item содержится ли репрезентация в самой системе или мы её ``вчитываем''?
	\medskip
	\item гипотеза о том, что анализ ``от наблюдаемого интеллектуального поведения'' и анализ ``от когнитивного субстрата'' должны где-то совпасть -- это вообще-то очень сильная гипотеза
	\medskip
	\item когнитивная непроницаемость, e.g. \textit{early vision} \\{\small(Pylyshyn, 1984)}
\end{itemize}
\end{frame}


\begin{frame}{}
    \thispagestyle{empty}
    \begin{center}
        {\large Спасибо!}
    \end{center}
\end{frame}


\end{document}
