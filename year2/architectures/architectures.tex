\documentclass{beamer}

\usepackage[T2A]{fontenc}
\usepackage[utf8]{inputenc}
\usepackage[english,russian]{babel}
\usepackage{amssymb,amsfonts,amsmath,mathtext}
\usepackage{cite,enumerate,float,indentfirst}

\usepackage{graphicx}
\usepackage{booktabs}
\usepackage{tabularx}

%% Gentzen style natural deduction proof trees
\usepackage{bussproofs}
\usepackage{latexsym}

\graphicspath{{images/}}

\usetheme{Rochester}
\usecolortheme{seagull}

\setbeamertemplate{footline}{\scriptsize{\hspace*{0.4cm}\insertframenumber}\vspace*{0.3cm}}
\beamertemplatenavigationsymbolsempty

\errorcontextlines 10000

\begin{document}
\title{\Large{Когнитивные архитектуры}}
\author{Константин Соколов}
\institute[]
{Mathlingvo, СПбГУ, Eventflow API\\ \bigskip  \url{http://nlu-rg.ru}}
\date{Санкт-Петербург, 2014} 
% Создание заглавной страницы
\begin{frame}
    \thispagestyle{empty}
    \titlepage
\end{frame}

\begin{frame}{Тексты}
\setcounter{framenumber}{1}
\begin{itemize}
	\item R. Brooks. Intelligence without representation. 1987.
    \medskip
    \item A. Newell. Unified theories of cognition. 1990.
    \medskip
    \item P. Langley et al. Cognitive architectures: Research issues and challenges. 2008.
\end{itemize}
\end{frame}

\begin{frame}{План}
    \begin{itemize}
        \item Обзор когнитивных архитектур
        \medskip
        \item Пример когнитивной архитектуры: Soar v. 9
        \medskip
        \item Проблема репрезентации
        \medskip
    \end{itemize}
\end{frame}

\begin{frame}{}
\begin{center}
	\textbf{Обзор когнитивных архитектур}
\end{center}
\end{frame}

% 1. Обзор когнитивных архитектур
%
% определение
% мотивация создания, назначение, задачи и функции: использование в задачах моделирования и объяснения когнитивных способностей человека (психология) -- экспериментальный аспект, использование для решения задач (искусственный интеллект, робототехника) -- практический аспект (промышленные применения)
% виды архитектур: символьные, коннекционистские, гибридные
% основные компоненты архитектуры: долговременная и кратковременная память, восприятие, поиск решений, осуществление действий, обучение
% понятие пространства задачи (problem space) и понятие оператора
% формы репрезентации знаний: продукционные правила, фреймы, семантические сети, графовые структуры и пр.; коннекционистские и распределенные представления знаний; виды памяти (долговременная, кратковременная, эпизодическая, семантическая и пр.) и потребность в различных видах памяти, проблема унифицированной репрезентации
% локализм и глобализм: какие параметры используются при решении задач (часть или все имеющиеся)
% обзор систем: Soar, ACT-R, etc.

% 2. Пример когнитивной архитектуры. Soar 9.
%
% Диаграмма компонентов
% Формализм представления знаний: продукционные правила; синтаксис
% Chunking (см. Newell, p. 7, the power law of practice) и его эффекты (экспоненциальный рост в начале, полиномиальный после)
% Виды памяти, семантическая память, эпизодическая память
% Планирование и поиск в пространстве задачи; impasse и его разрешение как подзадача со своим пространством задачи
% Теория аналогии по Holyoak, решение задач с помощью воображения (imagination)
% Демо

% 3. Проблема репрезентации
%
% Брукс. Кембрийский интеллект. Subsumption architecture. Creature hypothesis.
% Мир как своя наилучшая модель
% Уровневая структура поведения (avoid, wander, explore, seek); преимущества (real-time, no representation, no reasoning, robustness, real world)
% Эмерджентность интеллектуального поведения
% Способность создания репрезентаций как ключевой аспект интеллектуального поведения; создание систем на основе продукционных правил демонстрирует только интеллектуальное поведение их создателя, задающего репрезентацию -- сама система осуществляет только механический поиск; 
% Загадка репрезентации -- содержится ли она в самой системе или мы её "вчитываем"; можно ли надеяться на то, что при движении сверху вниз (от концепций к репрезентациям) и снизу вверх (от сигнала и восприятия к репрезентациям) эти движения вообще способны где-то встретиться? Может быть, они вообще в "параллельных пространствах" и гипотеза о том, что анализ "от наблюдаемого интеллектуального поведения" и анализ "от когнитивного субстрата" должны где-то совпать -- это вообще-то очень сильная гипотеза. (Ещё ср. с тезисом Чёрча-Тьюринга). Тестелец и "черные брызги".


\begin{frame}{Определение}
\begin{itemize}
	\item Когнитивная архитектура -- это ...
	\medskip
	\item 
	\medskip
	\item  
\end{itemize}
\end{frame}

\begin{frame}{Задачи и функции}
\begin{itemize}
	\item Когнитивна архитектура -- это ...
	\medskip
	\item 
	\medskip
	\item  
\end{itemize}
\end{frame}

\begin{frame}{Задачи и функции}
\begin{itemize}
	\item Когнитивна архитектура -- это ...
	\medskip
	\item 
	\medskip
	\item  
\end{itemize}
\end{frame}




\begin{frame}{}
\begin{center}
	\textbf{Часть первая}
\end{center}
\end{frame}

\begin{frame}{Часть первая}
\begin{itemize}
	\item 
	\medskip
	\item 
	\medskip
	\item  
\end{itemize}
\end{frame}




\begin{frame}{}
    \thispagestyle{empty}
    \begin{center}
        {\large Спасибо!}
    \end{center}
\end{frame}


\end{document}
