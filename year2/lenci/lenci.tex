\documentclass{beamer}

\usepackage[T2A]{fontenc}
\usepackage[utf8]{inputenc}
\usepackage[english,russian]{babel}
\usepackage{amssymb,amsfonts,amsmath,mathtext}
\usepackage{cite,enumerate,float,indentfirst}

\usepackage{graphicx}
\usepackage{booktabs}
\usepackage{tabularx}

\graphicspath{{images/}}

\usetheme{Rochester}
\usecolortheme{seagull}

\setbeamertemplate{footline}{\scriptsize{\hspace*{0.4cm}\insertframenumber}\vspace*{0.3cm}}
\beamertemplatenavigationsymbolsempty

\errorcontextlines 10000

\begin{document}
\title{\Large{\sc Дистрибутивная семантика}}
\subtitle{\small{\textit{обзор}}}
\author{Константин Соколов}
\institute[]
{СПбГУ\\ \bigskip  \url{http://nlu-rg.ru}}
\date{Санкт-Петербург, 2015} 

\begin{frame}
    \thispagestyle{empty}
    \titlepage
\end{frame}

\begin{frame}{Текст}
\setcounter{framenumber}{1}
\begin{center}
\small{Lenci A. Distributional semantics in linguistic and cognitive research.\\// Rivista di Linguistica 20.1 (2008), pp. 1-31.}
\end{center}
\end{frame}

\begin{frame}{}
\begin{center}
	\textbf{История вопроса}
\end{center}
\end{frame}

\iffalse
\begin{frame}{dsfsdf}
\begin{center}
	\begin{figure}[H]
		\includegraphics[scale=0.7]{harmonic1.png} 
	\end{figure}
\end{center}
\end{frame}
\fi

\begin{frame}{История вопроса - I}
\begin{itemize}
	\item Предпосылка: развитие корпусов
	\item Дистрибутивные методы требуют решения двух вопросов:
		\begin{itemize}
		    \item до какой степени лекические своства могут быть сведены к комбинаторным
		    \item причинно-следственные связи между контекстом и структурой семантических репрезентаций на когнитивном уровне
		\end{itemize}
\end{itemize}
\end{frame}


\begin{frame}{}
    \thispagestyle{empty}
    \begin{center}
        {\large Спасибо!}
    \end{center}
\end{frame}

\end{document}
