\documentclass{beamer}

\usepackage[T2A]{fontenc}
\usepackage[utf8]{inputenc}
\usepackage[english,russian]{babel}
\usepackage{amssymb,amsfonts,amsmath,mathtext}
\usepackage{cite,enumerate,float,indentfirst}

\usepackage{graphicx}
\usepackage{booktabs}
\usepackage{tabularx}

%% Gentzen style natural deduction proof trees
\usepackage{bussproofs}
\usepackage{latexsym}

\graphicspath{{images/}}

\usetheme{Rochester}
\usecolortheme{seagull}

\setbeamertemplate{footline}{\scriptsize{\hspace*{0.4cm}\insertframenumber}\vspace*{0.3cm}}
\beamertemplatenavigationsymbolsempty

\errorcontextlines 10000

\begin{document}
\title{\Large{Harmony Theory, \\Gradient Symbol Processing}}
\author{Константин Соколов}
\institute[]
{Mathlingvo, СПбГУ, Eventflow API\\ \bigskip  \url{http://nlu-rg.ru}}
\date{Санкт-Петербург, 2014} 
% Создание заглавной страницы
\begin{frame}
    \thispagestyle{empty}
    \titlepage
\end{frame}

\begin{frame}{Тексты}
\setcounter{framenumber}{1}
\begin{itemize}
	\item P. Smolensky (1986) Information Processing in Dynamical Systems: Foundations of Harmony Theory.
    \medskip
    \item A. Prince and P. Smolensky (1993) Optimality Theory.
    \item P. Smolensky et al. (2014) Gradient Symbol Processing.
\end{itemize}
\end{frame}

\begin{frame}{План}
    \begin{itemize}
        \item Предыстория: психолингвистика Хомского-Миллера и \\порождающая фонология Хомского-Халле
        \medskip
        \item Коннекционизм и гибридные когнитивные архитектуры
        \medskip
        \item Harmonic Grammar vs. Теория оптимальности
        \medskip
        \item Gradient Symbol Processing
    \end{itemize}
\end{frame}

1. Хомский-Халле, SPE, дистинктивные признаки, правила переписывания, глубинная и поверхностная фонологическая форма, проблема порядка применения правил, затемненность, абстрактность (критика, Кипарский)
2. Хомский-Миллер, анализ через синтез, психологическое обоснование -- время обработки неполного сигнала
3. Селфридж (1958?), Pandemonium, роль признаков (демоны признаков), связь с биологией зрения
4. Начало коннекционизма: перцептрон (1968?), нейронные сети
5. Harmony Theory: subsymbolic paradigm; необходимость совместить высокоуровневые когнитивные функции и восприятие, параллельная обрабока; Harmonium; обучение
6. Harmonic Grammar - P. Smolensy, G. Legendre, Y. Miyata (1990)  A Formal Multi-Level Connectionist Theory of Linguistic Well-Formedness: Theoretical Foundations.; HG Course at UMass (http://blogs.umass.edu/hgcourse/); переход от правил к системам ограничений (ср. с другими теориями в лингвистике, можно сослаться на Зубрицкую; constraint grammars в унификационных грамматиках, принципы и параметры, минимализм и пр.), well-formedness
7. Optimality Theory: отличие от HG -- в OT ранжирование ограничений, в HG взвешенные ограничения; плюсы и минусы; tableaux; пример анализа слоговой структуры в OT; факториальная типология; нетривиальные эффекты как следствие архитектуры: TETU и conspiracy
8. Gradient Symbol Processing: объединение символьных и статистических подходов, геометрия пространства (аттракторы)

\begin{frame}{}
\begin{center}
	\textbf{Часть первая}
\end{center}
\end{frame}

\begin{frame}{Часть первая}
\begin{itemize}
	\item 
	\medskip
	\item 
	\medskip
	\item  
\end{itemize}
\end{frame}




\begin{frame}{}
    \thispagestyle{empty}
    \begin{center}
        {\large Спасибо!}
    \end{center}
\end{frame}


\end{document}
