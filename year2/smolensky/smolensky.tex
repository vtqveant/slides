\documentclass{beamer}

\usepackage[T2A]{fontenc}
\usepackage[utf8]{inputenc}
\usepackage[english,russian]{babel}
\usepackage{amssymb,amsfonts,amsmath,mathtext}
\usepackage{cite,enumerate,float,indentfirst}

\usepackage{graphicx}
\usepackage{booktabs}
\usepackage{tabularx}

%% Gentzen style natural deduction proof trees
\usepackage{bussproofs}
\usepackage{latexsym}

\graphicspath{{images/}}

\usetheme{Rochester}
\usecolortheme{seagull}

\setbeamertemplate{footline}{\scriptsize{\hspace*{0.4cm}\insertframenumber}\vspace*{0.3cm}}
\beamertemplatenavigationsymbolsempty

\errorcontextlines 10000

\begin{document}
\title{\Large{Harmony Theory and ICS}}
\author{Константин Соколов}
\institute[]
{Mathlingvo, СПбГУ, Eventflow API\\ \bigskip  \url{http://nlu-rg.ru}}
\date{Санкт-Петербург, 2014} 
% Создание заглавной страницы
\begin{frame}
    \thispagestyle{empty}
    \titlepage
\end{frame}

\begin{frame}{Тексты}
\setcounter{framenumber}{1}
\begin{itemize}
	\item P. Smolensky (1986) Information Processing in Dynamical Systems: Foundations of Harmony Theory.
    \medskip
    \item P. Smolensky, G. Legendre and Y. Miyata (1990) Harmonic grammar: A formal multi-level connectionist theory of linguistic well-formedness: Theoretical foundations.
    \medskip
    \item A. Prince and P. Smolensky (1993) Optimality Theory.
    \medskip
    \item P. Smolensky et al. (2014) Gradient Symbol Processing.
\end{itemize}
\end{frame}

\begin{frame}{План}
    \begin{itemize}
        \item Предыстория: психолингвистика Хомского-Миллера и \\порождающая фонология Хомского-Халле
        \medskip
        \item Коннекционизм и гибридные когнитивные архитектуры
        \medskip
        \item Harmonic Grammar vs. Теория оптимальности
        \medskip
        \item Gradient Symbol Processing
    \end{itemize}
\end{frame}

% 5. Harmony Theory: subsymbolic paradigm; необходимость совместить высокоуровневые когнитивные функции и восприятие, параллельная обрабока; Harmonium; обучение
% 6. Harmonic Grammar - P. Smolensy, G. Legendre, Y. Miyata (1990)  A Formal Multi-Level Connectionist Theory of Linguistic Well-Formedness: Theoretical Foundations.; HG Course at UMass (http://blogs.umass.edu/hgcourse/); переход от правил к системам ограничений (ср. с другими теориями в лингвистике, можно сослаться на Зубрицкую; constraint grammars в унификационных грамматиках, принципы и параметры, минимализм и пр.), well-formedness
% 7. Optimality Theory: отличие от HG -- в OT ранжирование ограничений, в HG взвешенные ограничения; плюсы и минусы; tableaux; пример анализа слоговой структуры в OT; факториальная типология; нетривиальные эффекты как следствие архитектуры: TETU и conspiracy
% 8. Integrated Connectionist/Symbolic Cognitive Architecture (ICS)
% 9. Gradient Symbol Processing: объединение символьных и статистических подходов, геометрия пространства (аттракторы)

\begin{frame}{}
\begin{center}
	\textbf{Генеративная фонология и психолингвистика}
\end{center}
\end{frame}

\begin{frame}{Пандемониум - I}
\textit{Oliver Selfridge (1959)}, модель распознавания образов (букв)
\begin{center}
	\begin{figure}[H]
		\includegraphics[scale=0.5]{pandemonium_small.png} 
	\end{figure}
\end{center}
\bigskip
\tiny{П. Линдсей, Д. Норман, ``Переработка информации у человека'', М., 1974 -- стр. 129.}
\end{frame}

\begin{frame}{Пандемониум - II}
\begin{itemize}
	\item демоны узнавания изображения (регистрация сигнала)
	\medskip
	\item демоны выделения признаков (линии, углы, контуры и т.п.)
	\medskip
	\item демоны опознавания (отдельные буквы и пр.)
	\medskip
	\item демон принятия решения
\end{itemize}
\end{frame}

\begin{frame}{Пандемониум - III}
Биологическая обоснованность модели:
\bigskip
\begin{itemize}
	\item детекторы углов, линий и т.п. в нейронных структурах у животных
	\medskip
	\item обучение и влияние контекста
\end{itemize}
\end{frame}

\begin{frame}{Пандемониум - IV}
Сенсорная информация преобразуется в набор признаков (feature vector)
\bigskip
\begin{itemize}
	\item  отдельную букву кодирует лишь подмножество признаков
	\medskip
	\item добавление новых распознаваемых объектов может потребовать изменения набора признаков
	\medskip
	\item  ошибки ``несимметричны'' -- некоторые объекты распознавать проще, чем другие (задействуется больше признаков, проще принять решение)
\end{itemize}
\end{frame}

\begin{frame}{Пандемониум - V}
Мы свободны в своём выборе признаков при условии, что
\bigskip
\begin{itemize}
	\item совокупность признаков оказывается \\максимально простой
    \medskip
	\item выбор признаков гарантирует от ошибок
	\medskip
	\item признаки поддаются анализу
\end{itemize}
\end{frame}

\begin{frame}{Генеративная фонология - I}
\textit{Stevens, Kenneth and Morris Halle (1962) Speech Recognition: A Model and a Program for Research.}
\bigskip
\begin{itemize}
	\item модель распознавания речи на основе дистинктивных\\ признаков (экспертный подход ``от фонологии'')
	\medskip
	\item моторная концепция восприятии речи (подавление артикуляторной программы при восприятии)
    \medskip
	\item концепция ``анализа через синтез'' 
\end{itemize}
\end{frame}

\begin{frame}{Генеративная фонология - II}
\textit{Chomsky, Noam and Morris Halle (1968) Sound Pattern of English.}
\bigskip
\begin{itemize}
	\item концепция фонемы как пучка дифференциальных признаков заимствуется из Пражской школы
	\medskip
	\item концепция универсальной грамматики, общая для генеративизма
	\medskip 
	\item фонологический компонент преобразует глубинное (абстрактное фонологическое) представление слова в поверхностное (фонетическое)
\end{itemize}
\end{frame}

\begin{frame}{Генеративная фонология - III}
\begin{center}
	\begin{figure}[H]
		\includegraphics[scale=0.32]{phonemes.png} 
	\end{figure}
\end{center}
\bigskip
\tiny{Э. Черри, М. Халле, Р. Якобсон ``К вопросу о логическом описании языков в их фонологическом аспекте'', в кн.: ``Новое в лингвистике'', вып. II, М., 1962, стр. 286. (по П. Линдсей, Д. Норман, ``Переработка информации у человека'', М., 1974, стр. 142.)}
\end{frame}

\begin{frame}{Генеративная фонология - IV}
Устройство фонологического компонента: 
\bigskip
\begin{itemize}
	\item контекстно-зависимые правила переписывания вида $X \to Y / A \_ B$ (заменить $X$ на $Y$ в позиции между $A$ и $B$)
	\medskip
	\item правила манипулируют дифференциальными признаками
	\medskip
	\item процесс порождения поверхностной формы как последовательное применение правил
\end{itemize}
\begin{center}
	\begin{figure}[H]
		\includegraphics[scale=0.5]{alveolar_flap.png} 
	\end{figure}
\end{center}
\end{frame}

\begin{frame}{Генеративная психолингвистика - I}
\textit{George Miller (1920-2012)}
\bigskip
\begin{itemize}
	\item ``Language and Communication'' (1951)
	\medskip
	\item ``The Magical Number Seven, Plus or Minus Two: Some Limits on our Capacity for Processing Information'' (1956)
	\medskip
	\item совместные работы с Н. Хомским
	\medskip
	\item руководил созданием WordNet (1986)
\end{itemize}
\end{frame}

\begin{frame}{Генеративная психолингвистика - II}
Эксперименты Миллера:
\bigskip
\begin{itemize}
	\item распознавание зашумленных слов
	\medskip
	\item скорость реакции
	\medskip
	\item влияние избыточности языка на понимание
	\medskip
	\item роль внимания и краткосрочной памяти
\end{itemize}
\bigskip
Экспериментальное подтверждение концепции анализа через синтез, поддержка генеративизма \textit{(``американская психолингвистика второго поколения'' -- A. A. Леонтьев)}
\end{frame}

\begin{frame}{Анализ через синтез - I}
\begin{itemize}
	\item в сенсорной системе:
		\begin{itemize}
			\item реакция на внешние события
			\smallskip
			\item преобразование сенсорной информации в набор признаков
		\end{itemize}
	\medskip
	\item в памяти:
		\begin{itemize}
			\item запись прошлых событий
			\smallskip
			\item предположения, необходимые для интерпретации
		\end{itemize}
	\medskip
	\item распознавние осуществляется путем сопоставления сенсорных данных с данными в памяти с учетом \textit{ожидания} предстоящих сенсорных событий
	\item процесс активного синтеза: ожидания непрерывно пересматриваются по мере обработки сигнала
\end{itemize}
\end{frame}

\begin{frame}{Анализ через синтез - II}
Aктивный cинтез позволяет:
\bigskip
\begin{itemize}
	\item ограничить набор анализируемой информации в случаях, когда она хорошо согласуется с ожиданием, тем самым обработка ускоряется (но противоречие сенсорных данных ожиданиям увеличивает время обработки)
	\medskip
	\item восстанавливать недостающие части сигнала в соответствии с ожиданиями (внутренней моделью, основанной на опыте)
\end{itemize}
\end{frame}

\begin{frame}{Анализ через синтез - III}
\begin{itemize}
	\item анализ не обязательно протекает последовательно по мере поступления сигнала (возможность параллельной обработки)
	\medskip
	\item анализ может привлекать данные из различных источников (разные сенсорные каналы и детекторы, память, логический вывод и пр.) и комбинировать их
	\item потребности активного синтеза налагают требования на функции памяти и познания: необходимость кратковременной (оперативной) памяти
\end{itemize}
\end{frame}

\begin{frame}{Harmony Theory - I}
Harmony Theory как когнитивная архитектура:
\bigskip
\begin{itemize}
	\item представляет собой вычислительную модель когнитивного аппарата, опирающуюся на концепцию анализа через синтез
	\medskip
	\item согласуется с экспериментами
\end{itemize}
\end{frame}

\begin{frame}{Harmony Theory - II}
\begin{itemize}
	\item умеет работать по неполным и избыточным данным
	\medskip
	\item позволяет выполнять эксперименты с вычислительной моделью
	\medskip
	\item позволяет делать предсказания
	\medskip
	\item позволяет ставить обоснованные вопросы о природе когнитивного аппарата человека
\end{itemize}
\end{frame}

\begin{frame}{Harmonic Grammar and Optimality Theory}
\begin{center}
	\begin{figure}[H]
		\includegraphics[scale=0.40]{harmonic.jpg} 
	\end{figure}
\end{center}
\end{frame}

\begin{frame}{}
    \thispagestyle{empty}
    \begin{center}
        {\large Спасибо!}
    \end{center}
\end{frame}


\end{document}
