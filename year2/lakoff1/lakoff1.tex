\documentclass{beamer}

\usepackage[T2A]{fontenc}
\usepackage[utf8]{inputenc}
\usepackage[english,russian]{babel}
\usepackage{amssymb,amsfonts,amsmath,mathtext}
\usepackage{cite,enumerate,float,indentfirst}

\usepackage{graphicx}
\usepackage{booktabs}
\usepackage{tabularx}

%% Gentzen style natural deduction proof trees
\usepackage{bussproofs}
\usepackage{latexsym}

\graphicspath{{images/}}

\usetheme{Rochester}
\usecolortheme{seagull}

\setbeamertemplate{footline}{\scriptsize{\hspace*{0.4cm}\insertframenumber}\vspace*{0.3cm}}
\beamertemplatenavigationsymbolsempty

\errorcontextlines 10000

\begin{document}
\title{\Large{Когнитивная лингвистика, ч. I}}
\subtitle{\textit{критика формальной семантики}}
\author{Константин Соколов}
\institute[]
{Mathlingvo, СПбГУ, Eventflow API\\ \bigskip  \url{http://nlu-rg.ru}}
\date{Санкт-Петербург, 2014} 
% Создание заглавной страницы
\begin{frame}
    \thispagestyle{empty}
    \titlepage
\end{frame}

\begin{frame}{Текст}
\setcounter{framenumber}{1}
\begin{itemize}
	\item Дж. Лакофф. Женщины, огонь и опасные вещи. 1985 г.
	    \medskip
	    \begin{itemize}
	        \item глава 14. Формалистское предприятие
    	    \medskip
	        \item глава 15. Теорема Патнэма
	    \end{itemize}
\end{itemize}
\end{frame}

\begin{frame}{План}
    \begin{itemize}
        \item Краткая история формальной семантики
        \medskip
        \item Критика формалистского проекта
        	\smallskip
        	\begin{itemize}
				\item невозможность объективистского подхода в семантике
				\smallskip
				\item ``бомба'' X. Патнэма
        	\end{itemize}
    \end{itemize}
\end{frame}

\begin{frame}{}
\begin{center}
	\textbf{Краткая история формальной семантики}
\end{center}
\end{frame}

\begin{frame}{Краткая история формальной семантики}
\begin{itemize}
	\item Основы лингвистической формальной семантики были заложены в поздних работах Р. Монтегю в начале 70-х гг.
	\medskip
	\item Основа лингвистических конструкций Монтегю --\\ т.н. интенсиональная логика
	\medskip
	\item Монтегю опирается на предшествующие разработки таких авторов, как Р. Карнап и А. Чёрч
\end{itemize}
\end{frame}

\begin{frame}{Интенсиональная логика -- I}
\begin{itemize}
	\item \textit{Интенсиональная логика} -- это логика, содержащая в объектном языке имена интенсиональных сущностей
	\medskip
	\item Выражаясь нестрого, \textit{интенсионал} можно описать как название или наименование некоторого мыслимого содержания
	\medskip
	\item Интенсиональная логика позволяет формулировать утверждения об \textit{интенсиональных сущностях (интенсионалиях)}, выступающих денотатами интенсиональных понятий
	    \medskip
	    {\small
	    \begin{itemize}
	        \item Примеры: \textit{индивидный концепт, пропозиция}
	    \end{itemize}
	    }
\end{itemize}
\end{frame}

\begin{frame}{Интенсиональная логика -- II}
\begin{itemize}
    \item Косвенный контекст
    	\medskip
	    \begin{itemize}
	        \item \texttt{Иван думает, что снег бел}
	        \smallskip
	        \item \texttt{Иван думает, что фраза ``снег бел'' соответствует положению дел}
	    \end{itemize}    
	\medskip
    \item ``Навешивание'' квантора существования
    	\medskip
	    \begin{itemize}
	        \item \texttt{Иван сомневается, что снежный человек существует}
	        \smallskip
	        \item $^?$\texttt{Существует некто, в чьём существовании Иван сомневается}
	    \end{itemize}   	
\end{itemize}
\end{frame}

\begin{frame}{Интенсиональная логика -- II}
\begin{itemize}
    \item Референтная непрозрачность
    	\medskip
	    \begin{itemize}
	        \item \texttt{Мери полагает, что самый популярный ветеринар в Восточном Иллинойсе — мой друг}
	        \smallskip
	        \item \texttt{Самый популярный ветеринар в Восточном Иллинойсе -- Денвилл старший, страдающий от болезней ног}
	        \smallskip
	        \item \texttt{Мери полагает, что Денвилл старший, страдающий от болезней ног,— мой друг}
	    \end{itemize}   	
    \medskip
	\item Необходимо иметь возможность
    	\medskip
	    \begin{itemize}
	        \item различать мыслимое содержание и его выражение;
	        \item анализировать употребление в различных контекстах.
	    \end{itemize}   	
\end{itemize}
\end{frame}

\begin{frame}{Бикомпонентная семантика Г. Фреге}
\begin{itemize}
    \item Смысл и денотат
	\medskip
	\item Принцип взаимозаменяемости тождественных и косвенные контексты
	\medskip
	\item Принцип коэкстенсивности логически эквивалентных выражений
	\medskip
	\item Принцип композициональности
\end{itemize}
\end{frame}

\begin{frame}{Интенсиональная иерархия Р. Карнапа}
\begin{itemize}
    \item Метод интенсионала и экстенсионала
	\medskip
	\item Интенсиональная иерархия
\end{itemize}
\end{frame}

\begin{frame}{Логика смысла и денотата А. Чёрча -- I}

\begin{itemize}
    \item Простая типизация для $\lambda$-исчисления
    \medskip
    \item Моделирование интенсиональной иерархии средствами теоретико-типового формализма
   	\medskip
         \begin{itemize} 
	        \item базовые типы: $o$ -- истинностные значения, $\iota$ -- индивиды
	        \item функциональные типы: $\alpha_n \beta_m$, напр. $o \iota$ -- одноместные предикаты, $o(o \iota)$ -- дескрипции
	        \item интенсиональные типы: $o_1$ -- пропозиции, $\iota_1$ -- индивидные концепты, $o_2$ -- пропозициональные концепты, $\iota_2$ -- имена индивидных концептов, $oo_1$ -- тип модального оператора
	        \item трансфинитный интенсиональный тип $o_\omega$
	    \end{itemize}
	\medskip
	\item Процедура интенсионального восхождения
	\medskip
	\item Двухместный предикат $\Delta$ (``быть концептом'')
\end{itemize}
\end{frame}

\begin{frame}{Логика смысла и денотата А. Чёрча -- II}
\begin{itemize}
    \item ``Альтернативы''
    \medskip
        \begin{itemize}
            \item Альтернатива 0: смыслы различаются с точностью до $\alpha$-конверсии
            \item Альтернатива 1: смыслы различаются с точностью до $\beta$-редукции
            \item Альтернатива 2: смысл логически эквивалентных выражений совпадает
        \end{itemize}
    \item Майхилл показал (1958), что система Чёрча противоречива, и предложил свой бестиповый вариант интенсиональной логики
\end{itemize}
\end{frame}

\begin{frame}{Интенсиональная логика Р. Монтегю -- I}
\begin{itemize}
    \item Прагматика и интенсиональная логика (60-е гг.)
    \medskip
    \item Функциональная ($\lambda$) и интенсиональная (\,$\hat{} \, \alpha$) абстракция
    \medskip 
    \item Проблема ``псевдотипа'' $s$
    \medskip
    \item Семантика Монтегю-Скотта    
\end{itemize}
\end{frame}

\begin{frame}{Интенсиональная логика Р. Монтегю -- II}
\begin{itemize}
    \item Интенсиональная логика как ``семантический метаязык''
    \medskip
    \item Универсальная грамматика
    \medskip 
    \item PTQ Framework
\end{itemize}
\end{frame}

\begin{frame}{Формальные системы Д. Галлина}
\begin{itemize}
    \item Альтернативные формулировки ИЛ Монтегю
    \medskip
        \begin{itemize}
            \item устранение интенсиональной и функциональной абстракции
            \item устранение $s$
            \item чисто модальная система (высшего порядка)
        \end{itemize}            
    \medskip
    \item Семантика для интенсиональной логики
    \medskip
        \begin{itemize}
            \item окрестностная семантика (семантика Монтегю-Скотта)
            \item топологическая семантика Тарского-МакКинси
            \item алгебраическая семантика (Boolean Semantics)
        \end{itemize}            
\end{itemize}
\end{frame}

\begin{frame}{Поздние варианты топологической семантики}
\begin{itemize}
    \item Kripke-Joyal Semantics, Sheaf Semantics
    \medskip
    \item Семантика для FOS4 на основе теории пучков (Аводей-Кисида)
    \medskip
    \item Семантика для DRS на основе теории пучков (Абрамский-Садрзаде)
\end{itemize}
\end{frame}

\begin{frame}{}
\begin{center}
	\textbf{Критика формальной семантики}
\end{center}
\end{frame}

\begin{frame}{Критика формальной семантики -- I}
\begin{center}
	\begin{figure}[H]
		\includegraphics[scale=0.39]{cat_mat.jpg} 
	\end{figure}
\end{center}
\end{frame}

\begin{frame}{Критика формальной семантики -- II}
\begin{itemize}
    \item Принципы формальной семантики
    \medskip
        \begin{itemize}
            \item автономный синтаксис
            \medskip
            \item модельные множества и структуры
            \medskip
            \item принципы отображения синтаксиса на модельные структуры
        \end{itemize}
    \medskip
    \item Но можно ли построить на этой основе \textit{теорию значения?}
\end{itemize}
\end{frame}

\begin{frame}{}
\begin{center}
	\textbf{Теорема Х. Патнэма}
\end{center}
\end{frame}

\begin{frame}{Основное свойство теории значения -- I}
\begin{itemize}
	\item Положения, что (1) семантика характеризует отношение символов к сущностям в мире, и что (2) семантика характеризует значение -- несовместимы
	\item Чтобы теория значения была обоснованной, необходимо, чтобы значение сложного выражения зависело от ``элементарных референций'' (ср. с принципом композициональности)
\end{itemize}
\end{frame}

\begin{frame}{Основное свойство теории значения -- II}
\begin{itemize}	
    \item Если ``элементарные референции'' можно произвольно менять и это не влияет на значение сложных выражений, то либо это не теория значения, либо она не связана с реальностью (положением вещей в мире)
	\item Если это условие нарушается для любого предложения формального языка, будем говорить о \textit{сильной неопределенности}, если хотя бы для одного -- о \textit{слабой неопределенности}
	\item Х. Патнэм атакует формальную семантику, показывая неустранимость слабой неопределенности
\end{itemize}
\end{frame}

\begin{frame}{Теорема Лёвенгейма-Сколема о понижении мощности}
\begin{itemize}
	\item[] Если исчислимая совокупность предложений в языке первого порядка имеет модель, то она имеет счетную модель
	\bigskip
	\item \textit{Комментарии Лакоффа:} в теории моделей нельзя построить такой набор предложений, который описывает только то, что вы хотите (т.е. ``паразитные'' интерпретации неустранимы)
	\item \textit{Важное замечание:} т. Л-С относится только к FOL, но Патнэм (и Лакофф) используют её \textit{для иллюстрации}, а не для доказательства теорем
\end{itemize}
\end{frame}

\begin{frame}{Неустранимость неопределенности -- I}
\texttt{``кошка находится на коврике''}
\bigskip
\begin{itemize}
        \item термы: $C, M, A, T, C^*, M^*$
        \smallskip
        \item денотаты: $cat, mat, apple, tree, quark$
        \smallskip
        \item $d(C) = cat, d(M) = mat, d(A) = apple, d(T) = tree$
        \smallskip
        \item если $d(C) = d(C^*)$, то $C$ и $C^*$ -- коэкстенсивные термы
        \smallskip
        \item предложения: $on(C,M)$, $on(A,T)$
\end{itemize}
\end{frame}

\begin{frame}{Неустранимость неопределенности -- II}
\texttt{``кошка находится на коврике''}
\bigskip
\begin{itemize}
	\item Мы хотим подобрать значения $d(C^*)$ и $d(M^*)$ таким образом, чтобы:
	    \smallskip
	    \begin{itemize}
	        \item $on(C, M) = on(C^*, M^*)$
	        \smallskip
	        \item $d(C) \neq d(C^*)$ и/или $d(M) \neq d(M^*)$ хотя бы в одном из возможных миров
	    \end{itemize}
\end{itemize}
\end{frame}

\begin{frame}{Неустранимость неопределенности -- III}
\texttt{``кошка$^*$ находится на коврике$^*$''}
\bigskip
\begin{table}
    \begin{tabular}{|l|c|c|}
        \hline
        ~                & $on(C,M) = \top$                 & $on(C,M) = \bot$                  \\ \hline
        $on(A,T) = \top$ & \parbox[t]{3cm}{$d(C^*) = apple$\\ $d(M^*) = tree$} & \parbox[t]{3cm}{$d(C^*) = apple$\\ $d(M^*) = quark$} \\ 
        $on(A,T) = \bot$ & \parbox[t]{3cm}{$d(C^*) = cat$\\ $d(M^*) = mat$} & \parbox[t]{3cm}{$d(C^*) = apple$\\ $d(M^*) = tree$} \\
        \hline
    \end{tabular}
\end{table}
\bigskip
Всегда $on(C,M) = on(C^*,M^*)$, но в трех случаях из четырех $d(C) \neq d(C^*)$ и $d(M) \neq d(M^*)$
\end{frame}

\begin{frame}{Неустранимость неопределенности -- IV}
\texttt{``кошка$^*$ находится на коврике$^*$''}
\bigskip
\begin{itemize}
	\item Заменим $C$ на $C^*$, $M$ на $M^*$ (напр., пусть в возможном мире, где яблоко на дереве, а кошка на коврике, слово ``кошка'' обозначает яблоко)
	\item Референция поменялась, а значение предложения -- нет
\end{itemize}
\end{frame}

\begin{frame}{Выводы}
\begin{itemize}
    \item Теория моделей не может служить в качестве теории значения, т.к. неопределенность референции неустранима
    \item Незначимые структуры не могут дать значение незначимым символам
    \item Концепция автономного синтаксиса может быть подвергнута сомнению
    \item Объективно правильное описание референции невозможно
    \item Необходим отказ от \textit{экстерналистской} перспективы (человек вне мира)
\end{itemize}
\end{frame}


\begin{frame}{}
    \thispagestyle{empty}
    \begin{center}
        {\large Спасибо!}
    \end{center}
\end{frame}


%%% слайд помещается сюда
%% \begin{frame}{Заголовок}
%% \end{frame}

\end{document}
