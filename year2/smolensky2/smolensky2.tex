\documentclass{beamer}

\usepackage[T2A]{fontenc}
\usepackage[utf8]{inputenc}
\usepackage[english,russian]{babel}
\usepackage{amssymb,amsfonts,amsmath,mathtext}
\usepackage{cite,enumerate,float,indentfirst}

\usepackage{graphicx}
\usepackage{booktabs}
\usepackage{tabularx}

%% Gentzen style natural deduction proof trees
\usepackage{bussproofs}
\usepackage{latexsym}

\graphicspath{{images/}}

\usetheme{Rochester}
\usecolortheme{seagull}

\setbeamertemplate{footline}{\scriptsize{\hspace*{0.4cm}\insertframenumber}\vspace*{0.3cm}}
\beamertemplatenavigationsymbolsempty

\errorcontextlines 10000

\begin{document}
\title{\Large{Harmony Theory and ICS}}
\subtitle{\small{часть вторая}}
\author{Константин Соколов}
\institute[]
{Mathlingvo, СПбГУ, Eventflow API\\ \bigskip  \url{http://nlu-rg.ru}}
\date{Санкт-Петербург, 2014} 
% Создание заглавной страницы
\begin{frame}
    \thispagestyle{empty}
    \titlepage
\end{frame}

\begin{frame}{Тексты}
\setcounter{framenumber}{1}
\begin{itemize}
    \item P. Smolensky, G. Legendre and Y. Miyata (1990) Harmonic grammar: A formal multi-level connectionist theory of linguistic well-formedness: Theoretical foundations.
    \medskip
    \item A. Prince and P. Smolensky (1993) Optimality Theory.
    \medskip
	\item P. Smolensky, G. Legendre (2006) Hamonic Mind.
    \medskip
    \item P. Smolensky et al. (2014) Gradient Symbol Processing.
\end{itemize}
\end{frame}

\begin{frame}{План}
    \begin{itemize}
        \item Harmonic Grammar vs. Теория оптимальности
        \medskip
        \item ICS, Gradient Symbol Processing
    \end{itemize}
\end{frame}

% 5. Harmony Theory: subsymbolic paradigm; необходимость совместить высокоуровневые когнитивные функции и восприятие, параллельная обрабока; Harmonium; обучение
% 6. Harmonic Grammar - P. Smolensy, G. Legendre, Y. Miyata (1990)  A Formal Multi-Level Connectionist Theory of Linguistic Well-Formedness: Theoretical Foundations.; HG Course at UMass (http://blogs.umass.edu/hgcourse/); переход от правил к системам ограничений (ср. с другими теориями в лингвистике, можно сослаться на Зубрицкую; constraint grammars в унификационных грамматиках, принципы и параметры, минимализм и пр.), well-formedness
% 7. Optimality Theory: отличие от HG -- в OT ранжирование ограничений, в HG взвешенные ограничения; плюсы и минусы; tableaux; пример анализа слоговой структуры в OT; факториальная типология; нетривиальные эффекты как следствие архитектуры: TETU и conspiracy
% 8. Integrated Connectionist/Symbolic Cognitive Architecture (ICS)
% 9. Gradient Symbol Processing: объединение символьных и статистических подходов, геометрия пространства (аттракторы)

\begin{frame}{}
\begin{center}
	\textbf{Harmonic Grammar}
\end{center}
\end{frame}

\begin{frame}{Динамическая система - I}
\begin{center}
	\begin{figure}[H]
		\includegraphics[scale=0.7]{harmonic1.png} 
	\end{figure}
\end{center}
\bigskip
Динамика определяется системой уравнений\\
\medskip
$\frac{da_1}{dt} = -a_1 + w_1 - \lambda a_2$, $\frac{da_2}{dt} = -a_2 + w_2 - \lambda a_1$,\\
\medskip
максимизирующих $H(\textbf{a}) = a_1 w_1 + a_2 w_2 - \lambda a_1 a_2 - \frac{1}{2}(a_1^2 + a_2^2)$
\end{frame}

\begin{frame}{Динамическая система - II}
\begin{center}
	\begin{figure}[H]
		\includegraphics[scale=0.7]{harmonic2.png} 
	\end{figure}
\end{center}
\bigskip
Вектор активации $\textbf{a} = (a_1, a_2)$ стремится к $(0,79; -0,71)$
\end{frame}

\begin{frame}{Динамическая система - III}
\begin{center}
	\begin{figure}[H]
		\includegraphics[scale=0.7]{harmonic3.png} 
	\end{figure}
\end{center}
\bigskip
Эволюция системы в фазовом пространстве\\
\textit{(здесь двумерная проекция, т.к. входные значения постоянны)}
\end{frame}

\begin{frame}{Динамическая система - IV}
\begin{center}
	\begin{figure}[H]
		\includegraphics[scale=0.8]{harmonic4.png} 
	\end{figure}
\end{center}
\bigskip
\textit{Оптимизация как максимизация гармоничности}
\end{frame}

\begin{frame}{Система ограничений}
Динамика системы определяется набором \textit{конфликтующих} взвешенных ограничений:\\
\medskip
\begin{itemize}
	\item Узел №1 должен быть активен ($w_1 = 0,6$)
	\medskip
	\item Узел №2 должен быть активен ($w_2 = 0,5$)
	\medskip
	\item Узлы №1 и №2 не должны быть активны одновременно ($\lambda = -0,9$)
\end{itemize}
\bigskip
\textit{Оптимизация как достижение компромисса}
\end{frame}

\begin{frame}{Harmonic Grammar - I}
Грамматика задается как\\
\bigskip
\begin{itemize}
	\item набор ограничений $C_k$, $k = 1 \dots K$
	\medskip
	\item веса $w_k \in \mathbb{R}$ для каждого ограничения
	\medskip
	\item функция гармоничности $H = \sum_{k=1}^{K} s_k w_k$, где $s_k$ -- оценка (штраф) для каждого из ограничений
\end{itemize}
\end{frame}

\begin{frame}{Harmonic Grammar - II}
\begin{center}
	\begin{figure}[H]
		\includegraphics[scale=0.6]{hg_tableau1.png} 
	\end{figure}
\end{center}
\medskip
Оглушение в конце слова как эффект условия $w(^{*}\textsc{Coda-Voice}) > w(\textsc{Ident-Voice})$.
\end{frame}

\iffalse

\begin{frame}{Критика}
\begin{itemize}
	\item некорректные типологические предсказания
	\medskip
	\item 
	\medskip
	\item 
\end{itemize}
\end{frame}

\fi

\begin{frame}{}
\begin{center}
	\textbf{Optimality Theory}
\end{center}
\end{frame}

\begin{frame}{Несколько фактов}
\begin{itemize}
	\item A. Prince and P. Smolensky. Optimality Theory: Constraint Satisfaction in Generative Grammar.
		\medskip
		\begin{itemize}
			\item опубликовано в электронной форме в 1993 г.,\\в печатном виде в 2004 г.
		\end{itemize}
	\medskip
	\item Rutgers Optimality Archive (ROA) -- архив препринтов по теории оптимальности
		\medskip
		\begin{itemize}
			\item сейчас более 1200 работ
		\end{itemize}
\end{itemize}
\end{frame}


\begin{frame}{Архитектура - I}
\begin{itemize}
	\item компонент порождения форм \textit{(Gen)}
	\medskip
	\item набор ограничений \textit{(Con)}
	\medskip
	\item компонент оценки оптимальности \textit{(Eval)}
\end{itemize}
\end{frame}

\begin{frame}{Архитектура - II}
\begin{itemize}
	\item ограничения в OT упорядочены по отношению \textit{строгого доминирования} (весов нет)
	\medskip
	\item ограничения разбиты на два класса
		\medskip
		\begin{itemize}
			\item ограничения маркированности \textit{(markedness)} -- штрафуют определенные конфигурации в поверхностных формах
			\medskip
			\item ограничения ``верности'' \textit{(faithfulness)} -- штрафуют отклонения поверхностных форм от глубинных
		\end{itemize}
	\medskip
	\item ограничения считаются универсальными для всех языков, различия в грамматике связаны только с порядком ограничений
\end{itemize}
\end{frame}

\begin{frame}{Приложения}
\begin{itemize}
	\item Анализ слоговой структуры
	\medskip
	\item Фонотактика и просодическая морфология
	\medskip
	\item Факториальная типология и универсальная грамматика
\end{itemize}
\end{frame}


\begin{frame}{}
\begin{center}
	\textbf{Integrated Connectionist-Symbolic (ICS) Architecture}
\end{center}
\end{frame}

\iffalse

\begin{frame}{ICS Architecture - I}
локальные и распределенные репрезентации в коннекционистских архитектурах
\bigskip
\begin{itemize}
	\item 
	\medskip
	\item 
\end{itemize}
\end{frame}

\begin{frame}{ICS Architecture - II}
аргументы в пользу распределенных представлений
\bigskip
\begin{itemize}
	\item 2 эмпирических
	\medskip
	\item 2 теоретических
\end{itemize}
\end{frame}

\begin{frame}{ICS Architecture - III}
структурная (символьная) стратегия решения задач
\bigskip
\begin{itemize}
	\item пример -- анализ предложения как структурная декомпозиция
	\medskip
	\item обсуждение -- статус словаря в пр.
\end{itemize}
\end{frame}

\begin{frame}{ICS Architecture - IV}
Статус символов в ICS
\bigskip
\begin{itemize}
	\item 
	\medskip
	\item 
\end{itemize}
\end{frame}

\begin{frame}{ICS Architecture - V}
микроуровень и макроуровень
\bigskip
\begin{itemize}
	\item 
	\medskip
	\item 
\end{itemize}
\end{frame}

\fi

\begin{frame}{ICS Architecture - I}
\begin{center}
	\begin{figure}[H]
		\includegraphics[scale=0.4]{ics1.png} 
	\end{figure}
\end{center}
\end{frame}

\begin{frame}{ICS Architecture - II}
\begin{center}
	\begin{figure}[H]
		\includegraphics[scale=0.33]{ics2.png} 
	\end{figure}
\end{center}
\end{frame}

\begin{frame}{}
    \thispagestyle{empty}
    \begin{center}
        {\large Спасибо!}
    \end{center}
\end{frame}


\end{document}
