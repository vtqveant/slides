\documentclass{beamer}

\usepackage[T2A]{fontenc}
\usepackage[utf8]{inputenc}
\usepackage[english,russian]{babel}
\usepackage{amssymb,amsfonts,amsmath,mathtext}
\usepackage{cite,enumerate,float,indentfirst}

\usepackage{graphicx}
\usepackage{booktabs}
\usepackage{tabularx}

%% Gentzen style natural deduction proof trees
\usepackage{bussproofs}
\usepackage{latexsym}

\graphicspath{{images/}}

\usetheme{Rochester}
\usecolortheme{seagull}

\setbeamertemplate{footline}{\scriptsize{\hspace*{0.4cm}\insertframenumber}\vspace*{0.3cm}}
\beamertemplatenavigationsymbolsempty

\errorcontextlines 10000

\begin{document}
\title{\Large{Harmony Theory and ICS}}
\subtitle{\small{часть вторая}}
\author{Константин Соколов}
\institute[]
{Mathlingvo, СПбГУ, Eventflow API\\ \bigskip  \url{http://nlu-rg.ru}}
\date{Санкт-Петербург, 2014} 
% Создание заглавной страницы
\begin{frame}
    \thispagestyle{empty}
    \titlepage
\end{frame}

\begin{frame}{Тексты}
\setcounter{framenumber}{1}
\begin{itemize}
    \item P. Smolensky, G. Legendre and Y. Miyata (1990) Harmonic grammar: A formal multi-level connectionist theory of linguistic well-formedness: Theoretical foundations.
    \medskip
    \item A. Prince and P. Smolensky (1993) Optimality Theory.
    \medskip
	\item P. Smolensky, G. Legendre (2006) Hamonic Mind.
    \medskip
    \item P. Smolensky et al. (2014) Gradient Symbol Processing.
\end{itemize}
\end{frame}

\begin{frame}{План}
    \begin{itemize}
        \item Harmonic Grammar vs. Теория оптимальности
        \medskip
        \item ICS, Gradient Symbol Processing
    \end{itemize}
\end{frame}

% 5. Harmony Theory: subsymbolic paradigm; необходимость совместить высокоуровневые когнитивные функции и восприятие, параллельная обрабока; Harmonium; обучение
% 6. Harmonic Grammar - P. Smolensy, G. Legendre, Y. Miyata (1990)  A Formal Multi-Level Connectionist Theory of Linguistic Well-Formedness: Theoretical Foundations.; HG Course at UMass (http://blogs.umass.edu/hgcourse/); переход от правил к системам ограничений (ср. с другими теориями в лингвистике, можно сослаться на Зубрицкую; constraint grammars в унификационных грамматиках, принципы и параметры, минимализм и пр.), well-formedness
% 7. Optimality Theory: отличие от HG -- в OT ранжирование ограничений, в HG взвешенные ограничения; плюсы и минусы; tableaux; пример анализа слоговой структуры в OT; факториальная типология; нетривиальные эффекты как следствие архитектуры: TETU и conspiracy
% 8. Integrated Connectionist/Symbolic Cognitive Architecture (ICS)
% 9. Gradient Symbol Processing: объединение символьных и статистических подходов, геометрия пространства (аттракторы)

\begin{frame}{}
\begin{center}
	\textbf{Harmonic Grammar}
\end{center}
\end{frame}

\begin{frame}{Динамическая система - I}
\begin{center}
	\begin{figure}[H]
		\includegraphics[scale=0.7]{harmonic1.png} 
	\end{figure}
\end{center}
\bigskip
Динамика определяется системой уравнений\\
\medskip
$\frac{da_1}{dt} = -a_1 + w_1 - \lambda a_2$, $\frac{da_2}{dt} = -a_2 + w_2 - \lambda a_1$,\\
\medskip
максимизирующих $H(\textbf{a}) = a_1 w_1 + a_2 w_2 - \lambda a_1 a_2 - \frac{1}{2}(a_1^2 + a_2^2)$
\end{frame}

\begin{frame}{Динамическая система - II}
\begin{center}
	\begin{figure}[H]
		\includegraphics[scale=0.7]{harmonic2.png} 
	\end{figure}
\end{center}
\bigskip
Вектор активации $\textbf{a} = (a_1, a_2)$ стремится к $(0,79; -0,71)$
\end{frame}

\begin{frame}{Динамическая система - III}
\begin{center}
	\begin{figure}[H]
		\includegraphics[scale=0.7]{harmonic3.png} 
	\end{figure}
\end{center}
\bigskip
Эволюция системы в фазовом пространстве\\
\textit{(здесь двумерная проекция, т.к. входные значения постоянны)}
\end{frame}

\begin{frame}{Динамическая система - IV}
\begin{center}
	\begin{figure}[H]
		\includegraphics[scale=0.8]{harmonic4.png} 
	\end{figure}
\end{center}
\bigskip
Оптимизация как максимизация гармоничности
\end{frame}

\begin{frame}{Система ограничений - I}
Динамика системы определяется набором \textit{конфликтующих} взвешенных ограничений:\\
\medskip
\begin{itemize}
	\item Узел №1 должен быть активен ($w_1 = 0,6$)
	\medskip
	\item Узел №2 должен быть активен ($w_2 = 0,5$)
	\medskip
	\item Узлы №1 и №2 не должны быть активны одновременно ($\lambda = -0,9$)
\end{itemize}
\bigskip
Оптимизация как достижение компромисса
\end{frame}


\begin{frame}{}
\begin{center}
	\textbf{Optimality Theory}
\end{center}
\end{frame}


\begin{frame}{Optimality Theory - I}
\begin{itemize}
	\item 
	\medskip
	\item 
	\medskip
	\item 
\end{itemize}
\end{frame}


\begin{frame}{}
\begin{center}
	\textbf{Integrated Connectionist-Symbolic (ICS) Architecture}
\end{center}
\end{frame}

\begin{frame}{ICS Architecture - I}
sdfsdf
\bigskip
\begin{itemize}
	\item 
	\medskip
	\item 
\end{itemize}
\end{frame}


\begin{frame}{}
    \thispagestyle{empty}
    \begin{center}
        {\large Спасибо!}
    \end{center}
\end{frame}


\end{document}
