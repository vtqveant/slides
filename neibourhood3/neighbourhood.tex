\documentclass{beamer}

\usepackage[T2A]{fontenc}
\usepackage[utf8]{inputenc}
\usepackage[english,russian]{babel}
\usepackage{amssymb,amsfonts,amsmath,mathtext}
\usepackage{cite,enumerate,float,indentfirst}

\usepackage{diagrams}
\usepackage{extpfeil} % double-arrows

\usepackage{ragged2e} % text justifying

\graphicspath{{images/}}

\usetheme{Pittsburgh}
\usecolortheme{whale}

\setbeamertemplate{footline}{\scriptsize{\hspace*{0.4cm}\insertframenumber}\vspace*{0.3cm}}
\beamertemplatenavigationsymbolsempty

\errorcontextlines 10000

\begin{document}
\title{\Large{Окрестностная грамматика}}
\subtitle{\textit{вторая часть}}
\author{Константин Соколов}
\institute[]
{Mathlingvo, СПбГУ\\ \bigskip  \url{http://nlu-rg.ru}}
\date{Санкт-Петербург, 2014} 
% Создание заглавной страницы
\begin{frame}
    \thispagestyle{empty}
    \titlepage
\end{frame}

\begin{frame}{План}
\setcounter{framenumber}{1}
    \begin{itemize}
		\item теоретико-категорная трактовка композициональности
    \end{itemize}
\end{frame}

\begin{frame}{Тексты}
\begin{itemize}
	\item В. А. Лапшин. Топологии синтаксических отношений формальных языков. 2008. arXiv:0802.4181
\end{itemize}
\end{frame}

\begin{frame}{Пучок (повторение)}
\begin{small}
Пучок множеств $\mathcal{F}$ над топологическим пространством $(X, \mathcal{O}(X))$ -- это функтор $\mathcal{F} : \mathcal{O}(X)^{op} \to \textbf{Sets}$, для любого покрытия $U = \bigcup_{i \in I} U_i$ порождающий уравнитель
\begin{diagram}[labelstyle=\scriptstyle]
\mathcal{F}(U) & \rTo^{e} & & \prod_{i \in I} \mathcal{F}(U_i) &  & \pile{\rTo^f\\ \rTo_g} & \prod_{i,j \in I} \mathcal{F}(U_i \times_U U_j) \\
\end{diagram}\\
\medskip
т. е. для любого $t \in \mathcal{F}(U)$, $e(t) = \{ t \vert_{U_i} \; \vert \; i \in I \}$ и семейства $t_i \in \mathcal{F}(U_i)$\\ 
\begin{center}
$f(\{t_i\}) = \{t_i \vert_{U_i \cap U_j}\}$, $g(\{t_i\}) = \{t_j\vert_{U_i \cap U_j}\}$
\end{center}
\end{small}
\end{frame}

\begin{frame}{Пучок (повторение)}
Обобщение:\\
\medskip
\begin{small}
\begin{itemize}
	\item базовое пространство заменяется категорией
	\item понятие окрестности заменяется понятием \textit{решета}
	\item топологическая структура заменяется \textit{топологией Гротендика}
	\item топологическое пространство заменяется \textit{сайтом}
\end{itemize}
\end{small}
\end{frame}


\begin{frame}{Топология Гротендика (1)}
\begin{small}
\textit{Решето} на объекте $A$ -- это семейство стрелок $S = \{ f \; \vert \; Cod(f) = A \}$ таких, что если $f \in S$ и $h : B \to Dom(f)$, то $f \circ h \in S$\\
\medskip
Решето $S$, \textit{порожденное семейством стрелок} (с общим кодоменом) -- это наименьшее решето, содержащее все стрелки данного семейства.
\end{small}
\end{frame}

\begin{frame}{Топология Гротендика (2)}
\medskip
\begin{small}
Топология Гротендика $J$ сопоставляет каждому объекту $A \in Ob(\mathcal{C})$ множество решет $J(A)$ таких, что:\\
\bigskip
\begin{itemize}
	\item \textit{(максимальность)} максимальное решето $h_A = \{ f \; \vert \; Cod(f) = A \}$ принадлежит $J(A)$
	\item \textit{(стабильность)} если $S \in J(A)$ и $h : B \to A$ -- произвольный морфизм c концом в $A$, то решето $h^*(S) = \{ f \; \vert \; Cod(f) = B, h \circ f \in S \} \in J(B)$
	\item \textit{(транзитивность)} если $S \in J(A)$ и $R$ -- решето на $A$ такое, что $h^*(R) \in J(B)$ для всех $h : B \to A$, то $R \in J(A)$
\end{itemize}
\end{small}
\end{frame}

\begin{frame}{Топология Гротендика (3)}
\medskip
\begin{small}
Если $(X, \mathcal{O}(X))$ -- топологическое пространство, то
\bigskip
\begin{itemize}
	\item частично упорядоченное множество $\mathcal{O}(X)$ -- категория
	\item $S = \{ U_i \hookrightarrow U \; \vert \; i \in I \}$ -- решето над $U \in Ob(\mathcal{O}(X))$
	\item $J_{\mathcal{O}(X)}$ -- топология Гротендика, индуцированная введением топологической структуры на $X$
	\item $S \in J_{\mathcal{O}(X)}(U)$ тогда и только тогда, когда $\bigcup_{i \in I} U_i = U$
\end{itemize}
\end{small}
\end{frame}

\begin{frame}{Топология Гротендика (4)}
\begin{small}
\bigskip
\begin{itemize}
	\item Пара $(C, J)$, где $C$ -- категория, $J$ -- топология Гротендика, называется \textit{сайтом}
	\item Функтор $\mathcal{F} : C^{op} \to \textbf{Sets}$ -- предпучок над $C$
\end{itemize}
\end{small}
\end{frame}

\begin{frame}{Топология Гротендика (5)}
\begin{small}
Cечения здесь -- это $\mathcal{F}(A)$, $A \in Ob(C)$, мы же хотим определить набор сечений, некоторым образом ``согласованных'' с решетом $S(A) \in J$. Для этого определим \textit{согласованное семейство} (сечений) решета $S(A)$:
\bigskip
\begin{itemize}
	\item Каждой стрелке $f : B \to A$, $f \in S(A)$ сопоставляется \\элемент $x_f \in \mathcal{F}(B)$ такой, что для любой стрелки $g : C \to B$ выполнено	$\mathcal{F}(g)(x_f) = x_{f \circ g}$
	\item Элемент $x \in \mathcal{F}(A)$ т.ч. $\mathcal{F}(f)(x) = x_f$ для всех $f \in S(A)$ называется \textit{amalgamation}
	\item $\mathcal{F} : C^{op} \to \textbf{Sets}$ -- пучок, если для любого покрытия (решетами) любого объекта из $C$ все согласованные семейства сечений имеют единственную ``амальгаму''.
\end{itemize}
\end{small}
\end{frame}


% \begin{frame}{Топология Гротендика (6)}
% \medskip
% \begin{small}
% Пусть дан \textit{сайт} $(C, J)$. Предпучок $\mathcal{F} : C^{op} \to Sets$ называется \textit{пучком}, если для каждого объекта $A \in Ob(C)$ и каждого решета $S \in J(A)$, диаграмма\\
% \begin{diagram}[labelstyle=\scriptstyle,loose,height=.8em,width=2pt]
% \mathcal{F}(A) & \rTo^{e} & \prod \limits_{f \in S} \mathcal{F}(Dom(f)) & \pile{\rTo^p\\ \rTo_q} & \prod \limits_{f,g \; f \in S} \mathcal{F}(Dom(g)) \\
% \end{diagram}
% является уравнителем,\\
% \smallskip
% \begin{itemize}
% 	\item $Dom(f) = Cod(g)$
%	\item $e(x) = \{ \mathcal{F}(f)(x) \}_f$
%    \item произведение справа берется по всем $f$ и $g$, для которых определена композиция (откуда $f \circ g \in S$)
%    \item отображение $p$ определяется через значения $\mathcal{F}$ на композициях стрелок в $C$ (т.е. $p_{f,g}(x) = \mathcal{F}(f \circ g)(x)$ в предыдущем примере)
%    \item отображение $q$ определяется через действие $\mathcal{F}(g)$ на элементах $x_f$ (т.е. $q_{f,g}(x) = \mathcal{F}(g)(x_f)$ в предыдущем примере)
% \end{itemize}
%\end{small}
% \end{frame}

\begin{frame}{Принцип композициональности (1)}
\begin{small}
Объекты категории $Ext(D_G)$:\\
\medskip
\begin{itemize}
	\item мы рассматриваем множества диаграмм:
		\begin{itemize}
			\item $D$ -- все диаграммы над фиксированым ``алфавитом''
			\item $G \subseteq D$ -- фиксированный набор диаграмм (окрестностная грамматика)
			\item $D_G \subseteq D$ -- диаграммы, корректные относительно $G$
		\end{itemize}
	\item мы хотим, чтобы объектами категории были не просто диаграммы, но диаграммы вместе с их ``грамматическим разбором'', т.е. пары вида $(D, P)$, где $P$ -- синтаксическое покрытие (аналог дерева синтаксического разбора)
	\item т.е. $Ob(Ext(D_G)) = \{ (D_i, P_i) \; \vert \; D_i \in D_G \cup G \}$
\end{itemize}
\end{small}
\end{frame}

\begin{frame}{Принцип композициональности (2)}
\begin{small}
Морфизмы категории $Ext(D_G)$ должны реализовывать отображение включения диаграмм, уважающие их покрытия:\\
\medskip
\begin{itemize}
	\item если $A, B \in D_G$, то $Hom((A, P^A), (B, P^B))$ состоит из отображений включения $s : A \to B$, согласованных на покрытиях
	\item если $A \notin D_G$, $B \in D_G$, то $Hom((A, \varnothing), (B, P^B))$ состоит из отображений включения $A$ в диаграммы покрытия $P^B$
	\item если $A \in D_G$, $B \notin D_G$, то $Hom((A, P^A), (B, \varnothing)) = \varnothing$
	\item если $A \notin D_G$, $B \notin D_G$, то $Hom((A, \varnothing), (B, \varnothing)) = id$, если $A = B$ и $Hom((A, \varnothing), (B, \varnothing)) = \varnothing$, если $A \neq B$
\end{itemize}
\end{small}
\end{frame}

\begin{frame}{Принцип композициональности (3)}
\begin{small}
\textit{Синтаксическая топология} $J_G$ на основе окрестностной грамматики $G$ -- это топология Гротендика на $Ext(D_G)$, так что\\
\medskip
\begin{itemize}
	\item если $A \in D_G$, то $J_G(D)$ содержит максимальное решето на объекте $A$
	\item если $A = (D, P) \in D_G$, то семейство морфизмов элементов покрытия $P$ принадлежит $J_G(A)$
\end{itemize}
\end{small}
\end{frame}

\begin{frame}{Принцип композициональности (4)}
\textit{Принцип композициональности:}\\
\medskip
\begin{block}{}
\justifying
\begin{small}
Пусть $G = \{ G_a \; \vert \; a \in A \}$ -- окрестностная грамматика, $Ext(D_G)$ -- категория корректных диаграмм на множестве диаграмм $D$ и $J_G$ -- синтаксическая топология, определяемая грамматикой $G$. Предпучок значений $\mathcal{F} : Ext(D_G) \to \textbf{Sets}$ является пучком тогда и только тогда, когда каждое значение корректной диаграммы $D \in Ob(Ext(D_G))$ однозначно определяется соответствующим семейством значений на окрестностях его синтаксического покрытия.
\end{small}
\end{block}
\end{frame}

\begin{frame}{Замечания}
\medskip
\begin{small}
\begin{itemize}
	\item Окрестностная грамматика для КС-языков похожа на Tree Adjoining Grammar (Joshi, Levy, Takahashi. 1975).
	\item Линейно-индексируемые грамматики, вершинные грамматики, CCG и TAG слабо эквивалентны (Vijay-Shanker, Weir. 1994).
	\item Окрестностные грамматики и теоретико-модельный синтаксис.
\end{itemize}
\end{small}
\end{frame}



\begin{frame}{}
    \thispagestyle{empty}
    \begin{center}
        {\large Спасибо!}
    \end{center}
\end{frame}


%%% слайд помещается сюда
%% \begin{frame}{Заголовок}
%% \end{frame}

\end{document}
