\documentclass{beamer}

\usepackage[T2A]{fontenc}
\usepackage[utf8]{inputenc}
\usepackage[english,russian]{babel}
\usepackage{amssymb,amsfonts,amsmath,mathtext}
\usepackage{cite,enumerate,float,indentfirst}

\usepackage{diagrams}

\graphicspath{{images/}}

\usetheme{Pittsburgh}
\usecolortheme{whale}

\setbeamertemplate{footline}{\scriptsize{\hspace*{0.4cm}\insertframenumber}\vspace*{0.3cm}}
\beamertemplatenavigationsymbolsempty

\errorcontextlines 10000

\begin{document}
\title{\Large{Окрестностная грамматика}}
\author{Константин Соколов}
\institute[]
{Mathlingvo, СПбГУ, i-Free\\ \bigskip  \url{http://nlu-rg.ru}}
\date{Санкт-Петербург, 2014} 
% Создание заглавной страницы
\begin{frame}
    \thispagestyle{empty}
    \titlepage
\end{frame}

\begin{frame}{План}
\setcounter{framenumber}{1}
    \begin{itemize}
		\item окрестностная грамматика (Шрейдер, Борщев, Хомяков)
		\item синтаксические диаграммы и пучки (Лапшин)
    \end{itemize}
\end{frame}


% 1. Окрестностная грамматика

\begin{frame}{}
\begin{center}
	\textbf{Конструкции и понятия}
\end{center}
\end{frame}

% 2. Синтаксические диаграммы

\begin{frame}{Конструкции и понятия}
\begin{itemize}
	\item топологическое пространство {\small \textit{(повторение)}}
	\item предпучок и пучок {\small \textit{(повторение)}}
	\item немного алгебраической топологии:
		\begin{itemize}
			\item накрытие
			\item сечение
			\item поднятие
		\end{itemize}
	\item пучок непрерывных сечений накрытия
	\item гомоморфизм пучков
	\item расслоенное произведение пучков
\end{itemize}
\end{frame}

% 3. Теоретико-категорная формализация принципа композициональности

\begin{frame}{ТК}

\def\Assl{{\rm assl}}\def\Id{{\rm id}}

\begin{diagram}
A*(B*(C*D))			& \rTo^\Assl 	& 	(A*B)*(C*D) 	& \rTo^\Assl 	& ((A*B)*C)*D\\
\dTo^{\Id*\Assl} 	&				&	=				&				& \uTo_{\Assl*\Id}\\
A*((B*C)*D)			&				& \rTo^\Assl 		&				& (A*(B*C))*D\\
\end{diagram}

\end{frame}


\begin{frame}{Manin}

\begin{diagram}[labelstyle=\scriptstyle]
F(X) & \rTo^{f(X)} & G(X) \\
\dTo^{F(\phi)} & & \dTo_{G(\phi)} \\
F(Y) & \rTo_{f(Y)} & G(Y) \\
\end{diagram}

\end{frame}

\begin{frame}{pullback example}
\begin{diagram}[labelstyle=\scriptstyle]
U	&	&	&	&	\\
	& \rdTo~{(x,y)}\rdTo(4,2)^x\rdTo(2,4)_y &	&	&	\\
	&	& X\times_Z Y & \rTo_p &	X	\\
	&	&	\dTo^q	&	& \dTo_f \\
	&	&	Y	& \rTo^g &	Z	\\
\end{diagram}
\end{frame}



\begin{frame}{}
    \thispagestyle{empty}
    \begin{center}
        {\large Спасибо!}
    \end{center}
\end{frame}


%%% слайд помещается сюда
%% \begin{frame}{Заголовок}
%% \end{frame}

\end{document}
