\documentclass{beamer}

\usepackage[T2A]{fontenc}
\usepackage[utf8]{inputenc}
\usepackage[english,russian]{babel}
\usepackage{amssymb,amsfonts,amsmath,mathtext}
\usepackage{cite,enumerate,float,indentfirst}

\usepackage{diagrams}
\usepackage{extpfeil} % double-arrows

\usepackage{ragged2e} % text justifying

\graphicspath{{images/}}

\usetheme{Pittsburgh}
\usecolortheme{whale}

\setbeamertemplate{footline}{\scriptsize{\hspace*{0.4cm}\insertframenumber}\vspace*{0.3cm}}
\beamertemplatenavigationsymbolsempty

\errorcontextlines 10000

\begin{document}
\title{\Large{Окрестностная грамматика}}
\author{Константин Соколов}
\institute[]
{Mathlingvo, СПбГУ, i-Free\\ \bigskip  \url{http://nlu-rg.ru}}
\date{Санкт-Петербург, 2014} 
% Создание заглавной страницы
\begin{frame}
    \thispagestyle{empty}
    \titlepage
\end{frame}

\begin{frame}{План}
\setcounter{framenumber}{1}
    \begin{itemize}
		\item окрестностные грамматики
		\item синтаксические диаграммы
		\item теоретико-категорная трактовка композициональности
    \end{itemize}
\end{frame}

\begin{frame}{Тексты}
\begin{itemize}
	\item Ю. А. Шрейдер. Топологические модели языка. В сб.: ``Проблемы структурной лингвистики. 1971'', М., 1972.
	\item В. А. Лапшин. Лекции по математической лингвистике. М., 2010.
	\item В. А. Лапшин. Синтаксические диаграммы как формализм для представления синтаксических отношений формальных языков. 2008.  	arXiv:0802.3974
	\item В. А. Лапшин. Топологии синтаксических отношений формальных языков. 2008. arXiv:0802.4181
\end{itemize}
\end{frame}


% 1. Окрестностные грамматики

\begin{frame}{}
\begin{center}
	\textbf{Окрестностные грамматики}
\end{center}
\end{frame}

\begin{frame}{Окрестностные грамматики (1)}
\begin{small}
\begin{itemize}
	\item Пусть $A$ -- алфавит. Окрестность -- это цепочка над алфавитом $A \cup \{ \# \}$ с отмеченным вхождением некоторого знака из $A$ (центр окрестности)
		\begin{itemize}
			\item $a\underline{a}b$, $\underline{a}b\#$, $\#\underline{b}\#$
			\item в цепочке $X = abaac$ элемент $b$ имеет окрестности $U_1 = a\underline{b}$, $U_2 = a\underline{b}aa$, $U_3 = \#a\underline{b}a$ и т.д.
		\end{itemize}
\end{itemize}
\end{small}
\end{frame}

\begin{frame}{Окрестностные грамматики (2)}
\begin{small}
\begin{itemize}
	\item Пусть $M$ -- множество окрестностей над $A \cup \{ \# \}$. Цепочка $X$ \textit{непрерывна} относительно $M$, если любое вхождение в $X$ любого знака из $A$ имеет окрестность из $M$.
	\item Множество цепочек, непрерывных относительно $M$, называется \textit{простым окрестностным $M$-языком}
		\begin{itemize}
			\item система $U_1 = a\underline{a}$, $U_2 = \#\underline{a}$, $U_3 = a\underline{b}$ задаёт язык \texttt{a+b}
			\item система $U_1 = \underline{a}$, $U_2 = \underline{b}$ задаёт язык \texttt{(a|b)+}
		\end{itemize}
	\item $M$-языки являются автоматными языками (Борщев В. Б., 1967)
\end{itemize}
\end{small}
\end{frame}


% Синтаграммы 
%   - с отношением непосредственного предшествования
%   - сотношением неп. предш. и порождения (контекстно-своб.)
\begin{frame}{Синтаксические диаграммы (1)}
\smallskip
{\small Расширение идеи окрестностных грамматик:}
\bigskip
\begin{small}
\begin{itemize}
	\item \textit{текст} -- ``нелинейная'' последовательность символов
	\item \textit{окрестностная грамматика} -- система ограничений, определяющая множество \textit{корректных текстов}
	\item \textit{язык} -- множество текстов, задаваемых окрестностной грамматикой
\end{itemize}
\end{small}
\end{frame}

\begin{frame}{Синтаксические диаграммы (2)}
Простой окрестностный язык \texttt{(ab)+a}\\
\medskip
\begin{small}
\begin{itemize}
	\item алфавит $A = \{a, b\}$
	\item отношение непосредственного предшествования $\leftarrow \; \subseteq A \times A$
	\item $U(a) = \{\underline{a} \leftarrow b, b \leftarrow \underline{a}, b \leftarrow \underline{a} \leftarrow b\}$, $U(b) = \{ a \leftarrow \underline{b} \leftarrow a \}$
\end{itemize}
\end{small}
\end{frame}

\begin{frame}{Синтаксические диаграммы (3)}
Контекстно-свободный язык:\\
\medskip
\begin{small}
\begin{enumerate}
	\item[1)] КС-грамматика $G = \langle V, \Sigma, R, s \rangle$, где $V$ -- нетерминалы, $\Sigma$ -- терминалы, $R$ -- правила, $s \in V$ -- начальный символ
	\item[2)] диаграммы над алфавитом $A = V \cup \Sigma$ 
	\item[3)] множество отношений $S = \{ S_L, S_P \}$, где  
		\begin{itemize}
			\item $S_L$ -- отношение непосредственного предшествования, 
			\item $S_P$ -- отношение ``быть непосредственным потомком''
		\end{itemize}
\end{enumerate}
\end{small}
\end{frame}

\begin{frame}{Синтаксические диаграммы (3)}
\begin{small}
\begin{enumerate}
	\item[4)] окрестностная грамматика 
		\begin{itemize}	
			\item $U(a) = \{ \underline{a} \}$, $a \in \Sigma$
			\item $U(A)$ для каждого $A \in V$, стоящего слева в правиле вида $A \to X_1, X_2, \dots X_n$:
			\medskip
\begin{diagram}[labelstyle=\scriptstyle]
 & & \underline{A} & & & & \\
 & \ldTo^{S_P} & \dTo_{S_P} & \rdTo{S_P}(4,2) & & & \\
X_1 & \lDashto^{S_L} & X_2 & \lDashto^{S_L} & \dots & \lDashto^{S_L} & X_n \\
\end{diagram}
		\end{itemize}
\end{enumerate}
\end{small}
\end{frame}


% Синтаграммы - форм. определение
\begin{frame}{Синтаксические диаграммы (2)}
\begin{small}
Тройка $D = \langle A, \Gamma, F \rangle$, где\\
\medskip
\begin{itemize}
	\item алфавит $A = \{ a_1, a_2, \dots, a_n \}$
	\item размеченный мультиграф $\Gamma = \{ V, R, S, f \}$, где 
		\begin{itemize}
			\item $V$ -- вершины
			\item $R$ -- ребра, т.е. пары $(u, v)$, где $u, v \in V$, $u \neq v$
			\item $S$ -- пометки на ребрах (сорта)
			\item $f : R \to S$ приписывает ребрам пометки
		\end{itemize}
	\item $F : V \to A$ сопоставляет каждой вершине графа $\Gamma$ символ алфавита $A$
\end{itemize}
\medskip
называется \textit{синтаксической диаграммой}.
\end{small}
\end{frame}

% Синтаграммы - включение синтаграммы
\begin{frame}{Синтаксические диаграммы (3)}
\begin{small}
Даны две диаграммы $D_1 = \langle A, \Gamma_1, F_1 \rangle$, $D_2 = \langle A, \Gamma_2, F_2 \rangle$ над алфавитом $A$. Тройка отображений $s = (s_V, s_R, s_S)$, где $s_V : V_1 \to V_2$, $s_R : R_1 \to R_2$, $s_S : S_1 \to S_2$ таких, что\\
\medskip
\begin{itemize}
	\item $F_1(v) = F_2(s_V(v))$ для всех $v \in V_1$
	\item $s_S(f_1(u, v)) = f_2(s_R(u, v))$ для всех $(u, v) \in R_1$
	\item $s_R(u, v) = (s_V(u), s_V(u)) \in R_2$ для всех $(u, v) \in R_1$,
\end{itemize}
\medskip
определяет включение диаграммы $D_1$ в диаграмму $D_2$ в качестве поддиаграммы.
\end{small}\end{frame}

% Окрестность для синтаграмм
\begin{frame}{Синтаксические диаграммы (4)}
\begin{small}
\textit{Окрестность} $D_a$ символа $a \in A$ -- это пара  $(D_a, s_a)$, где $D_a$ -- синтаксическая диаграмма, а $s_a : a \to D_a$ -- включение диаграммы $a$, состоящей из одной вершины, помеченной символом $a$, в $D_a$.\\
\bigskip
\textit{Окрестностная грамматика} на множестве синтаксических диаграмм $\mathbf{D}$ -- это семейство окрестностей $G = \{ G_a \; \vert \; a \in A, G_a \in \mathbf{D} \}$
\end{small}
\end{frame}

\begin{frame}{Синтаксические диаграммы (5)}
\bigskip
\begin{small}
% Пусть $\mathbf{D}$ -- множество синтаксических диаграмм, $G = \{ G_a \; \vert \; a \in A \}$ -- окрестностная грамматика на $\mathbf{D}$ и $D \in \mathbf{D}$ -- некоторая диаграмма. 
\textit{Синтаксическое покрытие} диаграммы $D$ -- это семейство окрестностей $G_D = \{ D_v \; \vert \; v \in V, D_v \in G_{F(v)} \}$ такое, что\\
\medskip
\begin{itemize}
	\item $D_v$ -- поддиаграмма $D$ для всех вершин $v$ диаграммы $D$
	\item $s^v_R(z(D_v)(v)) = z(D)(v)$, где $z(D)(v)$ -- звезда диаграммы $D$ в вершине $v$
\end{itemize}

\end{small}
\begin{figure}[H]
\includegraphics[scale=0.25]{star.png} 
\end{figure}

\end{frame}




% 3. Теоретико-категорная формализация принципа композициональности

% Хорошо про топологию Гротендика - http://anhinga-anhinga.livejournal.com/59865.html
% Идея, что стрелки выражают не только функции, но и аппроксимацию - зачётная (думай про решето)

\begin{frame}{}
\begin{center}
	\textbf{Теоретико-категорная трактовка композициональности}
\end{center}
\end{frame}

\begin{frame}{Категорное определение пучка (1)}
Три варианта определения пучков:\\
\medskip
\begin{small}
\begin{itemize}
	\item топологическое пространство $(X, \mathcal{O})$ и сечения $\mathcal{F}(U)$, $U \in \mathcal{O}(X)$
	\item пучок непрерывных сечений накрытия $(\widetilde{\mathcal{F}}, p)$
	\item контравариантный функтор $\mathcal{F} : \mathcal{C}^{op} \to Sets$
\end{itemize}
\end{small}
\medskip
\dots и соответствующие условия
\end{frame}

\begin{frame}{Категорное определение пучка (2)}
Категорное определение позволяет обобщить конструкцию:\\
\medskip
\begin{small}
\begin{itemize}
	\item базовое пространство заменяется категорией
	\item понятие окрестности заменяется понятием \textit{решета}
	\item топологическая структура заменяется \textit{топологией Гротендика}
	\item топологическое пространство заменяется \textit{сайтом}
\end{itemize}
\end{small}
\end{frame}

\begin{frame}{Про pullback, уравнитель и аксиому пучка (1)}
\begin{diagram}[labelstyle=\scriptstyle]
 P          & \rTo^{p_2} & Y      &   & Q            & \rTo^{q_2} & Y \\
 \dTo^{p_1} &            & \dTo_g &   & \dTo^{q_1} &              & \dTo_g \\
 X          & \rTo_f     & Z      &   & X            & \rTo_f       & Z \\
\end{diagram}\\
\bigskip
{\small \textit{Универсальное свойство:}\\ существует и единственна стрелка $u : Q \to P$, $q_2 = p_2 \circ u$ и $q_1 = p_1 \circ u$}
\end{frame}

\begin{frame}{Про pullback, уравнитель и аксиому пучка (2)}
\begin{diagram}[labelstyle=\scriptstyle]
Q	&	&	&	&	\\
	& \rdDashto^u\rdTo(4,2)^{q}\rdTo(2,4)_{q} &	&	&	\\
	&	& X\times_Z X & \rTo_{p} &	X	\\
	&	&	\dTo^{p}	& 	& \dTo_g \\
	&	&	X	& \rTo^f &	Z	\\
\end{diagram}\\
\bigskip
{\small pullback можно охарактеризовать как уравнитель стрелок $f$ и $g$}
\end{frame}

\begin{frame}{Про pullback, уравнитель и аксиому пучка (3)}
\begin{diagram}[labelstyle=\scriptstyle]
 P & \rTo^{p} & X & \pile{\rTo^f\\ \rTo_g} & Z \\
 \uDashto^u & \ruTo^q & & & & \\
 Q & & & & & \\
\end{diagram}\\
\bigskip
\begin{small}
\begin{itemize}
	\item пусть $f, g : X \to Z$, уравнитель $Eq(f,g) = \{ x \; \vert \; f(x) = g(x) \}$
	\item пусть $F = \{ f_i : X \to Z \}$, $Eq(F) = \{ x \; \vert \; f_i, f_j \in F, f_i(x) = f_j(x) \}$
\end{itemize}
\end{small}
\end{frame}

\begin{frame}{Про pullback, уравнитель и аксиому пучка (4)}
\medskip
{\small Пусть $X, Y \in Ob(C)$ и $f, g \in Hom(X, Y)$.}\\
\medskip
\begin{diagram}[labelstyle=\scriptstyle]
 P & \rTo^{p} & X & \pile{\rTo^f\\ \rTo_g} & Y \\
 \uDashto^u & \ruTo^q & & & & \\
 Q & & & & & \\
\end{diagram}

\begin{small}
\begin{itemize}
	\item $f \circ p = g \circ p$
	\item для любого объекта $O \in Ob(C)$ и стрелки $q : Q \to X$, $q \circ f = q \circ g$ существует и единственна стрелка $u : Q \to P$,  $p \circ u = q$
\end{itemize}
\end{small}
\end{frame}

\begin{frame}{Про pullback, уравнитель и аксиому пучка (5)}
Аксиома пучка как уравнитель:\\
\bigskip
\begin{diagram}[labelstyle=\scriptstyle]
\mathcal{F}(U) & \rTo & \prod_{i \in I} \mathcal{F}(U_i) &  & \pile{\rTo^f\\ \rTo_g} & \prod_{i,j \in I} \mathcal{F}(U_i \times_U U_j) \\
\end{diagram}\\
\bigskip
{\small Интуиция: стрелки в категории можно рассматривать не как ``функции'', а как ``аппроксимацию''; все сечения составлены из одного ``материала''.}
\end{frame}

\begin{frame}{Про pullback, уравнитель и аксиому пучка (6)}
В категории топологических пространств $\textbf{Top}$:\\
\medskip
\begin{small}
\begin{itemize}
	\item $U_i \times_U U_j$ -- это $U_i \cap U_j$, 
	\item $\prod_{i \in I} \mathcal{F}(U_i) = \mathcal{F}(U_1) \times_U \mathcal{F}(U_2) \times_U \ldots \times_U \mathcal{F}(U_n)$, т.е. $\bigcap_{i \in I} \mathcal{F}(U_i)$.
\end{itemize}
\end{small}
\bigskip
Поэтому для пучка $\mathcal{F} : \mathcal{O}(X)^{op} \to \textbf{Sets}$ можно записать так:\\
\begin{diagram}[labelstyle=\scriptstyle]
\mathcal{F}(U) & \rTo & \bigcap_{i \in I} \mathcal{F}(U_i) &  & \pile{\rTo^f\\ \rTo_g} & \bigcap_{i,j \in I} \mathcal{F}(U_i \cap U_j) \\
\end{diagram}
\end{frame}

\begin{frame}{Про pullback, уравнитель и аксиому пучка (7)}
\begin{small}
Пучок множеств $\mathcal{F}$ над топологическим пространством $(X, \mathcal{O}(X))$ -- это функтор $\mathcal{F} : \mathcal{O}(X)^{op} \to \textbf{Sets}$, для любого покрытия $U = \bigcup_{i \in I} U_i$ порождающий уравнитель
\begin{diagram}[labelstyle=\scriptstyle]
\mathcal{F}(U) & \rTo^{e} & & \prod_{i \in I} \mathcal{F}(U_i) &  & \pile{\rTo^f\\ \rTo_g} & \prod_{i,j \in I} \mathcal{F}(U_i \times_U U_j) \\
\end{diagram}\\
\medskip
т. е. для любого $t \in \mathcal{F}(U)$, $e(t) = \{ t \vert_{U_i} \; \vert \; i \in I \}$ и семейства $t_i \in \mathcal{F}(U_i)$\\ 
\begin{center}
$f(\{t_i\}) = \{t_i \vert_{U_i \cap U_j}\}$, $g(\{t_i\}) = \{t_j\vert_{U_i \cap U_j}\}$
\end{center}
\end{small}
\end{frame}


\begin{frame}{Про pullback, уравнитель и аксиому пучка (8)}
\begin{small}
\begin{diagram}[labelstyle=\scriptstyle]
\mathcal{F}(U) & \rTo^{e} & & \mathcal{F}(U_i) \cap \mathcal{F}(U_j) &  & \pile{\rTo^f\\ \rTo_g} & \mathcal{F}(U_i \cap U_j) \\
\end{diagram}\\
\medskip
\begin{itemize}
	\item $Ker(\mathcal{F}(U_i) \cap \mathcal{F}(U_j) \mapsto \mathcal{F}(U_i \cap U_j))$ -- это множество точек в $U$, не содержащихся одновременно в $U_i$ и $U_j$
% 	\item если $t \in \mathcal{F}(U)$, $t' \in \mathcal{F}(U_i) \cap \mathcal{F}(U_j)$ и $t'' \in \mathcal{F}(U_i \cap U_j)$, то это ``одна и та же точка'' (можно отождествить $t$, $t'$ и $t''$)
\end{itemize}
\end{small}
\end{frame}


\begin{frame}{Топология Гротендика (1)}
\begin{small}
\textit{Решето} на объекте $A$ -- это семейство морфизмов $S = \{ f \; \vert \; Cod(f) = A \}$ таких, что если $f \in S$ и $h : B \to Dom(f)$, то $f \circ h \in S$
\end{small}
\end{frame}

\begin{frame}{Топология Гротендика (2)}
\medskip
\begin{small}
Топология Гротендика $J$ сопоставляет каждому объекту $A \in Ob(\mathcal{C})$ множество решет $J(A)$ таких, что:\\
\bigskip
\begin{itemize}
	\item максимальное решето $h_A = \{ f \; \vert \; Cod(f) = A \}$ принадлежит $J(A)$
	\item если $S \in J(A)$ и $h : B \to A$ - произвольный морфизм c концом в $A$, то решето $h^*(S) = \{ f \; \vert \; Cod(f) = B, h \circ f \in S \} \in J(B)$
	\item если $S \in J(A)$ и $R$ -- решето на $A$ такое, что $h^*(R) \in J(B)$ для всех $h : B \to A$, то $R \in J(A)$
\end{itemize}
\end{small}
\end{frame}

\begin{frame}{Топология Гротендика (3)}
\medskip
\begin{small}
Пусть дан \textit{сайт} $(C, J)$. Предпучок $\mathcal{F} : C^{op} \to Sets$ называется \textit{пучком}, если для каждого объекта $A \in Ob(C)$ и каждого решета $S \in J(A)$, диаграмма\\
\begin{diagram}[labelstyle=\scriptstyle,loose,height=.8em,width=2pt]
\mathcal{F}(A) & \rTo^{e} & \prod \limits_{f \in S} \mathcal{F}(Dom(f)) & \pile{\rTo^p\\ \rTo_q} & \prod \limits_{f,g \; f \in S} \mathcal{F}(Dom(g)) \\
\end{diagram}
является уравнителем,\\
\smallskip
\begin{itemize}
	\item $Dom(f) = Cod(g)$
	\item $e = \{ \mathcal{F}(f)(x) \}_f$
    \item произведение справа берется по всем $f$ и $g$, для которых определена композиция (т.е. $f \circ g \in S$)
    \item отображение $p$ определяется через значения $\mathcal{F}$ на композициях стрелок в $C$
    \item отображение $q$ определяется через действие $\mathcal{F}$, определяемое морфизмом $g$ на элементах $x_f$
\end{itemize}
\end{small}
\end{frame}


\begin{frame}{Принцип композициональности (1)}
\begin{small}
Объекты категории $Ext(D_G)$:\\
\medskip
\begin{itemize}
	\item мы рассматриваем множества диаграмм:
		\begin{itemize}
			\item $D$ -- все диаграммы над фиксированым ``алфавитом''
			\item $G \subseteq D$ -- фиксированный набор диаграмм (окрестностная грамматика)
			\item $D_G \subseteq D$ -- диаграммы, корректные относительно $G$
		\end{itemize}
	\item мы хотим, чтобы объектами категории были не просто диаграммы, но диаграммы вместе с их ``грамматическим разбором'', т.е. пары вида $(D, P)$, где $P$ -- синтаксическое покрытие (аналог дерева синтаксического разбора)
	\item т.е. $Ob(Ext(D_G)) = \{ (D_i, P_i) \; \vert \; D_i \in D_G \cup G \}$
\end{itemize}
\end{small}
\end{frame}

\begin{frame}{Принцип композициональности (2)}
\begin{small}
Морфизмы категории $Ext(D_G)$ должны реализовывать отображение включения диаграмм, уважающие их покрытия:\\
\medskip
\begin{itemize}
	\item если $A, B \in D_G$, то $Hom((A, P^A), (B, P^B))$ состоит из отображений включения $s : A \to B$, согласованных на покрытиях
	\item если $A \notin D_G$, $B \in D_G$, то $Hom((A, \varnothing), (B, P^B))$ состоит из отображений включения $A$ в диаграммы покрытия $P^B$
	\item если $A \in D_G$, $B \notin D_G$, то $Hom((A, P^A), (B, \varnothing)) = \varnothing$
	\item если $A \notin D_G$, $B \notin D_G$, то $Hom((A, \varnothing), (B, \varnothing)) = id$, если $A = B$ и $Hom((A, \varnothing), (B, \varnothing)) = \varnothing$, если $A \neq B$
\end{itemize}
\end{small}
\end{frame}

\begin{frame}{Принцип композициональности (3)}
\begin{small}
\textit{Синтаксическая топология} $J_G$ на основе окрестностной грамматики $G$ -- это топология Гротендика на $Ext(D_G)$, так что\\
\medskip
\begin{itemize}
	\item если $A \in D_G$, то $J_G(D)$ содержит максимальное решето на объекте $A$
	\item если $A = (D, P) \in D_G$, то семейство морфизмов элементов покрытия $P$ приналдежит $J_G(A)$
\end{itemize}
\end{small}
\end{frame}

\begin{frame}{Принцип композициональности (4)}
\textit{Принцип композициональности:}\\
\medskip
\begin{block}{}
\justifying
\begin{small}
Пусть $G = \{ G_a \; \vert \; a \in A \}$ -- окрестностная грамматика, $Ext(D_G)$ -- категория корректных диаграмм на множестве диаграмм $D$ и $J_G$ -- синтаксическая топология, определяемая грамматикой $G$. Предпучок значений $\mathcal{F} : Ext(D_G) \to \textbf{Sets}$ является пучком тогда и только тогда, когда каждое значение корректной диаграммы $D \in Ob(Ext(D_G))$ однозначно определяется соответствующим семейством значений на окрестностях его синтаксического покрытия.
\end{small}
\end{block}
\end{frame}

\begin{frame}{Замечания}
\medskip
\begin{small}
\begin{itemize}
	\item Окрестностная грамматика для КС-языков похожа на Tree Adjoining Grammar (Joshi, Levy, Takahashi. 1975).
	\item Линейно-индексируемые грамматики, вершинные грамматики, CCG и TAG слабо эквивалентны (Vijay-Shanker, Weir. 1994).
	\item Окрестностные грамматики и теоретико-модельный синтаксис.
\end{itemize}
\end{small}
\end{frame}



\begin{frame}{}
    \thispagestyle{empty}
    \begin{center}
        {\large Спасибо!}
    \end{center}
\end{frame}


%%% слайд помещается сюда
%% \begin{frame}{Заголовок}
%% \end{frame}

\end{document}
