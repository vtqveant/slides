\documentclass{beamer}

\usepackage[T2A]{fontenc}
\usepackage[utf8]{inputenc}
\usepackage[english,russian]{babel}
\usepackage{amssymb,amsfonts,amsmath,mathtext}
\usepackage{cite,enumerate,float,indentfirst}

\graphicspath{{images/}}

\usetheme{Pittsburgh}
\usecolortheme{whale}

\setbeamertemplate{footline}{\scriptsize{\hspace*{0.4cm}\insertframenumber}\vspace*{0.3cm}}
\beamertemplatenavigationsymbolsempty

\errorcontextlines 10000

\begin{document}
\title{\Large{Теоретико-игровая семантика}}
\author{Константин Соколов}
\institute[]
{Mathlingvo, СПбГУ\\ \bigskip  \url{http://nlu-rg.ru}}
\date{Санкт-Петербург, 2014} 
% Создание заглавной страницы
\begin{frame}
    \thispagestyle{empty}
    \titlepage
\end{frame}

\begin{frame}{План}
\setcounter{framenumber}{1}
    \begin{itemize}
		\item квантификация 
		\item теоретико-игровая семантика
    \end{itemize}
\end{frame}

\begin{frame}{Тексты}
\begin{itemize}
	\item J. Hintikka. \textit{Quantifiers in logic and quantifiers in natural languages.} 1979.
    \item J. Hintikka. \textit{Game-theoretical semantics: insights and prospects.} 1982.
\end{itemize}
\end{frame}

\begin{frame}{}
\begin{center}
Квантификация
\end{center}
\end{frame}

\begin{frame}{Квантификация (1)}
Г. Фреге, ``Begriffsschrift'', ``Смысл и денотат''\\
\bigskip
\begin{itemize}
	\item попытка разработать формальный однозначный язык науки, смысл выражений которого будет точно определен
	\item уточнение понятий ``смысл'' и ``значение''
	    \begin{itemize}
	        \item значение простого выражения -- объект 
	        \item значение сложного выражения -- истинностное значение
	    \end{itemize}
	\item принцип композициональности
\end{itemize}
\end{frame}

\begin{frame}{Квантификация (2)}
А. Тарский
\begin{itemize}
	\item смысл как условия истинности 
	\item объектный язык и метаязык
	\item Т-схема: '$P$' тогда и только тогда, когда $P$.
\end{itemize}
\end{frame}

\begin{frame}{Квантификация (3)}
Вопрос: что такое денотат (значение) выражения с квантором?\\
\bigskip
\begin{itemize}
	\item первоначальная мотивация введения кванторов в формальный язык -- формализация выражений естественного языка
	\begin{itemize}
    	\item ``для всех X верно, что...''
	    \item ``существует X такой, что...''
	    \item ``кто-то'', ``что-то''
	\end{itemize}
\end{itemize}
\end{frame}

\begin{frame}{}
\begin{center}
Квантификация по Хинтикке
\end{center}
\end{frame}

\begin{frame}{Квантификация (1)}
\begin{itemize}
	\item т.з., что логические кванторы -- формализация выражений естественного языка, ошибочна
	\item подход Монтегю недостаточен
	\item 
\end{itemize}
\end{frame}

\begin{frame}{Квантификация (2)}
\begin{itemize}
	\item семантическая игра, правила игры
	\item 
	\item 
\end{itemize}
\end{frame}

\begin{frame}{Квантификация (3)}
\begin{itemize}
	\item семантическая игра для формального языка
	\item 
	\item 
\end{itemize}
\end{frame}

\begin{frame}{Квантификация (4)}
\begin{itemize}
	\item семантическая игра для фрагмента естественного языка
	\item 
	\item 
\end{itemize}
\end{frame}




\begin{frame}{}
\begin{center}
Композициональность по Хинтикке
\end{center}
\end{frame}

\begin{frame}{Композициональность (1)}
\begin{itemize}
	\item scope ambiguities and context dependence
	\item 
	\item 
\end{itemize}
\end{frame}

\begin{frame}{Композициональность (2)}
\begin{itemize}
	\item GTS solution -- treat quantified phrases as nominals ($\exists x . \phi \to a . \phi$)
	\item 
	\item 
\end{itemize}
\end{frame}

\begin{frame}{Композициональность (3)}
\begin{itemize}
	\item example of analysis
	\item 
	\item 
\end{itemize}
\end{frame}

\begin{frame}{Композициональность (4)}
\begin{itemize}
	\item discussion: Hintikka's treatment of compositionality
	\item inside out, outside in
	\item 
\end{itemize}
\end{frame}


\begin{frame}{}
    \thispagestyle{empty}
    \begin{center}
        {\large Спасибо!}
    \end{center}
\end{frame}


%%% слайд помещается сюда
%% \begin{frame}{Заголовок}
%% \end{frame}

\end{document}
