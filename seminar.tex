\documentclass[12pt]{article}
\pagestyle{empty}

%%% Кодировки и шрифты %%%
\usepackage{cmap} % Улучшенный поиск русских слов в полученном pdf-файле
\usepackage[T2A]{fontenc} % Поддержка русских букв
\usepackage[utf8]{inputenc} % Кодировка utf8
\usepackage[english, russian]{babel} % Языки: русский, английский

%% графика
\ifx\pdfoutput\undefined
\usepackage{graphicx}
\else
\usepackage[pdftex]{graphicx}
\fi
\usepackage{wrapfig}
\usepackage{float}

%%% Математические пакеты %%%
\usepackage{amsthm,amsfonts,amsmath,amssymb,amscd} % Математические дополнения от AMS


\textheight 26cm
\textwidth 16.5cm

\oddsidemargin 0.96cm   % margins on all sides of 3.5 cm,
\evensidemargin 0.96cm  % correcting 1 inch default on left-hand
\topmargin -1.61cm      % side, and 1.5 inch default on top.

\begin{document}

\baselineskip 24pt

\parindent 0cm

\begin{wrapfigure}{r}{0.37\textwidth}
  \begin{center}
    \includegraphics[scale=0.70]{nlu-rg-logo.png}
  \end{center}
\end{wrapfigure}

\begin{center}
{\rm \Huge{Natural Language Understanding}}\\
\bigskip
{\rm \Large Reading Group}
\end{center}

\bigskip
\bigskip

\baselineskip 15pt

26 сентября 2013 года (четверг) состоится первая встреча в рамках семинара $\mathbb{NLU}/RG$, посвященного задаче понимания естественного языка. Встречи будут проходить на кафедре РВКС в ГПУ <<Политех>> раз в две недели и продолжаться 1,5 часа.\\

Занятия компьютерной лингвистикой требуют подготовки в области лингвистики, математики, логики, Computer Science, умение программировать и ставить эксперименты, поэтому одна из основных задач семинара -- собрать людей с разной базовой подготовкой, чтобы совместными усилиями разобраться в том, что удалось сделать в науке и в промышленности в задаче понимания естественного языка за последние годы.\\

Семинар будет проходить в формате <<Reading Group>>. Участники должны будут подготовиться: прочитать некоторый фрагмент из книги, статью или документацию (но не в первый раз). Тексты в основном на английском языке и относятся к следующим областям: лингвистическая семантика, формальная семантика и теория типов, матлогика и логическое программирование, математические методы. Мы планируем также рассмотреть реализации некоторых из существующих формализмов, поэтому может представиться возможность что-то запрограммировать.\\

Вход в корпус по пропускам. 26 сентября с 19:40 до 20:00 участников семинара будут встречать в фойе. При возникновении любых сложностей с поиском места семинара или проходом через вахту звоните по телефону +7-921-915-95-32 (Дмитрий Тимофеев).\\

\begin{minipage}{0.5\textwidth}
\begin{figure}[H]
\includegraphics[width=6.5cm]{map.png} 
%\caption{\label{fig:blue_rectangle} Rectangle}
\end{figure}
\end{minipage} \hfill
\begin{minipage}{0.45\textwidth}
\textbf{Место проведения:}\\
\\
Четверг, 26 сентября, 20:00\\
Политехническая улица, д. 21\\
9 учебный корпус СПбГПУ, ауд. 205\\
м. <<Площадь Мужества>>\\
\end{minipage}



\end{document}