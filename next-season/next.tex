\documentclass{beamer}

\usepackage[T2A]{fontenc}
\usepackage[utf8]{inputenc}
\usepackage[english,russian]{babel}
\usepackage{amssymb,amsfonts,amsmath,mathtext}
\usepackage{cite,enumerate,float,indentfirst}

\graphicspath{{images/}}

\usetheme{Pittsburgh}
\usecolortheme{whale}

\setbeamertemplate{footline}{\scriptsize{\hspace*{0.4cm}\insertframenumber}\vspace*{0.3cm}}
\beamertemplatenavigationsymbolsempty

\errorcontextlines 10000

\begin{document}
\title{\Large{$\mathbb{NLU}/RG$}}
\subtitle{\textit{обзор следующего сезона}}
\author{Константин Соколов}
\institute[]
{Mathlingvo, СПбГУ\\ \bigskip  \url{http://nlu-rg.ru}}
\date{Санкт-Петербург, 2014} 
% Создание заглавной страницы
\begin{frame}
    \thispagestyle{empty}
    \titlepage
\end{frame}

\begin{frame}{}
\begin{center}
2013 -- 2014
\end{center}
\end{frame}

\begin{frame}{Что было (1)}
\begin{small}
\begin{itemize}
    \item основные семантические концепции интенсиональной логики
    \begin{itemize}
        \item семантика смысла и денотата Г. Фреге
        \item теория объектов и пропозиций Б. Рассела
        \item концепция истины А. Тарского
        \item семантика возможных миров С. Крипке
        \item теоретико-типовая концепция К. Айдукевича    
    \end{itemize}
\end{itemize}
\end{small}
\end{frame}

\begin{frame}{Что было (2)}
\begin{small}
\begin{itemize}
    \item ликбез по лингвистической семантике
    \begin{itemize}
        \item грамматическая семантика
        \item лексическая семантика
        \item формальная семантика как семантика предложения
    \end{itemize}
    \item ликбез по математике и математической логике
    \begin{itemize}
        \item множества со структурой
        \item начала теории категорий
        \item начала теории моделей
        \item $\lambda$-исчисление
        \item начала теории типов и Agda
        \item логические исчисления и табличный метод
    \end{itemize}
\end{itemize}
\end{small}
\end{frame}

\begin{frame}{Что было (3)}
\begin{small}
\begin{itemize}
    \item лингвистические формализмы
    \begin{itemize}
        \item универсальная грамматика Р. Монтегю
        \item комбинаторные грамматики
        \item Dynamic Predicate Logic 
        \item Discourse Representation Theory
        \item Hybrid Logic Dependency Semantics
    \end{itemize}
    \item моделирование семантической композиции
    \begin{itemize}
        \item с помощью $\lambda$-исчисления
        \item с помощью унификации графовых структур
        \item топологическая трактовка принципа композициональности
    \end{itemize}
\end{itemize}
\end{small}
\end{frame}

\begin{frame}{}
\begin{center}
2014 -- 2015
\end{center}
\end{frame}

\begin{frame}{Что будет (1)}
\begin{small}
\begin{itemize}
    \item когнитивное направление в лингвистике и ИИ
    \begin{itemize}
        \item когнитивная семантика (Дж. Лакофф и др.)
        \item концептуальные пространства (П. Гэрденфорс и др.)
        \item моделирование когнитивных процессов (Б. Гёрцель и др.)
    \end{itemize}
	\item данные и их структура
	\begin{itemize}
    	\item Information Geometry (C.-И. Амари)
    	\item дифференциальная геометрия в машинном обучении
    	\item вычислительная топология
	\end{itemize}
\end{itemize}
\end{small}
\end{frame}

\begin{frame}{Что будет (2)}
\begin{small}
\begin{itemize}
	\item обнаружение структуры данных
	\begin{itemize}
    	\item Pattern Theory (У. Гренандер, Д. Мамфорд)
	    \item Representation Learning
	    \item Deep Learning
	\end{itemize}
	\item гибридные (логико-статистические) подходы
	\begin{itemize}
    	\item нейросимвольная интеграция
	    \item Gradient Symbol Processing (П. Смоленский)
	    \item композициональные дистрибутивные семантические модели (Митчелл, Лапата, Садрзаде и пр.)
	\end{itemize}
\end{itemize}
\end{small}
\end{frame}


\begin{frame}{}
    \thispagestyle{empty}
    \begin{center}
        {\large Спасибо!}
    \end{center}
\end{frame}


%%% слайд помещается сюда
%% \begin{frame}{Заголовок}
%% \end{frame}

\end{document}
