\documentclass{beamer}

\usepackage[T2A]{fontenc}
\usepackage[utf8]{inputenc}
\usepackage[english,russian]{babel}
\usepackage{amssymb,amsfonts,amsmath,mathtext}
\usepackage{cite,enumerate,float,indentfirst}

\graphicspath{{images/}}

\usetheme{Pittsburgh}
\usecolortheme{whale}

\setbeamertemplate{footline}{\scriptsize{\hspace*{0.4cm}\insertframenumber}\vspace*{0.3cm}}
\beamertemplatenavigationsymbolsempty

\errorcontextlines 10000

\begin{document}
\title{\Large{Обобщённые кванторы}}
\author{Константин Соколов}
\institute[]
{Mathlingvo, СПбГУ\\ \bigskip  \url{http://nlu-rg.ru}}
\date{Санкт-Петербург, 2014} 
% Создание заглавной страницы
\begin{frame}
    \thispagestyle{empty}
    \titlepage
\end{frame}

\begin{frame}{План}
\setcounter{framenumber}{1}
    \begin{itemize}
		\item История вопроса
		\item Определение по Мостовскому
		\item Обобщённые кванторы в формальной семантике
    \end{itemize}
\end{frame}

\begin{frame}{Тексты}
\begin{itemize}
    \item R. Montague. The Proper Treatment of Quantification in Ordinary English. 1973.
	\item J. Barwise and R. Cooper. Generalized Quantifiers and Natural Language. 1981
\end{itemize}
\end{frame}

\begin{frame}{}
\begin{center}
История вопроса
\end{center}
\end{frame}

\begin{frame}{История вопроса (1)}
Г. Фреге, ``Begriffsschrift''\\
\bigskip
\begin{itemize}
	\item Квантор всеобщности и квантор существования 
	    \medskip
	    \begin{itemize}
	        \item формализуют выражения естественного языка, такие как \textit{для всех X}, \textit{существует X}, \textit{кто-то}, \textit{что-то}, \textit{все}, \textit{любой} и т.п.
	        \medskip
        	\item cоответствуют общим и частным предложениям аристотелевской силлогистики
        \end{itemize}
    \bigskip
    \item Многие естественно-языковые конструкции нельзя выразить с их помощью
        \medskip
        \begin{itemize}
            \item \textit{больше половины}, \textit{бесконечно много}, \textit{most}
        \end{itemize}
\end{itemize}
\end{frame}

\begin{frame}{История вопроса (2)}
Б. Рассел, А. Уайтхед, ``Principia Mathematica``\\
\bigskip

\end{frame}

\begin{frame}{История вопроса (2)}
К. Айдукевич, ``О синтаксической связности'' (1935)
\bigskip
\begin{itemize}
	\item ``операторы'' -- это выражения, связывающие переменные 
	\medskip
	    \begin{itemize}
	        \item $\forall$, $\exists$, $\Sigma$, $\Pi$, $\int$
	    \end{itemize}
    \medskip
	\item реализация идеи разделения связывания переменной и квантификации
	    \medskip
	    \begin{itemize}
	        \item универсальный квантор заменяется \textit{универсальным оператором} $U$, параметризуемым выражением с оператором абстракции: $U((\hat{x}).fx) = \forall x \, . \, fx$
	        \medskip
	        \item отмечается существование многих универсальных функторов с различными категориями значения, зависящими от категории значений функтора, служащего для них аргументом.
	    \end{itemize}
\end{itemize}
\end{frame}

\begin{frame}{История вопроса (3)}
А. Чёрч, ``A formulation of the simple theory of types'' (1940)
\bigskip
\begin{itemize}
    \item универсальный квантор: $\Pi_{o(o\alpha)}[\lambda x_\alpha A_o] \equiv \forall x_\alpha A_o$
    \medskip
    \item реализация идеи \textit{определения} квантора с помощью характеристической функции (типа $\alpha \to o$).
\end{itemize}
\end{frame}


\begin{frame}{Квантификация (3)}

\end{frame}

\begin{frame}{}
\begin{center}
Определение по Мостовскому
\end{center}
\end{frame}

\begin{frame}{Определение (1)}
\begin{itemize}
	\item 
	\item 
	\item 
\end{itemize}
\end{frame}

\begin{frame}{Определение (2)}
\begin{itemize}
	\item 
	\item 
	\item 
\end{itemize}
\end{frame}

\begin{frame}{}
\begin{center}
Обобщённые кванторы в формальной семантике
\end{center}
\end{frame}

\begin{frame}{Квантификация (3)}
\begin{itemize}
	\item 
	\item 
	\item 
\end{itemize}
\end{frame}

\begin{frame}{Квантификация (4)}
\begin{itemize}
	\item 
	\item 
	\item 
\end{itemize}
\end{frame}



\begin{frame}{}
    \thispagestyle{empty}
    \begin{center}
        {\large Спасибо!}
    \end{center}
\end{frame}



\end{document}
