\documentclass{beamer}

\usepackage[T2A]{fontenc}
\usepackage[utf8]{inputenc}
\usepackage[english,russian]{babel}
\usepackage{amssymb,amsfonts,amsmath,mathtext}
\usepackage{cite,enumerate,float,indentfirst}

\graphicspath{{images/}}

\usetheme{Pittsburgh}
\usecolortheme{whale}

\setbeamertemplate{footline}{\scriptsize{\hspace*{0.4cm}\insertframenumber}\vspace*{0.3cm}}
\beamertemplatenavigationsymbolsempty

\errorcontextlines 10000

\begin{document}
\title{\Large{Обобщённые кванторы}}
\author{Константин Соколов}
\institute[]
{Mathlingvo, СПбГУ\\ \bigskip  \url{http://nlu-rg.ru}}
\date{Санкт-Петербург, 2014} 
% Создание заглавной страницы
\begin{frame}
    \thispagestyle{empty}
    \titlepage
\end{frame}

\begin{frame}{Тексты}
\setcounter{framenumber}{1}
\begin{itemize}
    \item R. Montague. The Proper Treatment of Quantification in Ordinary English. 1971
	\item J. Barwise and R. Cooper. Generalized Quantifiers and Natural Language. 1981
\end{itemize}
\end{frame}

\begin{frame}{План}
\begin{itemize}
    \item История вопроса
	\item Обобщённые кванторы по Мостовскому
	\item Обобщённые кванторы в формальной семантике
\end{itemize}
\end{frame}

\begin{frame}{}
\begin{center}
История вопроса
\end{center}
\end{frame}

\begin{frame}{История вопроса (1)}
Г. Фреге, ``Begriffsschrift'', 1879
\bigskip
\begin{itemize}
	\item квантор всеобщности и квантор существования 
	    \medskip
	    \begin{itemize}
	        \item формализуют выражения естественного языка, такие как \textit{для всех X}, \textit{существует X}, \textit{кто-то}, \textit{что-то}, \textit{все}, \textit{любой} и т.п.
	        \medskip
        	\item cоответствуют общим и частным предложениям аристотелевской силлогистики
        \end{itemize}
    \bigskip
    \item многие естественно-языковые конструкции нельзя выразить в логике первого порядка
        \medskip
        \begin{itemize}
            \item \textit{больше половины}, \textit{бесконечно много}, \textit{most}
        \end{itemize}
\end{itemize}
\end{frame}

\begin{frame}{История вопроса (2)}
Б. Рассел, А. Уайтхед, ``Principia Mathematica'', 1910-1913
\bigskip
\begin{itemize}
    \item \textit{оператор интенсиональной абстракции} $\hat{\cdot}$ с очевидной теоретико-множественной мотивацией
    \medskip
    \begin{itemize}
        \item если $fx$ -- символ неопределенного значения функции\\(т.е. $f(x) \in Im(f)$), то ``$f\hat{x}$'' - это сама функция (т.е. $f$)
        \medskip
        \item выражение ``$(\hat{x}).fx$'' имеет денотатом то же, что ``$f$'', т.е. то же, что ``$f\hat{x}$''
        \medskip
        \item выражения ``$(\hat{x}\hat{y}).fxy$'' и ``$f\hat{x}\hat{y}$'' эквавалентны (ср. $\lambda x.\lambda y.fxy$)
    \end{itemize}
    \medskip
    \item ср. с обозначением квантора всеобщности $(x).fx$ и с $\eta$-конверсией в $\lambda$-исчислении ($P = \lambda x . Px$)
\end{itemize}
\bigskip

\end{frame}

\begin{frame}{История вопроса (3)}
К. Айдукевич, ``О синтаксической связности'', 1935
\bigskip
\begin{itemize}
	\item ``операторы'' -- это выражения, связывающие переменные 
	\medskip
	    \begin{itemize}
	        \item $\forall$, $\exists$, $\Sigma$, $\Pi$, $\int$
	    \end{itemize}
    \medskip
	\item реализация идеи разделения связывания переменной и квантификации
	    \medskip
	    \begin{itemize}
	        \item универсальный квантор заменяется \textit{универсальным оператором} $U$, параметризуемым выражением с оператором абстракции: $U((\hat{x}).fx) = \forall x \, . \, fx$
	        \medskip
	        \item отмечается существование многих универсальных функторов с различными категориями значения, зависящими от категории значений функтора, служащего для них аргументом
	    \end{itemize}
\end{itemize}
\end{frame}

\begin{frame}{История вопроса (4)}
А. Чёрч, ``A formulation of the simple theory of types'', 1940
\bigskip
\begin{itemize}
    \item универсальный квантор: $\Pi_{o(o\alpha)}[\lambda x_\alpha A_o] \equiv \forall x_\alpha A_o$
    \medskip
    \item реализация идеи \textit{определения} квантора с помощью характеристической функции (типа $\alpha \to o$).
\end{itemize}
\end{frame}


\begin{frame}{}
\begin{center}
Обобщённые кванторы по Мостовскому
\end{center}
\end{frame}

\begin{frame}{Теория обобщённых кванторов}
А. Мостовский, ``On a Generalization of Quantifiers'', 1957
\bigskip
\begin{itemize}
	\item построил исчисление предикатов с нестандартными кванторами и показал его неполноту
	\item дал теоретико-модельное определение квантора
\end{itemize}
\bigskip
\bigskip
Tеория обобщённых кванторов -- более двухсот работ \\к началу 80-х годов
\end{frame}

\begin{frame}{Определение (1)}
\begin{itemize}
    \item квантор задаёт подмножество множества присваиваний значений переменных, отличающихся только значением связанной этим квантором переменной
    \medskip
    \begin{itemize}
        \item напр., $M \models \forall x \psi(x, b_1, \dots, b_n) \Leftrightarrow M \models \psi(a, b_1, \dots, b_n)$ \\для каждого $a \in D$, где $D$ -- домен интерпретации
    \end{itemize}
    \medskip
	\item можно определить такое множество явным образом как $\psi(x, b_1, \dots, b_n)^{M,x} = \{ a \in D \; | \; M \models \psi(a, b_1, \dots, b_n) \}$
\end{itemize}
\end{frame}

\begin{frame}{Определение (2)}
\begin{itemize}
	\item обобщённый квантор $Q$ -- это отображение, сопоставляющее произвольному непустому множеству (связанных переменных) множество $Q_D$ подмножеств домена интерпретации $D$ такое, что $M \models Q x \phi(x, b_1, \dots, b_n) \Leftrightarrow \phi(x, b_1, \dots, b_n)^{M,x} \in Q_D$
    \bigskip
    \item примеры кванторов:
    \medskip
    \begin{itemize}
        \item $\forall$, для которого $\forall_D = \{ D \}$
        \medskip
        \item $\exists$, для которого $\exists_D = \{ A \subseteq D \; | \; A \neq \varnothing \}$
        \medskip
        \item $\exists_{\geq 5}$ со значением ``существует не менее пяти'',\\ для которого $(\exists_{\geq 5})_D = \{ A \subseteq D \; | \; |A| \geq 5 \}$
    \end{itemize}	
\end{itemize}
\end{frame}

\begin{frame}{Определение (3)}
\begin{itemize}
	\item можно определить кванторы, связывающие более одной переменной в нескольких формулах \textit{(опр. опускаем)}
	\medskip
	\item важные частные случаи:
	\medskip
	\begin{itemize}
        \item квантор типа $\langle 1 \rangle$ связывает переменную в одной формуле
        \medskip
        \item квантор типа $\langle 1, 1 \rangle$ связывает переменную в двух формулах
    \end{itemize}
\end{itemize}
\end{frame}

\begin{frame}{}
\begin{center}
Обобщённые кванторы в формальной семантике
\end{center}
\end{frame}

\begin{frame}{PTQ (1)}
Р. Монтегю, ``The Proper Treatment of Quantification in Ordinary English'', 1971
\bigskip
\begin{itemize}
    \item анализ употребления слов \textit{some} и \textit{every}, именных групп \\и имен собственных, в т.ч. в интенсиональных контекстах
	\item строится формализм второго порядка, но обобщённые кванторы в смысле Мостовского не используются
	\item именные группы, включая имена собственные, рассматриваются как кванторы
\end{itemize}
\end{frame}

\begin{frame}{PTQ (2)}
Как формализовать выражение \textit{Джон умен}
\bigskip
\begin{itemize}
   	\item обычный подход: денотат слова \textit{Джон} -- индивид, признаки задают множества индивидов, т.е. \textit{Джон из числа умных}
   	\medskip
   	\item подход PTQ: отдельные признаки -- индивиды, имя -- множество признаков или символ предиката,\\ т.е. \textit{быть умным -- один из признаков Джона}
\end{itemize}
\end{frame}

\begin{frame}{PTQ (3)}
\begin{itemize}
	\item универсальный квантор трактуется как пересечение эксенсионалов, квантор существования -- как объединение экстенсионалов (задаваемых характеристической функцией)
    \medskip
    \item \textit{every man}: $\lambda P . \forall x . [ man(x) \to P(x) ]$ 
        \medskip
        \begin{itemize}
            \item[] т.е. множество признаков, задаваемое характеристической функцией $\lambda x . man(x)$, есть подмножество множества признаков, задаваемого характеристической функцией $\lambda x . Px$
        \end{itemize}
\end{itemize}
\end{frame}

\begin{frame}{Обобщённые кванторы в семантике (1)}
J. Barwise and R. Cooper. Generalized Quantifiers and Natural Language. 1981
\bigskip
\begin{itemize}
	\item начало применения теории обобщённых кванторов к явлениям естественного языка
	\item необходимость расширения формальных средств по причине невыразимости многих выражений в логике первого порядка
	\item потребность в улучшении соответствия синтаксиса формального языка синтаксису естественного языка
	\item анализ свойств обобщенных кванторов
\end{itemize}
\end{frame}

\begin{frame}{Обобщённые кванторы в семантике (2)}
\begin{itemize}
	\item \textit{some} или \textit{most} -- кванторы типа $\langle 1, 1 \rangle$
	\medskip
	\item именные группы вроде \textit{most students} -- кванторы типа $\langle 1 \rangle$
	\medskip
	\item образование именной группы посредством присоединения существительного к детерминативу соответствует каррированию квантора типа $\langle 1, 1 \rangle$
	\medskip
	\item такая трактовка именных групп отвечает принципу композициональности и синтаксической интуиции
	\medskip
	\begin{itemize}
	    \item напр., \textit{Иван и трое студентов} -- квантор типа $\langle 1 \rangle$
    \end{itemize}
\end{itemize}
\end{frame}

\begin{frame}{Свойства кванторов (1)}
\begin{itemize}
	\item \textit{консервативность}: для любого домена интерпретации $D$ и $A, B \subseteq D$ выполняется $Q_D(A, B) \Leftrightarrow Q_D(A, A \cap B)$
	\medskip
	\begin{itemize}
	    \item \textit{большинство студентов курит} =\\ \textit{большинство студентов -- курящие студенты}
    \end{itemize}
    \bigskip
	\item \textit{симметричность}: $Q_D(A, B) \Leftrightarrow Q_D(B, A)$
	\medskip
	\begin{itemize}
	    \item квантор \textit{некоторые} симметричен, ср. \textit{некоторые автомобили -- грузовики} и \textit{некоторые грузовики -- автомобили}
	    \medskip
	    \item квантор \textit{большинство} -- нет, ср. \textit{большинство автомобилей -- грузовики} и \textit{большинство грузовиков -- автомобили}
    \end{itemize}	
	\end{itemize}
\end{frame}

\begin{frame}{Свойства кванторов (2)}
\begin{itemize}
	\item квантор $Q$ типа $\langle 1, 1 \rangle$ называется \textit{возрастающим (убывающим) по правому аргументу} (обозначается $mon \! \uparrow$ и $mon \! \downarrow$), если для любого домена интерпретации $D$ и $A, B, B' \subseteq D$, $B \subseteq B'$ ($B' \subseteq B$) выполняется $Q_D(A, B) \Rightarrow Q_D(A, B')$. 
	\medskip
	\item возрастающий или убывающий квантор называется \textit{монотонным}
\end{itemize}
\end{frame}

\begin{frame}{Свойства кванторов (3)}
\begin{itemize}
   	\item большинство несоставных именных групп в английском языке монотонны
   	\item почти все детерминативы, т.е. кванторы типа $\langle 1, 1 \rangle$, монотонны по правому аргументу
   	\item имена собственные возрастают
   	\item в выражениях с \textit{there is} обычно употребляются симметричные кванторы, ср. \textit{there are at least five men in the garden} vs. \textit{$^*$there are most men in the garden} 
\end{itemize}
\end{frame}





\begin{frame}{}
    \thispagestyle{empty}
    \begin{center}
        {\large Спасибо!}
    \end{center}
\end{frame}



\end{document}
