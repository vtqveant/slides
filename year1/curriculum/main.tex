\documentclass[a4paper,12pt]{article}
\input{packages}
\input{styles}

\title{Natural Language Understanding\\ Reading Group}
\author{
        Константин Соколов\\
        e-mail: vtqveant@gmail.com\\
        twitter: @vtqveant
}
\date{\today}

\begin{document}
\maketitle

\section{Общая концепция}

Семинар посвящен пониманию естественного языка. Основная задача, которую преследует цикл в целом -- подойти к обсуждению вопроса о возможных путях интеграции двух различных подходов к задачам лингвистической семантики и обработки естественного языка: подхода на основе символических (логических) формализмов и в известной степени противоречащему ему в современной практике подхода на основе статистического машинного обучения. Предполагается, что современные достижения в целом ряде смежных областей могут позволить выработать интегральный подход, потребность в котором остро ощущается в большом числе практических приложений. 

Рассматриваемые темы включают введение в лингвистическую семантику, подходы к ее формализации, обсуждение выразительной силы классической логики первого порядка и различных вариантов неклассической логики, теорию моделей, логическое программирование и ряд приложений к задачам понимания естественного языка. Другая линия относится к   рассмотрению современных разработок и подходов в области машинного обучения, в особенности ориентированных на обнаружение признаков или нахождение наилучшего представления данных, а также включает работы, посвященные задаче совместного использования символических и статистических методов в рамках интегрированных систем. Отдельный блок посвящен математическим методам и теориям, используемым в рассматриваемых задачах. Хотя задачи глубокого изучения различных математических и логических теорий не ставится, ожидается, что участники семинара смогут освоиться с нотацией и своеобразным <<математическим метаязыком>>, часто затрудняющими понимание даже вполне естественных и интуитивных логических соображений.

Целевая аудитория -- участники <<R\&D-группы>> и все интересующиеся. Мероприятие предполагается в формате <<reading group>>, частота проведения -- раз в две недели. Перед встречей участникам будет предложена статья или фрагмент книги для ознакомления. Предполагаемый регламент на самом мероприятии: а) краткое введение в проблематику, обоснование выбора статьи (ведущий, 20 минут), б) доклад одним из участников по тексту статьи (40 минут), в) обсуждение (30 минут). Ориентировочное время на подготовку участников к встрече (самостоятельное прочтение статьи) - около двух часов. Тексты для чтения преимущественно на английском языке, некоторая часть на русском. Предпочтение отдается основополагающим работам и кратким обзорам. Возможно проведение демонстраций программных систем (от участников может потребоваться наличие ноутбука для выполнения заданий). Требования к начальной подготовке: интерес к проблематике, способность читать.


\section{Примерный список работ и тем}

\subsection{Формальная семантика}

\textit{Задачи блока: предоставить краткое введение в формальную семантику без претензии на полноту и глубину; закрыть пробелы, ввести основную терминологию, понять проблематику, познакомиться с историческим контекстом, узнать несколько имен.}

\subsubsection{<<Фон>>}
\begin{enumerate}
  \item Основные идеи. 
    \subitem [Фреге, <<Смысл и денотат>>, 1892] (денотат и референт)
    \subitem [Огден, Ричардс. <<Значение значения>> 1923] (<<семантический треугольник>>, субстанциональное и реляционное понимание значения)
    \subitem [Карнап Р. Эмпиризм, семантика и онтология. 1950] (отрывок из книги <<Значение и необходимость>>: онтологический статус абстрактных понятий в семантике, языковой каркас, экстенсионал и интенсионал)
    \subitem [Тарский А. <<Семантическая концепция истины и основания семантики>>, 1944]
    \subitem что-нибудь из Ч. Пирса про абдукцию
  \item Грамматическая семантика. 
    \subitem [Плунгян В. A. Введение в грамматическую семантику: грамматические значения и грамматические системы языков мира: Учеб. пособие. М.: РГГУ, 2011. -- стр. 17-44, 79-122] (глава 1 <<Понятие грамматического значения>>, глава 2 <<Проблемы описания семантики грамматических показателей>>)
  \item Лексическая семантика. 
    \subitem [Ю.Д.Апресян. Избранные труды. Том 1. Лексическая семантика. Синонимические средства языка. М., 1995, стр. 3-69.]
  \item Теория референции. 
    \subitem [Арутюнова Н.Д. Лингвистические проблемы референции. // В сб. <<Новое в зарубежной лингвистике. Выпуск XIII. Логика и лингвистика (Проблемы референции)>>, М.: "Радуга", 1982 -- стр. 5-40]
  \item Модальности в лингвистике. 
    \subitem [Плунгян В. A. Введение в грамматическую семантику: грамматические значения и грамматические системы языков мира: Учеб. пособие. М.: РГГУ, 2011. -- стр. 423-440, 449-489] (глава 7 <<Основные семантические граммемы глагола>>, параграфы <<Модальность и наклонение>> и <<Эвиденциальность>>)
  \item Контекст, дискурс, прагматика. 
    \subitem [Allegranza, 1995. Formal Semantics for Natural Language: a Non-Technical Overview. In <<Semantics and Discourse: an NLP Perspective>>, Van Eynde, Allegranza (eds.), vol.9, 1995, pp. 11-43]
  \item Альтернативные подходы к построению формальной семантики
    \subitem Proof-theoretical semantics (Gentzen, Prawitz, Dummett)
    \subitem Game-theoretical semantics [J. Hintikka and G. Sandu, 2009, <<Game-Theoretical Semantics>> in Keith Allan (ed.) Concise Encyclopedia of Semantics, Elsevier, ISBN 0-08095-968-7, pp. 341–343]
    \subitem Catastrophe-theoretical semantics (Ren\'{e} Thom) [Wolfgang Wildgen, 1982. Catastrophe Theoretic Semantics. An elaboration and application of Ren\'{e} Thom's theory.]
\end{enumerate}

\subsubsection{<<Фигуры>>}
\begin{enumerate}
  \item Грамматика Монтегю. 
    \subitem [Montague, Richard, 1973. <<The Proper Treatment of Quantification in Ordinary English>>]
  \item Теория типов в формальной семантике
    \subitem [Muskens, 2011. Type-logical Semantics]
  \item Динамическая семантика 
    \subitem [Dekker, 2008. A Guide to Dynamic Semantics]
  \item Категориальная грамматика. 
    \subitem [Казенин, <<Категориальная грамматика>> // Тестелец Я. Г. Введение в общий синтаксис. М.: РГГУ, 2001 -- стр. 664-692].
\end{enumerate}


\subsection{Логика и логическое программирование}

\textit{Задачи блока: узнать об основных подходах к моделированию семантики естественных языков; сформировать четкое представление о сущности семантики через обращение к теории моделей. Последовательно рассмотреть ряд конструкций, предлагавшихся для моделирования семантики (от простых табличных методов для классической логики высказываний к сложным теоретико-категорным конструкциям для неклассических логик); рассмотреть ряд практических методов и реализаций из области логического программирования, в т.ч. в приложении к обработке естественного языка.}

\subsubsection{Теория моделей в приложении к лингвистической семантике}
\begin{enumerate}
  \item Логики для описания фрагментов естественного языка: какие подходят и какие именно естественно-языковые явления способны выразить. 
    \subitem [Moss, Tiede.  Applications of modal logic in linguistics.] 
    \subitem [Borschev, Partee. 1999. Semantika genitivnoj konstrukcii: raznye podxody k formalizacii.] (<<Московская школа>>).
  \item Теория доказательств vs. теория моделей. Как теория моделей позволяет определить семантику. 
      \subitem [Schroeder-Heister, P. <<Proof-theoretic Versus Model-theoretic Consequence.>> The Logica Yearbook, 2007, pp. 1–12.] 
      \subitem [Geoffrey K. Pullum, Barbara C. Scholz, On the Distinction between Model-Theoretic and Generative-Enumerative Syntactic Frameworks, Proceedings of the 4th International Conference on Logical Aspects of Computational Linguistics, p.17-43, June 27-29, 2001]
  \item Обзор про неклассические логики. Интуиционизм, многозначные логики, модальные логики, немонотонные логики, параконсистентные логики. 
    \subitem [Верещагин, Шень. Языки и исчисления. М.: Издательство МЦНМО, 2008. -- стр. 69-86] (глава <<Интуиционисткая пропозициональная логика>>), 
    \subitem [Restall, G. <<Relevant and Substructural Logics.>> Handbook of the History of Logic (2006): 1–105. Web. 23 Jan. 2013.] (большой обзор как по теории доказательств, так и по теории моделей; можно брать только часть про теорию моделей -- pp. 51-101.)
  \item Табличные методы. 
    \subitem [Антонова. Табличные методы в логике. СПб, 2003 -- стр. 40-52, 152-158] (главы <<Табличный метод и история его развития>> и <<Семантический анализ логики методом таблиц. Семантические таблицы Крипке и модельные множества Хинтикки.>>)
  \item Семантика Крипке для модальной и интуиционистcкой логики. 
    \subitem [Kripke, 1965. Semantical analysis of intuitionistic logic I. Normal Modal Propositional Calculi.]
%%    \subitem [Kripke, 1965. Semantical analysis of intuitionistic logic II. Normal Modal Propositional Calculi.] 
  \item Модели на основе теории пучков. 
    \subitem [Hilken, Rydeheard. A First Order Modal Logic and its Sheaf Models] 
    \subitem [Dale, 2011. Semantics with Algebra and Geometry.]
  \item Окрестностная семантика.
    \subitem [Шрейдер Ю. А. Топологические модели языка. // В сб. <<Проблемы структурной лингвистики. 1971>>. М.: Наука, 1972. -- стр. 47-67] 
    \subitem [Борщев В. Б. Окрестностная модель языка -- логика и/или топология (история одной идеи Ю. А. Шрейдера). Системы и модели: границы интерпретаций : cборник трудов Всероссийской научной конференции с международным участием. Москва – Томск, 5–7 ноября 2008 г. – Томск : Издательство Томского государственного педагогического университета, 2008. -- стр. 9-18.]
  \item H\'{e}rmeneutique formelle О. Прозорова.\footnote{Ср.: [Mormann, 1995. Trope Sheaves. A Topological Ontology of Tropes]}
    \subitem [Oleg Prosorov, 2005. Formal Hermeneutics Based on Frege Duality.]


\end{enumerate}

\subsubsection{Приложения}
\begin{enumerate}
  \item Reasoners и автоматическое доказательство теорем.
  \item Solvers и поиск моделей.
  \item Дедуктивные базы данных и абдукция. 
    \subitem [Дейт, К. Дж. Введение в системы баз данных, 8-е издание -- М.: Издательский дом "Вильямс", 2008 -- стр. 971-1014] (глава <<Логические системы управления базами данных>>), 
    \subitem [Vrain, C, and D Laurent. <<Updates, Induction and Abduction in Deductive Databases.>> Proc. ECAI] и вспомнить про Ч. Пирса и понятие абдукции.
  \item Circumscription и <<common sense reasoning>> (Дж. МакКарти)  
    \subitem [McCarthy, J. (April 1980). "Circumscription -- A form of non-monotonic reasoning". Artificial Intelligence 13, pp. 27–39.]
  \item Stable Model Semantics и Answer Set Programming. 
    \subitem [Eiter, Ianni, Krennwallner. 2009. Answer set programming: A primer.]
  \item Использование ASP и ATP в задаче RTE. 
    \subitem [Recognising Textual Entailment with Robust Logical Inference. J. Bos, K. Markert (2006), Machine Learning Challenges, MLCW 2005, LNAI (Vol 3944).]
\end{enumerate}

\subsection{Статистическое машинное обучение}

\textit{Задачи блока: познакомиться с идеей отыскания представления данных в структурном (символическом,  алгебраическом) виде с помощью машинного обучения.}

\begin{enumerate}
  \item Pattern Theory. 
    \subitem [U. Grenander. Pattern Theory: From Representation to Inference.]
  \item Representation Learning. 
    \subitem [Bengio, Yoshua, Aaron C Courville, and Pascal Vincent. <<Represenation Learning: a Review and New Perspectives.>> CoRR abs/1206.5.1993 (2012), pp. 1–34]
  \item Дифференциальная геометрия в машинном обучении. 
    \subitem [der Maaten, et al. <<Dimensionality Reduction: A Comparative Review.>> Journal of Machine Learning Research 10. 2008, pp. 1–41.]
  \item Вычислительная топология в машинном обучении. 
    \subitem [Edelsbrunner, Herbert, and John Harer. <<Persistent Homology--a survey.>> Contemporary mathematics (2008), pp. 1–26]
  \item Information Geometry (приложения дифференциальной геометрии к статистике)\footnote{См. тж.: Information Theory of Cognitive Systems Research Group.}. 
    \subitem [Amari, S. «Differential Geometrical Theory of Statistics.» ... Monograph vol. 10, Differential Geometry in Statistical ... (1987), pp. 19–94.] 
    \subitem [Goertzel, Ikle. 2011. Three hypotheses About the Geometry of Mind] (Information Geometry в когнитивной архитектуре в AI)
  \item Глубокие архитектуры. 
    \subitem [Bengio, Yoshua. <<Learning Deep Architectures for AI.>> Foundations and Trends in Machine Learning 2.1 (2009), pp. 1–127]
  \item Нейросимволическая интеграция и когнитивные архитектуры. 
    \subitem [Bader, Hitzler. 2005. Dimensions of neural-symbolic integration--a structured survey.]
    \subitem [Fenstad, Jens Erik (2006). Grammar, Geometry and Brain, In Torgrim Solstad; Atle Gronn \& Dag Trygve Truslew Haug (ed.),  A Festschrift for Kjell Johan Saebo: In Partial Fulfilment of the Requirements for the Celebration of his 50th Birthday.  Forfatterne.  ISBN 978-82-997289-0-4.  Kapittel.  s 59 - 73]
  \item Gradient Symbol Processing 
    \subitem [Smolensky, Goldrick, Mathis. Optimization and quantization in gradient symbol systems: A framework for integrating the continuous and the discrete in cognition. Cognitive Science (in press, 2012)]
\end{enumerate}

\subsection{Матметоды}

\textit{Задачи блока: познакомиться с рядом математических методов, используемых для решения рассматривавшихся задач; осознать близость методов и взаимосвязь задач.}\footnote{<<Чистая математика работает с моделями высокого уровня абстрактности и малой сложности (математики эту малую сложность любят называть элегантностью).  Прикладная математика работает с моделями более конкретными, но высокого уровня сложности (много переменных, уравнений и т.д.). Интересные применения идей современной чистой математики скорее всего лежат в области высокой абстрактности и высокой сложности. Эта область сейчас практически недоступна, во многом из-за ограниченной способности человеческого мозга работать с такими моделями. Когда мы научимся использовать компьютеры для работы с абстрактными математическими объектами, эта проблема постепенно отойдет на второй план и появятся интересные приложения идей сегодняшней абстрактной математики.>> (В. Воеводский)}

\begin{enumerate}
  \item Множества со структурой. 
    \subitem [Н. Бурбаки, Архитектура математики.]
  \item Введение в теорию категорий 
    \subitem [Kostecki, 2011. <<An Introduction to Topos Theory.>> pp. 1-48]
  \item Теория топосов, связь теории категорий с логикой. 
    \subitem [Kostecki, 2011. <<An Introduction to Topos Theory.>> pp. 48-93]  
  \item Пучки и когомологии с коэффициентами в пучках.\footnote{Можно заменить как [Kostecki 2011], так и что-то по пучкам и пр. книгой [Голдблатт Р. Топосы. Категорный анализ логики. М.: Мир, 1983]}
  \item Homotopy Type Theory 
    \subitem [Awodey, Pelayo, Warren, 2013. Voevodsky's Univalence Axiom in homotopy type theory]
\end{enumerate}


%% \section{Замечание об истории теоретико-модельной и топологической семантики в России}
%% В России идеи соотнесения теоретико-модельных представлений в семантике с топологией имеют свою любопытную историю. В 1967 г. Ю. А. Шрейдер предложил т.н. <<окрестностную модель языка>>, которая легла в основу разработок в области <<окрестностных грамматик>> на основе теоретико-модельного подхода к описанию синтаксиса в работах B. Б. Борщева и М. В. Хомякова, выходивших с 1967 по 1974 гг.\footnote{См.: Борщев В. Б. Окрестностная модель языка -- логика и/или топология (история одной идеи Ю. А. Шрейдера). Системы и модели: границы интерпретаций : cборник трудов Всероссийской научной конференции с международным участием. Москва – Томск, 5–7 ноября 2008 г. – Томск : Издательство Томского государственного педагогического университета, 2008. -- стр. 9-18.} Позже под влиянием этих работ В. А. Лапшин в своей диссертации\footnote{Лапшин В.А. Особенности синтаксического анализа открытых интерфейсных контекстно-свободных языков. Дисс. ... кандидата физико-математических наук. 2005.; Лапшин В.А. Языки синтаксических диаграмм. Окрестностные грамматики и их топологические интерпретации: Рукопись. РГГУ. М., 2008. Окрестностные грамматики также рассматриваются в учебнике: Лапшин В. А. Лекции по математической лингвистике, М., Научный мир. 2010.} предложил использование топологического и теоретико-категорного подхода в синтаксисе. В настоящее время В. А. Лапшин -- руководитель группы семантического поиска в компании ABBYY (Москва) и можно предположить, что развитие этих идей продолжается и сейчас в практических разработках.


\end{document}
