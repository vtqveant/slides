\documentclass{beamer}

\usepackage[T2A]{fontenc}
\usepackage[utf8]{inputenc}
\usepackage[english,russian]{babel}
\usepackage{amssymb,amsfonts,amsmath,mathtext}
\usepackage{cite,enumerate,float,indentfirst}

%% Gentzen style natural deduction proof trees
\usepackage{bussproofs}
\usepackage{latexsym}

\graphicspath{{images/}}

\usetheme{Pittsburgh}
\usecolortheme{whale}

\setbeamertemplate{footline}{\scriptsize{\hspace*{0.4cm}\insertframenumber}\vspace*{0.3cm}}
\beamertemplatenavigationsymbolsempty

\errorcontextlines 10000

\begin{document}
\title{\huge{$\mathbb{NLU}/RG$, \textit{pt. ???}}}
\author{Константин Соколов}
\institute[]
{Mathlingvo, СПбГУ, i-Free\\ \bigskip  \url{http://nlu-rg.ru}
}
\date{Санкт-Петербург, 2013} 
% Создание заглавной страницы
\begin{frame}
    \thispagestyle{empty}
    \titlepage
\end{frame}

%%% 0. План
\begin{frame}{План}
    \setcounter{framenumber}{1}
    \begin{itemize}
        \item 
        \item 
        \item 
    \end{itemize}
\end{frame}


\begin{frame}{Естественный вывод для пропозициональной логики (1)}
\begin{itemize}
  \item Впервые предложен Я. Лукасевичем и С. Яськовским.
  \item Мы рассмотрим вариант Г. Генцена.
  \item Исчисления такого типа (``синтаксические'') - инструмент \textit{теории доказательств}.
\end{itemize}
\end{frame}

\begin{frame}{Естественный вывод для пропозициональной логики (2)}
\textit{Формальная система} или \textit{исчисление} задается с помощью
\begin{enumerate}
  \item формального языка
  \item множества аксиом
  \item правил вывода
  \item определения формального вывода
\end{enumerate}
\end{frame}

\begin{frame}{Естественный вывод для пропозициональной логики (3)}
\begin{itemize}
  \item Пропозициональная система естественного вывода основана на языке логики высказываний
  \item Множество аксиом пусто
  \item Правила вывода: правила введения и удаления связок (классическая и интуиционистсткая пропозициональная логика) + правило снятия двойного отрицания (для классической логики высказываний)
  \item Определениe формального вывода: \textit{чуть позже}
\end{itemize}
\end{frame}

\begin{frame}{Язык логики высказываний (Алфавит)}
\begin{itemize}
  \item $\land$, $\lor$, $\supset$, $\neg$, $($, $)$, $p$.
  \item $p_n = (p...p)$ (n раз) - переменные
\end{itemize}
\end{frame}

\begin{frame}{Язык логики высказываний (Формула)}
Формулы языка логики высказываний определяются \textit{рекурсивно}:
\begin{itemize}
  \item $p_n$ - формула
  \item если $A$ и $B$ - формулы, то $(A \land B)$, $(A \lor B)$, $(A \supset B)$ и $(\neg A)$ - формулы
  \item все прочие слова в том же алфавите - не формулы\\ (напр. $(ppppp) \neg (ppp) \supset$ - не формула)
\end{itemize}
\end{frame}

\begin{frame}{Правила вывода}

%% введение и удаление конъюнкции

\begin{prooftree}
  \AxiomC{$A$}
  \AxiomC{$B$}
  \RightLabel{\scriptsize $\land_I$}
  \BinaryInfC{$A \land B$}
\end{prooftree}

\begin{prooftree}
  \AxiomC{$A \land B$}
  \RightLabel{\scriptsize $\land_E$}
  \UnaryInfC{$A$}
\end{prooftree}

\begin{prooftree}
  \AxiomC{$A \land B$}
  \RightLabel{\scriptsize $\land_E$}
  \UnaryInfC{$B$}
\end{prooftree}

%% введение и удаление дизъюнкции

\begin{prooftree}
  \AxiomC{$A$}
  \RightLabel{\scriptsize $\lor_I$}
  \UnaryInfC{$A \lor B$}
\end{prooftree}

\begin{prooftree}
  \AxiomC{$B$}
  \RightLabel{\scriptsize $\lor_I$}
  \UnaryInfC{$A \lor B$}
\end{prooftree}

\begin{prooftree}
\AxiomC{$A \lor B$}
\AxiomC{[$A$]}
\noLine
\UnaryInfC{$C$}
\AxiomC{[$B$]}
\noLine
\UnaryInfC{$C$}
\RightLabel{\scriptsize $\lor_E$}
\TrinaryInfC{$C$}
\end{prooftree}


\end{frame}


\begin{frame}{Правила вывода}

%% введение и удаление импликации

%% введение и удаление отрицания

%% удаление двойного отрицания (для классической системы)

\end{frame}

\begin{frame}{Дерево вывода}

$(A \supset B) \supset (\neg B \supset \neg A)$\\

\begin{prooftree}
\AxiomC{$[A]^3$}
\AxiomC{$[A \supset B]^1$}
\RightLabel{\scriptsize $\supset_E$}
\BinaryInfC{$B$}

\AxiomC{$[\neg B]^2$}
\RightLabel{\scriptsize $\neg_I$ (3)}
\BinaryInfC{$\neg A$}

\RightLabel{\scriptsize $\supset_I$ (2)}
\UnaryInfC{$\neg B \supset \neg A$}

\RightLabel{\scriptsize $\supset_I$ (1)}
\UnaryInfC{$(A \supset B) \supset (\neg B \supset \neg A)$}
\end{prooftree}

\end{frame}


\begin{frame}{Выводимость}
Формула $F$ выводима из конечного множества формул $\Gamma$ ($\Gamma \vdash F$), если существует вывод для формулы $F$ такой, что его множество \textit{зеленых листьев} $\Delta$ есть подмножество $\Gamma$ (т.е. $\Delta \subseteq \Gamma$).\\
\bigskip
Если $\Delta = \varnothing$, пишут $\vdash F$.\\
\bigskip
\textit{Зеленый лист} - открытое (существенное) допущение.\\
\bigskip
\textit{Увядший лист} - закрытое (промежуточное) допущение.\\

\end{frame}


\begin{frame}{}
    \thispagestyle{empty}
    \begin{center}
        {\large Спасибо!}
    \end{center}
\end{frame}


%%% слайд помещается сюда
%% \begin{frame}{Заголовок}
%% \end{frame}

\end{document}
