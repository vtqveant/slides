\documentclass{beamer}

\usepackage[T2A]{fontenc}
\usepackage[utf8]{inputenc}
\usepackage[english,russian]{babel}
\usepackage{amssymb,amsfonts,amsmath,mathtext}
\usepackage{cite,enumerate,float,indentfirst}

\graphicspath{{images/}}

\usetheme{Pittsburgh}
\usecolortheme{whale}

\setbeamertemplate{footline}{\scriptsize{\hspace*{0.4cm}\insertframenumber}\vspace*{0.3cm}}
\beamertemplatenavigationsymbolsempty

\errorcontextlines 10000

\begin{document}
\title{\huge{$\mathbb{NLU}/RG$, \textit{pt. 3}}}
\author{Константин Соколов}
\institute{Mathlingvo, СПбГУ, i-Free}
\date{Санкт-Петербург, 2013} 
% Создание заглавной страницы
\begin{frame}
    \thispagestyle{empty}
    \titlepage
\end{frame}

%%% 0. План
\begin{frame}{План}
    \setcounter{framenumber}{1}
    \begin{itemize}
        \item Домашнее задание
        \item Обоснование выбора статей
        \item Огден, Ричардс ``Значение значения''
        \item Карнап ``Значение и необходимость''
        \item Голдблатт ``Теория топосов'', гл. 1
        \item Отображения, морфизмы, стрелки
    \end{itemize}
\end{frame}

% 1. Домашнее задание
\begin{frame}{Домашнее задание}
\end{frame}

% 2. Обоснование выбора статей
\begin{frame}{Обоснование выбора статей (1)}
Пять основных концепций ``логики смысла'':\\
  \begin{itemize}
    \item Бикомпонентная теория значения Г. Фреге
    \item Теория объекта и пропозиции Б. Рассела
    \item Концепция истины А. Тарского
    \item Концепция возможных миров С. Крипке
    \item Теоретико-типовая концепция Б. Рассела, К. Айдукевича
  \end{itemize}
\end{frame}

% 2. Обоснование выбора статей
\begin{frame}{Обоснование выбора статей (2)}
  \begin{itemize}
    \item Огден, Ричардс о магии слов
    \item Ранний Карнап об интенсионалиях
    \item Поздний Карнап о языковом каркасе
  \end{itemize}
\end{frame}


% 3. Огден, Ричардс ``Значение значения''
\begin{frame}{Огден, Ричардс ``Значение значения''}
  \begin{itemize}
    \item Треугольник отнесенности: символ, мысль, вещь.
    \item Контексты внутренние (психологические) и внешние
    \item Теория знаковых ситуаций
  \end{itemize}
\end{frame}

% 4. Карнап 
\begin{frame}{Ранний Карнап}
  \begin{itemize}
    \item Интенсионал и экстенсионал
    \item Универсальная иерархия понятий
  \end{itemize}
\end{frame}

\begin{frame}{Карнап ``Значение и необходимость''}
  \begin{itemize}
    \item Статус абстрактных понятий
    \item Языковой каркас
  \end{itemize}
\end{frame}

% 5. Голдблатт ``Теория топосов'', гл. 1
\begin{frame}{Голдблатт ``Теория топосов'', гл. 1}
\end{frame}

% 6. Отображения, морфизмы, стрелки
\begin{frame}{Отображения, морфизмы, стрелки (2)}
Пусть $A$ и $B$ - множества, отображение $f : A \to B$ сопоставляет элементу из $A$ элемент из $B$\\
  \begin{itemize}
    \item $A = Dom(f)$ - домен (область отправления)
    \item $B = Cod(f)$ - кодомен (область прибытия) 
    \item $Im(f) = \{f(x) : x \in Dom(f)\}$ - образ 
    \item $Ker(f) = \{x \in Dom(f) : f(x) = 0\}$ - ядро 
  \end{itemize}
\end{frame}

\begin{frame}{Отображения, морфизмы, стрелки (2)}
Отображение $f : A \to B$
  \begin{itemize}
    \item сюръективно, если $Cod(f) = Im(f)$
    \item инъективно, если $f(x) = f(y) \Rightarrow x = y$, где $x, y \in Dom(f)$
    \item биективно, если оно сюръективно и инъективно
  \end{itemize}
\end{frame}

\begin{frame}{Отображения, морфизмы, стрелки (3)}
Пусть $A$ и $B$ - множества со структурой\\ \bigskip
Гомоморфизм (морфизм, стрелка) - это отображение $f : A \to B$, сохраняющее структуру\\
  \begin{itemize}
    \item эпиморфизм (эпистрелка) - сюръективный морфизм ($A \twoheadrightarrow B$)
    \item мономорфизм (монострелка) - инъективный морфизм ($A \rightarrowtail B$)
    \item изоморфизм (изострелка) - биективный морфизм 
  \end{itemize}
\end{frame}

\begin{frame}{}
    \thispagestyle{empty}
    \begin{center}
        {\large Спасибо!}
    \end{center}
\end{frame}


%%% слайд помещается сюда
%% \begin{frame}{Заголовок}
%% \end{frame}

\end{document}
