\documentclass{beamer}

\usepackage[T2A]{fontenc}
\usepackage[utf8]{inputenc}
\usepackage[english,russian]{babel}
\usepackage{amssymb,amsfonts,amsmath,mathtext}
\usepackage{cite,enumerate,float,indentfirst}

\usepackage{graphicx}
\usepackage{booktabs}
\usepackage{tabularx}

% Attribute-Value Matrices
\usepackage{avm}
\avmfont{\sc}
\avmoptions{sorted,active}
\avmvalfont{\rm}
\avmsortfont{\scriptsize\it}

% CCG parse trees
\newcommand{\deriv}[2]
{  \renewcommand{\arraystretch}{.5}
$\begin{array}[t]{*{#1}{c}}
     #2
   \end{array}$ }
\newcommand{\gf}[1]{\textsf{\textsl{#1}}}
\newcommand{\cf}[1]{\mbox{\ensuremath{\cfont{#1}}}}
\newcommand{\uline}[1]
{\mc{#1}{\hrulefill} }
\newcommand{\mc}[2]
  {\multicolumn{#1}{c}{#2}}
\newcommand{\cfont}{\mathsf}
\newcommand{\bs}{\backslash}
\newcommand{\subsa}[1]{\hspace{-0.75mm}_{_{#1}}}
\newcommand{\subsb}[1]{\hspace{-0.10mm}_{_{#1}}}
\newcommand{\subs}[1]{\hspace{-0.40mm}_{#1}}
\newcommand{\subsf}[1]{\hspace{-0.75mm}_{_{#1}}}
\newcommand{\supsa}[1]{\hspace{-1.75mm}^{^{#1}} }
\newcommand{\supsb}[1]{\hspace{-0.80mm}^{^{#1}}  }
\newcommand{\sups}[1]{\hspace{-0.40mm}^{#1}}


\graphicspath{{images/}}

\usetheme{Pittsburgh}
\usecolortheme{whale}

\setbeamertemplate{footline}{\scriptsize{\hspace*{0.4cm}\insertframenumber}\vspace*{0.3cm}}
\beamertemplatenavigationsymbolsempty

\errorcontextlines 10000

\begin{document}
\title{\Large{Формализация локальных свойств\\в структурах типизированных признаков}}
\author{Константин Соколов}
\institute[]
{Mathlingvo, СПбГУ, i-Free\\ \bigskip  \url{http://nlu-rg.ru}}
\date{Санкт-Петербург, 2014} 
% Создание заглавной страницы
\begin{frame}
    \thispagestyle{empty}
    \titlepage
\end{frame}

\begin{frame}{План}
    \setcounter{framenumber}{1}
    \begin{itemize}
        \item Структуры типизированных признаков (TFS)
        \item Логическое описание TFS (языки $L^{KR}$, $HL(@)$)
        \item Графовое представление TFS как шкала Крипке
        \item Конструкция накрытия графа, локальность
        \item Логическая характеристика свойства локальности
        \item Лингвистические приложения конструкции  % (HLDS, underspecification и др.)
    \end{itemize}
\end{frame}

% 1. Структуры типизированных признаков (TFS)
\begin{frame}{}
\begin{center}
	\textbf{Структуры типизированных признаков}
\end{center}
\end{frame}

\begin{frame}[fragile]
\frametitle{Структуры типизированных признаков}
HLDS в виде AVM:\\
\begin{center}
	\begin{avm}
	[{action} predicate & on \cr
    	      Mood & imp \cr 
        	  Actor & @{1} \cr 
	          Patient & [{thing} predicate & @{2} лампа \cr
    	                         Num & sg \cr 
        	                     Modifier & [{q-color} predicate & красный\_adj ]]]
	\end{avm}
\end{center}	
\end{frame}

% 2. Логическое описание TFS (языки $L^{KR}$, $HL(@)$)
\begin{frame}{}
\begin{center}
	\textbf{Логическое описание TFS}
\end{center}
\end{frame}


\begin{frame}{Мультимодальная логика}
\begin{itemize}
	\item Набор модальных операторов $\Box_i$ (соотв., $\Diamond_i$)
	\item Шкала Крипке - размеченный граф
\end{itemize}
\end{frame}

\begin{frame}{Гибридная логика (1)}
\begin{itemize}
	\item Будем писать $\langle \pi \rangle$ и $[\pi]$ вместо $\Diamond_\pi$ и $\Box_\pi$
	\item $\langle \pi \rangle \alpha \; \equiv \; \neg [\pi] \neg \alpha$
	\item Введем дополнительно класс \textit{номиналов} (обозн. $i, j, k$)
	\item Введем оператор $@_i$ со значением ``истинно в точке $i$''
\end{itemize}
\bigskip
Язык гибридной логики $HL(@)$:\\
\begin{center}
$WFF_{HL(@)} := \top \; \vert \; i \; \vert \; p \; \vert \; \neg \alpha \; | \; \alpha \wedge \beta \; \vert \; \langle \pi \rangle \alpha \; \vert \; @_i \alpha$
\end{center}
\end{frame}

\begin{frame}{Гибридная логика (2)}
Гибридная модель Крипке:\\
\bigskip
\begin{center}
$\mathcal{M} = (W, \{R_\pi \vert \pi \in MOD\}, V)$, где\\
\bigskip
\begin{itemize}
	\item $\mathcal{F} = (W, \{R_\pi \vert \pi \in MOD\})$ - шкала Крипке
	\item $V : PROP \cup NOM \to 2^W$
	\item $V(i)$ - синглетон, т.е. $\vert V(i) \vert = 1$
\end{itemize}
\end{center}
\end{frame}

\begin{frame}{Гибридная логика (3)}
Денотационная семантика для $HL(@)$:\\
\bigskip
\begin{itemize}
	\item $\mathcal{M}, w \models i \; \Leftrightarrow \; w = V(i)$
	\item $\mathcal{M}, w \models @_i \alpha \; \Leftrightarrow \; \mathcal{M}, w' \models \alpha$ и $w' = V(i)$
\end{itemize}
\end{frame}


% 3. Графовое представление TFS как шкала Крипке
\begin{frame}{}
\begin{center}
	\textbf{Графовое представление TFS как шкала Крипке}
\end{center}
\end{frame}

\begin{frame}{Реляционная семантика (1)}
\textit{Шкала Крипке} (Kripke frame):\\
\bigskip
$\mathcal{F} = (W, R)$, где $W$ - непустое множество, $R \subseteq W \times W$.\\
\bigskip
\begin{itemize}
  \item $w_i \in W$ - миры, состояния, точки отнесенности
  \item $R$ - отношение достижимости
  \item если $wRv$, то говорят, что $v$ \textit{возможен относительно} $w$
\end{itemize}
\bigskip
Легко заметить, что шкала Крипке - это граф.
\end{frame}

\begin{frame}{Реляционная семантика (2)}
\textit{Модель Крипке} $\mathcal{M} = (\mathcal{F}, V)$, где  
\bigskip
\begin{itemize}
  \item $\mathcal{F}$ - шкала Крипке
  \item $V : P \to 2^W$ - функция оценивания, т.е. отображение из атомарных выражений в подмножества множества миров
\end{itemize}
\bigskip
\end{frame}

\begin{frame}{Реляционная семантика (3)}
Пусть $\mathcal{M} = (\mathcal{F}, V)$, $w \in W$, $\phi$ - формула, $p \in P$\\
\bigskip
Истинность $\phi$ в модели $\mathcal{M}$ в точке $w$ определяется рекурсивно\\
\bigskip
\begin{itemize}
  \item $\mathcal{M}, w \models p \; \Longleftrightarrow \; w \in V(p)$
  \item $\mathcal{M}, w \models \neg \phi \; \Longleftrightarrow \; \mathcal{M}, w \not\models \phi$
  \item $\mathcal{M}, w \models \phi \vee \psi \; \Longleftrightarrow \; \mathcal{M}, w \models \phi$ или $\mathcal{M}, w \models \psi$
  \item $\mathcal{M}, w \models \Box \phi \; \Longleftrightarrow \; \forall v \in W \; . \; (w R v \to \mathcal{M}, v \models \phi)$
  \item $\mathcal{M}, w \not\models \perp$
\end{itemize}
\bigskip
Если $\mathcal{M}, w \models \phi$, говорят, что $\phi$ \textit{логически следует} из $\mathcal{M}, w$
\end{frame}

\begin{frame}{Гибридная логика и проверка моделей}
Дана гибридная модель $\mathcal{M}$ и формула $\alpha$, найти все узлы в $\mathcal{M}$, в которых $\alpha$ истинна:
\bigskip
\begin{center}
    $T(\mathcal{M}, \alpha) = \{ w \in W \; | \; \mathcal{M}, w \models \alpha \}$
\end{center}
\bigskip
Известные результаты:\\  % уточнить
\bigskip
\begin{itemize}
	\item сложность в худшем случае для $HL(@, \downarrow, \textbf{E})$ - $PSPACE$
	\item сложность для $H(@)$ 
		\begin{itemize}
			\item экспоненциальная по длине формулы
			\item полиномиальная по размеру модели
		\end{itemize}
\end{itemize}
\end{frame}



% 4. Конструкция накрытия графа, локальность
\begin{frame}{}
\begin{center}
	\textbf{Конструкция накрытия графа, локальность}
\end{center}
\end{frame}


% 5. Логическая характеристика свойства локальности
\begin{frame}{}
\begin{center}
	\textbf{Логическая характеристика свойства локальности}
\end{center}
\end{frame}


% 6. Лингвистические приложения конструкции  % (HLDS, underspecification и др.)
\begin{frame}{}
\begin{center}
	\textbf{Лингвистические приложения конструкции}
\end{center}
\end{frame}

\begin{frame}{Hybrid Logic Dependency Semantics}
\begin{itemize}
	\item Композициональный семантический формализм
	\item Гибридная логика и структуры зависимостей
	\item Транслируется в \textit{HL}
\end{itemize}
\end{frame}

\begin{frame}[fragile]
\frametitle{HLDS (1)}
Компактная форма:
{\footnotesize \begin{verbatim}
@w0:action(ON ^
           <Mood>imp ^
           <Actor>x1:entity ^
           <Patient>(w2:thing ^ лампа ^
                     <Num>sg ^
                     <Modifier>(w1:q-color ^ красный-adj) ^
                     <Modifier>(w3:m-location ^ на ^
                                <Anchor>(w4:e-place ^ кухня ^
                                         <Num>sg))))
\end{verbatim}}
\end{frame}

\begin{frame}[fragile]
\frametitle{HLDS (2)}
Линеаризованная форма:
{\footnotesize \begin{verbatim}
@E_0:action(CLOSE) ^ 
@E_0:action(<Mood>imp) ^
@E_0:action(<Actor>S_0:entity) ^ 
@E_0:action(<Patient>T_1:thing) ^ 
@M_3:m-location(в) ^ 
@M_3:m-location(<Anchor>T_4:e-place) ^ 
@T_1:thing(шторы) ^ 
@T_1:thing(<Num>pl) ^ 
@T_1:thing(<Modifier>M_3:m-location) ^ 
@T_4:e-place(прихожая) ^ 
@T_4:e-place(<Num>sg))
\end{verbatim}}
\end{frame}

\begin{frame}[fragile]
\frametitle{Пример в формализме dotCCG (1)}
Определение семейства слов:
{\footnotesize \begin{verbatim}
family tv(V) {
    entry : s \! np / np : 	E:event(* <Actor>(S:entity) 
                                      <Patient>(X:entity));
}
\end{verbatim}}
\bigskip
\bigskip
Запись в словарной части:
{\footnotesize \begin{verbatim}
word лампа:Noun(thing, pred=лампа) {
    лампа:  s-sg nom fem;
    лампу:  s-sg acc fem;
    лампы:  s-pl nom fem;
    лампы:  s-pl acc fem;
}
\end{verbatim}}
\end{frame}

\begin{frame}[fragile]
\frametitle{Пример в формализме dotCCG (2)}
Правила изменения типа:
{\footnotesize \begin{verbatim}
rule {
    typechange: s<10> [E NUM PERS MOOD POL FIN VFORM vf-to-imp] \! 
                np<9> [S nom NUM PERS nf-real] / 
                np<2> [X acc]
             => s<~10>[E fin-full s-imp] / 
                np<2> [X acc] : E:event(<Mood>(imp) 
                                        <Subject>(S:entity 
                                                 addressee));
}
\end{verbatim}}
\end{frame}





% ----------



\begin{frame}{Теоретический минимум}
\begin{itemize}
	\item Multimodal Combinatory Categorial Grammar (MMCCG)
	\item Hybrid Logic Dependency Semantics (HLDS)
	\item Hybrid Logic Model Checking (HLMC)
\end{itemize}
\end{frame}



\begin{frame}[fragile]
\frametitle{Гибридная логика и XML (1)}
{\footnotesize 
\begin{verbatim}
<biblio>
    <book id="b1">
        <author>Marx</author>
        <author>de Rijke</author>
        <title>Hybrid Logics</title>
        <date>1998</date>
        <cites idref="b1"/>
    </book>
    <book id="b2">
        <author>Franceschet</author>
        <title>Model Checking</title>
        <date>2000</date>
        <cites idref="a1"/>
    </book>
</biblio>
\end{verbatim}
}
\end{frame}

\begin{frame}{Гибридная логика и XML (2)}
XML документ - модель, формула гибридной логики - запрос.\\
\bigskip
\begin{itemize}
\item Существует ли в точности одна книга автора Franceschet?
	\begin{itemize}
		{\scriptsize \item $@_{root} \langle biblio \rangle \langle book \rangle \downarrow x \; . \; \langle author \rangle Franceschet \; \wedge \; @_{root} \langle biblio \rangle [book] x$}
	\end{itemize}
\bigskip	
\item Существует ли две разных книги?	
	\begin{itemize}
		{\scriptsize \item $@_{root} \langle biblio \rangle \langle book \rangle \downarrow x \; . \; @_{root} \langle biblio \rangle \langle book \rangle \neg x$}
	\end{itemize}
\end{itemize}
\end{frame}



\begin{frame}{Технологии}
\begin{itemize}
	\item OpenCCG (J. Baldridge et al.)
	\item Грамматика Moloko (DFKI)
	\item HLMC (L. Dragone)
\end{itemize}
\end{frame}

\begin{frame}{}
\begin{center}
	\textbf{Формализмы}\\
\end{center}
\end{frame}


\begin{frame}{dotCCG (1)}
\begin{itemize}
	\item DSL для создания MMCCG-грамматик
	\item Транслируется в XML-формат OpenCCG
	\item MOLOKO - около 4 kLOC 
	\item Мой прототип - около 0.5 kLOC (из-за морфологии)
\end{itemize}
\end{frame}



\begin{frame}[fragile]
\frametitle{Пример анализа (3)}
Первый вариант разбора (HLDS):\\
\bigskip
\begin{center}
{\scriptsize \begin{verbatim}
@w0:action(ON ^ 
              <Mood>imp ^ 
              <Actor>x1:entity ^ 
              <Patient>(w2:entity ^ и ^ 
                        <Num>pl ^ 
                        <First>(w1:thing ^ лампа ^ 
                                <Num>sg) ^ 
                        <Next>(w3:thing ^ подсветка ^ 
                               <Num>sg ^ 
                               <Modifier>(w4:m-location ^ на ^ 
                                          <Anchor>(w5:e-place ^ кухня ^ 
                                                   <Num>sg))) ^ 
                        <Num>pl))
\end{verbatim}
}                        
\end{center}
\end{frame}

\begin{frame}[fragile]
\frametitle{Пример анализа (4)}
Первый вариант разбора (HLMC):\\
\bigskip
\begin{center}
{\scriptsize \begin{verbatim}
>   ON (lampa & <num>(sg)) : x1_node x6_node 

>   ON (podsvetka & <num>(sg) & 
        <modifier>((na & <anchor>((kuhnya & <num>(sg)))))) : x3_node 
\end{verbatim}
}                        
\end{center}
\end{frame}

\begin{frame}{Пример анализа (5)}
Второй вариант разбора (MMCCG):\\
\bigskip
\begin{center}
\deriv{6}{
\gf{включи} & \gf{лампу} & \gf{и} & \gf{подсветку} & \gf{на} & \gf{кухне} \\
\uline{1} & \uline{1} & \uline{1} & \uline{1} & \uline{1} & \uline{1} \\
\cf{s\bs \supsa{-} np/ np} & \cf{np} & \cf{n/ np\bs \subsb{*} np} & \cf{np} & \cf{pp/ \subsa{\diamond} np} & \cf{np} \\
& \mc{2} {\hrulefill_{<}} \\
& \mc{2}{\cf{n/ np}} \\
&&&& \mc{2} {\hrulefill_{>}} \\
&&&& \mc{2}{\cf{pp}} \\
& \mc{3} {\hrulefill_{>}} \\
& \mc{3}{\cf{n}} \\
&&&& \mc{2} {\hrulefill_{t\mathbf{ypechange-6}}}\\
&&&& \mc{2}{\cf{n\bs \subsb{*} n}} \\
& \mc{5} {\hrulefill_{<}} \\
& \mc{5}{\cf{n}} \\
& \mc{5} {\hrulefill_{t\mathbf{ypechange-4}}}\\
& \mc{5}{\cf{np}} \\
 \mc{6} {\hrulefill_{>}} \\
 \mc{6}{\cf{s\bs \supsa{-} np}} \\
}
\end{center}
\end{frame}

\begin{frame}[fragile]
\frametitle{Пример анализа (6)}
Второй вариант разбора (HLDS):\\
\bigskip
\begin{center}
{\scriptsize \begin{verbatim}
@w0:action(ON ^ 
              <Mood>imp ^ 
              <Actor>x1:entity ^ 
              <Patient>(w2:entity ^ и ^ 
                        <Num>pl ^ 
                        <First>(w1:thing ^ лампа ^ 
                                <Num>sg) ^ 
                        <Modifier>(w4:m-location ^ на ^ 
                                   <Anchor>(w5:e-place ^ кухня ^ 
                                            <Num>sg)) ^ 
                        <Next>(w3:thing ^ подсветка ^ 
                               <Num>sg) ^ 
                        <Num>pl))
\end{verbatim}
}                        
\end{center}
\end{frame}

\begin{frame}[fragile]
\frametitle{Пример анализа (7)}
Второй вариант разбора (HLMC):\\
\bigskip
\begin{center}
{\scriptsize \begin{verbatim}
>   ON (lampa & <num>(sg) & 
        <modifier>((na & <anchor>((kuhnya & <num>(sg)))))) : x1_node 

>   ON (podsvetka & <num>(sg) & 
        <modifier>((na & <anchor>((kuhnya & <num>(sg)))))) : x3_node 
\end{verbatim}
}                        
\end{center}
\end{frame}


\begin{frame}{}
    \thispagestyle{empty}
    \begin{center}
        {\large Спасибо!}
    \end{center}
\end{frame}


%%% слайд помещается сюда
%% \begin{frame}{Заголовок}
%% \end{frame}

\end{document}
