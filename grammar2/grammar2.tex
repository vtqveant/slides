\documentclass{beamer}

\usepackage[T2A]{fontenc}
\usepackage[utf8]{inputenc}
\usepackage[english,russian]{babel}
\usepackage{amssymb,amsfonts,amsmath,mathtext}
\usepackage{cite,enumerate,float,indentfirst}

%% Gentzen style natural deduction proof trees
\usepackage{bussproofs}
\usepackage{latexsym}

\graphicspath{{images/}}

\usetheme{Pittsburgh}
\usecolortheme{whale}

\setbeamertemplate{footline}{\scriptsize{\hspace*{0.4cm}\insertframenumber}\vspace*{0.3cm}}
\beamertemplatenavigationsymbolsempty

\errorcontextlines 10000

\begin{document}
\title{\huge{$\mathbb{NLU}/RG$, \textit{pt. 6}}}
\author{Константин Соколов}
\institute[]
{Mathlingvo, СПбГУ, i-Free\\ \bigskip  \url{http://nlu-rg.ru}
}
\date{Санкт-Петербург, 2013} 
% Создание заглавной страницы
\begin{frame}
    \thispagestyle{empty}
    \titlepage
\end{frame}

%%% 0. План
\begin{frame}{План}
    \setcounter{framenumber}{1}
    \begin{itemize}
        \item Апресян, лексическая семантика
        \item Лямбда-исчисление
        \item А. Чёрч, простая типизация для лямбда-исчисления
        \item Спойлер! Compositional Distributional Semantic Models
    \end{itemize}
\end{frame}

\begin{frame}{Домашнее задание}
\begin{itemize}
  \item Приведите сигнатуры для языков, позволяющих описать
    \begin{itemize}
      \item группу
      \item абелеву группу
      \item свободную абелеву группу
      \item частично упорядоченное множество
      \item линейное пространство
    \end{itemize}
  \item Приведите примеры доменов для их интерпретации
\end{itemize}
\end{frame}

\begin{frame}{Плунгян, грамматические значения (1)}
\begin{itemize}
  \item Критерий обязательности.
  \item Грамматическая категория как ``ярлык'' для отношений между грамматическими значениями и показателями.
  \item Зависимость от языка и универсальный набор.
\end{itemize}
\end{frame}

\begin{frame}{Плунгян, грамматические значения (2)}
\begin{itemize}
  \item Понятие части речи как семантическое с опорой на грамматическую сочетаемость. Класс сочетаемости. 
  \item Акциональная классификация предикатов. Состояния и ситуации. Ситуации как процессы и события. Предельные и непредельные процессы.
  \item Семантические категории и словоформы. Акциональная характеристика предиката как множество всех таксономических категорий для всех форм данного предиката.
\end{itemize}
\end{frame}

\begin{frame}{Плунгян, грамматические значения (3)}
Замечания о семантике в рамках ``уровневой теории''.\\
\bigskip
\begin{itemize}
  \item Формальная семантика как семантика предложения.
  \item Принцип композициональности и естественные языки (``значение частей'', как далеко можно и нужно спускаться при анализе ``значений частей сложного выражения и способа их композиции'').
  \item Грамматикализация и формальная семантика.
\end{itemize}
\end{frame}

\begin{frame}{Плунгян, грамматические значения (4)}
Замечания о частях речи и синтаксических категориях в формальной семантике.\\
\bigskip
\begin{itemize}
  \item Определяя синтаксические категории, Монтегю следует Айдукевичу. 
  \item Объект как носитель набора свойств, прилагательные как предикаты, экстенсионал как характеристическая функция.
  \item Трансляция синтаксических категорий в семантические классы.
  \item Возможно ли однозначно приписать слову синтаксическую категорию?
  \item Возможно ли ввести синтаксические категории без привязки к конкретному языку?
\end{itemize}
\end{frame}


\begin{frame}{Айдукевич, ``О синтаксической связности'' (1)}
\begin{itemize}
  \item Выяснение условий синтаксической связности.
  \item Развитие концепции С. Лесьневского о ``категориях значения'' (термин ввел Э. Гуссерль).
  \item Критерий принадлежности к одной категории - способность взаимно заменять в контексте.
  \item ``Лестница категорий значения'' - ``грамматическо-\\семантический эквивалент'' упрощенной иерархии логических типов.
\end{itemize}
\end{frame}

\begin{frame}{Айдукевич, ``О синтаксической связности'' (2)}
\begin{itemize}
  \item Два вида категорий значения: подстановочные и функторные.
  \item Две основные (подстановочные) категории: категории предложений ($s$) и имен ($n$).
  \item Неограниченная иерархия функторных категорий: $s/n$, $s/nn$, $s/n/s/n$  и т.п.
\end{itemize}
\bigskip
\textit{Замечание:} категории значений как типы: $s$, $n$, $n \to s$, $n \to (n \to s)$, $(n \to s) \to (n \to s)$
\end{frame}

\begin{frame}{Айдукевич, ``О синтаксической связности'' (3)}
\includegraphics[scale=0.50]{ajdukiewicz2.png}\\
\bigskip
\includegraphics[scale=0.50]{ajdukiewicz1.png}\\
\bigskip
\textit{``sehr'' :} $((n \to s) \to (n \to s)) \to ((n \to s) \to (n \to s))$
\end{frame}

\begin{frame}{Айдукевич, ``О синтаксической связности'' (4)}
Проблема неоднозначности. Как определить ``однозначность'' и каковы её условия.\\
\bigskip
\begin{itemize}
  \item Предложение может быть правильно составленным и неоднозначным.
  \item Необходимо различать понятия ``правильно составленное выражение'' \textit{(ср.: п.п.ф.)} и ``синтаксически связное выражение''.
  \item Синтаксически связное выражение должно удовлетворять двум условиям:
    \begin{itemize}
      \item быть правильно составленным 
      \item индекс всего выражения должен быть $s$ и должен иметься алгоритм проверки (единственность главного функтора и пр.)
    \end{itemize}
\end{itemize}
\end{frame}

\begin{frame}{Айдукевич, ``О синтаксической связности'' (5)}
Проблемы при определении категорий значений для операторов.\\
\bigskip
\begin{itemize}
  \item Оператор - то, что связывает переменные ($\forall$, $\exists$, $\Sigma$, $\Pi$, $\int$)
  \item Анализ операторов. Смешивание функций. 
  \item Оператор, функция которого состоит только в связывании.
\end{itemize}
\end{frame}

\begin{frame}{Айдукевич, ``О синтаксической связности'' (6)}
Знак ``$\,\hat{\cdot}\,$'' (введен Расселом и Уайтхедом)\\
\bigskip
\begin{itemize}
  \item Если ``fx'' - символ неопределенного значения функции (т.е. $f(x) \in Im(f)$), то ``$f\hat{x}$'' - это сама функция (т.е. $f$) 
  \item Выражение ``$(\hat{x}).fx$'' имеет денотатом то же, что ``f'', т.е. то же, что ``$f\hat{x}$''
  \item Выражения ``$(\hat{x}\hat{y}).fxy$'' и ``$f\hat{x}\hat{y}$'' эквавалентны \\(ср. $\lambda x.\lambda y.fxy$)
\end{itemize}
\bigskip
\textit{Замечание:} ср. в PTQ: ``The expression $[\string^\alpha]$ is regarded as denoting (or having as its \textit{extension}) the \textit{intension} of the expression $\alpha$.'') и понимание эксенсионала как характеристической функции.
\end{frame}

\begin{frame}{Айдукевич, ``О синтаксической связности'' (7)}
Анализ квантора всеобщности\\
\bigskip
\begin{itemize}
  \item Две функции: связывание и квантификация
  \item Попытка ``распутать'' квантор $\forall$, введя ``универсальный функтор'' $U$ с индексом $s/s/n$ (или $(n \to s) \to s)$), чтобы $U(f) \approx \forall x.fx$, т.е. можно писать $U((\hat{x}).fx)$.
  \item Т.о. роль квантора всеобщности удалось бы заменить комбинацией ролей универсального функтора и оператора ``$\hat{x}$''.
  \item Очевидно, существует много универсальных функторов с различными категориями значения в зависимости от категории значения функтора, служащего для них аргументом.
\end{itemize}
\end{frame}


\begin{frame}{}
    \thispagestyle{empty}
    \begin{center}
        {\large Спасибо!}
    \end{center}
\end{frame}


%%% слайд помещается сюда
%% \begin{frame}{Заголовок}
%% \end{frame}

\end{document}
