\documentclass{beamer}

\usepackage[T2A]{fontenc}
\usepackage[utf8]{inputenc}
\usepackage[english,russian]{babel}
\usepackage{amssymb,amsfonts,amsmath,mathtext}
\usepackage{cite,enumerate,float,indentfirst}

%% Gentzen style natural deduction proof trees
\usepackage{bussproofs}
\usepackage{latexsym}

\graphicspath{{images/}}

\usetheme{Pittsburgh}
\usecolortheme{whale}

\setbeamertemplate{footline}{\scriptsize{\hspace*{0.4cm}\insertframenumber}\vspace*{0.3cm}}
\beamertemplatenavigationsymbolsempty

\errorcontextlines 10000

\begin{document}
\title{\huge{$\mathbb{NLU}/RG$, \textit{pt. 6}}}
\author{Константин Соколов}
\institute[]
{Mathlingvo, СПбГУ, i-Free\\ \bigskip  \url{http://nlu-rg.ru}
}
\date{Санкт-Петербург, 2013} 
% Создание заглавной страницы
\begin{frame}
    \thispagestyle{empty}
    \titlepage
\end{frame}

%%% 0. План
\begin{frame}{План}
    \setcounter{framenumber}{1}
    \begin{itemize}
        \item Апресян, лексическая семантика
        \item $\lambda$-исчисление
        \item Church. A Formulation of the Simple Theory of Types. 1940.
    \end{itemize}
\end{frame}

\begin{frame}{Апресян, лексическая семантика (1)}
\begin{itemize}
  \item Значение слова и принцип композициональности
  \item Лексическая семантика и семантический метаязык
\end{itemize}
\end{frame}

\begin{frame}{Апресян, лексическая семантика (2)}
Лексикографическая традиция:\\
\bigskip
\begin{itemize}
  \item наивное понятие о значении слова как о ``сущности'', подлежащей выяснению \textit{(толковый словарь)} 
  \item раскрытие этой сущности в толковании \textit{(словарная статья)}
  \item ``синтаксический аспект'' значения - слова в языке соединяются не вполне свободно \textit{(словосочетания)}
  \item значение словосочетаний - не простая сумма, а ``сложный продукт'' \textit{(фразеология, идиомы)}
\end{itemize}
\end{frame}

\begin{frame}{Апресян, лексическая семантика (3)}
Структуралистская традиция:\\
\bigskip
\begin{itemize}
  \item компонентная структура лексических значений\\ {\footnotesize (\texttt{мужчина = человек + самец + взрослый})}
  \item признаки
    \begin{itemize}
      \item дифференциальные: {\footnotesize \texttt{cын vs. дочь}}
      \item интегральные: {\footnotesize \texttt{дети vs. племянники}}
      \item ассоциативные: {\footnotesize \texttt{дядя vs. дяденька \textit{(обращение)}}}
    \end{itemize}
  \item привативные и эквиполентные оппозиции
  \item иерархическая организация семантических компонентов значения
\end{itemize}
\end{frame}

\begin{frame}{Апресян, лексическая семантика (4)}
Философская традиция:\\
\begin{itemize}
  \item выяснение наиболее общих признаков/категорий (в аристотелевском смысле)
\end{itemize}
\end{frame}

\begin{frame}{Апресян, лексическая семантика (5)}
Логическая традиция:\\
\bigskip
\begin{itemize}
  \item имена vs. предикаты
  \item предикаты высшего порядка
  \item местность предикатов (ср. валентность глаголов, глубинные падежи)
  \item анализ модальных (эпистемических, темпоральных и пр.) аспектов
\end{itemize}
\end{frame}

\begin{frame}{Апресян, лексическая семантика (6)}
Почему необходим и что должен позволять метаязык?\\
\bigskip
\begin{itemize}
  \item строить правильные предложения по заданным значениям
  \item извлекать значения из заданных предложений
  \item осуществлять семантически инвариантные трансформации (перефразирование) предложений
  \item оценивать предложение с т.з. семантической связности
\end{itemize}
\end{frame}

\begin{frame}{Чёрч, лямбда-исчисление}
\begin{center}
\texttt{This page intentionally left blank}
\end{center}
\end{frame}

\begin{frame}{Повторение}
\textit{Формальная система} или \textit{исчисление} задается\\
\begin{itemize}
  \item набором базовых символов и способом их комбинации
  \item набором аксиом
  \item правилами вывода
  \item определением понятия выводимости
\end{itemize}
\bigskip

\end{frame}

\begin{frame}{[Church 1940], $\lambda^\to$ (1)}
Нотация:\\
\bigskip
\begin{itemize}
  \item Базовые типы $\alpha, \beta$, функциональные типы $(\alpha\beta) = \beta \to \alpha$
  \item $\lambda x_\alpha \; . \; A_{\beta\alpha}$
  \item Константы $true$ и $false$ имеют тип $o$
  \item Функции с типом $(o\alpha) = \alpha \to o$
  \item Каррирование: $(o\alpha o\beta) = \beta \to (o \to (\alpha \to o))$
  \item Универсальный квантор: $\Pi_{o(o\alpha)}[\lambda x_\alpha A_o] \equiv \forall x_\alpha A_o$
  \item $[MN]$ - применение функции $M$ к аргументу $N$
\end{itemize}
\end{frame}

\begin{frame}{[Church 1940], $\lambda^\to$ (2)}
Язык:\\
\bigskip
\begin{itemize}
  \item Переменная или константа типа $\alpha$ - п.п.ф. типа $\alpha$ 
  \item Если $A_{\alpha\beta}$ и $B_{\beta}$ - п.п.ф. соответствующих типов, то $[A_{\alpha\beta}B_\beta]$ - п.п.ф. типа $\beta$.
  \item Если $x_\beta$ - переменная типа $\beta$ и $A_\alpha$ - п.п.ф., то $[\lambda x_\beta A_\alpha]$ - п.п.ф. типа $(\alpha \beta)$.
\end{itemize}
\bigskip
Замечания:\\
\begin{itemize}
  \item $\sim_{(oo)}$ - п.п.ф. типа $(oo)$, $[\sim_{(oo)}A_o]$ - п.п.ф. типа $o$.
  \item $Q_{o \alpha \alpha} = [\lambda x_\alpha \lambda y_\alpha \forall f_{o \alpha}[f_{o \alpha} x_\alpha \supset f_{o \alpha} y_\alpha]]$ - если $y$ обладает всеми свойствами $x$, то $y = x$.
\end{itemize}
\end{frame}

\begin{frame}{[Church 1940], $\lambda^\to$ (3)}
Правила вывода:\\
\bigskip
\begin{itemize}
  \item $\alpha$-конверсия: переименование связанных переменных
  \item $\beta$-редукция: замена $[[\lambda x_\beta M_\alpha] N_\beta]$ на $M_\alpha[x_\beta/N_\beta]$
  \item $\beta$-экспансия: $D \vdash C$, если $D$ получается из $C$ однократным применением $\beta$-редукции
  \item подстановка: $F_{(o \alpha)} x_\alpha \vdash F_{(o \alpha)} A_\alpha$
  \item modus ponens: $[A_o \supset B_o], A_o \vdash B_o$ 
  \item обобщение: $F_{(o \alpha)} x_\alpha \vdash \Pi_{o (o \alpha)} F_{(o \alpha)}$
\end{itemize}
\end{frame}

\begin{frame}{[Church 1940], $\lambda^\to$ (4)}
Аксиомы исчисления выказываний:\\
\begin{itemize}
  \item $p \vee p \supset p$
  \item $p \supset p \vee p$
  \item $p \vee q \supset q \vee p$
  \item $(p \supset q) \supset (r \vee p \supset r \vee q)$
\end{itemize}
\bigskip
Аксиомы логического функционального исчисления:\\
\begin{itemize}
  \item $\Pi_{o (o \alpha)} f_{o \alpha} \supset f_{o \alpha} x_\alpha$
  \item $\forall x_\alpha [p_o \vee f_{o \alpha}] \supset [p_o \vee \Pi_{o(o \alpha)} f_{o \alpha}]$
\end{itemize}
\end{frame}

\begin{frame}{[Church 1940], $\lambda^\to$ (5)}
Аксиома экстенсиональности:\\
\bigskip
\begin{itemize}
  \item $\forall x_\beta [f_{\alpha \beta} x_\beta = g_{\alpha \beta} x_\beta] \supset f_{\alpha \beta} = g_{\alpha \beta}$
\end{itemize}
\bigskip
Аксиома дескрипции:\\
\begin{itemize}
  \item $\exists ! x_\alpha [p_{o \alpha} x_\alpha] \supset p_{o \alpha} [\iota_{\alpha(o \alpha)}p_{o \alpha}]$, где $\exists ! x_\alpha A_o =_{def} [\lambda p_{o \alpha} \exists y_\alpha[p_{o \alpha} y_\alpha \wedge \forall z_\alpha[p_{o \alpha} z_\alpha \supset z_\alpha = y_\alpha]]][\lambda x_\alpha A_o]$
\end{itemize}
\bigskip
Неформально, $A_o$ описывает $x_\alpha$; \textit{оператор дескрипции} $\iota_{\alpha (o \alpha)}$ сопоставляет одноэлементному множеству его (единственный) элемент.
\end{frame}

\begin{frame}{[Church 1940], $\lambda^\to$ (6)}
Аксиома выбора:\\
\bigskip
\begin{itemize}
  \item $f_{o \alpha} x_\alpha \supset f_{o \alpha} [\iota_{\alpha (o \alpha)} f_{o \alpha}].$
\end{itemize}
\bigskip
Неформально, \textit{оператор выбора} $\iota_{\alpha (o \alpha)}$ сопоставляет непустому множеству какой-то его элемент.\\
\bigskip
Аксиома выбора влечет аксиому дескрипции.
\end{frame}

\begin{frame}{[Church 1940], $\lambda^\to$ (7)}
Формальный вывод:\\
\bigskip
\begin{itemize}
  \item \textit{Доказательство} формулы $B_o$ в предположении формул $A_o^1, A_o^2, ..., A_o^n$ - это конечная последовательность формул, последняя из которых - $B_o$, а остальные - либо одна из формул $A_o^1, A_o^2, ..., A_o^n$, вариант схемы аксиом, либо формула, полученная из предыдущих формул последовательности применением правил вывода.
  \item Если такое доказательство существует, будем писать $A_o^1, A_o^2, ..., A_o^n \vdash B_o$
  \item \textit{Теорема о дедукции:} если $A_o^1, A_o^2, ..., A_o^n \vdash B_o$, то $A_o^1, A_o^2, ..., A_o^{n-1} \vdash A_o^n \supset B_o$
\end{itemize}
\end{frame}

\begin{frame}{Разное}
\begin{itemize}
  \item Семантику для $\lambda^\to$ построил Генкин в 1950 г.
  \item Категорную семантику для $\lambda^\to$ построил Ламбек в 1980 г.
  \item Монтегю использовал многое из [Church 1940] \textit{(обобщенные кванторы, оператор дескрипции, $\lambda$-оператор, лейбницевское (экстенсиональное) равенство предикатов, теорию типов)}, но строил семантику по-другому.
\end{itemize}
\end{frame}

\begin{frame}{}
    \thispagestyle{empty}
    \begin{center}
        {\large Спасибо!}
    \end{center}
\end{frame}


%%% слайд помещается сюда
%% \begin{frame}{Заголовок}
%% \end{frame}

\end{document}
