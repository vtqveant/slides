\documentclass{beamer}

\usepackage[T2A]{fontenc}
\usepackage[utf8]{inputenc}
\usepackage[english,russian]{babel}
\usepackage{amssymb,amsfonts,amsmath,mathtext}
\usepackage{cite,enumerate,float,indentfirst}

\graphicspath{{images/}}

\usetheme{Pittsburgh}
\usecolortheme{whale}

\setbeamertemplate{footline}{\scriptsize{\hspace*{0.4cm}\insertframenumber}\vspace*{0.3cm}}
\beamertemplatenavigationsymbolsempty

\begin{document}
\title{\huge{$\mathbb{NLU}/RG$}}
\author{Константин Соколов}
\institute{Mathlingvo, СПбГУ, i-Free}
\date{Санкт-Петербург, 2013} 
% Создание заглавной страницы
\begin{frame}
\thispagestyle{empty}
\titlepage
\end{frame}


%%% 0. Пара слов про себя: Mathlingvo, i-Free Innovations, AINL 2012-2013, СПбГУ
\begin{frame}{Пара слов о себе}
\setcounter{framenumber}{1}
\begin{itemize}
  \item Mathlingvo 
  \item i-Free Innovations
  \item AINL 2012-2013
  \item СПбГУ
  \item @vtqveant
\end{itemize}
\end{frame}

%%% 1. Что: откуда у семинара ноги растут - R&D в i-Free Innovations
\begin{frame}{Откуда $\mathbb{NLU}/RG$}
\begin{itemize}
  \item R\&D в i-Free Innovations
\end{itemize}
\end{frame}

%%% 2. Зачем: максимально быстро выйти на актуальные проблемы. Больше вширь, чем вглубь. Но одному не разобраться.
\begin{frame}{Зачем $\mathbb{NLU}/RG$}
\begin{itemize}
  \item максимально быстро выйти на актуальные проблемы
  \item больше вширь, чем вглубь
  \item одному не разобраться
\end{itemize}
\end{frame}

%%% 3. Формат: Reading Group - решение проблемы. Чтобы было у кого спросить, если чего сам не понимаешь; и рассказать, если знаешь.
\begin{frame}{Формат $\mathbb{NLU}/RG$}
\begin{itemize}
  \item Reading Group
  \item чтобы было у кого спросить, если чего сам не понимаешь
  \item ...и рассказать, если знаешь
\end{itemize}
\end{frame}

%%% 4. Что такое "понимание естественного языка". Курс в Стенфорде. Несколько учебников. Что-то вроде NLP. Что-то про семантику. Гораздо менее понятно, что вообще значит "понимание" в инженерном контексте.
\begin{frame}[fragile]
\frametitle{Что такое $\mathbb{NLU}$}
\begin{itemize}
  \item курс в Стенфорде ({\scriptsize CS 224U, http://www.stanford.edu/class/cs224u})     
  \item несколько учебников ({\scriptsize [Allen, Natural Language Understanding, 1994], [Blackburn, Bos]})
  \item что-то вроде NLP
  \item что-то про семантику 
  \item гораздо менее понятно, что вообще значит ''понимание'' в инженерном контексте
\end{itemize}
\end{frame}

%%% 5-1. Что такое "актуальный уровень". Не (обязательно) недавнее. Не (обязательно) модное. Не (обязательно) революционное. Даже не (обязательно) успешное. Пример из прошлого: неевклидова геометрия (Гаусс, Больяи, Лобачевский). "Идеи носятся в воздухе". Актуальное - это когда "время пришло".
\begin{frame}[fragile]
\frametitle{Что такое \textit{актуальное} в $\mathbb{NLU}$}
\begin{itemize}
  \item не (обязательно) недавнее
  \item не (обязательно) модное
  \item не (обязательно) революционное
  \item даже не (обязательно) успешное
\end{itemize}
\end{frame}

%%% 5-2. Что такое "актуальный уровень". Не (обязательно) недавнее. Не (обязательно) модное. Не (обязательно) революционное. Даже не (обязательно) успешное. Пример из прошлого: неевклидова геометрия (Гаусс, Больяи, Лобачевский). "Идеи носятся в воздухе". Актуальное - это когда "время пришло".
\begin{frame}[fragile]
\frametitle{Что такое \textit{актуальное} в $\mathbb{NLU}$}
\begin{itemize}
  \item пример из прошлого: неевклидова геометрия (''идеи носятся в воздухе'')
  \item актуальное - это когда ''время пришло''
\end{itemize}
\end{frame}

%%% 6. Прорывы последнего времени. Representation Learning, Manifold Learning, Deep Learning, etc.
\begin{frame}[fragile]
\frametitle{State of the Art в NLP}
Прорывы последнего времени:\bigskip
\begin{itemize}
  \item Representation Learning
  \item Manifold Learning
  \item Deep Learning
  \item \&c.
\end{itemize}
\end{frame}

%%% 7. Почему это модно? Хорошо работают; можно считать на компьютерах; big data; бизнес видит ценность(?). Но главное: эти подходы отвечают (или обещают ответить) на проблему, которая ''носится в воздухе''.
\begin{frame}{State of the Art в NLP}
Почему это модно?\bigskip
\begin{itemize}
  \item хорошо работает
  \item можно считать на компьютерах
  \item Big Data
  \item бизнес видит ценность (?) 
  \item отвечают (или обещают ответить) на проблему, которая ''носится в воздухе'' 
\end{itemize}
\end{frame}

%%% 8. Актуальная проблема. Norvig on Chomsky (http://norvig.com/chomsky.html) Статистические методы vs. олдскульная лингвистика? Возможен ли синтез. Имеет ли он смысл?
\begin{frame}[fragile]
\frametitle{Актуальные проблемы в NLP}
\begin{itemize}
  \item Norvig on Chomsky (http://norvig.com/chomsky.html) 
  \item статистические методы vs. ''олдскульная'' лингвистика 
  \item Возможен ли синтез? Имеет ли он смысл?
\end{itemize}
\end{frame}

%%% 9. Программа. Принципы при составлении программы: актуальное (не модное, не недавнее, не революционное); двигаться вширь; просить помощи; искать приложения в конкретных задачах (в научных и промышленных). Изначально рассчитывалась на год при еженедельных встречах.
\begin{frame}[fragile]
\frametitle{Программа $\mathbb{NLU}/RG$}
Принципы при составлении программы:\bigskip 
\begin{itemize}
  \item актуальное (не модное, не недавнее, не революционное)
  \item двигаться вширь
  \item просить помощи
  \item искать приложения в конкретных задачах (в научных и промышленных)
\end{itemize}
\bigskip  
\textit{Изначально рассчитывалась на год при еженедельных встречах.}
\end{frame}

%%% 10. Обзор программы
\begin{frame}{}
\thispagestyle{empty}
\begin{center}
{\large Обзор программы}
\end{center}
\end{frame}

%%% 11. В целом. Основные линии: лингвистическая, логическая, инженерная, актуальные проблемы, матметоды.
\begin{frame}[fragile]
\frametitle{Обзор программы}
Основные линии:\bigskip
\begin{itemize}
  \item лингвистическая
  \item логическая
  \item инженерная
  \item актуальные проблемы
  \item матметоды
\end{itemize}
\end{frame}

%%% 12. Лингвистическая. Предоставить краткое введение в формальную семантику; закрыть пробелы; ввести основную терминологию; понять проблематику; познакомиться с историческим контекстом; узнать несколько имен. 
\begin{frame}[fragile]
\frametitle{Обзор программы}
Лингвистическая линия:\bigskip
\begin{itemize}
  \item предоставить краткое введение в формальную семантику
  \item закрыть пробелы
  \item ввести основную терминологию
  \item понять проблематику
  \item познакомиться с историческим контекстом
  \item узнать несколько имен
\end{itemize}
\end{frame}

%%% 13. Логическая. узнать об основных подходах к моделированию семантики естественных языков; сформировать четкое представление о сущности семантики через обращение к теории моделей. Последовательно рассмотреть ряд конструкций, предлагавшихся для моделирования семантики;
\begin{frame}{Обзор программы}
Логическая линия:\bigskip
\begin{itemize}
  \item узнать об основных подходах к моделированию семантики естественных языков
  \item сформировать четкое представление о сущности семантики через обращение к теории моделей
  \item последовательно рассмотреть ряд конструкций, предлагавшихся для моделирования семантики
\end{itemize}
\end{frame}

%%% 14. Инженерная. Рассмотреть ряд практических методов и реализаций из области (логического) программирования. Приложения (формальной) семантики к програмированию, обработке и представлению данных. Приложение программирования к обработке естественного языка.
\begin{frame}[fragile]
Инженерная линия:\bigskip
\frametitle{Обзор программы}
\begin{itemize}
  \item рассмотреть ряд практических методов и реализаций из области (логического) программирования
  \item приложение (формальной) семантики к програмированию, обработке и представлению данных
  \item приложение (логического) программирования к обработке естественного языка
\end{itemize}
\end{frame}

%%% 15. Актуальная проблема: поиск порядка в хаосе. Познакомиться с идеей отыскания представления данных в структурном (символическом, алгебраическом) виде с помощью машинного обучения.
\begin{frame}{Обзор программы}
Актуальные проблемы:\bigskip 
\begin{itemize}
  \item познакомиться с идеей нахождения представления данных в структурном (символьном, алгебраическом) виде с помощью машинного обучения.
\end{itemize}
\end{frame}

%%% 16. Матметоды. По сути, выбрана только одна линия: от понятия множества со структурой к теории топосов и гомотопической теории типов. В направлении, где смыкаются логика и геометрия, где встречаются порядок структур и аморфность данных, где встречаются высокая сложность и высокая абстрактность.
\begin{frame}[fragile]
\frametitle{Обзор программы}
Матметоды: от понятия множества со структурой к теории топосов и HoTT.\bigskip
\begin{itemize}
  \item где смыкаются логика и геометрия
  \item где встречаются порядок структур и аморфность данных
  \item где одновременно присутствуют высокая сложность и высокая абстрактность \textit{(В. Воеводский)}
\end{itemize}
\end{frame}

%%% 17. Административное. Порядок встреч, регламент; гугл-группа, сайт
\begin{frame}{Административное}
\begin{itemize}
  \item порядок встреч: раз в две недели, СПбГПУ
  \item регламент: 1,5 часа, обоснование темы, доклады, обсуждение
  \item гугл-группа: \textit{nlu-rg@googlegroups.com}
  \item сайт: \textit{http://nlu-rg.ru}
  \item wiki, репозитории
\end{itemize}
\end{frame}

%%% 18. Слово аудитории. Кто откуда. Кто с чем из указанного знаком (кто с чем готов помочь другим разобраться). Кто готов программировать и чего хотел бы.
\begin{frame}{Слово аудитории}
\begin{itemize}
  \item кто откуда
  \item кто с чем из указанного знаком 
  \item кто с чем готов помочь другим разобраться
  \item кто готов программировать и чего хотел бы
\end{itemize}
\end{frame}

\begin{frame}{}
\thispagestyle{empty}
\begin{center}
{\large Спасибо!}
\end{center}
\end{frame}


%%% слайд помещается сюда
%% \begin{frame}{Заголовок}
%% \end{frame}

\end{document}
