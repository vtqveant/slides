\documentclass{beamer}

\usepackage[T2A]{fontenc}
\usepackage[utf8]{inputenc}
\usepackage[english,russian]{babel}
\usepackage{amssymb,amsfonts,amsmath,mathtext}
\usepackage{cite,enumerate,float,indentfirst}

\graphicspath{{images/}}

\usetheme{Pittsburgh}
\usecolortheme{whale}

\setbeamertemplate{footline}{\scriptsize{\hspace*{0.4cm}\insertframenumber}\vspace*{0.3cm}}
\beamertemplatenavigationsymbolsempty

\begin{document}
\title{\huge{$\mathbb{NLU}/RG$}}
\author{Константин Соколов}
\institute{Mathlingvo, СПбГУ, i-Free}
\date{Санкт-Петербург, 2013} 
% Создание заглавной страницы
\begin{frame}
\thispagestyle{empty}
\titlepage
\end{frame}

%%% 0. Пара слов про себя: Mathlingvo, i-Free Innovations, AINL 2012-2013, СПбГУ
%%% 1. Что: откуда у семинара ноги растут - R&D в i-Free Innovations
%%% 2. Зачем: максимально быстро выйти на актуальные проблемы. Больше вширь, чем вглубь. Но одному не разобраться.
%%% 3. Формат: Reading Group - решение проблемы. Чтобы было у кого спросить, если чего сам не понимаешь; и рассказать, если знаешь.
%%% 4. Что такое "понимание естественного языка". Курс в Стенфорде. Несколько учебников. Что-то вроде NLP. Что-то про семантику. Гораздо менее понятно, что вообще значит "понимание" в инженерном контексте.
%%% 5. Что такое "актуальный уровень". Не (обязательно) недавнее. Не (обязательно) модное. Не (обязательно) революционное. Даже не (обязательно) успешное. Пример из прошлого: неевклидова геометрия (Гаусс, Больяи, Лобачевский). "Идеи носятся в воздухе". Актуальное - это когда "время пришло".
%%% 6. Прорывы последнего времени. Representation Learning, Manifold Learning, Deep Learning, etc.
%%% 7. Почему это модно? Хорошо работают; можно считать на компьютерах; big data; бизнес видит ценность(?). Но главное: эти подходы отвечают (или обещают ответить) на проблему, которая ''носится в воздухе''.
%%% 8. Актуальная проблема. Norvig on Chomsky (http://norvig.com/chomsky.html) Статистические методы vs. олдскульная лингвистика? Возможен ли синтез. Имеет ли он смысл?
%%% 9. Программа. Принципы при составлении программы: актуальное (не модное, не недавнее, не революционное); двигаться вширь; просить помощи; искать приложения в конкретных задачах (в научных и промышленных). Изначально рассчитывалась на год при еженедельных встречах.
%%% 10. Обзор программы
%%% 11. В целом. Основные линии: лингвистическая, логическая, инженерная, актуальные проблемы, матметоды.
%%% 12. Лингвистическая. Предоставить краткое введение в формальную семантику; закрыть пробелы; ввести основную терминологию; понять проблематику; познакомиться с историческим контекстом; узнать несколько имен. 
%%% 13. Логическая. узнать об основных подходах к моделированию семантики естественных языков; сформировать четкое представление о сущности семантики через обращение к теории моделей. Последовательно рассмотреть ряд конструкций, предлагавшихся для моделирования семантики; 
%%% 14. Инженерная. Рассмотреть ряд практических методов и реализаций из области (логического) программирования. Приложения (формальной) семантики к програмированию, обработке и представлению данных. Приложение программирования к обработке естественного языка.
%%% 15. Актуальная проблема: поиск порядка в хаосе. Познакомиться с идеей отыскания представления данных в структурном (символическом, алгебраическом) виде с помощью машинного обучения. 
%%% 16. Матметоды. По сути, выбрана только одна линия: от понятия множества со структурой к теории топосов и гомотопической теории типов. В направлении, где смыкаются логика и геометрия, где встречаются порядок структур и аморфность данных, где встречаются высокая сложность и высокая абстрактность.
%%% 17. Административное. Порядок встреч, регламент; гугл-группа, сайт
%%% 18. Слово аудитории. Кто откуда. Кто с чем из указанного знаком (кто с чем готов помочь другим разобраться). Кто готов программировать и чего хотел бы.

\begin{frame}{Теория моделей как способ изучать семантику}
\setcounter{framenumber}{1}
\begin{itemize}
  \item Теория моделей сопоставляет формальному языку (или формализованному фрагменту естественного языка) математический объект
    \begin{itemize}
      \item в языке колец $L_{ring} = \{0, 1, +, \cdot, -\}$ терм -- это многочлен $P(x_1, ..., x_n)$ в $\mathbb{Z}[x_1, ..., x_n]$ при некотором $n$.
    \end{itemize}
  \item Изучая язык, лучше понимаем объект (и наоборот)
    \begin{itemize}
      \item $L_{ring}$-структура $A$ -- коммутативное кольцо если и только если в $A$ выполнено $\forall x$ $\forall y$ . $x + y = y + x$
    \end{itemize}
\end{itemize}
\end{frame}

\begin{frame}{Интерпретации и модели (1)}
\begin{itemize}
  \item \textbf{Сигнатура} - набор предикатных и функциональных символов
  \item \textbf{Замкнутые формулы} - формулы без параметров
  \item \textbf{Теория} - множество замкнутых формул данной сигнатуры
  \item \textbf{Интерпретация} - способ приписать истинностное значение замкнутым формулам данной сигнатуры
  \item \textbf{Модель} - интерпретация, в которой все формулы теории истинны
  \item \textbf{Совместная теория} - теория, имеющая модель
\end{itemize}
\end{frame}

\begin{frame}{Интерпретации и модели (2)}
\begin{itemize}
  \item \textbf{синтаксис:} формула $\phi$ \textit{выводима} в теории $T$ ($T \vdash \phi$), если ее можно получить из формул теории $T$ по правилам вывода
  \item \textbf{семантика:} $\phi$ \textit{семантически следует} из $T$ ($T \models \phi$), если она истинна в любой модели $T$
  \item $T \vdash \phi \Leftrightarrow T \models \phi$
\end{itemize}
\end{frame}

\begin{frame}{ASP-программа как спецификация задачи}
\begin{itemize}
  \item данные -- атомы
  \item отношения и ограничения -- предикаты
  \item переменные -- <<классы>> атомов и предикатов
\end{itemize}\bigskip
Результат выполнения программы -- <<возможные миры>> (один или несколько), в которых описанная в программе ситуация непротиворечива, или несовместность.
\end{frame}

\begin{frame}{AnsProlog}
AnsProlog - язык, поддерживаемый lparse | smodels.\bigskip
\begin{itemize}
  \item константы: \texttt{a. b. c.}
  \item переменные: \texttt{X, Lengthy}
  \item функциональные символы: \texttt{f, g}
  \item предикаты: \texttt{p, q}
  \item термы: переменные, константы и $f(t_1, ..., t_n)$, где $t_i$ -- терм
  \item атомы: $p(t_1, ..., t_n)$, где $t_i$ -- терм, напр. \texttt{color(red).}
  \item $\neg$ и $not$
  \item хорновские дизъюнкты:
    \begin{itemize}
      \item \texttt{a :- b, c, not d.}
      \item \texttt{:- predicate(X).    \% ограничение}
    \end{itemize}
\end{itemize}\bigskip
ASP-программа -- это множество хорновских дизъюнктов (и немного синтаксического сахара)
\end{frame}


\begin{frame}[fragile]
\frametitle{Универсум Эрбрана}
\textit{Универсум Эрбрана} $HU(S)$ множества формул $S$ содержит все \textbf{термы}, образованные из констант и функциональных символов в $S$.\\
\textit{Базис Эрбрана} $HB(S)$ содержит все \textbf{атомы}, образованные из констант, функциональных символов и предикатов в $S$.\\
\bigskip

{\footnotesize \begin{verbatim}
    X, Y - переменные
    a, b - константы
    f    - одноместный функциональный символ
    p    - одноместный предикат
    
    HU = {  a,    b,    f(a),    f(b),    f(f(a)),    f(f(b)),  ...}
    HB = {p(a), p(b), p(f(a)), p(f(b)), p(f(f(a))), p(f(f(b))), ...} 
\end{verbatim}}
\end{frame}

\begin{frame}[fragile]
\frametitle{Модель Эрбрана}
\textit{Эрбрановская интерпретация} - интерпретация на \textit{универсуме Эрбрана}\\
\textit{Модель Эрбрана} - это такая структура на \textit{универсуме Эрбрана}...\bigskip

Иными словами, 
{\footnotesize \begin{verbatim}
    natural(zero).
    natural(s(N)) :- natural(N).
\end{verbatim}}
интерпретируется не в $\mathbb{N}$, а в {\footnotesize \texttt{\{zero, s(zero), s(s(zero)), ...\}}}\\
\bigskip
Модель:\\ {\footnotesize \texttt{\{natural(zero), natural(s(zero)), natural(s(s(zero))), ...\}}}
\end{frame}


\begin{frame}{Как это работает}
$ground(r, L)$ - множество правил, полученных из $r$ заменой переменных в $r$ элементами из $HU(L)$ всеми возможными способами.

\begin{itemize}
  \item 1-й шаг: grounder (<<уничтожение>> переменных)
  \item 2-й шаг: solver (поиск моделей)
\end{itemize}\bigskip
\end{frame}


\begin{frame}[fragile]
\frametitle{Пример программы: раскраска графа (1)}
{\footnotesize \begin{verbatim}
$ cat graph.lp
color(red; blue; yellow).
node(a; b; c; d).
edge(a, b).  edge(b, c).  edge(c, d).  edge(d, a).

col(X, red) :- node(X), not col(X, blue), not col(X, yellow).
col(X, blue) :- node(X), not col(X, red), not col(X, yellow).
col(X, yellow) :- node(X), not col(X, blue), not col(X, red).

:- edge(X, Y), color(C), col(X, C), col(Y, C).

#hide.
#show col/2.
\end{verbatim}}
\end{frame}

\begin{frame}[fragile]
\frametitle{Пример программы: раскраска графа (2)}
{\footnotesize \begin{verbatim}
$ gringo graph.lp | clasp -n 0
clasp version 2.0.4
Reading from stdin
Solving...
Answer: 1
col(d,red) col(c,yellow) col(b,blue) col(a,yellow)
Answer: 2
col(d,red) col(c,blue) col(b,yellow) col(a,blue)
...
Models      : 18
\end{verbatim}}
\end{frame}


\begin{frame}[fragile]
\frametitle{Positive logic program}
\begin{itemize}
  \item множество хорновских дизъюнктов без отрицания
  \item минимальная модель (least model) существует и единственна
\end{itemize}\bigskip

Пример: 
{\footnotesize \begin{verbatim} 
    a. 
    b :- a.
\end{verbatim}}

Модель: \texttt{\{a, b\}}
\end{frame}

\begin{frame}[fragile]
\frametitle{Stratified logic program}
\begin{itemize}
  \item множество хорновских дизъюнктов с отрицанием, допускающее стратификацию
  \item отрицание -- это <<negation as failure>>
  \item стратификация -- это разбиение мн-ва дизъюнктов на подмножества, такие что...
  \item стратифицированная программа имеет уникальную минимальную модель (perfect model)
\end{itemize}\bigskip

Пример:
{\footnotesize \begin{verbatim} 
    a.
    b :- not a.
\end{verbatim}}

Модель: \texttt{\{a\}}
\end{frame}

\begin{frame}[fragile]
\frametitle{Normal logic program}
\begin{itemize}
  \item множество хорновских дизъюнктов с отрицанием
  \item нестратифицированная программа имеет несколько моделей (stable model)
  \item stable model -- обобщение perfect model. Eсли программа стратифицируема, то ее стабильная модель единственна и совпадает с perfect model
\end{itemize}\bigskip

Пример: 
{\footnotesize \begin{verbatim} 
    a :- not b. 
    b :- not a.
\end{verbatim}}

Модели две: \texttt{\{a\}}, \texttt{\{b\}}
\end{frame}

\begin{frame}{ASP и немонотонная логика (1)}
\begin{itemize}
  \item свойство монотонности: если $\Gamma \vdash A$, то $\Gamma, B \vdash A$
  \item немонотонный вывод (неформально): добавление новых знаний может потребовать пересмотра картины мира (<<логика здравого смысла>>)
  \item семантика стабильной модели -- упрощенный вариант автоэпистемической логики, т.е. логики с модальным оператором $\Box$, трактуемым как <<известно, что>>
  \item $\neg \Box F \rightarrow \neg F$ \textit{(Negation as Failure)}
\end{itemize}
\end{frame}

\begin{frame}[fragile]
\frametitle{ASP и немонотонная логика (2)}
Пример \textit{(Дж. МакКарти)}:
{\scriptsize \begin{verbatim}
  cross :- not train. % можно перейти, если неизвестно, едет поезд или нет.
  cross :- -train.    % можно перейти, если известно, что поезд не едет.
\end{verbatim}}

Пример:
{\scriptsize \begin{verbatim}
  flies(X) :- bird(X), not -flies(X). % нет доказательств, 
                                      % что эта птица не умеет летать.
\end{verbatim}}

\end{frame}

\begin{frame}{Action Languages}
\textbf{Action languages} -- формальные модели фрагментов естественного языка, используемые для рассуждений об эффектах действий. \textit{(Gelfond, Lifschitz)}\bigskip

\begin{itemize}
  \item языки описания действий (STRIPS, PDDL, Language A, Language B, Language C, Language K)
  \item языки запросов (Language P, Language Q, Language R)
\end{itemize}\bigskip

Трансляторы из AL в ASP: coala, $DLV^K$
\end{frame}

\begin{frame}[fragile]
\frametitle{The Blocksworld (Language $K$)}
{\scriptsize \begin{verbatim}
fluents: on(B,L) requires block(B), location(L).
         occupied(B) requires location(B).

actions: move(B,L) requires block(B), location(L).

always:  executable move(B,L) if not occupied(B), not occupied(L), B <> L.
         inertial on(B,L).
         caused occupied(B) if on(B1,B), block(B).
         caused on(B,L) after move(B,L).
         caused -on(B,L1) after move(B,L), on(B,L1), L <> L1.

noConcurrency.
\end{verbatim}}
\end{frame}

\begin{frame}{ASP и Semantic Web}
\begin{itemize}
  \item немонотонный вывод наряду с DL-ризонерами
  \item новые возможности для языка запросов (SPARQL)
  \item интеграция данных (разрешение семантических конфликтов)
  \item слияние онтологий
\end{itemize}
\end{frame}

\begin{frame}{Recognizing Textual Entailment (1)}
\begin{itemize}
  \item логический вывод по фрагменту текста
  \item общая задача для Question Answering (QA), Information Extraction (IE), Machine Translation (MT), автореферирования, диалоговых систем и пр.
  \item RTE Track
    \begin{itemize}
      \item даны пары текст-гипотеза, требуется установить, следует ли гипотеза из текста
      \item возможные ответы: \textit{entails}, \textit{contradicts}, \textit{unknown}.
    \end{itemize}
\end{itemize}
\end{frame}

\begin{frame}{Recognizing Textual Entailment (2)}
\textbf{T:} The emerging historical record now shows White House officials downgraded threats from al Quaeda prior to 9/11 to focus on other priorities.\\
\textbf{H:} White House ignores the threat of attack.\bigskip 

\textbf{T:} Turnout for the historic vote for the first time since the EU took in 10 new members in May has hit a record low of 45.3\%.\\
\textbf{H:} New members joined the EU.\bigskip

\textbf{T:} In Paris, on March 15th, John packed his laptop in the carry-on luggage and took a plane to Baghdad.\\
\textbf{H:} His laptop was in Baghdad on March 16th.
\end{frame}

\begin{frame}{ASP в задаче RTE: Nutcracker}
\texttt{[Bos, Markert 2005]} -- \textit{Nutcracker}\\
\texttt{[Lierler, Lifschitz, 2008]} -- \textit{показали, как реализовать этот подход с помощью ASP-солвера (DLV)}\bigskip

\begin{itemize}
  \item прувер доказывает $T \wedge BK \to H$ $\Rightarrow$ \textit{entailment}
  \item прувер доказывает $\neg (T \wedge BK) \wedge H$ $\Rightarrow$ \textit{inconsistent}
  \item model builder строит модель для $T \wedge BK \wedge \neg H$ $\Rightarrow$ \textit{no entailment}
\end{itemize}\bigskip

{\footnotesize $T$ - текст, $BK$ - фоновые знания, $H$ - гипотеза}

\end{frame}

\begin{frame}{Диалоговые системы}
Как говорят люди:\bigskip
\begin{footnotesize}
\begin{itemize}
  \item терпимы к противоречиям (параконсистентность)
  \item умеют менять точку зрения 
  \item употребляют одновременно несколько модальностей
  \item умеют отслеживать происхождение знаний (эвиденциальность)
  \item способны к рефлексии
  \item умеют делать умозаключения (индукция, дедукция)
  \item умеют формулировать и проверять гипотезы (абдукция)
  \item могут обманывать и обманываться
  \item пользуются обширными знаниями и умеют учиться
  \item планируют диалог 
  \item преследуют коммуникативные цели
  \item умеют управлять вниманием
  \item умеют отслеживать изменения среды
\end{itemize}
\end{footnotesize}
\end{frame}

\begin{frame}{}
\thispagestyle{empty}
\begin{center}
{\large Спасибо!}
\end{center}
\end{frame}


%%% слайд помещается сюда
%% \begin{frame}{Заголовок}
%% \end{frame}

\end{document}
