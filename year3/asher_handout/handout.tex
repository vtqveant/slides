\documentclass[10pt]{article} 

\usepackage[T2A]{fontenc}
\usepackage[utf8]{inputenc}
\usepackage[english,russian]{babel}
\usepackage{amssymb,amsfonts,amsmath,amsthm,amscd,mathtext}

\usepackage[a4paper]{geometry} 
\geometry{verbose,tmargin=2cm,bmargin=2cm,lmargin=2cm,rmargin=2cm}

\sloppy
\clubpenalty=10000
\widowpenalty=10000

%%% оформление заголовков глав, секций и подсекций

\usepackage{titlesec}

\makeatletter
  \renewcommand{\section}{\@startsection{section}{1}{5.5ex}{-3.5ex plus -1ex minus -.2ex}{2.3ex plus.2ex}{\raggedright\hyphenpenalty=10000\normalfont\bfseries}}
\makeatother

\usepackage{indentfirst}      % красная строка
\setlength{\parindent}{5.5ex} % 5 символов

%%% biblatex

\usepackage{csquotes} % recommended by pdflatex
\usepackage[%
    style = gost-authoryear-min,
    singletitle = false,
    autolang = other,
    language = auto,
    backend = biber,
    defernumbers = true,
    sortlocale = ru
]{biblatex}
\addbibresource{bibliography.bib}

\defbibenvironment{bibliography}
{\enumerate{}
{\setlength{\leftmargin}{\bibhang}%
\setlength{\itemindent}{-\leftmargin}%
\setlength{\itemsep}{\bibitemsep}%
\setlength{\parsep}{\bibparsep}}}
{\endenumerate}
{\item}

\usepackage{hyperref}
\hypersetup{%
    colorlinks = true
}

%%% math env

\newtheoremstyle{example-style}% name
{5pt}  % space above 
{5pt}  % space below 
{}     % body font
{\parindent}  % indent 
{\bfseries}  % theorem head font
{.}  % punctuation after theorem head
{.5em}  % space after theorem head 
{}  % theorem head spec (can be left empty, meaning 'normal')

\theoremstyle{example-style}
\newtheorem{example}{Пример}

%%% linguistic stuff

% Gentzen style natural deduction proof trees
\usepackage{bussproofs}
\usepackage{latexsym}

\usepackage{tikz}
\usepackage{tikz-qtree,tikz-qtree-compat}   % regular trees (e.g. GB style)
\usepackage{tikz-dependency}                % dependency trees (bracket style)
\usetikzlibrary{matrix,arrows}              % for commutative diagrams

\usepackage{avm}
\avmsortfont{\footnotesize\it}
\avmvalfont{\rm}
\avmfont{\sc}

\usepackage{drs}
\usepackage{linguex}     % numbered linguistic examples and glosses

\begin{document} 

\title{Dot types}
\author{\textit{Константин Соколов {\small (СПбГУ, СПбПУ)}} \smallskip \\ {\small \texttt{vtqveant@gmail.com}}}
\date{}

\maketitle

\section{Тексты}

\parencite{asher2011lexical}, \parencite{pustejovsky2013type}, \parencite{pustejovsky1995generative}. Дополнительные лингвистические примеры по большей части заимствуются из \parencite{plungjan2011introduction} и \parencite{paducheva2014dynamic}.

\section{Type Composition Logic (TCL)}

\begin{itemize}
  \item Продолжает Пустейовского (полурешетка типов, subtyping, dot types, идея соотнесения предикатно-аргументной структуры с лексическим значением предиката)
  \item Идея контекста, <<влияющего>> на значение слова
  \item Теоретико-типовой анализ предикатно-аргументной структуры: ограничения на сочетаемость (т.е. дистрибуцию), тип предиката должен учитывать типы аргументов
\end{itemize}

\section{Напоминание о теории репрезентации дискурса (DRT)}

Для фрагмента связного текста (дискурса) строится т.н. \textit{структура репрезентации дискурса} (DRS), в которой для каждого референта заводится переменная. При последовательной обработке каждый новый фрагмент анализируется с учетом набора референтов, сформированного на предыдущем этапе. В примере \ref{drt:smile} местоимение в анафорическом употреблении трактуется как референциальная переменная.

\ex.\label{drt:smile} Всякий мужчина, встретив женщину$_i$, улыбается ей$_i$\\
	$\langle [\,], [\langle [x, y] , [man(x), woman(y), meet(x,y)] \rangle \Rightarrow \langle [\,], [smiles\_at(x,y)] \rangle ] \rangle$

Множество референциальных переменных называется \textit{контекстом}. В процессе включения новых фрагментов дискурса в общую структуру переменные согласуются друг с другом по особым правилам.

Пресуппозиция -- совокупность условий, необходимых для того, чтобы высказывание было правильно понято в своём прямом смысле. В динамической семантике. Например, в DRT пресуппозиция может пониматься как требования, предъявляемые к контексту -- наличие специальной переменной и т.п.

\section{Система типов TCL}

\begin{itemize}
  \item базовые типы \textit{(primitive types)}: {\sc e, t, physical\_object}, etc.
  \item типы пресуппозиции \textit{(presuppositional types)}: $\Pi$
  \item дизъюнктивные типы \textit{(disjunctive types)}: если $\sigma, \tau$ и $\rho$ -- типы и $\sigma \sqsubseteq \rho, \tau \sqsubseteq \rho$, то $(\sigma \vee \tau)$ -- тип
  \item функциональные типы \textit{(functional types)}: если $\sigma$ и $\tau$ -- типы, то $(\sigma \Rightarrow \tau)$ -- тип
  \item кванторные типы \textit{(quantificational types)}: если $\sigma$ -- базовый тип, $\tau$ -- произвольный тип, $x$ -- переменная по типам, то $\exists x \sqsubseteq \sigma \, \tau$ -- тип  (т.е. $x$ -- подтип $\sigma$)
  \item имеется $\bot$ \textit{(absurd type)}, но $\top$ нет, т.е. здесь не полная решетка типов, а semi-lattice; всегда определен \textit{infimum}, но \textit{supremum} -- не всегда
\end{itemize}

\section{Subtyping}

\textbf{Важное замечание.} Теоретико-множественная интерпретация подтипов функциональных типов неадекватна. E.g., экстенсионал типа {\sc (p $\Rightarrow$ t)} (физические свойства объектов) и экстенсионал типа {\sc (e $\Rightarrow$ t)} (подмножества объектов) не пересекаются, хотя {\sc p $\sqsubseteq$ e} (и потому можно было бы ожидать {\sc (p $\Rightarrow$ t) = (e $\Rightarrow$ t)$|_{\textsc{p}}$}). Корректную интерпретацию TCL можно построить в рамках теории топосов.

Подтипы для функциональных типов

\ex.\label{tcl:subtyping}
  \begin{prooftree}
    \AxiomC{$\alpha \sqsubseteq \alpha^\prime$}
    \AxiomC{$\beta \sqsubseteq \beta^\prime$}
    \RightLabel{}
    \BinaryInfC{$(\alpha^\prime \Rightarrow \beta) \sqsubseteq (\alpha \Rightarrow \beta^\prime)$}
  \end{prooftree}

Т.о., нельзя получить {\sc (p $\Rightarrow$ t) $\sqsubseteq$ (e $\Rightarrow$ t)}, поэтому потребуется изменить понятие <<свойство первого порядка>> (т.е. то, что раньше было $\langle e, t \rangle$) и сделать его тип $\exists x \sqsubseteq {\textsc{e}} \, (x \Rightarrow {\textsc{t}})$.

Конструкторы типов для выражений с подтипами (если угодно, \textit{subtyping as deduction})

\ex.\label{tcl:intro}
  \begin{prooftree}
    \AxiomC{$\beta \sqsubseteq \alpha$}
    \RightLabel{$I_\exists$}
    \UnaryInfC{$A \sqsubseteq (\exists x \sqsubseteq \alpha \; A[x / \beta])$}
  \end{prooftree}

\ex.\label{tcl:elim}
  \begin{prooftree}
    \AxiomC{$\beta \sqsubseteq \alpha$}
    \AxiomC{$A \sqsubseteq B$}
    \RightLabel{$E_\exists$}
    \BinaryInfC{$(\exists x \sqsubseteq \alpha \; A[x / \beta]) \sqsubseteq B$}
  \end{prooftree}

Пример: тип ${\textsc{black}} = ({\textsc{p}} \Rightarrow {\textsc{t}})$, тип (абстрактного) физического свойства ${\textsc{physical\_property}} = \exists x \sqsubseteq {\textsc{p}} \; (x \Rightarrow {\textsc{T}})$, поэтому {\sc black} $\sqsubseteq$ {\sc physical\_property}.

\section{Пресуппозиции типов}

Нужно реализовать проверку типов при применении предиката к аргументам, но т.к. здесь типы нетривиальные, может потребоваться их модификация или изменение в процессе композиции под влиянием контекста.\footnote{Ср.: \texttt{float + int = float, int + int = int}.} 


Чтобы этого добиться, Ашер вводит в каждый тип <<параметр пресуппозиции>> \textit{(presuppositional parameter)}, чтобы отразить <<динамическую>> адаптацию к контексту. Для этого в базовую логическу форму \ref{presup:basic} вводится дополнительная переменная $\pi$ специального типа, получается расширенная форма \ref{presup:param}.

\ex.\label{presup:basic} tree $\vdash \lambda x \; . \; $tree(x)

\ex.\label{presup:param} tree $\vdash \lambda x : {\textsc{p}} \; . \; \lambda \pi : \Pi \; . \;$tree(x, $\Pi$)

Помимо этого, нужно добиться особого поведения пресуппозиций при композиции: <<большое дерево>> -- это всё равно <<дерево>>, поэтому пресуппозиции из терма <<дерево>>, должны <<протаскиваться>> наружу и согласовываться с модификатором. В форме  \ref{presup:percloation} переменная $\mathcal{P}$ имеет тип модификатора ${\textsf{MOD}} = \exists x \sqsubseteq {\textsc{e}} \, (x \Rightarrow (\Pi \Rightarrow {\textsc{t}})) \Rightarrow \exists x \sqsubseteq {\textsc{e}} \, (x \Rightarrow (\Pi \Rightarrow {\textsc{t}}))$, отображающее одно свойство первого порядка (тип $\exists x \sqsubseteq {\textsc{e}} \, (x \Rightarrow (\Pi \Rightarrow {\textsc{t}}))$) в другое. Здесь ${\textsf{ARG}}^{\textsf{tree}}_1$ -- первый аргумент tree.

\ex.\label{presup:percloation} tree $\vdash \lambda \mathcal{P} : {\textsf{MOD}} \; . \; \lambda x : P \; . \; \lambda \pi : \Pi \; . \; \mathcal{P} ( \pi \ast {\textsf{ARG}}^{\textsf{tree}}_1 : P)(x)( \lambda v \; . \; \lambda \pi^\prime \; . \; {\textsf{tree}}(v, \pi^\prime))$



\section{Формальная система TCL}

Две группы правил: простые правила и правила для работы с пресуппозицией.\\

\textbf{Простые правила}

\ex.\label{rule:application} Application\\
 \begin{prooftree}
    \AxiomC{$\lambda x : \alpha \; . \; \phi [t : \gamma]$}
    \AxiomC{$\lambda x : \alpha \; . \; \phi : \alpha \Rightarrow \beta$}
    \AxiomC{$\gamma \sqsubseteq \alpha$}
%    \RightLabel{$$}
    \TrinaryInfC{$\phi [t / x] : \beta$}
  \end{prooftree}

\ex.\label{rule:application} Application\\
 \begin{prooftree}
    \AxiomC{$$}
    \AxiomC{$$}
    \RightLabel{$$}
    \BinaryInfC{$$}
  \end{prooftree}



\iffalse

\section{Проблемы лексической семантики}

Проблемы лексической семантики: creative use of words, etc.;

лексикализм в лингвистике вообще и в генеративной лингвистике в частности

Ещё раз о требованиях к семантическому представлению
\begin{itemize}
	\item возможность сочетания (композициональной) семантики предложения и лексической семантики
	\item возможность учета семантических отношений (гипо-гиперонимических и др.) -- грузовик $\sqsubseteq$ автомобиль (подтипы)
	\item возможность реализации фреймовой семантики (т.к. реализация семантических ролей отражается в значении слова, 
 ср. ``вода течет в лодку'', ``лодка течет'')
	\item решение проблемы сопредикации (copredication) -- необходимости приписывать слову два типа для
 отражения различных аспектов значения, притом что два эти типа не могут быть одновременно
 (lunch was delicious but took forever -- both food and event?)
	\item логическая полисемия (термин Пустейовского) -- разные, но взаимосвязанные значения, раскрывающиеся в контексте
 (a heavy book (физический объект) vs. a boring book (носитель информации)
	\item компонентная семантика (холостяк = [+ мужчина] [+ взрослый] [- женатый]
	\item co-composition и функциональное применение (стандартный подход в композиционной семантике -- одно слово ``применяется'' к другому; нужно учесть явления ``взаимообусловленности значений''
	\item новообразования (термин Падучевой) -- метафорическое, метонимическое употребление и пр., языковое творчество 
\end{itemize} 


\section{Основы формализма}

Унификационные грамматики и типизированные структуры признаков. Логическое описание структур признаков (Каспер-Раундс, Джонсон, гибридная логика). Графовое представление структур признаков, reentrancy. Subsumption relation и решетка типов; связь с системами типов с подтипами (subtyping coerсion); связь с системами множественного наследования

\begin{avm} 
\[\asort{$\alpha$}
  argstr & \[arg1 & $x$ \\
             \dots \] \\
  eventstr & \[E1 & $e_1$ \\
             \dots   
             \] \\
  qualia & \[const & \textit{what x is made of} \\
             formal & \textit{what x is} \\
             telic & \textit{function of x} \\
             agentive & \textit{how x came into being}
    \]
\]               
\end{avm}


\section{Конструкция записи в ГЛ}
Общая конструкция записи в GL. Концепт (слово) как тип.\\

Пояснения к конструкции и (лингвистический) бэкграунд
\begin{enumerate}
    \item argument structure в GL -- аргументно-предикатная структура, фреймы, PropBank, тематические роли, тета-критерий, диатеза и актантная деривация
    \item event structure в GL -- грамматическая семантика и основные граммемы глагола (основное деление -- аспект vs. модальность); аспектология; event semantics Дэвидсона как развитие формальной семантики;
    \item qualia structure -- учение о причинах у Аристотеля; компонентный анализ значения; 
    \item inheritance structure -- гиперо-гипонимичские и пр. отношения в структуре словаря; тезаурус и семантические поля; WordNet
\end{enumerate}

\section{Лексическая декомпозиция}

Основная идея: аргументно-предикатная структура и декомпозиция лексического значения -- по сути, два взгляда на одно и то же явление. 

Лексическая декомпозиция (``агрументно-предикатный уровень''): можно ли определить значение предиката через типы аргументов? Четыре способа построить отображение между аргументно-предикатной структурой и синтаксисом. Эти способы с примерами. GL -- один из эти способов.

Модель, с помощью которой иллюстрируются различные способы соотнесения аргументно-предикатной структуры и синтаксиса

\begin{exe}
\ex{$\lambda x_i . \Phi$} \label{pred:atomic}
\end{exe}

Известные способы декомпозиции можно разбить на классы (один тривиальный + четыре):

Тривиальный подход -- \textbf{атомарная предикация}. Параметры отображаются в глагольные аргументы непосредственно
\begin{exe}
\ex{$\lambda x_n \dots \lambda x_1 . \Phi \Rightarrow Verb(Arg_1, \dots, Arg_n)$} \label{pred:atomic:ex1}
\end{exe}

$\theta$-теория в GB

hit : $\lambda y . \lambda x . hit(x,y)$

[the car]$_i$ hit [the wall]$_j$

тета-решетка (theta-grid):

\begin{table}[ht]
\centering
\begin{tabular}{|c|c|}
\hline
агенс & пациенс \\ \hline
i     & j       \\ \hline
\end{tabular}
\end{table}

Тета-решетка -- это способ задать соответствие между аргументами и элементами синтаксической структуры. В GB формулируется условие, что это соответствие должно быть одно-однозначным (т.н. тета-критерий). 

\begin{exe}
  \ex[*]{Мальчик читает книгу газету} \label{pred:atomic:ex2}
\end{exe}

Пример (\ref{pred:atomic:ex2}) нарушает $\theta$-критерий и потому неграмматичен.

\textbf{Параметрическая декомпозиция.} Атомарное тело выражения, в список аргументов вводятся новые

\begin{exe}
\ex{$\lambda x_m \dots \lambda x_{n+1} \lambda x_n \dots \lambda x_1 . \Phi$}
\end{exe}

Пример: дэвидсоновская event semantics.

\begin{exe}
\ex 
  \begin{xlist}
    \ex Mary ate the soup.
    \ex Mary ate the soup with a spoon.
    \ex Mary ate the soup with a spoon in the kitchen.
    \ex Mary ate the soup with a spoon in the kitchen at 3:00 pm.
  \end{xlist}
\ex 
  \begin{xlist}
    \ex{$\exists e$[\textit{eat}($e$, $m$, the\_soup)]} 
    \ex{$\exists e$[\textit{eat}($e$, $m$, the\_soup) $\wedge$ \textit{with}($e$, a\_spoon)]} 
    \ex{$\exists e$[\textit{eat}($e$, $m$, the\_soup) $\wedge$ \textit{with}($e$, a\_spoon) $\wedge$ \textit{in}($e$, the\_kitchen)]} 
    \ex{$\exists e$[\textit{eat}($e$, $m$, the\_soup) $\wedge$ \textit{with}($e$, a\_spoon) $\wedge$ \textit{in}($e$, the\_kitchen) $\wedge$ \textit{at}($e$, 3:00pm)]} 
  \end{xlist}
\end{exe}

В данном случае содержание предиката не меняется, но одно-однозначное соответствие между агрументами и синтаксисом не сохраняется (теория должна сформулировать способ указать, сохраняется или удаляется ли конкретный аргумент).

\textbf{Простая предикативная декомпозиция.} Тело предиката представляется как сложное выражение.

\begin{exe}
\ex{$\lambda x . [\Phi_1, \dots , \Phi_k]$}
\end{exe}

Пример: компонентный анализ значения.

\begin{exe}
  \ex $\forall x$[\textit{bachelor}(x) $\Rightarrow$ [\textit{male}(x) $\wedge$ \textit{adult}(x) $\wedge \neg$ \textit{married}(x)]]  
  \ex $\forall x$[\textit{die}(x) $\Rightarrow$ [\textit{become}($\neg$ \textit{alive}(x)]]
\end{exe}

В данном случае помимо идентификации аргумента, требуется соотнести субпредикаты с одним синтаксическим материалом.

\textbf{Полная предикативная декомпозиция.} Расширение списка аргументов и сложная структура тела предиката.

\begin{exe}
\ex{$\lambda x_m \dots \lambda x_{n+1} \lambda x_n \dots \lambda x_1 . [\Phi_1, \dots, \Phi_k]$}
\end{exe}

\begin{exe}
  \ex kill : $\lambda y \lambda x \lambda e_1 \lambda e_2$ [\textit{act}(e_1, x, y) $\wedge \neg$ \textit{dead}(e_1, y) $\wedge$ \textit{dead}(e_2, x) $\wedge$ e_1 < e_2]  
\end{exe}

Пример: ГЛ

\textbf{Супралексическая декомпозиция.} Выражение не меняется, но множество параметров снабжается некоторой структурой путем введения дополнительных операторов.

\begin{exe}
\ex{$\lambda f_\sigma \lambda x_1 [R(f)(x_1)](\lambda x . [\Phi_1, \dots, \Phi_k])_\sigma$}
\end{exe}

Пример: залог и актантная деривация

Залог -- глагольная категория, граммемы которой указывают на определенное изменение коммуникативного ранга участников ситуации (при сохранении их числа!)\\
\medskip
\begin{itemize}
    \item \textit{активный залог} отражает базовую ранговую структуру
    \item \textit{косвенные залоги} отражают передачу статуса от одного глагольного аргумента к другому
\end{itemize}

\begin{itemize}
    \item \textit{пассив} -- лишение исходного подлежащего наивысшего коммуникативного ранга
        \begin{itemize}
            \item пассивная конструкция с нулевым агенсом (\textit{разговор был прерван})
            \item пассивная конструкция без повышения статуса агенса (\textit{в газете сообщалось о параде})
        \end{itemize}
    \item \textit{пермутатив} -- не затрагивает статус подлежащего, но перераспределяет статус дополнений (\textit{царь одарил его шубой} из \textit{царь подарил ему шубу})
    \item \textit{транзитиватив} -- повышение косвенного дополнения до прямого (\textit{мухи обсели абажур}) или понижение прямого дополнения до косвенного (\textit{ветер швырялся песком})
\end{itemize}

Для реализации залога нужно добавить процедуру изменения коммуникативного ранга актантов как набор отношений между актантоами, \textit{внешний} по отношению к значению глагола.



\section{Система типов в GL}

Мотивация: \textit{испечь картошку}, \textit{испечь хлеб}; но (??)\textit{выпекать картошку}, \textit{выпекать хлеб}. Искусственные vs. естественные вещи.

типизация для учета лексических классов:
sleep : anim $\to$ t (ср. с $<e,t>$

Различные стратегии декомпозиции:
\begin{itemize}
  \item (атомарная) $\lambda x$[sleep(x)] : e $\to$ t
  \item (предикативная) $\lambda x$[sleep(x) $\wedge$ animate(x)] : e $\to$ t
  \item (расширенная типизация) $\lambda x$:\textit{anim}[sleep(x)] : anim $\to$ t
\end{itemize}

Таксономия типов: естественные (ср. natural kind и Рош), искусственные и сложные типы. Их определение через qualia. Естественные содержат только формальные и материальные качества; искусстенные содержат материальные качества; сложные содержат информацию об отношениях между типами. Qualia unification.\\

Система типов в GL. Структура наследования типов и qualia как система ограничений на типы структур признаков. Три конструктора типов (стрелка, точка, тензор). 

\section{Механизм композиции в ГЛ}

Операции, учитывающие типы, для реализации ``выбора'' (selection) значений требуют особых типов и особых механизмов композиции. Композициональность по Монтегю -- функциональное применение (реализуется как конкатенация лямбда-термов с последующей редукцией). Расширение композициональных механизмов у Пустейовского (так что получаются ``механизмы выбора''): а) pure selection (type matching), б) accomodation в) type coercion (два варианта -- exploitation и introduction). Примеры действия этих механизмов.

\section{Структура события}

Event structure. Внутренняя структура событий; события используются для задания ограничений на qualia; аспектуальность и лексическая семантика каузации (глава 9)

\section{Заключение}

Заключение. Отличительные особенности GL. Семантика синтаксическими средствами (внутри формального языка, конечно) в GL vs. теоретико-модельная семантика (Монтегю и пр.). Динамический характер лексикона в GL vs. статический лексикон как в дистрибутивной семантике, так и в CCG и других лексикализованных концепциях. GL и DPL (GL как динамическая семантика)\\


\fi

\nocite{*}
\printbibliography[resetnumbers=true]

\end{document}