\documentclass[10pt]{article} 

\usepackage[T2A]{fontenc}
\usepackage[utf8]{inputenc}
\usepackage[english,russian]{babel}
\usepackage{amssymb,amsfonts,amsmath,amsthm,amscd,mathtext}

\usepackage[a4paper]{geometry} 
\geometry{verbose,tmargin=2cm,bmargin=2cm,lmargin=2cm,rmargin=2cm}

\sloppy
\clubpenalty=10000
\widowpenalty=10000

%%% оформление заголовков глав, секций и подсекций

\usepackage{titlesec}

\makeatletter
  \renewcommand{\section}{\@startsection{section}{1}{5.5ex}{-3.5ex plus -1ex minus -.2ex}{2.3ex plus.2ex}{\raggedright\hyphenpenalty=10000\normalfont\bfseries}}
\makeatother

\usepackage{indentfirst}      % красная строка
\setlength{\parindent}{5.5ex} % 5 символов

%%% biblatex

\usepackage{csquotes} % recommended by pdflatex
\usepackage[%
    style = gost-authoryear-min,
    singletitle = false,
    autolang = other,
    language = auto,
    backend = biber,
    defernumbers = true,
    sortlocale = ru
]{biblatex}
\addbibresource{bibliography.bib}

\defbibenvironment{bibliography}
{\enumerate{}
{\setlength{\leftmargin}{\bibhang}%
\setlength{\itemindent}{-\leftmargin}%
\setlength{\itemsep}{\bibitemsep}%
\setlength{\parsep}{\bibparsep}}}
{\endenumerate}
{\item}

\usepackage{hyperref}
\hypersetup{%
    colorlinks = true
}

%%% math env

\newtheoremstyle{example-style}% name
{5pt}  % space above 
{5pt}  % space below 
{}     % body font
{\parindent}  % indent 
{\bfseries}  % theorem head font
{.}  % punctuation after theorem head
{.5em}  % space after theorem head 
{}  % theorem head spec (can be left empty, meaning 'normal')

\theoremstyle{example-style}
\newtheorem{example}{Пример}

%%% linguistic stuff

% Gentzen style natural deduction proof trees
\usepackage{bussproofs}
\usepackage{latexsym}

\usepackage{tikz}
\usepackage{tikz-qtree,tikz-qtree-compat}   % regular trees (e.g. GB style)
\usepackage{tikz-dependency}                % dependency trees (bracket style)
\usetikzlibrary{matrix,arrows}              % for commutative diagrams

\usepackage{avm}
\avmsortfont{\footnotesize\it}
\avmvalfont{\rm}
\avmfont{\sc}

\usepackage{drs}
\usepackage{linguex}     % numbered linguistic examples and glosses

\begin{document} 

\title{Type Composition Logic}
\author{\textit{Константин Соколов {\small (СПбГУ, СПбПУ)}} \smallskip \\ {\small \texttt{vtqveant@gmail.com}}}
\date{}

\maketitle

\section{Предварительные замечания}

\begin{itemize}
  \item Основной текст -- \parencite{asher2011lexical}, глава 4 (Type Composition Logic).
  \item Во многом продолжает Пустейовского (полурешетка типов, subtyping, dot types, идея соотнесения предикатно-аргументной структуры с лексическим значением предиката).
  \item Идея контекста, <<влияющего>> на значение слова.
  \item Теоретико-типовой анализ предикатно-аргументной структуры: ограничения на сочетаемость (т.е. дистрибуцию), тип предиката должен учитывать типы аргументов.
\end{itemize}

\section{Система типов}

\begin{itemize}
  \item базовые типы \textit{(primitive types)}: {\sc e, t, physical\_object}, etc.
  \item типы пресуппозиции \textit{(presuppositional types)}: $\Pi$
  \item дизъюнктивные типы \textit{(disjunctive types)}: если $\sigma, \tau$ и $\rho$ -- типы и $\sigma \sqsubseteq \rho, \tau \sqsubseteq \rho$, то $(\sigma \vee \tau)$ -- тип
  \item функциональные типы \textit{(functional types)}: если $\sigma$ и $\tau$ -- типы, то $(\sigma \Rightarrow \tau)$ -- тип
  \item кванторные типы \textit{(quantificational types)}: если $\sigma$ -- базовый тип, $\tau$ -- произвольный тип, $x$ -- переменная по типам, то $\exists x \sqsubseteq \sigma \, \tau$ -- тип  (т.е. $x$ -- подтип $\sigma$)
  \item имеется $\bot$ \textit{(absurd type)}, но $\top$ нет, т.е. здесь не полная решетка типов, а полурешетка; всегда определен \textit{infimum}, но \textit{supremum} -- не всегда
\end{itemize}

\section{Подтипы}

Теоретико-множественная интерпретация подтипов функциональных типов неадекватна. E.g., экстенсионал типа {\sc (p $\Rightarrow$ t)} (физические свойства объектов) и экстенсионал типа {\sc (e $\Rightarrow$ t)} (подмножества объектов) не пересекаются, хотя {\sc p $\sqsubseteq$ e} (и потому можно было бы ожидать {\sc (p $\Rightarrow$ t) = (e $\Rightarrow$ t)$|_{\textsc{p}}$}).\footnote{Интерпретацию TCL можно построить с помощью теории топосов, что Ашер делает дальше.} Подтипы для функциональных типов представляют собой проблему.

\ex.\label{tcl:subtyping}
  \begin{prooftree}
    \AxiomC{$\alpha \sqsubseteq \alpha^\prime$}
    \AxiomC{$\beta \sqsubseteq \beta^\prime$}
    \RightLabel{}
    \BinaryInfC{$(\alpha^\prime \Rightarrow \beta) \sqsubseteq (\alpha \Rightarrow \beta^\prime)$}
  \end{prooftree}

Т.о., нельзя получить {\sc (p $\Rightarrow$ t) $\sqsubseteq$ (e $\Rightarrow$ t)}, поэтому потребуется изменить понятие <<свойство первого порядка>> (т.е. то, что раньше было $\langle e, t \rangle$) и сделать его тип $\exists x \sqsubseteq {\textsc{e}} \, (x \Rightarrow {\textsc{t}})$.

Конструкторы типов для выражений с подтипами (если угодно, \textit{subtyping as deduction})

\ex.\label{tcl:intro}
  \begin{prooftree}
    \AxiomC{$\beta \sqsubseteq \alpha$}
    \RightLabel{$I_\exists$}
    \UnaryInfC{$A \sqsubseteq (\exists x \sqsubseteq \alpha \; A[x / \beta])$}
  \end{prooftree}

\ex.\label{tcl:elim}
  \begin{prooftree}
    \AxiomC{$\beta \sqsubseteq \alpha$}
    \AxiomC{$A \sqsubseteq B$}
    \RightLabel{$E_\exists$}
    \BinaryInfC{$(\exists x \sqsubseteq \alpha \; A[x / \beta]) \sqsubseteq B$}
  \end{prooftree}

Пример: тип ${\textsc{black}} = ({\textsc{p}} \Rightarrow {\textsc{t}})$, тип (абстрактного) физического свойства ${\textsc{physical\_property}} = \exists x \sqsubseteq {\textsc{p}} \; (x \Rightarrow {\textsc{T}})$, поэтому {\sc black} $\sqsubseteq$ {\sc physical\_property}.

\section{Напоминание: пресуппозиция в теории репрезентации дискурса}

Для фрагмента связного текста (дискурса) строится т.н. \textit{структура репрезентации дискурса} (DRS), в которой для каждого референта заводится переменная. При последовательной обработке каждый новый фрагмент анализируется с учетом набора референтов, сформированного на предыдущем этапе. В примере \ref{drt:smile} местоимение в анафорическом употреблении трактуется как референциальная переменная.

\ex.\label{drt:smile} Всякий мужчина, встретив женщину$_i$, улыбается ей$_i$\\
	$\langle [\,], [\langle [x, y] , [man(x), woman(y), meet(x,y)] \rangle \Rightarrow \langle [\,], [smiles\_at(x,y)] \rangle ] \rangle$

Множество референциальных переменных называется \textit{контекстом}. В процессе включения новых фрагментов дискурса в общую структуру переменные согласуются друг с другом по особым правилам.

Пресуппозиция -- совокупность условий, необходимых для того, чтобы высказывание было правильно понято в своём прямом смысле. Концепция проекции пресуппозиции -- идея, что условия могут эксплицироваться таким же образом, как и референциальные переменные, т.е. что пресуппозиция может анализироваться с помощью того же аппарата, что и анафора. В динамической семантике пресуппозиция может пониматься как требования, предъявляемые к контексту -- наличие в контексте особой переменной и т.п.

\section{Пресуппозиции типов}

Нужно реализовать проверку типов при применении предиката к аргументам, но т.к. здесь типы нетривиальные, может потребоваться их модификация или изменение в процессе композиции под влиянием контекста.\footnote{Ср.: \texttt{float + int = float, int + int = int}.} 

Чтобы этого добиться, Ашер вводит в каждый тип <<параметр пресуппозиции>> \textit{(presuppositional parameter)}, чтобы отразить <<динамическую>> адаптацию к контексту. Для этого в базовую логическу форму \ref{presup:basic} вводится дополнительная переменная $\pi$ специального типа, получается расширенная форма \ref{presup:param}.

\ex.\label{presup:basic} tree $\vdash \lambda x \; . \; $tree(x)

\ex.\label{presup:param} tree $\vdash \lambda x : {\textsc{p}} \; . \; \lambda \pi : \Pi \; . \;$tree(x, $\pi$)

Помимо этого, нужно добиться особого поведения пресуппозиций при композиции: <<большое дерево>> -- это всё равно <<дерево>>, поэтому пресуппозиции из терма <<дерево>> должны <<протаскиваться>> наружу и согласовываться с модификатором. В форме  \ref{presup:percloation} переменная $\mathcal{P}$ имеет тип модификатора ${\textsf{MOD}} = \exists x \sqsubseteq {\textsc{e}} \, (x \Rightarrow (\Pi \Rightarrow {\textsc{t}})) \Rightarrow \exists x \sqsubseteq {\textsc{e}} \, (x \Rightarrow (\Pi \Rightarrow {\textsc{t}}))$, отображающее одно свойство первого порядка (тип $\exists x \sqsubseteq {\textsc{e}} \, (x \Rightarrow (\Pi \Rightarrow {\textsc{t}}))$) в другое. Здесь ${\textsf{ARG}}^{\textsf{tree}}_1$ -- первый аргумент tree, выражение $\pi \ast {\textsf{ARG}}^{\textsf{tree}}_1 : P$ означает, что в контекст $\pi$ добавляется ограничение на тип первого аргумента tree (он должен иметь тип $P$).\footnote{Ср.: форма записи $\Gamma, A$ в логике.}

\ex.\label{presup:percloation} tree $\vdash \lambda \mathcal{P} : {\textsf{MOD}} \; . \; \lambda x : {\textsc{e}} \; . \; \lambda \pi : \Pi \; . \; \mathcal{P} ( \pi \ast {\textsf{ARG}}^{\textsf{tree}}_1 : P)(x)( \lambda v \; . \; \lambda \pi^\prime \; . \; {\textsf{tree}}(v, \pi^\prime))$



\section{Формальная система TCL}

Две группы правил: простые правила (применение, абстракция, подстановка) и правила для работы с пресуппозицией.\\

\ex.\label{rule:application} Применение \textit{(application)}\\
 \begin{prooftree}
    \AxiomC{$\lambda x : \alpha \; . \; \phi [t : \gamma]$}
    \AxiomC{$\lambda x \; . \; \phi : \alpha \Rightarrow \beta$}
    \AxiomC{$\gamma \sqsubseteq \alpha$}
    \TrinaryInfC{$\phi [t / x] : \beta$}
  \end{prooftree}

\ex.\label{rule:abstraction} Абстракция \textit{(abstraction)}\\
 \begin{prooftree}
    \AxiomC{$x : \alpha$}
    \AxiomC{$t : \beta$}
    \BinaryInfC{$\lambda x \; . \; t : \alpha \Rightarrow \beta$}
  \end{prooftree}

\ex.\label{rule:substitution} Подстановка \textit{(substitution)}\\
 \begin{prooftree}
    \AxiomC{$t \to_\beta t^\prime$}
    \UnaryInfC{$\phi \to_\beta \phi [t^\prime / t]$}
  \end{prooftree}

\ex.\label{rule:binding} Связывание пресуппозиций \textit{(binding presuppositions)}\\
 \begin{prooftree}
    \AxiomC{$\alpha \sqsubseteq \beta$}
    \AxiomC{${\textsf{ARG}}^P_i : \alpha$}
    \AxiomC{${\textsf{ARG}}^P_i : \beta$}
    \TrinaryInfC{${\textsf{ARG}}^P_i : \alpha$}
  \end{prooftree}

\ex.\label{rule:accomodation} Простое согласование типов \textit{(simple type accomodation)}\\
 \begin{prooftree}
    \AxiomC{$\alpha \sqcap \beta \neq \bot$}
    \AxiomC{${\textsf{ARG}}^P_i : \alpha \ast {\textsf{ARG}}^P_i : \beta$}
    \BinaryInfC{${\textsf{ARG}}^P_i : \alpha \sqcap \beta$}
  \end{prooftree}
  
\textbf{Замечания.} Правило подстановки выводимо из \ref{rule:abstraction} и \ref{rule:application}, но вводится для удобства. Связывание -- частный случай простого согласования типов, т.к. если $\alpha \sqsubseteq \beta$, то $\alpha \sqcap \beta = \alpha$.

\section{Пример вывода}

Присоединение тривиального (нулевого) модификатора. Форма \ref{inf:tree} имеет нереализованную пресуппозицию типа, поэтому может присоединять модификаторы (напр., образовать форму big tree). 

\ex.\label{inf:tree} tree $\vdash \lambda \mathcal{P} : {\textsf{MOD}} \; . \; \lambda x : {\textsc{e}} \; . \; \lambda \pi : \Pi \; . \; \mathcal{P} ( \pi \ast {\textsf{ARG}}^{\textsf{tree}}_1 : P)(x)( \lambda v \; . \; \lambda \pi^\prime \; . \; {\textsf{tree}}(v, \pi^\prime))$

\ex.\label{inf:null} $\varnothing \vdash \lambda P : \exists x \sqsubseteq {\textsc{e}} \, (x \Rightarrow (\Pi \Rightarrow {\textsc{t}})) \; . \; \lambda x_1 : {\textsc{e}} \; . \; \lambda \pi_1 \; . \; P(\pi_1)(x_1)$

Нулевой модификатор \ref{inf:null} <<удовлетворяет>> требованиям пресуппозиции, не модифицируя значения выражения. Редуцировав форму <<$\varnothing$ tree>> по правилам применения и подстановки, получаем форму \ref{inf:sattree}, которую далее модифицировать нельзя (но можно присоединить к ней артикль, например).

\ex.\label{inf:sattree} ${\mathsf{tree}}^\prime \vdash \lambda w \; . \; \lambda \pi \; . \; tree(w, \pi \ast {\textsf{ARG}}^{\textsf{tree}}_1 : P)$

\nocite{*}
\printbibliography[resetnumbers=true]

\end{document}