\documentclass[10pt]{article} 

\usepackage[T2A]{fontenc}
\usepackage[utf8]{inputenc}
\usepackage[english,russian]{babel}
\usepackage{amssymb,amsfonts,amsmath,amsthm,amscd,mathtext}

\usepackage[a4paper]{geometry} 
\geometry{verbose,tmargin=2cm,bmargin=2cm,lmargin=2cm,rmargin=2cm}

\sloppy
\clubpenalty=10000
\widowpenalty=10000

%%% оформление заголовков глав, секций и подсекций

\usepackage{titlesec}

\makeatletter
  \renewcommand{\section}{\@startsection{section}{1}{5.5ex}{-3.5ex plus -1ex minus -.2ex}{2.3ex plus.2ex}{\raggedright\hyphenpenalty=10000\normalfont\bfseries}}
\makeatother

\usepackage{indentfirst}      % красная строка
\setlength{\parindent}{5.5ex} % 5 символов

%%% biblatex

\usepackage{csquotes} % recommended by pdflatex
\usepackage[%
    style = gost-authoryear-min,
    singletitle = false,
    autolang = other,
    language = auto,
    backend = biber,
    defernumbers = true,
    sortlocale = ru
]{biblatex}
\addbibresource{bibliography.bib}

\defbibenvironment{bibliography}
{\enumerate{}
{\setlength{\leftmargin}{\bibhang}%
\setlength{\itemindent}{-\leftmargin}%
\setlength{\itemsep}{\bibitemsep}%
\setlength{\parsep}{\bibparsep}}}
{\endenumerate}
{\item}

\usepackage{hyperref}
\hypersetup{%
    colorlinks = true
}

%%% math env

\newtheoremstyle{example-style}% name
{5pt}  % space above 
{5pt}  % space below 
{}     % body font
{\parindent}  % indent 
{\bfseries}  % theorem head font
{.}  % punctuation after theorem head
{.5em}  % space after theorem head 
{}  % theorem head spec (can be left empty, meaning 'normal')

\theoremstyle{example-style}
\newtheorem{example}{Пример}

%%% linguistic stuff

% Gentzen style natural deduction proof trees
\usepackage{bussproofs}
\usepackage{latexsym}

\usepackage{tikz}
\usepackage{tikz-qtree,tikz-qtree-compat}   % regular trees (e.g. GB style)
\usepackage{tikz-dependency}                % dependency trees (bracket style)
\usetikzlibrary{matrix,arrows}              % for commutative diagrams

\usepackage{avm}
\avmsortfont{\footnotesize\it}
\avmvalfont{\rm}
\avmfont{\sc}

\usepackage{gb4e} % пакет для лингвистических примеров
\let\eachwordtwo=\small
\let\eachwordthree=\small

\begin{document} 

\title{Генеративный лексикон Дж.~Пустейовского}
\author{\textit{Константин Соколов {\small (СПбГУ, СПбПУ)}} \smallskip \\ {\small \texttt{vtqveant@gmail.com}}}
\date{}

\maketitle

\section{Тексты}

В основном материал излагается по статье \parencite{pustejovsky2013type} и немного по книге \parencite{pustejovsky1995generative} (про qualia и структуру события). Дополнительные лингвистические примеры по большей части заимствуются из учебника \parencite{plungjan2011introduction}.

\section{Проблемы лексической семантики}

Проблемы лексической семантики: creative use of words, etc.;

лексикализм в лингвистике вообще и в генеративной лингвистике в частности

Ещё раз о требованиях к семантическому представлению
\begin{itemize}
	\item возможность сочетания (композициональной) семантики предложения и лексической семантики
	\item возможность учета семантических отношений (гипо-гиперонимических и др.) -- грузовик $\sqsubseteq$ автомобиль (подтипы)
	\item возможность реализации фреймовой семантики (т.к. реализация семантических ролей отражается в значении слова, 
 ср. ``вода течет в лодку'', ``лодка течет'')
	\item решение проблемы сопредикации (copredication) -- необходимости приписывать слову два типа для
 отражения различных аспектов значения, притом что два эти типа не могут быть одновременно
 (lunch was delicious but took forever -- both food and event?)
	\item логическая полисемия (термин Пустейовского?) -- разные, но взаимосвязанные значения, раскрывающиеся в контексте
 (a heavy book (физический объект) vs. a boring book (носитель информации)
	\item компонентная семантика (холостяк = [+ мужчина] [+ взрослый] [- женатый]
	\item co-composition и функциональное применение (стандартный подход в композиционной семантике -- одно слово ``применяется'' к другому; нужно учесть явления ``взаимообусловленности значений''
	\item новообразования (термин Падучевой) -- метафорическое, метонимическое употребление и пр., языковое творчество 
\end{itemize} 


\section{Основы формализма}

Унификационные грамматики и типизированные структуры признаков. Логическое описание структур признаков (Каспер-Раундс, Джонсон, гибридная логика). Графовое представление структур признаков, reentrancy. Subsumption relation и решетка типов; связь с системами типов с подтипами (subtyping coerсion); связь с системами множественного наследования

\begin{avm} 
\[\asort{some-type}
  conj & \textsc{and0} \\
  pred & `\textsc{think}<($\uparrow$ \textsc{subj})($\uparrow$ \textsc{comp})>' \\
  subj & \[\asort{some-other-type} pred & `\textsc{pro}'\] \\
  comp & \[{}
          pred & `\textsc{cheat}<($\uparrow$ \textsc{subj})($\uparrow$ \textsc{obj})>' \\
          subj & \[{} pred & `\textsc{pro}'\] \\
          obj & \[{} pred & `\textsc{pro}'\] \\
         \] \\
\]
\end{avm}


\section{Конструкция записи в ГЛ}
Общая конструкция записи в GL. Концепт (слово) как тип.\\

Пояснения к конструкции и (лингвистический) бэкграунд
\begin{enumerate}
    \item argument structure в GL -- аргументно-предикатная структура, фреймы, PropBank, тематические роли, тета-критерий, диатеза и актантная деривация
    \item event structure в GL -- грамматическая семантика и основные граммемы глагола (основное деление -- аспект vs. модальность); аспектология; event semantics Дэвидсона как развитие формальной семантики;
    \item qualia structure -- учение о причинах у Аристотеля; компонентный анализ значения; 
    \item inheritance structure -- гиперо-гипонимичские и пр. отношения в структуре словаря; тезаурус и семантические поля; WordNet
\end{enumerate}

\section{Лексическая декомпозиция}

Основная идея: аргументно-предикатная структура и декомпозиция лексического значения -- по сути, два взгляда на одно и то же явление. 

Лексическая декомпозиция (``агрументно-предикатный уровень''): можно ли определить значение предиката через типы аргументов? Четыре способа построить отображение между аргументно-предикатной структурой и синтаксисом. Эти способы с примерами. GL -- один из эти способов.

Модель, с помощью которой иллюстрируются различные способы соотнесения аргументно-предикатной структуры и синтаксиса

\begin{exe}
\ex\label{pred:atomic}
$\lambda x_i . \Phi$
\end{exe}

Известные способы декомпозиции можно разбить на четыре класса:

Тривиальный подход -- атомарная предикация. Параметры отображаются в глагольные аргументы непосредственно
\begin{exe}
\ex\label{pred:atomic:ex1}{$\lambda x_n \dots \lambda x_1 . \Phi \Rightarrow Verb(Arg_1, \dots, Arg_n)$}
\end{exe}

$\theta$-теория в GB

hit : $\lambda y . \lambda x . hit(x,y)$

[the car]$_i$ hit [the wall]$_j$

тета-решетка (theta-grid):

\begin{table}[ht]
\centering
\begin{tabular}{|c|c|}
\hline
агенс & пациенс \\ \hline
i     & j       \\ \hline
\end{tabular}
\end{table}

Тета-решетка -- это способ задать соответствие между аргументами и элементами синтаксической структуры. В GB формулируется условие, что это соответствие должно быть одно-однозначным (т.н. тета-критерий). 

\begin{exe}
  \ex[*]{Мальчик читает книгу газету} \label{pred:atomic:ex2}
\end{exe}

Пример (\ref{pred:atomic:ex2}) нарушает $\theta$-критерий и потому неграмматичен.

\section{Система типов в GL}

Таксономия типов: естественные, искусственные и сложные типы. Их определение через qualia. Естественные содержат только формальные и материальные качества; искусстенные содержат материальные качества; сложные содержат информацию об отношениях между типами. Qualia unification.\\

Система типов в GL. Структура наследования типов и qualia как система ограничений на типы структур признаков. Три конструктора типов (стрелка, точка, тензор). 

\section{Механизм композиции в ГЛ}

Операции, учитывающие типы, для реализации ``выбора'' (selection) значений требуют особых типов и особых механизмов композиции. Композициональность по Монтегю -- функциональное применение (реализуется как конкатенация лямбда-термов с последующей редукцией). Расширение композициональных механизмов у Пустейовского (так что получаются ``механизмы выбора''): а) pure selection (type matching), б) accomodation в) type coercion (два варианта -- exploitation и introduction). Примеры действия этих механизмов.

\section{Структура события}

Event structure. Внутренняя структура событий; события используются для задания ограничений на qualia; аспектуальность и лексическая семантика каузации (глава 9)

\section{Заключение}

Заключение. Отличительные особенности GL. Семантика синтаксическими средствами (внутри формального языка, конечно) в GL vs. теоретико-модельная семантика (Монтегю и пр.). Динамический характер лексикона в GL vs. статический лексикон как в дистрибутивной семантике, так и в CCG и других лексикализованных концепциях. GL и DPL (GL как динамическая семантика)\\

\hrulefill

Ossetic has a construction where the coordinating conjunction \textit{3m3} `and' is used to introduce a complement clause:

\begin{exe} 
 \ex\label{ex:compl}
 \gll 3\v{z} 3nq3l d3n \textbf{3m3} [ d@ =m3 A-\v sAjd-t-Aj ]\\
 1Sg.\textsc{nom} (think) be.\textsc{prs.1sg} and {} 2Sg.\textsc{nom} 1SgEncl.\textsc{acc} \textsc{pv}-cheat-\textsc{tr-pst.2sg} {}\\
 \trans `I think that you've cheated me'
\end{exe}

The construction in (\ref{ex:compl}) is c-coordinate but f-subordinate, and thus has the following c- and f-structures:

\Tree [.S [.S \edge[roof]; 3 z 3nq3l d3n ] [.Cnj \textbf{3m3} ] [.S \edge[roof]; d@ =m3 A ] ] 




\begin{prooftree}
  \AxiomC{$\Gamma, B, \Gamma' \vdash C$}
  \AxiomC{$\Delta \vdash A$}
  \RightLabel{{\small $\textbackslash_h$}}
  \BinaryInfC{$\Gamma, A, A \textbackslash B, \Gamma' \vdash C$}
\end{prooftree}

\begin{prooftree}
  \AxiomC{$A, \Gamma \vdash C$}
  \RightLabel{{\small $\textbackslash_i, \Gamma \neq \epsilon$}}
  \UnaryInfC{$\Gamma \vdash A \textbackslash C$}
\end{prooftree}



\nocite{*}
\printbibliography[resetnumbers=true]

\end{document}