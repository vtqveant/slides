\documentclass[10pt]{article} 

\usepackage[T2A]{fontenc}
\usepackage[utf8]{inputenc}
\usepackage[english,russian]{babel}
\usepackage{amssymb,amsfonts,amsmath,mathtext}

\usepackage[a4paper]{geometry} % задание полей для документа
\geometry{verbose,tmargin=2cm,bmargin=2cm,lmargin=2cm,rmargin=2cm}

\usepackage{authblk}
\renewcommand\Affilfont{\itshape\small}

\sloppy
\clubpenalty=10000
\widowpenalty=10000

%%% linguistic stuff

\usepackage{tipa} % пакет для ввода символов МФА

%% Gentzen style natural deduction proof trees
\usepackage{bussproofs}
\usepackage{latexsym}

\usepackage{tikz}
\usepackage{tikz-qtree,tikz-qtree-compat}   % regular trees (e.g. GB style)
\usepackage{tikz-dependency}                % dependency trees (bracket style)
\usetikzlibrary{matrix,arrows}              % for commutative diagrams

\usepackage{avm}
\avmsortfont{\footnotesize\it}
\avmvalfont{\rm}
\avmfont{\sc}

\usepackage{gb4e} % пакет для лингвистических примеров
\let\eachwordone=\tipaencoding % строка примеров - в МФА
\let\eachwordtwo=\small
\let\eachwordthree=\small

\begin{document} 

\title{Syntactic Annotation by Interactive Proof Search}
\author[]{Konstantin Sokolov}
\affil[]{Department of Mathematical Linguistics, St.~Petersburg State University, St.~Petersburg, Russia}
\date{}

\maketitle

\section{Introduction}
The properties that are considered to characterize a given construction as ``subordinate'' or ``coordinate'' do not always align as neatly as one would expect it from familiar cases.

\section{Some problems}
Ossetic has a construction where the coordinating conjunction \textit{\textipa{3m3}} `and' is used to introduce a complement clause:

\begin{exe} % глоссированный пример
 \ex\label{ex:compl}
 \gll 3\v{z} 3nq3l d3n \textbf{3m3} [ d@ =m3 A-\v sAjd-t-Aj ]\\
 1Sg.\textsc{nom} (think) be.\textsc{prs.1sg} and {} 2Sg.\textsc{nom} 1SgEncl.\textsc{acc} \textsc{pv}-cheat-\textsc{tr-pst.2sg} {}\\
 \trans `I think that you've cheated me'
\end{exe}

\section{Analysis}
The construction in (\ref{ex:compl}) is c-coordinate but f-subordinate, and thus has the following c- and f-structures:

\Tree [.S [.S \edge[roof]; \textipa{3\v z 3nq3l d3n} ] [.Cnj \textbf{\textipa{3m3}} ] [.S \edge[roof]; \textipa{d@ =m3 A-\v sAjd-t-Aj} ] ] 

\begin{avm} 
\[\asort{}
  conj & \textsc{and0} \\
  pred & `\textsc{think}<($\uparrow$ \textsc{subj})($\uparrow$ \textsc{comp})>' \\
  subj & \[\asort{} pred & `\textsc{pro}'\] \\
  comp & \[\asort{}
          pred & `\textsc{cheat}<($\uparrow$ \textsc{subj})($\uparrow$ \textsc{obj})>' \\
          subj & \[{} pred & `\textsc{pro}'\] \\
          obj & \[{} pred & `\textsc{pro}'\] \\
         \] \\
\]
\end{avm}

\end{document}