\documentclass{beamer}

\usepackage[T2A]{fontenc}
\usepackage[utf8]{inputenc}
\usepackage[english,russian]{babel}
\usepackage{amssymb,amsfonts,amsmath,mathtext}
\usepackage{cite,enumerate,float,indentfirst}

\usepackage{graphicx}
\usepackage{booktabs}
\usepackage{tabularx}

% Attribute-Value Matrices
\usepackage{avm}
\avmfont{\sc}
\avmoptions{sorted,active}
\avmvalfont{\rm}
\avmsortfont{\scriptsize\it}

% CCG parse trees
\newcommand{\deriv}[2]
{  \renewcommand{\arraystretch}{.5}
$\begin{array}[t]{*{#1}{c}}
     #2
   \end{array}$ }
\newcommand{\gf}[1]{\textsf{\textsl{#1}}}
\newcommand{\cf}[1]{\mbox{\ensuremath{\cfont{#1}}}}
\newcommand{\uline}[1]
{\mc{#1}{\hrulefill} }
\newcommand{\mc}[2]
  {\multicolumn{#1}{c}{#2}}
\newcommand{\cfont}{\mathsf}
\newcommand{\bs}{\backslash}
\newcommand{\subsa}[1]{\hspace{-0.75mm}_{_{#1}}}
\newcommand{\subsb}[1]{\hspace{-0.10mm}_{_{#1}}}
\newcommand{\subs}[1]{\hspace{-0.40mm}_{#1}}
\newcommand{\subsf}[1]{\hspace{-0.75mm}_{_{#1}}}
\newcommand{\supsa}[1]{\hspace{-1.75mm}^{^{#1}} }
\newcommand{\supsb}[1]{\hspace{-0.80mm}^{^{#1}}  }
\newcommand{\sups}[1]{\hspace{-0.40mm}^{#1}}


\graphicspath{{images/}}

\usetheme{Pittsburgh}
\usecolortheme{whale}

\setbeamertemplate{footline}{\scriptsize{\hspace*{0.4cm}\insertframenumber}\vspace*{0.3cm}}
\beamertemplatenavigationsymbolsempty

\errorcontextlines 10000

\begin{document}
\title{\Large{Генеративный лексикон Дж. Пустейовского}}
\author{Константин Соколов}
\institute[]
{СПбГУ, ИИТУ СПбПУ\\ \bigskip  \url{http://nlu-rg.ru}}
\date{Санкт-Петербург, 2015} 
% Создание заглавной страницы
\begin{frame}
    \thispagestyle{empty}
    \titlepage
\end{frame}

% 1. Бэкграунд 
% унификационные грамматики, HPSG, LFG
% основы формализма -- типизированные структуры признаков
% Логическое описание структур признаков (Каспер-Раундс, Джонсон, гибридная логика)
% Графовое представление структур признаков, reentrancy
% subsumption relation и решетка типов; связь с системами типов с подтипами (subtyping coersion)
% 
% 2. Ещё раз о требованиях к семантическому представлению
% а) возможность сочетания семантики предложения и 
% 
%
% 3. Решение Пустейовского
% семантическое представление как типизированный объект (структура признаков) особой структуры
% четыре компонента семантического представления -- argument structure, qualia structure etc.
% 
%
\begin{frame}{План}
    \setcounter{framenumber}{1}
    \begin{itemize}
        \item Типизированные структуры признаков (TFS)
        \item 
    \end{itemize}
\end{frame}

% 1. Структуры типизированных признаков (TFS)
\begin{frame}{}
\begin{center}
	\textbf{Структуры типизированных признаков}
\end{center}
\end{frame}

\begin{frame}[fragile]
\frametitle{Структуры типизированных признаков}
Матрица ``атрибут-значение'' (AVM):\\
\begin{center}
	\begin{avm}
	[{action} predicate & on \cr
    	      Mood & imp \cr 
        	  Actor & @{1} \cr 
	          Patient & [{thing} predicate & @{2} лампа \cr
    	                         Num & sg \cr 
        	                     Modifier & [{q-color} predicate & красный\_adj ]]]
	\end{avm}
\end{center}	
\end{frame}

\begin{frame}{Формальное определение (Copestake, 2001)}
Зададим конечное множество признаков $\textbf{Feat}$ и иерархию типов $\langle \textbf{Type}, \sqsubseteq \rangle$ (частично упорядоченное множество)\\
\bigskip
TFS - это четверка $\langle Q, r, \delta, \theta \rangle$, где
\begin{itemize}
	\item $Q$ - конечное множество вершин
	\item $r \in Q$ - выделенная вершина
	\item $\theta : Q \to \textbf{Type}$ - частичная функция типизации
	\item $\delta : Q \times \textbf{Feat} \to Q$ - частичная функция, сопоставляющая признаку при вершине другую вершину
\end{itemize}
\bigskip
{\footnotesize \textit{...и некоторые ограничения}}
\end{frame}

% 3. Графовое представление TFS как шкала Крипке
\begin{frame}{}
\begin{center}
	\textbf{Графовое представление TFS как шкала Крипке}
\end{center}
\end{frame}

\begin{frame}{Графовое представление}
\begin{center}
	\begin{figure}[H]
		\includegraphics[scale=0.5]{tfs_avm.png} 
		\includegraphics[scale=0.5]{tfs_graph.png} 
	\end{figure}
\end{center}
\end{frame}


\begin{frame}{Реляционная семантика (1)}
\textit{Шкала Крипке} (Kripke frame):\\
\bigskip
$\mathcal{F} = (W, R)$, где $W$ - непустое множество, $R \subseteq W \times W$.\\
\bigskip
\begin{itemize}
  \item $w_i \in W$ - миры, состояния, точки отнесенности
  \item $R$ - отношение достижимости
  \item если $wRv$, то говорят, что $v$ \textit{возможен относительно} $w$
\end{itemize}
\end{frame}

\begin{frame}{Реляционная семантика (2)}
\textit{Модель Крипке} $\mathcal{M} = (\mathcal{F}, V)$, где  
\bigskip
\begin{itemize}
  \item $\mathcal{F}$ - шкала Крипке
  \item $V : P \to 2^W$ - функция оценивания, т.е. отображение из атомарных выражений в подмножества множества миров
\end{itemize}
\bigskip
\end{frame}

\begin{frame}{Реляционная семантика (3)}
Пусть $\mathcal{M} = (\mathcal{F}, V)$, $w \in W$, $\phi$ - формула, $p \in P$\\
\bigskip
Истинность $\phi$ в модели $\mathcal{M}$ в точке $w$ определяется рекурсивно\\
\bigskip
\begin{itemize}
  \item $\mathcal{M}, w \models p \; \Longleftrightarrow \; w \in V(p)$
  \item $\mathcal{M}, w \models \neg \phi \; \Longleftrightarrow \; \mathcal{M}, w \not\models \phi$
  \item $\mathcal{M}, w \models \phi \vee \psi \; \Longleftrightarrow \; \mathcal{M}, w \models \phi$ или $\mathcal{M}, w \models \psi$
  \item $\mathcal{M}, w \models \Box \phi \; \Longleftrightarrow \; \forall v \in W \; . \; (w R v \to \mathcal{M}, v \models \phi)$
  \item $\mathcal{M}, w \not\models \perp$
\end{itemize}
\end{frame}



% 2. Логическое описание TFS (языки $L^{KR}$, $HL(@)$)
\begin{frame}{}
\begin{center}
	\textbf{Логическое описание TFS}
\end{center}
\end{frame}

\begin{frame}{Логическое описание TFS}
\begin{itemize}
	\item логика Каспера-Раундса $L^{KR}$
	\item гибридная логика $HL(@)$
\end{itemize}
\end{frame}

\begin{frame}{$L^{KR}$ (1)}
Язык $L^{KR}$ с сигнатурой $\langle L, A \rangle$, где $L$ - множество признаков, \\$A$ - множество пропозициональных переменных:\\
\bigskip
\begin{itemize}
	\item $a$ для каждого $a \in A$
	\item пропозициональные связки и $\top$
	\item знак равенства путей $\approx$
\end{itemize}
\bigskip	
Eсли $\phi$ и $\psi$ - формулы, то также формулы и 
\begin{itemize}
	\item $\phi \wedge \psi$,
	\item $\phi \vee \psi$,
	\item $l : \phi$ или $\langle l \rangle \phi$ для $l \in L$
\end{itemize}
\end{frame}

\begin{frame}{$L^{KR}$ (2)}
Уравнения по путям (цепочкам модальных операторов):\\
\bigskip
\begin{itemize}
	\item $\langle VP \; VERB \; HEAD \; NUM \rangle \; sing$
	\item $\langle VP \; HEAD \rangle \approx \langle VP \; VERB \; HEAD \rangle$
	\item $0$ - путь нулевой длины
\end{itemize}
\bigskip
Пример: $\neg(0 \approx \langle F_1 \rangle ... \langle F_n \rangle)$ - условие ацикличности
\end{frame}

\begin{frame}{$HL(@)$ (1)}
Гибридная логика:\\
\bigskip
\begin{itemize}
	\item Будем писать $\langle \pi \rangle$ и $[\pi]$ вместо $\Diamond_\pi$ и $\Box_\pi$
	\item $\langle \pi \rangle \alpha \; \equiv \; \neg [\pi] \neg \alpha$
	\item Введем дополнительно класс \textit{номиналов} (обозн. $i, j, k$)
	\item Введем оператор $@_i$ со значением ``истинно в точке $i$''
\end{itemize}
\bigskip
Язык гибридной логики $HL(@)$:\\
\begin{center}
$WFF_{HL(@)} := \top \; \vert \; i \; \vert \; p \; \vert \; \neg \alpha \; | \; \alpha \wedge \beta \; \vert \; \langle \pi \rangle \alpha \; \vert \; @_i \alpha$
\end{center}
\end{frame}

\begin{frame}{$HL(@)$ (2)}
Гибридная модель Крипке:\\
\bigskip
\begin{center}
$\mathcal{M} = (W, \{R_\pi \vert \pi \in MOD\}, V)$, где\\
\bigskip
\begin{itemize}
	\item $\mathcal{F} = (W, \{R_\pi \vert \pi \in MOD\})$ - шкала Крипке
	\item $V : PROP \cup NOM \to 2^W$
	\item $V(i)$ - синглетон
\end{itemize}
\end{center}
\end{frame}

\begin{frame}{$HL(@)$ (3)}
Денотационная семантика для $HL(@)$:\\
\bigskip
\begin{itemize}
	\item $\mathcal{M}, w \models i \; \Leftrightarrow \; w = V(i)$
	\item $\mathcal{M}, w \models @_i \alpha \; \Leftrightarrow \; \mathcal{M}, w' \models \alpha$ и $w' = V(i)$
\end{itemize}
\end{frame}


\begin{frame}{Гибридная логика и проверка моделей}
Дана гибридная модель $\mathcal{M}$ и формула $\alpha$, найти все узлы в $\mathcal{M}$, в которых $\alpha$ истинна:
\bigskip
\begin{center}
    $T(\mathcal{M}, \alpha) = \{ w \in W \; | \; \mathcal{M}, w \models \alpha \}$
\end{center}
\end{frame}


% 4. Конструкция накрытия графа
\begin{frame}{}
\begin{center}
	\textbf{Конструкция накрытия графа}
\end{center}
\end{frame}

% Let G = (V, E) and C = (V2, E2) be two graphs, and let f: V2 → V be a surjection. Then f is a covering map from C to G if for each v ∈ V2, the restriction of f to the neighbourhood of v is a bijection onto the neighbourhood of f(v) ∈ V in G. Put otherwise, f maps edges incident to v one-to-one onto edges incident to f(v).
\begin{frame}{Накрытие графа (3)}
$G_1 = (V_1, E_1)$, $G_2 = (V_2, E_2)$ - два графа, $\pi: V_2 \to V_1$ - сюръекция. Если ограничение $\pi \vert_U$ на окрестность вершины $v \in U \subset V_2$ биективно, то $\pi$ - накрытие.\\
\bigskip
Иначе, накрытие - локально биективный гомоморфизм.
\end{frame}


% 5. Логическая характеристика свойства локальности
\begin{frame}{}
\begin{center}
	\textbf{Логическая характеристика свойства локальности}
\end{center}
\end{frame}

\begin{frame}{Локальность (1)}
\begin{center}
\textbf{Задача}\\
\bigskip
Выразить локальные свойства модели \\в описывающем её формализме
\end{center}
\end{frame}

\begin{frame}{Локальность (2)}
Основное соотношение конструкции:\\
\bigskip
\begin{center}
$[ \! [ \pi^{-1}(\phi) ] \! ]^{M_1, g} = [ \! [ \psi ] \! ]^{M_2, g}$, где
\end{center}
\bigskip
\begin{itemize}
	\item $\pi : M_2 \to M_1$ - накрытие,
	\item $\phi$ и $\psi$ - формулы языка гибридной логики $HL(@)$,
	\item $g$ - функция означивания,
	\item домены моделей $M_1$ и $M_2$ совпадают.
\end{itemize}
\bigskip
Тогда выражение $\pi^{-1}(\phi)$ назовем \textit{локализацией} формулы $\phi$ при накрытии $\pi$.
\end{frame}


% 6. Лингвистические приложения конструкции  % (HLDS, underspecification и др.)
\begin{frame}{}
\begin{center}
	\textbf{Лингвистические приложения конструкции}
\end{center}
\end{frame}

\begin{frame}{HLDS (1)}
Hybrid Logic Dependency Semantics:\\
\bigskip
\begin{itemize}
	\item Композициональный семантический формализм 
	\item Описание семантических структур зависимостей с помощью выражений гибридной логики
	\item Cемантическая композиция реализуется как унификация логических форм (ср. с формализмами на основе $\lambda$-исчисления: конкатенация с последующей редукцией)
	\item Реализован в системе OpenCCG (Baldridge et al., 2007)
\end{itemize}
\end{frame}

\begin{frame}{HLDS (2)}
\begin{center}
$@_X\phi$
\end{center}
\bigskip
\begin{itemize}
	\item $X$ интерпретируется как референциальная переменная
	\item $\phi$ - высказывание
	\item $\langle \cdot \rangle$ - модальный оператор, выражающий отношение семантической зависимости
\end{itemize}	
\end{frame}

\begin{frame}{HLDS (3)}
Пример словарной записи:\\
\bigskip
\begin{center}
$flower \vdash n_{sg,X:thing} : @_{X:thing}(\textbf{flower} \wedge \langle NUM \rangle sg)$
\end{center}
\bigskip
\begin{itemize}
	\item $n_{sg,X:thing}$ - синтаксическая категория в MMCCG
	\item $@_{X:thing}(\textbf{flower} \wedge \langle NUM \rangle sg)$ - логическая форма в HLDS
\end{itemize}	
\end{frame}


\begin{frame}[fragile]
\frametitle{HLDS (4)}
Компактная форма:
{\footnotesize \begin{verbatim}
@w0:action(ON ^
           <Mood>imp ^
           <Actor>x1:entity ^
           <Patient>(w2:thing ^ лампа ^
                     <Num>sg ^
                     <Modifier>(w1:q-color ^ красный-adj) ^
                     <Modifier>(w3:m-location ^ на ^
                                <Anchor>(w4:e-place ^ кухня ^
                                         <Num>sg))))
\end{verbatim}}
\end{frame}

\begin{frame}[fragile]
\frametitle{HLDS (5)}
Линеаризованная форма:
{\footnotesize \begin{verbatim}
@E_0:action(CLOSE) ^ 
@E_0:action(<Mood>imp) ^
@E_0:action(<Actor>S_0:entity) ^ 
@E_0:action(<Patient>T_1:thing) ^ 
@M_3:m-location(в) ^ 
@M_3:m-location(<Anchor>T_4:e-place) ^ 
@T_1:thing(шторы) ^ 
@T_1:thing(<Num>pl) ^ 
@T_1:thing(<Modifier>M_3:m-location) ^ 
@T_4:e-place(прихожая) ^ 
@T_4:e-place(<Num>sg))
\end{verbatim}}
\end{frame}

\begin{frame}[fragile]
\frametitle{DotCCG (1)}
Определение семейства слов:
{\footnotesize \begin{verbatim}
family tv(V) {
    entry : s[E] \! np[S] / np[X] : E:event(* <Actor>  (S:entity) 
                                              <Patient>(X:entity));
}
\end{verbatim}}
\bigskip
\bigskip
Запись в словарной части:
{\footnotesize \begin{verbatim}
word включать:tv(action, pred=ON) {
    включи: imp vf-to-imp;
}
\end{verbatim}}
\end{frame}

\begin{frame}[fragile]
\frametitle{DotCCG (2)}
Правила изменения типа:
{\footnotesize \begin{verbatim}
rule {
    typechange: s<10> [E NUM PERS MOOD POL FIN VFORM vf-to-imp] \! 
                np<9> [S nom NUM PERS nf-real] / 
                np<2> [X acc]
             => s<~10>[E fin-full s-imp] / 
                np<2> [X acc] : E:event(<Mood>(imp) 
                                        <Subject>(S:entity 
                                                  addressee));
}
\end{verbatim}}
\end{frame}

\begin{frame}{Локализация в HLDS (1)}
\begin{itemize}
	\item Необходимость контроля области видимости референциальных переменных в процессе семантической композиции
	\item Необходимость реализации порождения нескольких семантических представлений в результате одного синтаксического разбора (ср. \textit{scope ambiguity})
\end{itemize}
\end{frame}

\begin{frame}{Локализация в HLDS (2)}
Неоднозначность:\\
\bigskip
\begin{itemize}
	\item \texttt{включи лампу на столе и свет в комнате}
		\begin{itemize}
			\item {\footnotesize лампа на столе, стол не в комнате}
			\item {\footnotesize лампа на столе в комнате}
		\end{itemize}
	\bigskip
	\item \texttt{включи лампу на кухне и свет в комнате}
		\begin{itemize}
			\item {\footnotesize лампа на кухне}
			\item {\footnotesize $^?$лампа на кухне в комнате}
		\end{itemize}
	\bigskip
	\item \texttt{включи свет в комнате и лампу на столе}
		\begin{itemize}
			\item {\footnotesize свет в комнате}
			\item {\footnotesize $^{??}$свет в комнате на столе}
		\end{itemize}
\end{itemize}
\end{frame}

\begin{frame}[fragile]
\frametitle{Локализация в HLDS (3)}
\texttt{включи [[[лампу на кухне] и подсветку] в прихожей]}\\
\begin{center}
{\scriptsize \begin{verbatim}
@w0:action(ON ^ 
              <Mood>imp ^ 
              <Actor>x1:entity ^ 
              <Patient>(w4:entity ^ и ^ 
                        <Num>pl ^ 
                        <First>(w1:thing ^ лампа ^ 
                                <Num>sg ^ 
                                <Modifier>(w2:m-location ^ на ^ 
                                           <Anchor>(w3:e-place ^ stol ^ 
                                                    <Num>sg))) ^ 
                        <Modifier>(w6:m-location ^ в ^ 
                                   <Anchor>(w7:e-place ^ прихожая ^ 
                                            <Num>sg)) ^ 
                        <Next>(w5:thing ^ подсветка ^ 
                               <Num>sg) ^ 
                        <Num>pl))
\end{verbatim}
}                        
\end{center}
\end{frame}


\begin{frame}{Локализация в HLDS (4)}
Синтаксическое управление процедурой семантической композиции приводит к появлению некорректной логической формы.\\
\bigskip
Варианты решения:\\
\bigskip
\begin{itemize}
	\item пометки в грамматике (внутри синтаксических категорий)
	\item проверка модели (model checking) и внешняя онтология
	\item препятствия на уровне согласования логических форм
\end{itemize}
\end{frame}

\begin{frame}{Локализация в HLDS (5)}
Реализация ограничений на унификацию логических форм в виде накрытия как вариант неполной специфицикации (\textit{underspecification}) для HLDS.\\
\bigskip
\begin{itemize}
	\item устранение вычислительно трудных решений по схеме ``порождение гипотез и фильтрация''
	\item устранение необходимости добавления ad hoc правил в синтаксический компонент
	\item дополнительный механизм наряду с типизацией (типизация касается вершин, локализация - ребер)
	\item возможность адаптации к дискурсу
\end{itemize}
\end{frame}


\begin{frame}{}
    \thispagestyle{empty}
    \begin{center}
        {\large Спасибо!}
    \end{center}
\end{frame}


%%% слайд помещается сюда
%% \begin{frame}{Заголовок}
%% \end{frame}

\end{document}
