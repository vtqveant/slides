\documentclass{beamer}

\usepackage[T2A]{fontenc}
\usepackage[utf8]{inputenc}
\usepackage[english,russian]{babel}
\usepackage{amssymb,amsfonts,amsmath,mathtext}
\usepackage{cite,enumerate,float,indentfirst}

%% Gentzen style natural deduction proof trees
\usepackage{bussproofs}
\usepackage{latexsym}

\graphicspath{{images/}}

\usetheme{Pittsburgh}
\usecolortheme{whale}

\setbeamertemplate{footline}{\scriptsize{\hspace*{0.4cm}\insertframenumber}\vspace*{0.3cm}}
\beamertemplatenavigationsymbolsempty

\errorcontextlines 10000

\begin{document}
\title{\huge{$\mathbb{NLU}/RG$, \textit{pt. 8}}}
\author{Константин Соколов}
\institute[]
{Mathlingvo, СПбГУ, i-Free\\ \bigskip  \url{http://nlu-rg.ru}
}
\date{Санкт-Петербург, 2013} 
% Создание заглавной страницы
\begin{frame}
    \thispagestyle{empty}
    \titlepage
\end{frame}

%%% 0. План
\begin{frame}{План}
    \setcounter{framenumber}{1}
    \begin{itemize}
        \item Монтегю, универсальная грамматика и PTQ
        \item Compositional Distributional Semantic Models
    \end{itemize}
\end{frame}

%% Старые заметки про М.
%% 1. 
%%  а) Об универсальном кванторе у Айдукевича
%%  б) Теория обобщенных кванторов: Мостовский, Линдстрём, Монтегю
%%  в) Семантика Генкина, в частности, домены/фреймы и пр. Интересный пример: К-схема $(\forall(\phi \to \psi) \to (\forall \phi \to \forall \psi)$
%%  what the schema actually *says* is that the universal quntifier is preserved by supersets, a form of monotonicity that might in principle pertain to other quantifiers as well.

%% 2. Семантика в PTQ - это реляционная семантика для (мульти)модальной логики, а не семантика для типизированного лямбда-исчисления. Почему?
%% Типизированное лямбда-исчисление в стиле (Чёрч 1940) используется как синтаксический формализм (ср. Гурасимова)

%% 3. Понятие алгебры множеств и универсальной алгебры. Алгебраическая структура в теории моделей (для языка с сигнатурой, содержащей функциональные символы, но не содержащая реляционных символов). Гомоморфизм алгебр в UG. Нетривиальность соображеиня, что синтаксические категории t/e могут быть сопоставлены с типами <e,t>. Это сопоставление *индуцируется* гомоморфизмом алгебр, т.к. алгебра определена через операции (F_0 и пр.), а типы порождаются из базовых типов, то первичной процедурой здесь является гомоморфизм, т.е. отображение, сохраняющее "композициональную" структуру. В свою очередь "образ" синтаксической структуры приводит к появлению семантических типов. Схожесть нотации  может мысль, что первично соответствие межу синтаксическими категориями и семантическими типами, но это не так; первичен гомоморфизм алгебраических структур, сохраняющий композициональность. Поскольку это гомоморфизм, а не изоморфизм, иерархии типов и синтаксических категорий не обязаны совпадать.

%% 4. Интересовал ли Монтегю вопрос о рекурсивных функциях в определении иерархии типов, функции с неподвижной точкой и пр., комбинатор неподвижной точки (т.е. проблемы, которые вынужден был решать Генкин).

%% 5. Знал ли М. о соответствии Хиндли-Милнера?

%% 6. Насколько общими могут быть предикаты в MG? (Типа loves_a_lot(x) vs. loves(x)) Почему это не проблема метаязыка? Предикаты выступают здесь как labels для множеств, в этом смысле они не выражают сами по себе никакого смысла/значения, их смысл/значение должен вычисляться. Поэтому здесь нет необходимости организовывать "словарь" метаязыка на манер Вежбицкой, по идее, предикаты здесь можно было бы вообще просто занумеровать.

%% 7. Смысл как интенсионал, интенсионал как нечто, чему может быть приписано значение *только* со ссылкой на point of reference. Отсюда включение пары $(i, j) \in \mathcal{I} \times \mathcal{J}$ в число аргументов интенсионального предиката.

\begin{frame}{Монтегю, Универсальная грамматика (1)}
\begin{itemize}
  \item Universal Grammar, 1970
  \item \textit{универсальная грамматика} у Монтегю - \\
  грамматика, сформулированная в максимальной общности\\
  (ср. ``универсальная алгебра'' )
  \item \textit{универсальная грамматика} у Хомского - \\
  грамматика ``человеческого языка''
\end{itemize}
\end{frame}

\begin{frame}{Монтегю, Универсальная грамматика (2)}
Пример алгебры:\\
\bigskip
Пусть $X$ - множество, $2^X$ - множество подмножеств $X$.\\
\textit{Алгебра множеств} - это система подмножеств $\mathcal{A} \subseteq 2^X$,\\ таких что:\\
\begin{itemize}
  \item $\varnothing \in \mathcal{A}$
  \item если $A \in \mathcal{A}$, то $X \! \setminus \! A \in \mathcal{A}$
  \item если $A \in \mathcal{A}$ и $B \in \mathcal{A}$, то $A \cup B \in \mathcal{A}$
\end{itemize}
\end{frame}

\begin{frame}{Монтегю, Универсальная грамматика (3)}
\textit{Сигнатура} или \textit{тип подобия} - это тройка $\sigma = (S_{func}, S_{rel}, ar)$\\
\bigskip
\begin{itemize}
  \item $S_{func}$ - набор функциональных символов (напр.: $+, \times, 0, 1$)
  \item $S_{rel}$ - множество реляционных символов (напр.: $\in, \leq$)
  \item $ar$ - функция арности: $S_{func} \cup S_{rel} \to \mathbb{N}$
\end{itemize}
\bigskip
Структура - это тройка $\mathcal{A} = (A, \sigma, I)$, где $A$ - домен (носитель), $\sigma$ - сигнатура, $I$ - функция интерпретации сигнатуры в домене.\\
\bigskip
Структура, определяемая языком с сигнатурой, содержащей только функциональные символы, называется \textit{алгеброй}.
\end{frame}

\begin{frame}{Монтегю, Универсальная грамматика (4)}
Иными словами, на множестве $X$ можно ввести структуру алгебры, определив на нем одну или несколько операций.\\
\bigskip
Например, структурные операции для алгебры множеств:
\begin{itemize}
  \item дополнение $F_1 : 2^X \to 2^X$
  \item объединение $F_2 : 2^X \times 2^X \to 2^X$ 
\end{itemize}
\end{frame}

\begin{frame}{Монтегю, Универсальная грамматика (5)}
\begin{itemize}
  \item Р. Монтегю определяет формальный язык, порождающую строки (\textit{собственные выражения}, \textit{proper expressions}),
  \item снабжает множество собственных выражений структурой алгебры путем задания \textit{структурных операций} $F_\gamma$,
  \item определяет множество \textit{осмысленных выражений} как ``правильно типизированное'' подмножество собственных выражений (следуя К. Айдукевичу),
  \item определяет интерпретацию как отображение из множества осмысленных выражений в множество значений, сопоставляющее структурным операциям формального языка операции семантической композиции\\ (т.е. \textit{гомоморфизм алгебр}).
\end{itemize}
\end{frame}

\begin{frame}{Монтегю, Универсальная грамматика (6)}
Формальный язык строится как типизированное исчисление.\\
\bigskip
\bigskip
Чтобы задать формальную систему, нужно:\\
\begin{itemize}
  \item задать набор символов и правила построения выражений
  \item определить правила вывода
  \item задать набор аксиом
  \item определить понятие выводимости
\end{itemize}
\end{frame}

\begin{frame}{Монтегю, Универсальная грамматика (7)}
Терминальные символы и выражения формального языка:\\
\begin{itemize}
  \item ограниченный лексикон, т.е. \textit{слова}
  \item структурные операции, напр. 
    \begin{itemize}
      \item $F_1(\alpha, \beta) = \alpha \beta$
      \item $F_2(\alpha) = every \; \alpha$
      \item $F_3(\alpha) = not \; \alpha$
    \end{itemize}
\end{itemize}
\end{frame}

\begin{frame}{Монтегю, Универсальная грамматика (8)}
``Аксиомы'':\\
\begin{itemize}
  \item определяются синтаксические категории (по Айдукевичу)
    \begin{itemize}
      \item элементарные: \texttt{e} \textit{(entities)}, \texttt{t} \textit{(truth values)}
      \item производные: \texttt{t/e}, \texttt{t/e/t/e} и т.п.
    \end{itemize}
  \item задаются наборы базовых выражений, относящихся к конкретным синтаксическим категориям.
\end{itemize}
\bigskip
У Айдукевича было \texttt{n} вместо \texttt{e}, \texttt{s} вместо \texttt{t}. Монтегю использует обозначение \texttt{t} для высказываний, денотат которых (по Фреге) - истинностное значение.
\end{frame}

\begin{frame}{Монтегю, Универсальная грамматика (9)}
Понятие ``выводимости'':\\
\begin{itemize}
  \item определяется рекурсивными правилами в соответствии со структурными операциями с учетом синтаксических категорий.
    \begin{itemize}
      \item $< \! F_\gamma, < \! \alpha_1, \alpha_2, ... , \alpha_n \! >, \epsilon \!>$, где $\alpha_i$ - типы аргументов, $\epsilon$ - тип результата структурной операции $F_\gamma$.
    \end{itemize}
\end{itemize}
\end{frame}

\begin{frame}{Монтегю, Универсальная грамматика (10)}
Цель Монтегю - построить интерпретацию формального языка, удовлетворяющую принципу композициональности.\\
\bigskip
Он осуществляет это путем построения гомоморфизма алгебр из ``алгебры синтаксиса'' в ``алгебру семантики''.\\
\bigskip
``Алгебра семантики'' ещё не построена, потому приходится идти окольным путем.
\end{frame}




\begin{frame}{CDSM (1)}
\begin{itemize}
  \item 
  \item 
\end{itemize}
\end{frame}

\begin{frame}{SemEval-2014, Track 1}
\begin{itemize}
  \item 
  \item 
\end{itemize}
\end{frame}

\begin{frame}{}
    \thispagestyle{empty}
    \begin{center}
        {\large Спасибо!}
    \end{center}
\end{frame}


%%% слайд помещается сюда
%% \begin{frame}{Заголовок}
%% \end{frame}

\end{document}
