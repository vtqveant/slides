\documentclass{beamer}

\usepackage[T2A]{fontenc}
\usepackage[utf8]{inputenc}
\usepackage[english,russian]{babel}
\usepackage{amssymb,amsfonts,amsmath,mathtext}
\usepackage{cite,enumerate,float,indentfirst}

%% Gentzen style natural deduction proof trees
\usepackage{bussproofs}
\usepackage{latexsym}

\graphicspath{{images/}}

\usetheme{Pittsburgh}
\usecolortheme{whale}

\setbeamertemplate{footline}{\scriptsize{\hspace*{0.4cm}\insertframenumber}\vspace*{0.3cm}}
\beamertemplatenavigationsymbolsempty

\errorcontextlines 10000

\begin{document}
\title{\huge{$\mathbb{NLU}/RG$, \textit{pt. 8}}}
\author{Константин Соколов}
\institute[]
{Mathlingvo, СПбГУ, i-Free\\ \bigskip  \url{http://nlu-rg.ru}
}
\date{Санкт-Петербург, 2013} 
% Создание заглавной страницы
\begin{frame}
    \thispagestyle{empty}
    \titlepage
\end{frame}

%%% 0. План
\begin{frame}{План}
    \setcounter{framenumber}{1}
    \begin{itemize}
        \item Монтегю, универсальная грамматика и PTQ
        \item Compositional Distributional Semantic Models
    \end{itemize}
\end{frame}

\begin{frame}{Монтегю, универсальная грамматика (1)}
\begin{itemize}
  \item 
  \item 
\end{itemize}
\end{frame}

%% Старые заметки про М.
%% 1. 
%%  а) Об универсальном кванторе у Айдукевича
%%  б) Теория обобщенных кванторов: Мостовский, Линдстрём, Монтегю
%%  в) Семантика Генкина, в частности, домены/фреймы и пр. Интересный пример: К-схема $(\forall(\phi \to \psi) \to (\forall \phi \to \forall \psi)$
%%  what the schema actually *says* is that the universal quntifier is preserved by supersets, a form of monotonicity that might in principle pertain to other quantifiers as well.

%% 2. Семантика в PTQ - это реляционная семантика для (мульти)модальной логики, а не семантика для типизированного лямбда-исчисления. Почему?
%% Типизированное лямбда-исчисление в стиле (Чёрч 1940) используется как синтаксический формализм (ср. Гурасимова)

%% 3. Понятие алгебры множеств и универсальной алгебры. Алгебраическая структура в теории моделей (для языка с сигнатурой, содержащей функциональные символы, но не содержащая реляционных символов). Гомоморфизм алгебр в UG. Нетривиальность соображеиня, что синтаксические категории t/e могут быть сопоставлены с типами <e,t>. Это сопоставление *индуцируется* гомоморфизмом алгебр, т.к. алгебра определена через операции (F_0 и пр.), а типы порождаются из базовых типов, то первичной процедурой здесь является гомоморфизм, т.е. отображение, сохраняющее "композициональную" структуру. В свою очередь "образ" синтаксической структуры приводит к появлению семантических типов. Схожесть нотации  может мысль, что первично соответствие межу синтаксическими категориями и семантическими типами, но это не так; первичен гомоморфизм алгебраических структур, сохраняющий композициональность. Поскольку это гомоморфизм, а не изоморфизм, иерархии типов и синтаксических категорий не обязаны совпадать.

%% 4. Интересовал ли Монтегю вопрос о рекурсивных функциях в определении иерархии типов, функции с неподвижной точкой и пр., комбинатор неподвижной точки (т.е. проблемы, которые вынужден был решать Генкин).

%% 5. Знал ли М. о соответствии Хиндли-Милнера?

%% 6. Насколько общими могут быть предикаты в MG? (Типа loves_a_lot(x) vs. loves(x)) Почему это не проблема метаязыка? Предикаты выступают здесь как labels для множеств, в этом смысле они не выражают сами по себе никакого смысла/значения, их смысл/значение должен вычисляться. Поэтому здесь нет необходимости организовывать "словарь" метаязыка на манер Вежбицкой, по идее, предикаты здесь можно было бы вообще просто занумеровать.

%% 7. Смысл как интенсионал, интенсионал как нечто, чему может быть приписано значение *только* со ссылкой на point of reference. Отсюда включение пары $(i, j) \in \mathcal{I} \times \mathcal{J}$ в число аргументов интенсионального предиката.

\begin{frame}{Монтегю, PTQ (1)}
\begin{itemize}
  \item 
  \item 
\end{itemize}
\end{frame}

\begin{frame}{CDSM (1)}
\begin{itemize}
  \item 
  \item 
\end{itemize}
\end{frame}

\begin{frame}{SemEval-2014, Track 1}
\begin{itemize}
  \item 
  \item 
\end{itemize}
\end{frame}

\begin{frame}{}
    \thispagestyle{empty}
    \begin{center}
        {\large Спасибо!}
    \end{center}
\end{frame}


%%% слайд помещается сюда
%% \begin{frame}{Заголовок}
%% \end{frame}

\end{document}
