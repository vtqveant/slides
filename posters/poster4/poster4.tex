\documentclass[12pt]{article}
\pagestyle{empty}

%%% Кодировки и шрифты %%%
\usepackage{cmap} % Улучшенный поиск русских слов в полученном pdf-файле
\usepackage[T2A]{fontenc} % Поддержка русских букв
\usepackage[utf8]{inputenc} % Кодировка utf8
\usepackage[english, russian]{babel} % Языки: русский, английский

%% графика
\ifx\pdfoutput\undefined
\usepackage{graphicx}
\else
\usepackage[pdftex]{graphicx}
\fi
\usepackage{wrapfig}
\usepackage{float}

\usepackage{setspace}

\usepackage{tabularx}
\usepackage{bigstrut}  % row spacing in tables

\textheight 26cm
\textwidth 16.5cm

\usepackage[margin=2.1cm]{geometry}  % exact margins

% fancy separator, http://altermundus.com/pages/tkz/ornament/index.html
\usepackage[object=vectorian]{pgfornament} 

\newcommand{\sectionline}[1]{%
  \nointerlineskip \vspace{.45\baselineskip}\hspace{\fill}
  {\resizebox{0.5\linewidth}{1.0ex}
    {\pgfornament{#1}
    }}%
    \hspace{\fill}
    \par\nointerlineskip \vspace{.45\baselineskip}
  }


\begin{document}

%\baselineskip 15pt

\parindent 0mm


\begin{center}
{\rm \Huge \textsc{понимание естественного языка}}\\
\bigskip
{\rm \Large Natural Language Understanding Reading Group}\\
\bigskip
{\rm \Large \texttt{http://nlu-rg.ru}}
\end{center}

\bigskip
\medskip

\begin{spacing}{1.2}
\hfill
\parbox{0.986\textwidth}{
Осенний семестр 2015--2016 гг. посвящен теоретико-типовой лексической семантике.
}
\end{spacing}

\medskip

\begin{table}[h!]
\begin{tabularx}{\textwidth}{cX}
10.09 & Теория типов и лексическая семантика (обзор) \bigstrut[t] \\
      & {\scriptsize \textsc{?}}  \bigstrut[b] \\
24.09 & Генеративный лексикон -- I \bigstrut[t] \\
      & {\scriptsize \textsc{Pustejovsky, 1995. Generative Lexicon.}} \bigstrut[b] \\
08.10 & Генеративный лексикон -- II \bigstrut[t] \\
      & {\scriptsize \textsc{Pustejovsky, 2013. Type Theory and Lexical Decomposition.}} \bigstrut[b] \\
22.10 & Теоретико-типовая грамматика \bigstrut[t] \\
      & {\scriptsize \textsc{Ranta, 1994. Type-theoretical Grammar.}}     \bigstrut[b] \\
05.11 & Type Theory with Records (TTR) \bigstrut[t] \\
      & {\scriptsize \textsc{Cooper, 2005. Records and Record Types in Semantic Theory.}} \bigstrut[b] \\
19.11 & Type Composition Logic (TCL)  \bigstrut[t] \\
      & {\scriptsize \textsc{Asher 2011. Lexical Meaning in Context: A Web of Words.}} \bigstrut[b] \\
03.12 & Dependent Types Semantics (DTS)  \bigstrut[t] \\
      & {\scriptsize \textsc{Luo, 2011. Type-theoretical Semantics with Coercive Subtyping.}} \bigstrut[b] \\
17.12 & Montagovian Generative Lexicon (MGL) \bigstrut[t] \\ 
      & {\scriptsize \textsc{Retor\'{e}, 2013. A Type Theoretical Framework for Natural Language Semantics: the Montagovian Generative Lexicon.}} \bigstrut[b] \\ 
24.12 & Линейные типы \bigstrut[t] \\
      & {\scriptsize \textsc{Mery, 2015. Lexical Semantics with Linear Types.}} \bigstrut[b] \\      
\end{tabularx}
\end{table}

\sectionline{88}

\bigskip
\hfill
\begin{minipage}{0.41\textwidth}
\textsc{Время и место:}\\
\\
м. <<Площадь Мужества>>, \\ул. Политехническая, д. 21, \\9-й учебный корпус СПбГПУ,\\ ауд. 106, начало в 20:00
\end{minipage} 
\hfill
\begin{minipage}{0.4\textwidth}
\begin{figure}[H]
\includegraphics[width=6.3cm]{map.png} 
\end{figure}
\end{minipage}


\end{document}
