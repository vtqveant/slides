\documentclass[12pt]{article}
\pagestyle{empty}

%%% Кодировки и шрифты %%%
\usepackage{cmap} % Улучшенный поиск русских слов в полученном pdf-файле
\usepackage[T2A]{fontenc} % Поддержка русских букв
\usepackage[utf8]{inputenc} % Кодировка utf8
\usepackage[english, russian]{babel} % Языки: русский, английский

%% графика
\ifx\pdfoutput\undefined
\usepackage{graphicx}
\else
\usepackage[pdftex]{graphicx}
\fi
\usepackage{wrapfig}
\usepackage{float}

\usepackage{setspace}

\usepackage{tabularx}
\usepackage{bigstrut}  % row spacing in tables

\textheight 26cm
\textwidth 16.5cm

\usepackage[margin=2.1cm]{geometry}  % exact margins

% fancy separator, http://altermundus.com/pages/tkz/ornament/index.html
\usepackage[object=vectorian]{pgfornament} 

\newcommand{\sectionline}[1]{%
  \nointerlineskip \vspace{.45\baselineskip}\hspace{\fill}
  {\resizebox{0.5\linewidth}{1.0ex}
    {\pgfornament{#1}
    }}%
    \hspace{\fill}
    \par\nointerlineskip \vspace{.45\baselineskip}
  }


\begin{document}

%\baselineskip 15pt

\parindent 0cm


\begin{center}
{\rm \Huge \textsc{понимание естественного языка}}\\
\bigskip
{\rm \Large Natural Language Understanding Reading Group}\\
\bigskip
{\rm \Large \texttt{http://nlu-rg.ru}}
\end{center}

\bigskip
\medskip

%\baselineskip 15pt

\begin{spacing}{1.15}
Весенний семестр 2014--2015 гг. будет посвящен композициональным дистрибутивным семантическим моделям (compositional distributional semantic models).
\end{spacing}

\medskip

\begin{table}[h!]
\begin{tabularx}{\textwidth}{cX}
19.02 & Дистрибутивная семантика (обзор) \bigstrut[t] \\
      & {\scriptsize \textsc{Lenci, 2008. Distributional Semantics in Linguistic and Cognitive Research.}}  \bigstrut[b] \\
05.03 & Латентный семантический анализ  \bigstrut[t] \\
      & {\scriptsize \textsc{Landauer, Dumais, 1997. A Solution to Plato's problem: the latent semantic analysis theory of acquisition, induction and representation of knowledge.}}  \bigstrut[b] \\
19.03 & Аддитивные и мультипликативные методы в CDSM \bigstrut[t] \\
      & {\scriptsize \textsc{Mitchell, Lapata, 2010. Composition in Distributional Models of Semantics.}} \bigstrut[b] \\
02.04 & Функциональная композиция в CDSM \bigstrut[t] \\
      & {\scriptsize \textsc{Baroni et al., 2014. Frege in Space: A Program for Compositional Distributional Semantics.}}                \bigstrut[b] \\
16.04 & CDSM с помощью искусственных нейронных сетей \bigstrut[t] \\
      & {\scriptsize \textsc{Socher et al., 2013. Recursive Deep Models for Semantic Compositionality Over a Sentiment Treebank.}} \bigstrut[b] \\
30.04 & Полилинейная алгебра и тензорные представления \bigstrut[t] \\
      & {\scriptsize \textsc{Van de Cruys et al., 2013. A Tensor-Based Factorization Model of Semantic Compositionality.}}\\
      & {\scriptsize \textsc{Maillard et al., 2014. A Type-Driven Tensor-Based Semantics for CCG.}} \bigstrut[b] \\
14.05 & Pregroup Grammar и языки диаграмм для тензорного исчисления \bigstrut[t] \\
      & {\scriptsize \textsc{Lambek, 2001. Type Grammars as Pregroups.}}  \\
      & {\scriptsize \textsc{Joyal, Street, 1991. The Geometry of Tensor Calculus - I.}} \bigstrut[b] \\
28.05 & Теоретико-категорная модель CDSM \bigstrut[t] \\ 
      & {\scriptsize \textsc{Coecke et al., 2010. Mathematical Foundations for a Compositional Distributional Model of Meaning.}} \bigstrut[b] \\ 
\end{tabularx}
\end{table}

\sectionline{88}

\bigskip
\hfill
\begin{minipage}{0.41\textwidth}
\textsc{Время и место:}\\
\\
м. <<Площадь Мужества>>, \\ул. Политехническая, д. 21, \\9-й учебный корпус СПбГПУ,\\ ауд. 106, начало в 20:00
\end{minipage} 
\hfill
\begin{minipage}{0.4\textwidth}
\begin{figure}[H]
\includegraphics[width=6.3cm]{map.png} 
\end{figure}
\end{minipage}


\end{document}
