\documentclass[12pt]{article}
\pagestyle{empty}

%%% Кодировки и шрифты %%%
\usepackage{cmap} % Улучшенный поиск русских слов в полученном pdf-файле
\usepackage[T2A]{fontenc} % Поддержка русских букв
\usepackage[utf8]{inputenc} % Кодировка utf8
\usepackage[english, russian]{babel} % Языки: русский, английский

%% графика
\ifx\pdfoutput\undefined
\usepackage{graphicx}
\else
\usepackage[pdftex]{graphicx}
\fi
\usepackage{wrapfig}
\usepackage{float}

\usepackage{booktabs}

\textheight 26cm
\textwidth 16.5cm

\oddsidemargin 0.96cm   % margins on all sides of 3.5 cm,
\evensidemargin 0.96cm  % correcting 1 inch default on left-hand
\topmargin -1.61cm      % side, and 1.5 inch default on top.

\begin{document}

\baselineskip 24pt

\parindent 0cm


\begin{center}
{\rm \Huge{Понимание естественного языка}}\\
\medskip
{\rm \Large \textit{Исследовательский семинар}}\\
\medskip
{\rm \Large \texttt{http://nlu-rg.ru}}
\end{center}

\bigskip
\bigskip

\baselineskip 15pt

Весенний семестр 2014-2015 гг. посвящен композициональным дистрибутивным семантическим моделям (CDSM).\\

Встречи проходят раз в две недели по четвергам в 20:00 по адресу:\\ м. ``Площадь Мужества'', Политехническая ул., д. 21, 9-й учебный корпус СПбГПУ, ауд. 106.\\

Первая встреча состоится 19 февраля 2015 г.\\

\bigskip
\bigskip
\bigskip

\begin{table}[h]
\centering
\begin{tabular}{@{}ll@{}}
\toprule
\textit{Дата}  & \textit{Тема}                                                        \\ \midrule
19.02 & Дистрибутивная семантика (обзор)                            \\
05.03 & Латентный семантический анализ                               \\
19.03 & Аддитивные и мультипликативные методы в CDSM                \\
02.04 & Функциональная композиция в CDSM                            \\
16.04 & CDSM с помощью нейронных сетей                              \\
30.04 & Полилинейная алгебра и тензорные представления              \\
14.05 & Pregroup Grammar и языки диаграмм для тензорного исчисления \\
28.05 & Теоретико-категорная модель CDSM (Oxford Quantum Group)     \\ \bottomrule 
\end{tabular}
\end{table}

\bigskip
\bigskip

\begin{minipage}{0.5\textwidth}
\begin{figure}[H]
\includegraphics[width=7.5cm]{map.png} 
%\caption{\label{fig:blue_rectangle} Rectangle}
\end{figure}
\end{minipage} \hfill
\begin{minipage}{0.45\textwidth}
Политехническая улица, д. 21\\
9-й учебный корпус СПбГПУ, ауд. 106\\
м. <<Площадь Мужества>>\\
\end{minipage}






\end{document}
