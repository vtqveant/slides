\documentclass[12pt]{article}
\pagestyle{empty}

%%% Кодировки и шрифты %%%
\usepackage{cmap} % Улучшенный поиск русских слов в полученном pdf-файле
\usepackage[T2A]{fontenc} % Поддержка русских букв
\usepackage[utf8]{inputenc} % Кодировка utf8
\usepackage[english, russian]{babel} % Языки: русский, английский

%% графика
\ifx\pdfoutput\undefined
\usepackage{graphicx}
\else
\usepackage[pdftex]{graphicx}
\fi
\usepackage{wrapfig}
\usepackage{float}

\usepackage{setspace}

\usepackage{tabularx}
\usepackage{bigstrut}  % row spacing in tables

\textheight 26cm
\textwidth 16.5cm

\oddsidemargin 0.96cm   % margins on all sides of 3.5 cm,
\evensidemargin 0.96cm  % correcting 1 inch default on left-hand
\topmargin -2.2cm      % side, and 1.5 inch default on top.

\begin{document}

\baselineskip 15pt

\parindent 0cm


\begin{center}
{\rm \Huge{Понимание естественного языка}}\\
\medskip
{\rm \Large \textit{Natural Language Understanding Reading Group}}\\
\medskip
{\rm \Large \texttt{http://nlu-rg.ru}}
\end{center}

\bigskip

\baselineskip 12pt

\begin{spacing}{1.15}
Весенний семестр 2014-2015 гг. посвящен композициональным дистрибутивным \\семантическим моделям (Compositional Distributional Semantic Models). Встречи \\проходят \textit{раз в две недели по четвергам в 20:00} по адресу: м. <<Площадь Мужества>>, Политехническая ул., д. 21, 9-й учебный корпус СПбГПУ, ауд. 106.
\end{spacing}

% \bigskip

\begin{table}[h!]
%\caption{Расписание}
\begin{tabularx}{\textwidth}{cX}
\textit{Дата}  & \textit{Тема} \bigstrut \\ \hline
19.02 & Дистрибутивная семантика (обзор) \bigstrut[t] \\
      & {\scriptsize \textsc{Lenci, 2008. Distributional Semantics in Linguistic and Cognitive Research.}}  \bigstrut[b] \\
05.03 & Латентный семантический анализ  \bigstrut[t] \\
      & {\scriptsize \textsc{Landauer, Dumais, 1997. A Solution to Plato's problem: the latent semantic analysis theory of acquisition, induction and representation of knowledge.}}  \bigstrut[b] \\
19.03 & Аддитивные и мультипликативные методы в CDSM \bigstrut[t] \\
      & {\scriptsize \textsc{Mitchell, Lapata, 2010. Composition in Distributional Models of Semantics.}} \bigstrut[b] \\
02.04 & Функциональная композиция в CDSM \bigstrut[t] \\
      & {\scriptsize \textsc{Baroni et al., 2014. Frege in Space: A Program for Compositional Distributional Semantics.}}                \bigstrut[b] \\
16.04 & CDSM с помощью нейронных сетей \bigstrut[t] \\
      & {\scriptsize \textsc{Socher et al., 2013. Recursive Deep Models for Semantic Compositionality Over a Sentiment Treebank.}} \bigstrut[b] \\
30.04 & Полилинейная алгебра и тензорные представления \bigstrut[t] \\
      & {\scriptsize \textsc{Van de Cruys et al., 2013. A Tensor-based Factorization Model of Semantic Compositionality.}}\\
      & {\scriptsize \textsc{Maillard et al., 2014. A Type-Driven Tensor-Based Semantics for CCG.}} \bigstrut[b] \\
14.05 & Pregroup Grammar и языки диаграмм для тензорного исчисления \bigstrut[t] \\
      & {\scriptsize \textsc{Lambek, 2001. Type Grammars as Pregroups.}}  \\
      & {\scriptsize \textsc{Joyal, Street, 1991. The Geometry of Tensor Calculus - I.}} \bigstrut[b] \\
28.05 & Теоретико-категорная модель CDSM \bigstrut[t] \\ 
      & {\scriptsize \textsc{Coecke et al. 2010. Mathematical Foundations for a Compositional Distributional Model of Meaning.}} \bigstrut[b] \\ 
\hline
\end{tabularx}
\end{table}

\begin{minipage}{0.5\textwidth}
\begin{figure}[H]
\includegraphics[width=6.5cm]{map.png} 
%\caption{\label{fig:blue_rectangle} Rectangle}
\end{figure}
\end{minipage} \hfill
\begin{minipage}{0.45\textwidth}
\textbf{Время и место:}\\
\\
м. <<Площадь Мужества>>,\\
Политехническая ул., д. 21,\\
9-й учебный корпус СПбГПУ,\\
ауд. 106, начало в 20:00\\
\end{minipage}






\end{document}
