\documentclass{beamer}

\usepackage[T2A]{fontenc}
\usepackage[utf8]{inputenc}
\usepackage[english,russian]{babel}
\usepackage{amssymb,amsfonts,amsmath,mathtext}
\usepackage{cite,enumerate,float,indentfirst}

%% Gentzen style natural deduction proof trees
\usepackage{bussproofs}
\usepackage{latexsym}

\graphicspath{{images/}}

\usetheme{Pittsburgh}
\usecolortheme{whale}

\setbeamertemplate{footline}{\scriptsize{\hspace*{0.4cm}\insertframenumber}\vspace*{0.3cm}}
\beamertemplatenavigationsymbolsempty

\errorcontextlines 10000

\begin{document}
\title{\huge{$\mathbb{NLU}/RG$, \textit{pt. 7}}}
\author{Константин Соколов}
\institute[]
{Mathlingvo, СПбГУ, i-Free\\ \bigskip  \url{http://nlu-rg.ru}
}
\date{Санкт-Петербург, 2013} 
% Создание заглавной страницы
\begin{frame}
    \thispagestyle{empty}
    \titlepage
\end{frame}

%%% 0. План
\begin{frame}{План}
    \setcounter{framenumber}{1}
    \begin{itemize}
        \item Плунгян, модальность и эвиденциальность
        \item Крипке, реляционная семантика
    \end{itemize}
\end{frame}

\begin{frame}{Плунгян, модальность и эвиденциальность (1)}
\begin{itemize}
  \item 
  \item 
\end{itemize}
\end{frame}

%% системы Льюиса
\begin{frame}{Крипке, реляционная семантика (1)}
\begin{itemize}
  \item 
  \item 
  \item 
  \item 
\end{itemize}
\end{frame}

%% системы модальной логики: K, S4, S5 и др.
%% формальная система S4 (язык, аксиомы, правила вывода, понятие выводимости)
%% нормальные и ненормальные системы
\begin{frame}{Крипке, реляционная семантика (2)}
\begin{itemize}
  \item 
  \item 
  \item 
  \item 
\end{itemize}
\end{frame}

%% Теория моделей для модальной логики: возможность относительно, возможный мир
\begin{frame}{Крипке, реляционная семантика (3)}
\begin{itemize}
  \item 
  \item 
  \item 
  \item 
\end{itemize}
\end{frame}

%% Kripke frame
\begin{frame}{Крипке, реляционная семантика (4)}
\begin{itemize}
  \item 
  \item 
  \item 
  \item 
\end{itemize}
\end{frame}

%% табличный метод, эквивалентность таблиц и моделей
\begin{frame}{Крипке, реляционная семантика (5)}
\begin{itemize}
  \item 
  \item 
  \item 
  \item 
\end{itemize}
\end{frame}

%% полнота
\begin{frame}{Крипке, реляционная семантика (6)}
\begin{itemize}
  \item 
  \item 
  \item 
  \item 
\end{itemize}
\end{frame}

%% матрицы (раздел 5.2), характеристические функции
\begin{frame}{Крипке, реляционная семантика (7)}
\begin{itemize}
  \item 
  \item 
  \item 
  \item 
\end{itemize}
\end{frame}

%% несколько слов об окрестностной семантике Монтегю-Скотта
\begin{frame}{Крипке, реляционная семантика (8)}
\begin{itemize}
  \item отношение достижимости заменяется отношением ... $2^{2^W}$ 
  \item 
  \item 
  \item 
\end{itemize}
\end{frame}


\begin{frame}{}
    \thispagestyle{empty}
    \begin{center}
        {\large Спасибо!}
    \end{center}
\end{frame}


%%% слайд помещается сюда
%% \begin{frame}{Заголовок}
%% \end{frame}

\end{document}
