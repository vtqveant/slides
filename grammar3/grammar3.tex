\documentclass{beamer}

\usepackage[T2A]{fontenc}
\usepackage[utf8]{inputenc}
\usepackage[english,russian]{babel}
\usepackage{amssymb,amsfonts,amsmath,mathtext}
\usepackage{cite,enumerate,float,indentfirst}

%% Gentzen style natural deduction proof trees
\usepackage{bussproofs}
\usepackage{latexsym}

\graphicspath{{images/}}

\usetheme{Pittsburgh}
\usecolortheme{whale}

\setbeamertemplate{footline}{\scriptsize{\hspace*{0.4cm}\insertframenumber}\vspace*{0.3cm}}
\beamertemplatenavigationsymbolsempty

\errorcontextlines 10000

\begin{document}
\title{\huge{$\mathbb{NLU}/RG$, \textit{pt. 7}}}
\author{Константин Соколов}
\institute[]
{Mathlingvo, СПбГУ, i-Free\\ \bigskip  \url{http://nlu-rg.ru}
}
\date{Санкт-Петербург, 2013} 
% Создание заглавной страницы
\begin{frame}
    \thispagestyle{empty}
    \titlepage
\end{frame}

%%% 0. План
\begin{frame}{План}
    \setcounter{framenumber}{1}
    \begin{itemize}
        \item Плунгян, модальность и эвиденциальность
        \item Крипке, реляционная семантика
    \end{itemize}
\end{frame}

\begin{frame}{Плунгян, модальность и эвиденциальность (1)}
\begin{itemize}
  \item 
  \item 
\end{itemize}
\end{frame}

%% системы Льюиса
\begin{frame}{Крипке, реляционная семантика (1)}
\begin{itemize}
  \item 
  \item 
  \item 
  \item 
\end{itemize}
\end{frame}

%% системы модальной логики: K, S4, S5 и др.
%% формальная система S4 (язык, аксиомы, правила вывода, понятие выводимости)
%% нормальные и ненормальные системы
\begin{frame}{Система Льюиса S4}
S4 - пропозициональная логика с (единственным) модальным оператором $\Box$.\\
\bigskip
Аксиомы (только для $\Box$):
\begin{itemize}
  \item \textbf{K:} $\Box (p \supset q) \supset ( \Box p \supset \Box q)$
  \item \textbf{T:} $\Box p \supset p$
  \item \textbf{4:} $\Box p \supset \Box \Box p$
\end{itemize}
\end{frame}

%% Теория моделей для модальной логики: возможность относительно, возможный мир
\begin{frame}{Крипке, реляционная семантика (3)}
\begin{itemize}
  \item 
  \item 
  \item 
  \item 
\end{itemize}
\end{frame}

%% Kripke frame
\begin{frame}{Крипке, реляционная семантика (4)}
\begin{itemize}
  \item 
  \item 
  \item 
  \item 
\end{itemize}
\end{frame}

%% табличный метод, эквивалентность таблиц и моделей
\begin{frame}{Семантические таблицы (1)}
\textit{Табличный метод} - это \\
\bigskip
\begin{itemize}
  \item формальная процедура
    \begin{itemize}
      \item процедура доказательства (синтаксический аспект)
      \item процедура поиска контрмодели (семантический аспект)
    \end{itemize}
  \item процедура состоит в формировании таблицы с помощью 
    \begin{itemize}
      \item расширения исходной таблицы путем выписывания подформул \textit{по правилам}
      \item перехода к подтаблицам для рассмотрения случаев
    \end{itemize}  
  \item процесс завершается 
    \begin{itemize}
      \item получением \textit{закрытой} таблицы, демонстрирующей противоречие (исходная формула выводима)
      \item получением открытой таблицы (исходная формула невыводима)
      \item или не завершается
    \end{itemize}
\end{itemize}
\end{frame}

%% табличный метод, эквивалентность таблиц и моделей
\begin{frame}{Семантические таблицы (2)}
Правила (для пропозициональной логики):\\
\begin{prooftree}
  \AxiomC{$T \! X$} \AxiomC{$F \! X$}
  \noLine
  \BinaryInfC{}
\end{prooftree}

\begin{prooftree}
  \AxiomC{$T \! \sim \! X$}
  \UnaryInfC{$FX$}
\end{prooftree}

\begin{prooftree}
  \AxiomC{$F \! \sim \! X$}
  \UnaryInfC{$TX$}
\end{prooftree}

\begin{prooftree}
  \AxiomC{$T \! X \supset Y$}
  \UnaryInfC{$FX \; / \; TY$}
\end{prooftree}

\begin{prooftree}
  \AxiomC{$F \! X \supset Y$}
  \UnaryInfC{$TX$}
  \noLine
  \UnaryInfC{$FY$}
\end{prooftree}

\end{frame}


%% табличный метод, эквивалентность таблиц и моделей
\begin{frame}{Семантические таблицы (3)}
Доказательство закона де Моргана:\\
\bigskip
\begin{itemize}
  \item $F \! \sim (p \wedge q) \supset (\sim p \; \vee \sim q)$
  \item $T \! \sim (p \wedge q)$
  \item $F \! \sim p \; \vee \sim q$
  \item $F \! \sim p$
  \item $F \! \sim q$
  \item $T \! p$
  \item $T \! q$
  \item $F \! p \wedge q$
  \item $F \! p \; / \; F \! q$
\end{itemize}
\end{frame}

%% полнота
\begin{frame}{Семантические таблицы (4)}
Табличный метод для модальной логики:\\
\bigskip
\begin{itemize}
  \item Предполагаем, что $A_1, A_2, ..., A_n \supset B$ ложно, т.е.
    \begin{table}
      \begin{tabular*}{1.5cm}{c|c}
        T   & F \\ \hline
        $A_1$ & $B$ \\
        $A_2$ & ~ \\
        $\vdots$ & ~ \\
        $A_n$ & ~ \\
      \end{tabular*}
    \end{table}
  \item Аналогично, предполагаем, что $A$ ложно, т.е. 
    \begin{table}
      \begin{tabular*}{1.3cm}{c|c}
        T   & F \\ \hline
        ~ & $A$ \\
      \end{tabular*}
    \end{table}
\end{itemize}
\end{frame}

\begin{frame}{Семантические таблицы (5)}
Правила пострения таблиц (для связок):\\
\bigskip
\begin{itemize}
  \item 
      \begin{tabular}{p{0.75cm}|p{0.75cm}}
        \hline
        $\sim \! A$ & ~ \\
      \end{tabular}
      $\; \to \;$
      \begin{tabular}{p{0.75cm}|p{0.75cm}}
        \hline
        $\sim \! A$ & $A$ \\
      \end{tabular}\\
      \bigskip
  \item 
      \begin{tabular}{p{0.75cm}|p{0.75cm}}
        \hline
        ~ & $\sim \! A$ \\
      \end{tabular}
      $\; \to \;$
      \begin{tabular}{p{0.75cm}|p{0.75cm}}
        \hline
        $A$ & $\sim \! A$ \\
      \end{tabular}\\
      \bigskip
  \item и т. п.
\end{itemize}
\end{frame}

\begin{frame}{Семантические таблицы (6)}
Правила пострения таблиц (для кванторов):\\
\bigskip
\begin{itemize}
  \item 
      \begin{tabular}{p{1.25cm}|p{1.25cm}}
        \hline
        $\forall x . A(x)$ & ~ \\
      \end{tabular}
      $\; \to \;$
      \begin{tabular}{p{1.25cm}|p{1.25cm}}
        \hline
        $\forall x . A(x)$ & ~ \\
        $A(a)$ & ~ \\
      \end{tabular}\\
      \bigskip
      {\scriptsize где $a$ - уже встречавшаяся свободная переменная}\\
      \bigskip
  \item 
      \begin{tabular}{p{1.25cm}|p{1.25cm}}
        \hline
        ~ & $\forall x . A(x)$ \\
      \end{tabular}
      $\; \to \;$
      \begin{tabular}{p{1.25cm}|p{1.25cm}}
        \hline
        ~ & $\forall x . A(x)$ \\
        ~ & $A(a)$ \\
      \end{tabular}\\
      \bigskip
      {\scriptsize где $a$ - новая переменная}
\end{itemize}
\end{frame}

\begin{frame}{Семантические таблицы (7)}
Правила пострения таблиц (для модального оператора):\\
\bigskip
\begin{itemize}
  \item 
      \begin{tabular}{p{0.75cm}|p{0.75cm}}
        \hline
        $\Box A$ & ~ \\
      \end{tabular}
      $\; \to \;$
      \begin{tabular}{p{0.75cm}|p{0.75cm}}
        \hline
        $\Box A$ & ~ \\
        $A$ & ~ \\
      \end{tabular}\\
      \bigskip
      {\scriptsize для всех таблиц \textit{множества альтернатив}}\\
      \bigskip
  \item 
      \begin{tabular}{p{0.75cm}|p{0.75cm}}
        \hline
        ~ & $\Box A$ \\
      \end{tabular}
      $\; \to \;$
      \begin{tabular}{p{0.75cm}|p{0.75cm}}
        \hline
        ~ & $\Box A$ \\
      \end{tabular} +
      \begin{tabular}{p{0.75cm}|p{0.75cm}}
        \hline
        ~ & $A$ \\
      \end{tabular}\\
      \bigskip
      {\scriptsize т.е. добавляется новая \textit{вспомогательная таблица}}\\
\end{itemize}
\end{frame}

\begin{frame}{Семантические таблицы (8)}
\begin{itemize}
  \item \textit{таблица замкнута}, если
      \begin{tabular}{c|c}
        T & F \\ \hline
        $\vdots$ & $\vdots$ \\
        $A$ & $A$ ~ \\
      \end{tabular}\\
      \bigskip
  \item \textit{множество альтернатив} замкнуто, если одна из таблиц этого множества замкнута
  \item вся \textit{конструкция замкнута}, если в ней появляется замкнутая система альтернатив
\end{itemize}
\end{frame}


\begin{frame}{Крипке, реляционная семантика (1)}
\textit{Шкала Крипке} (Kripke frame) позволяет охарактеризовать множество таблиц с т.з. их отношений.\\
\bigskip
$\mathcal{F} = (W, R)$, где $W$ - непустое множество, $R \subset W \times W$.\\
\bigskip
\begin{itemize}
  \item $w_i \in W$ - миры, состояния, точки отнесенности
  \item $R$ - отношение достижимости
  \item если $wRv$, то говорят, что $v$ \textit{возможен относительно} $w$
\end{itemize}
\bigskip
Легко заметить, что шкала Крипке - это граф.
\end{frame}

\begin{frame}{Крипке, реляционная семантика (2)}
\textit{Модель Крипке} $\mathcal{M} = (\mathcal{F}, V)$, где  
\bigskip
\begin{itemize}
  \item $\mathcal{F}$ - шкала Крипке
  \item $V : P \to 2^W$ - функция оценивания, т.е. отображение из атомарных выражений в подмножества множества миров
\end{itemize}
\bigskip
Важный технический результат: конструкция для $A$ замкнута тогда и только тогда, когда $A$ общезначима.\\
\bigskip
Иными словами, таблицы эквивалентны моделям.
\end{frame}

\begin{frame}{Крипке, реляционная семантика (3)}
Пусть $\mathcal{M} = (\mathcal{F}, V)$, $w \in W$, $\phi$ - формула, $p \in P$\\
\bigskip
Истинность $\phi$ в модели $\mathcal{M}$ в точке $w$ определяется рекурсивно\\
\bigskip
\begin{itemize}
  \item $\mathcal{M}, w \models p \; \Longleftrightarrow \; w \in V(p)$
  \item $\mathcal{M}, w \models \neg \phi \; \Longleftrightarrow \; \mathcal{M}, w \not\models \phi$
  \item $\mathcal{M}, w \models \phi \vee \psi \; \Longleftrightarrow \; \mathcal{M}, w \models \phi$ или $\mathcal{M}, w \models \psi$
  \item $\mathcal{M}, w \models \Box \phi \; \Longleftrightarrow \; \forall v \in W \; . \; w R v$, $\mathcal{M}, v \models \phi$
  \item $\mathcal{M}, w \not\models \perp$
\end{itemize}
\bigskip
Если $\mathcal{M}, w \models \phi$, говорят, что $\phi$ \textit{логически следует} из $\mathcal{M}, w$
\end{frame}

\begin{frame}{Крипке, реляционная семантика (4)}
Функция оценивания $V: P \to 2^W$ может быть преобразована разными способами:\\
\bigskip
\begin{itemize}
  \item $V': P \times W \to 2$ задает множество $V' \subseteq P \times W$, элементы которого - пары вида $(p, w)$, для которых $w \in V(p)$
  \item $V'': W \times P \to 2$, тогда $\tilde{V} : W \to 2^P$ задает\\ множество $\tilde{V}$ истинных утверждений о мире $w$
\end{itemize}
\end{frame}



%% несколько слов об окрестностной семантике Монтегю-Скотта
\begin{frame}{Крипке, реляционная семантика (8)}
\begin{itemize}
  \item отношение достижимости заменяется отношением ... $2^{2^W}$ 
  \item 
  \item 
  \item 
\end{itemize}
\end{frame}


\begin{frame}{}
    \thispagestyle{empty}
    \begin{center}
        {\large Спасибо!}
    \end{center}
\end{frame}


%%% слайд помещается сюда
%% \begin{frame}{Заголовок}
%% \end{frame}

\end{document}
