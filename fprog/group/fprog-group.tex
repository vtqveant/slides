\documentclass{beamer}

\usepackage[T2A]{fontenc}
\usepackage[utf8]{inputenc}
\usepackage[english,russian]{babel}
\usepackage{amssymb,amsfonts,amsmath,mathtext}
\usepackage{cite,enumerate,float,indentfirst}

\usepackage{graphicx}
\usepackage{booktabs}
\usepackage{tabularx}

% CCG parse trees
\newcommand{\deriv}[2]
{  \renewcommand{\arraystretch}{.5}
$\begin{array}[t]{*{#1}{c}}
     #2
   \end{array}$ }
\newcommand{\gf}[1]{\textsf{\textsl{#1}}}
\newcommand{\cf}[1]{\mbox{\ensuremath{\cfont{#1}}}}
\newcommand{\uline}[1]
{\mc{#1}{\hrulefill} }
\newcommand{\mc}[2]
  {\multicolumn{#1}{c}{#2}}
\newcommand{\cfont}{\mathsf}
\newcommand{\bs}{\backslash}
\newcommand{\subsa}[1]{\hspace{-0.75mm}_{_{#1}}}
\newcommand{\subsb}[1]{\hspace{-0.10mm}_{_{#1}}}
\newcommand{\subs}[1]{\hspace{-0.40mm}_{#1}}
\newcommand{\subsf}[1]{\hspace{-0.75mm}_{_{#1}}}
\newcommand{\supsa}[1]{\hspace{-1.75mm}^{^{#1}} }
\newcommand{\supsb}[1]{\hspace{-0.80mm}^{^{#1}}  }
\newcommand{\sups}[1]{\hspace{-0.40mm}^{#1}}


% Attribute-Value Matrices
\usepackage{avm}
\avmfont{\sc}
\avmoptions{sorted,active}
\avmvalfont{\rm}
\avmsortfont{\scriptsize\it}

%% Gentzen style natural deduction proof trees
\usepackage{bussproofs}
\usepackage{latexsym}

\graphicspath{{images/}}

\usetheme{Pittsburgh}
\usecolortheme{whale}

\setbeamertemplate{footline}{\scriptsize{\hspace*{0.4cm}\insertframenumber}\vspace*{0.3cm}}
\beamertemplatenavigationsymbolsempty

\errorcontextlines 10000

\begin{document}
\title{\Large{Компьютерная лингвистика на кафедре РВКС}}
\author{Константин Соколов}
\institute[]
{СПбГУ, ИИТУ СПбПУ\\ \bigskip  \url{http://nlu-rg.ru}}
\date{} 

\begin{frame}
    \thispagestyle{empty}
    \titlepage
\end{frame}

% 10 min - представление -- о себе, о семинаре
%          вводная часть - компьютерная лингвистика, моделирование
% 10 мин - формальная семантика
%          Фреге -- смысл и значение, принцип композициональности
%          Карнап -- интенсионал/экстенсионал, возможные миры
%          Чёрч -- просто типизированное лямбда-исчисление, логика смысла и денотата
%          Монтегю -- синтактико-семантический интерфейс, интенсиональная логика
% 5 мин  - CCG/HLDS, model checking
% 5 мин  - теоретико-типовая семантика
%          анонс семинара
% 10 мин - разметка корпуса, исчисление Ламбека, coq

\begin{frame}{РВКС}
\setcounter{framenumber}{1}
\begin{small}
\begin{itemize}
    \item Семинар ``Понимание естественного языка'' (\textit{2013 -- н.~в.})
    \item Курсы 
        \begin{itemize}
            \item Анализ текстов на естественных языках (\textit{весна 2015})
            \item Машинное обучение в NLP (\textit{планируется})
        \end{itemize}
    \item Проекты
        \begin{itemize}
        	    \item Голосовое управление оборудованием
            \item Извлечение процессов из текстовых данных
            \item Синтаксическая разметка средствами интерактивного поиска доказательств
        \end{itemize}
\end{itemize}
\end{small}
\end{frame}

\begin{frame}{}
\begin{center}
	\textbf{``Понимание естественного языка''}
\end{center}
\end{frame}

\begin{frame}{``Понимание естественного языка'' (I)}
\begin{small}
\begin{itemize}
    \item формат ``Reading Group''
    \item каждый второй четверг в 20:00
    \item \texttt{http://www.nlu-rg.ru}
    \item материалы семинара вошли в пособие\\ \textit{Андреев и др., ``Введение в формальную семантику'', СПб, 2015}
\end{itemize}
\end{small}
\end{frame}

\begin{frame}{``Понимание естественного языка'' (II)}
\begin{small}
\begin{itemize}
    \item основные семантические концепции интенсиональной логики
    \item формальная, когнитивная, дистрибутивная семантика
    \item лингвистические формализмы
    \item моделирование семантической композиции
    \begin{itemize}
        \item с помощью $\lambda$-исчисления
        \item с помощью унификации графовых структур
        \item топологическая трактовка принципа композициональности
    \end{itemize}
    \item когнитивные архитектуры
    \item композициональные дистрибутивные семантические модели
\end{itemize}
\end{small}
\end{frame}

\begin{frame}{NLP, NLU (I)}
\begin{small}
\begin{itemize}
        \item математическая лингвистика
        \item компьютерная лингвистика (Computational Linguistics)
        \item прикладная лингвистика
        \item обработка естественного языка (Natural Language Processing)
        \item Text Processing, Text Mining
        \item Cognitive Computing
        \item (автоматическое) понимание естественного языка
        \item вычислительная семантика
\end{itemize}
\end{small}
\end{frame}

\begin{frame}{NLP, NLU (II)}
\begin{small}
\begin{itemize}
    \item задача -- моделирование аспектов естественного языка
    \begin{itemize}
        \item логические, алгебраические, геометрические методы
        \item теория формальных языков
        \item теория вычислений, теория типов
        \item когнитивное моделирование (символьные, коннекционистские, гибридные подходы)
        \item статистика, машинное обучение
        \item нейросети
    \end{itemize}    
\end{itemize}
\end{small}
\end{frame}

\begin{frame}{}
\thispagestyle{empty}
\begin{center}
	\textbf{Формальная семантика}
\end{center}
\end{frame}

\begin{frame}{Формальная семантика}
\begin{small}
\begin{itemize}
	\item основы лингвистической формальной семантики были заложены в поздних работах Р. Монтегю в начале 70-х гг.
	\item Монтегю опирается на предшествующие разработки\\ Р. Карнапа, А. Чёрча и др.
\end{itemize}
\end{small}
\end{frame}

\begin{frame}{Интенсиональная логика (I)}
\begin{itemize}
	\item \textit{Интенсиональная логика} -- это логика, содержащая в объектном языке имена интенсиональных сущностей
	\medskip
	\item выражаясь нестрого, \textit{интенсионал} можно описать как название или наименование некоторого мыслимого содержания
	\medskip
	\item интенсиональная логика позволяет формулировать утверждения об \textit{интенсиональных сущностях (интенсионалиях)}, выступающих денотатами интенсиональных понятий
	    \medskip
	    {\small
	    \begin{itemize}
	        \item Примеры: \textit{индивидный концепт, пропозиция}
	    \end{itemize}
	    }
\end{itemize}
\end{frame}

\begin{frame}{Интенсиональная логика -- II}
\begin{itemize}
    \item косвенный контекст
    	\medskip
	    \begin{itemize}
	        \item \texttt{Иван думает, что снег бел}
	        \smallskip
	        \item \texttt{Иван думает, что фраза ``снег бел'' соответствует положению дел}
	    \end{itemize}    
	\medskip
    \item ``навешивание'' квантора существования
    	\medskip
	    \begin{itemize}
	        \item \texttt{Иван сомневается, что снежный человек существует}
	        \smallskip
	        \item $^?$\texttt{Существует некто, в чьём существовании Иван сомневается}
	    \end{itemize}   	
\end{itemize}
\end{frame}

\begin{frame}{Интенсиональная логика -- II}
\begin{itemize}
    \item референтная непрозрачность
    	\medskip
	    \begin{itemize}
	        \item \texttt{Мери полагает, что самый популярный ветеринар в Восточном Иллинойсе — мой друг}
	        \smallskip
	        \item \texttt{Самый популярный ветеринар в Восточном Иллинойсе -- Денвилл старший, страдающий от болезней ног}
	        \smallskip
	        \item \texttt{Мери полагает, что Денвилл старший, страдающий от болезней ног,— мой друг}
	    \end{itemize}   	
    \medskip
	\item необходимо иметь возможность
    	\medskip
	    \begin{itemize}
	        \item различать мыслимое содержание и его выражение
	        \item анализировать употребление в различных контекстах
	    \end{itemize}   	
\end{itemize}
\end{frame}

\begin{frame}{Г. Фреге}
\begin{center}
	\begin{figure}[H]
		\includegraphics[scale=0.5]{frege.jpg} 
	\end{figure}
\end{center}
\end{frame}

\begin{frame}{Бикомпонентная семантика Г. Фреге}
\begin{itemize}
    \item смысл и денотат
	\item косвенный контекст
	\item треугольник Фреге
	\item принцип композициональности
\end{itemize}
\end{frame}

\begin{frame}{Р. Карнап}
\begin{center}
	\begin{figure}[H]
		\includegraphics[scale=0.44]{carnap.jpg} 
	\end{figure}
\end{center}
\end{frame}

\begin{frame}{Интенсиональная иерархия Р. Карнапа}
\begin{itemize}
    \item метод интенсионала и экстенсионала
    \begin{itemize}
        \item описание состояния и \textit{действительное} описание состояние
        \item $L$-истинность -- логическая истинность
        \item $F$-истинность -- соответствие положению вещей
    \end{itemize}
	\item интенсиональная иерархия
	\item интенсиональный (косвенный) контекст и возможные миры
\end{itemize}
\end{frame}

\begin{frame}{А. Чёрч}
\begin{center}
	\begin{figure}[H]
		\includegraphics[scale=0.6]{church.jpg} 
	\end{figure}
\end{center}
\end{frame}

\begin{frame}{$\lambda_\to$ (I)}
\textit{исчисление} задается как\\
\begin{itemize}
  \item набор базовых символов и способ их комбинации
  \item множество аксиом
  \item правила вывода
  \item определение понятия выводимости
\end{itemize}
\bigskip
систему простой типизации для лямбда-исчисления построил Алонзо Чёрч в 1940 г.
\end{frame}

\begin{frame}{$\lambda_\to$ (II)}
\textit{нотация}\\
\medskip
\begin{itemize}
  \item базовые типы $\alpha, \beta$, функциональные типы $(\alpha\beta) = \beta \to \alpha$
  \item $\lambda x_\alpha \; . \; A_{\beta\alpha}$
  \item константы $true$ и $false$ имеют тип $o$
  \item функции с типом $(o\alpha) = \alpha \to o$
  \item каррирование: $(o\alpha o\beta) = \beta \to (o \to (\alpha \to o))$
  \item универсальный квантор: $\Pi_{o(o\alpha)}[\lambda x_\alpha A_o] \equiv \forall x_\alpha A_o$
  \item $[MN]$ - применение функции $M$ к аргументу $N$
\end{itemize}
\end{frame}

\begin{frame}{$\lambda_\to$ (III)}
\textit{язык}\\
\medskip
\begin{itemize}
  \item переменная или константа типа $\alpha$ -- п.п.ф. типа $\alpha$ 
  \item если $A_{\alpha\beta}$ и $B_{\beta}$ -- п.п.ф. соответствующих типов, то $[A_{\alpha\beta}B_\beta]$ -- п.п.ф. типа $\alpha$
  \item если $x_\beta$ -- переменная типа $\beta$ и $A_\alpha$ -- п.п.ф., то $[\lambda x_\beta A_\alpha]$ -- п.п.ф. типа $(\alpha \beta)$
\end{itemize}
\bigskip
\textit{замечания}\\
\begin{itemize}
  \item $\sim_{(oo)}$ - п.п.ф. типа $(oo)$, $[\sim_{(oo)}A_o]$ - п.п.ф. типа $o$
  \item $Q_{o \alpha \alpha} = [\lambda x_\alpha \lambda y_\alpha \forall f_{o \alpha}[f_{o \alpha} x_\alpha \supset f_{o \alpha} y_\alpha]]$ - если $y$ обладает всеми свойствами $x$, то $y = x$
\end{itemize}
\end{frame}

\begin{frame}{$\lambda_\to$ (IV)}
\textit{правила вывода}\\
\bigskip
\begin{itemize}
  \item $\alpha$-конверсия: переименование связанных переменных
  \item $\beta$-редукция: замена $[[\lambda x_\beta M_\alpha] N_\beta]$ на $M_\alpha[x_\beta/N_\beta]$
  \item $\beta$-экспансия: $D \vdash C$, если $D$ получается из $C$ однократным применением $\beta$-редукции
  \item подстановка: $F_{(o \alpha)} x_\alpha \vdash F_{(o \alpha)} A_\alpha$
  \item modus ponens: $[A_o \supset B_o], A_o \vdash B_o$ 
  \item обобщение: $F_{(o \alpha)} x_\alpha \vdash \Pi_{o (o \alpha)} F_{(o \alpha)}$
\end{itemize}
\end{frame}

\begin{frame}{$\lambda_\to$ (V)}
\textit{аксиомы исчисления выказываний}\\
\begin{itemize}
  \item $p \vee p \supset p$
  \item $p \supset p \vee p$
  \item $p \vee q \supset q \vee p$
  \item $(p \supset q) \supset (r \vee p \supset r \vee q)$
\end{itemize}
\bigskip
\textit{аксиомы логического функционального исчисления}\\
\begin{itemize}
  \item $\Pi_{o (o \alpha)} f_{o \alpha} \supset f_{o \alpha} x_\alpha$
  \item $\forall x_\alpha [p_o \vee f_{o \alpha}] \supset [p_o \vee \Pi_{o(o \alpha)} f_{o \alpha}]$
\end{itemize}
\end{frame}

\begin{frame}{$\lambda_\to$ (VI)}
\textit{аксиома экстенсиональности}\\
\bigskip
\begin{itemize}
  \item $\forall x_\beta [f_{\alpha \beta} x_\beta = g_{\alpha \beta} x_\beta] \supset f_{\alpha \beta} = g_{\alpha \beta}$
\end{itemize}
\bigskip
\textit{аксиома дескрипции}\\
\begin{itemize}
  \item $\exists ! x_\alpha [p_{o \alpha} x_\alpha] \supset p_{o \alpha} [\iota_{\alpha(o \alpha)}p_{o \alpha}]$, где $\exists ! x_\alpha A_o =_{def} [\lambda p_{o \alpha} \exists y_\alpha[p_{o \alpha} y_\alpha \wedge \forall z_\alpha[p_{o \alpha} z_\alpha \supset z_\alpha = y_\alpha]]][\lambda x_\alpha A_o]$
\end{itemize}
\bigskip
неформально, $A_o$ описывает $x_\alpha$; \textit{оператор дескрипции} $\iota_{\alpha (o \alpha)}$ сопоставляет одноэлементному множеству его (единственный) элемент
\end{frame}

\begin{frame}{$\lambda_\to$ (VII)}
\textit{аксиома выбора}\\
\bigskip
\begin{itemize}
  \item $f_{o \alpha} x_\alpha \supset f_{o \alpha} [\iota_{\alpha (o \alpha)} f_{o \alpha}].$
\end{itemize}
\bigskip
неформально, \textit{оператор выбора} $\iota_{\alpha (o \alpha)}$ сопоставляет непустому множеству какой-то его элемент\\
\bigskip
аксиома выбора влечет аксиому дескрипции
\end{frame}

\begin{frame}{$\lambda_\to$ (VIII)}
\textit{формальный вывод}\\
\bigskip
\begin{itemize}
  \item \textit{доказательство} формулы $B_o$ в предположении формул $A_o^1, A_o^2, ..., A_o^n$ - это конечная последовательность формул, последняя из которых - $B_o$, а остальные - либо одна из формул $A_o^1, A_o^2, ..., A_o^n$, вариант схемы аксиом, либо формула, полученная из предыдущих формул последовательности применением правил вывода.
  \item если такое доказательство существует, будем писать $A_o^1, A_o^2, ..., A_o^n \vdash B_o$
  \item \textit{теорема дедукции:} если $A_o^1, A_o^2, ..., A_o^n \vdash B_o$, то $A_o^1, A_o^2, ..., A_o^{n-1} \vdash A_o^n \supset B_o$
\end{itemize}
\end{frame}

\begin{frame}{Логика смысла и денотата}
\begin{itemize}
    \item моделирование интенсиональной иерархии средствами теоретико-типового формализма
   	\medskip
         \begin{itemize} 
	        \item базовые типы: $o$ -- истинностные значения, $\iota$ -- индивиды
	        \item функциональные типы: $\alpha_n \beta_m$, напр. $o \iota$ -- одноместные предикаты, $o(o \iota)$ -- дескрипции
	        \item интенсиональные типы: $o_1$ -- пропозиции, $\iota_1$ -- индивидные концепты, $o_2$ -- пропозициональные концепты, $\iota_2$ -- имена индивидных концептов, $oo_1$ -- тип модального оператора
	        \item трансфинитный интенсиональный тип $o_\omega$
	    \end{itemize}
	\medskip
	\item процедура интенсионального восхождения
	\medskip
	\item двухместный предикат $\Delta$ (``быть концептом'')
\end{itemize}
\end{frame}

\begin{frame}{Р. Монтегю}
\begin{center}
	\begin{figure}[H]
		\includegraphics[scale=0.5]{montague.jpg} 
	\end{figure}
\end{center}
\begin{flushleft}
I reject the contention that an important theoretical difference exists between formal and natural languages.
\end{flushleft}
\begin{flushright}
\textit{English as a Formal Language (1970)}
\end{flushright}
\end{frame}

\begin{frame}{Интенсиональная логика Р. Монтегю (I)}
прагматика и интенсиональная логика (60-е гг.)\\
\bigskip
\begin{itemize}
    \item совместил подход Карнапа (интенсиональный контекст как класс возможных миров) и подход Чёрча (моделирование интенсиональной иерархии средствами теории типов)
    \item функциональная абстракция $\lambda$ и интенсиональная абстракция $\hat{}$
    \item проблема ``псевдотипа'' $s$
    \item окрестностная семантика (семантика Монтегю-Скотта)
\end{itemize}
\end{frame}

\begin{frame}{Интенсиональная логика Р. Монтегю (II)}
грамматика Монтегю (1970--1971 гг.)\\
\bigskip
\begin{itemize}
    \item три лингвистические работы
    \begin{itemize}
        \item English as a Formal Language
        \item Universal Grammar
        \item Proper Treatment of Quantification in Ordinary English
    \end{itemize}
    \item метод трансляции -- интенсиональная логика как ``семантический метаязык''
    \item синтактико-семантический интерфейс
\end{itemize}
\end{frame}

\begin{frame}{Синтактико-семантический интерфейс (I)}
\begin{small}
Анализ предложения \textit{some woman admires every man}\\
\begin{itemize}
	\item в процессе порождения получены синтаксические формы
	\begin{itemize}
		\item $\alpha_1$ : some woman admires him$_n$
		\item $\alpha_2$ : every man
	\end{itemize}
	\item и соответствующие им термы 
	\begin{itemize}
		\item $\beta_1 : \exists x . [woman(x) \wedge admire(x,y)]$
		\item $\beta_2 : \lambda B . \forall y . [man(y) \to B(y)]$
	\end{itemize}
\end{itemize}
\end{small}
\end{frame}

\begin{frame}{Синтактико-семантический интерфейс (II)}
\begin{small}
\begin{itemize}
	\item синтаксическое и трансляционное правила применяются одновременно
	\begin{itemize}
		\item синтаксическое правило $S_{14}$: подстановка другого выражения вместо him$_n$
		\item трансляционное правило $T_{14}$: $\beta = \beta_2(\lambda x_n . \beta_1)$
	\end{itemize}
	\item результат порождения 
	\begin{itemize}
		\item $\alpha$ : some woman admires every man
		\item $\beta : \forall y [man(y) \to \exists x [woman(x) \wedge admire(x,y)]]$
	\end{itemize}
\end{itemize}
\end{small}
\end{frame}

\begin{frame}{CCG/FOL -- I}
\begin{small}
  \begin{itemize}
    \item каждому слову вместе с категорией можно приписать $\lambda$-терм
      \begin{eqnarray*}
        ``John'' &\vdash& np : john \\
        ``Mary'' &\vdash& np : mary \\
        ``loves'' &\vdash& (s \backslash np) / np : \lambda x.\lambda y.love(y,x)
      \end{eqnarray*}
    \item композиция термов задается правилами
      \begin{eqnarray*}
        (X / Y) : f~~~Y : x &\rightarrow_{>}& X : fx \\
        Y : x~~~(X \backslash Y) : f &\rightarrow_{<}& X : fx
      \end{eqnarray*}
  \end{itemize}
\end{small}  
\end{frame} 

\begin{frame}[fragile]{CCG/FOL -- II}
\begin{verbatim}
 John        loves                         Mary
 np : john   (s\np)/np : \x.\y.love(y,x)   np : mary
            ----------------------------------------->
             s\np : \y.love(y,mary)
---------------------------------------<
 s : love(john,mary)
\end{verbatim}
\end{frame} 

\begin{frame}[fragile]{CCG/FOL -- III}
\begin{footnotesize}
\begin{verbatim}
 library             of                                    New York
 n : \x.library(x)   (n\n)/np : \y.\f.\x.f(x) & loc(x,y)   np : nyc
                    ------------------------------------------------>
                     n\n : \f.\x.f(x) & loc(x,nyc)
--------------------------------------------------<
 n : \x.library(x) & loc(x,nyc)                     
\end{verbatim}
\end{footnotesize}
\end{frame} 

\begin{frame}[fragile]{MMCCG/HLDS -- I}
Пример словарной записи:\\
\smallskip
\begin{center}
$flower \vdash n_{sg,X:thing} : @_{X:thing}(\textbf{flower} \wedge \langle NUM \rangle sg)$
\end{center}
\bigskip
\begin{small}
\begin{verbatim}
family tv(V) {
    entry : s \! np / np : 	E:event(* <Actor>(S:entity) 
                                      <Patient>(X:entity));
}
\end{verbatim}
\end{small}
\end{frame}

\begin{frame}[fragile]{MMCCG/HLDS -- II}
\begin{center}
\deriv{6}{
\gf{включи} & \gf{лампу} & \gf{и} & \gf{подсветку} & \gf{на} & \gf{кухне} \\
\uline{1} & \uline{1} & \uline{1} & \uline{1} & \uline{1} & \uline{1} \\
\cf{s\bs \supsa{-} np/ np} & \cf{np} & \cf{n/ np\bs \subsb{*} np} & \cf{np} & \cf{pp/ \subsa{\diamond} np} & \cf{np} \\
& \mc{2} {\hrulefill_{<}} \\
& \mc{2}{\cf{n/ np}} \\
&&&& \mc{2} {\hrulefill_{>}} \\
&&&& \mc{2}{\cf{pp}} \\
& \mc{3} {\hrulefill_{>}} \\
& \mc{3}{\cf{n}} \\
&&&& \mc{2} {\hrulefill_{t\mathbf{ypechange-6}}}\\
&&&& \mc{2}{\cf{n\bs \subsb{*} n}} \\
& \mc{5} {\hrulefill_{<}} \\
& \mc{5}{\cf{n}} \\
& \mc{5} {\hrulefill_{t\mathbf{ypechange-4}}}\\
& \mc{5}{\cf{np}} \\
 \mc{6} {\hrulefill_{>}} \\
 \mc{6}{\cf{s\bs \supsa{-} np}} \\
}
\end{center}
\end{frame} 

\begin{frame}[fragile]{MMCCG/HLDS -- III}
\begin{center}
\begin{footnotesize}
\begin{verbatim}
@w0:action(ON ^ 
              <Mood>imp ^ 
              <Actor>x1:entity ^ 
              <Patient>(w2:entity ^ и ^ 
                        <Num>pl ^ 
                        <First>(w1:thing ^ лампа ^ 
                                <Num>sg) ^ 
                        <Modifier>(w4:m-location ^ на ^ 
                                   <Anchor>(w5:e-place ^ кухня ^ 
                                            <Num>sg)) ^ 
                        <Next>(w3:thing ^ подсветка ^ 
                               <Num>sg) ^ 
                        <Num>pl))
\end{verbatim}
\end{footnotesize}
\end{center}
\end{frame} 

\begin{frame}{Теоретико-типовая семантика}
\begin{small}
\begin{itemize}
	\item проблемы с подходом Монтегю
	\begin{itemize}
    		\item проблема псевдотипа $s$
	    \item отсутствие подтипов (напр. у $\langle e \rangle$)
	    	\item контринтуитивность (что такое ``мужчина''?)
	\end{itemize}
	\item \textit{A. Ranta, Type-Theoretical Grammar (1994)}
	\begin{itemize}
    		\item использует теорию зависимых типов Мартин-Лёфа для моделирования семантики естественного языка
	    \item Types as Propositions
	\end{itemize}
\end{itemize}
\end{small}
\end{frame}

\begin{frame}{Интерактивные доказательства}
\begin{small}
\begin{itemize}
	\item Grail III -- прувер для неассоциативного исчисления Ламбека (сети доказательств)
	\item Icharate -- библиотека для Coq (категориальные грамматики)
	\item Chatzikyriakidis and Luo -- серия работ по Natural Language Inference (NLI) в Coq
\end{itemize}
\end{small}
\end{frame}

\begin{frame}{}
\begin{center}
	\textbf{Синтаксическая разметка\\средствами интерактивного поиска доказательств}
\end{center}
\end{frame}

\begin{frame}{Разметка (I)}
OpenCorpora (\texttt{http://opencorpora.org})\\
\bigskip
\begin{itemize}
    \item создание общедоступного корпуса средствами сообщества
    \item с 2010 г.
    \item около 20 участников, более 4000 разметчиков
    \item 1,3 млн. слов, ок. 95 тыс. предложений
    \item сейчас только морфологическая разметка (готово ~50\%)
\end{itemize}
\end{frame}

\begin{frame}{Разметка (II)}
\begin{center}
	\begin{figure}[H]
		\includegraphics[scale=0.285]{annotation1.png} 
	\end{figure}
\end{center}
\end{frame}

\begin{frame}{Разметка (III)}
\begin{center}
	\begin{figure}[H]
		\includegraphics[scale=0.285]{annotation2.png} 
	\end{figure}
\end{center}
\end{frame}

\begin{frame}{Parsing as Deduction (I)}
\begin{small}
Предложен в работе \textit{Pereira, Warren, Parsing as Deduction. (1983)}\\
\medskip
Алгоритм Кока-Янгера-Касами для КС-грамматики в виде системы гильбертовского типа\\
\begin{itemize}
	\item продукционные правила двух видов: $A \to BC$ и $A \to a$, где $A, B, C$ -- нетерминалы, $a$ -- терминал.
	\item формальный язык включает выражения вида $[A, \; i, \; j]$, где $A$ -- нетерминал, $i$ и $j$ -- индексы начала и конца
	\item два правила вывода: \textit{scan} и \textit{complete}
\end{itemize}

\begin{columns}[c,onlytextwidth]
\column{0.5\textwidth}
  \begin{prooftree}
    \AxiomC{ }
    \RightLabel{{\small $A \to w_i \in P$}}
    \UnaryInfC{$[A, i-1, i]$}
  \end{prooftree}
\column{0.5\textwidth}
  \begin{prooftree}
      \AxiomC{$[B, i, j]$}
      \AxiomC{$[C, j, k]$}
      \RightLabel{{\small $A \to BC \in P$}}
      \BinaryInfC{$[A, i, k]$}
  \end{prooftree}
\end{columns}

\end{small}
\end{frame}

\begin{frame}{Parsing as Deduction (II)}
\begin{small}
Грамматика:\\
\bigskip
\begin{columns}[c,onlytextwidth]
\column{0.25\textwidth}
    $S \to NP \; VP$\\
    $NP \to D \; N$
\column{0.25\textwidth}
    $VP \to V \; NP$\\
    $N \to man$
\column{0.25\textwidth}
    $NP \to Mary$\\
    $V \to saw$
\column{0.25\textwidth}
    $D \to the$
\end{columns}

\bigskip
Разбор предложения \textit{Mary saw the man}:\\
\medskip
\begin{itemize}
	\item[] $\lbrack NP, \; 0, \; 1 \rbrack$~~\textit{scan (Mary)}
	\item[] $\lbrack V, \; 1, \; 2 \rbrack$~~\textit{scan (saw)}
	\item[] $\lbrack D, \; 2, \; 3 \rbrack$~~\textit{scan (the)}
	\item[] $\lbrack N, \; 3, \; 4 \rbrack$~~\textit{scan (man)}
	\item[] $\lbrack NP, \; 2, \; 4 \rbrack$~~\textit{complete 3,4}
	\item[] $\lbrack VP, \; 1, \; 4 \rbrack$~~\textit{complete 2,5}
	\item[] $\lbrack S, \; 0, \; 4 \rbrack$~~\textit{complete 1,6}
\end{itemize}
\end{small}
\end{frame}

\begin{frame}{Исчисление Ламбека (I)}
\begin{footnotesize}
Ассоциативное исчисление Ламбека с умножением

\begin{columns}[c,onlytextwidth]
\column{0.5\textwidth}

\begin{prooftree}
  \AxiomC{$\Gamma, B, \Gamma' \vdash C$}
  \AxiomC{$\Delta \vdash A$}
  \RightLabel{{\small $\textbackslash_h$}}
  \BinaryInfC{$\Gamma, A, A \textbackslash B, \Gamma' \vdash C$}
\end{prooftree}

\begin{prooftree}
  \AxiomC{$A, \Gamma \vdash C$}
  \RightLabel{{\small $\textbackslash_i, \Gamma \neq \epsilon$}}
  \UnaryInfC{$\Gamma \vdash A \textbackslash C$}
\end{prooftree}

\begin{prooftree}
  \AxiomC{$\Gamma, B, \Gamma' \vdash C$}
  \AxiomC{$\Delta \vdash A$}
  \RightLabel{{\small $/_h$}}
  \BinaryInfC{$\Gamma, B/A, \Delta, \Gamma' \vdash C$}
\end{prooftree}

\begin{prooftree}
  \AxiomC{$\Gamma, A \vdash C$}
  \RightLabel{{\small $/_i, \Gamma \neq \epsilon$}}
  \UnaryInfC{$\Gamma \vdash C/A$}
\end{prooftree}

\column{0.5\textwidth}

\begin{prooftree}
  \AxiomC{$\Gamma, A, B, \Gamma' \vdash C$}
  \RightLabel{{\small $\bullet_h$}}
  \UnaryInfC{$\Gamma, A \bullet B, \Gamma' \vdash C$}
\end{prooftree}

\begin{prooftree}
  \AxiomC{$\Delta \vdash A$}
  \AxiomC{$\Gamma \vdash B$}
  \RightLabel{{\small $\bullet_i$}}
  \BinaryInfC{$\Delta, \Gamma \vdash A \bullet B$}
\end{prooftree}

\begin{prooftree}
  \AxiomC{$\Gamma \vdash A$}
  \AxiomC{$\Delta_1, A, \Delta_2 \vdash B$}
  \RightLabel{{\small \textit{сечение}}}
  \BinaryInfC{$\Delta_1, \Gamma, \Delta_2 \vdash B$}
\end{prooftree}

\begin{prooftree}
  \AxiomC{}
  \RightLabel{{\small \textit{аксиома}}}
  \UnaryInfC{$A \vdash A$}
\end{prooftree}

\end{columns}
\end{footnotesize}
\end{frame}

\begin{frame}{Исчисление Ламбека (II)}
Словарь: {\small $John \vdash np$, $Mary \vdash np$, $loves \vdash (np \textbackslash s)/np$}\\
Доказать, что строка \textit{John loves Mary} -- предложение\\
\medskip

\begin{prooftree}
  \AxiomC{$s \vdash s$}
  \AxiomC{$np \vdash np$}
  \RightLabel{{\small $\textbackslash_h$}}
      \BinaryInfC{$np, \, np \textbackslash s \vdash s$}
      \AxiomC{$np \vdash np$}
      \RightLabel{{\small $/_h$}}      
      \BinaryInfC{$np, \, (np \textbackslash s)/np, \, np \vdash s$}
\end{prooftree}
\end{frame}

\begin{frame}{Разметка (V)}
Цели проекта:\\
\bigskip
\begin{itemize}
    \item создание proof assistant для варианта исчисления Ламбека
    \item быстрая синтаксическая разметка существующих корпусов
    \item реализовать статистический парсер для русского языка
\end{itemize}
\end{frame}

\begin{frame}{}
\begin{center}
Спасибо!
\end{center}
\end{frame}

\end{document}
