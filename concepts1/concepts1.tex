\documentclass{beamer}

\usepackage[T2A]{fontenc}
\usepackage[utf8]{inputenc}
\usepackage[english,russian]{babel}
\usepackage{amssymb,amsfonts,amsmath,mathtext}
\usepackage{cite,enumerate,float,indentfirst}

\graphicspath{{images/}}

\usetheme{Pittsburgh}
\usecolortheme{whale}

\setbeamertemplate{footline}{\scriptsize{\hspace*{0.4cm}\insertframenumber}\vspace*{0.3cm}}
\beamertemplatenavigationsymbolsempty

\errorcontextlines 10000

\begin{document}
\title{\huge{$\mathbb{NLU}/RG$, \textit{pt. 2}}}
\author{Константин Соколов}
\institute{Mathlingvo, СПбГУ, i-Free}
\date{Санкт-Петербург, 2013} 
% Создание заглавной страницы
\begin{frame}
    \thispagestyle{empty}
    \titlepage
\end{frame}

% Что ещё нужно сделать
%   1. Собрать инфу про SemEval - какие были задачи, сайт, прошлые публикации, что будет в следующем году, сроки.
%   2. 


%%% 0. План
\begin{frame}{План}
    \setcounter{framenumber}{1}
    \begin{itemize}
        \item Краткое введение для тех, кто не был в прошлый раз
        \item Обоснование выбора статей, пара слов о Фреге и Бурбаки
        \item Обсуждение статьи Фреге ``Смысл и денотат''
        \item Обсуждение статьи Бурбаки ``Архитектура математики''
        \item Рассказ о лабораторных
        \item Рассказ про RTE и SemEval'2014
    \end{itemize}
\end{frame}

% 1. Краткое введение для тех, кто не был в прошлый раз. 
\begin{frame}{Краткое введение (1)}
    \begin{itemize}
        \item $\mathbb{NLU}$ -- что это? ``Понимание'' -- что это?
        \item Инженерный подход (не философия, не тест Тьюринга, не премия Лёбнера)
        \item Междисциплинарный характер
        \item Формат Reading Group
        \item Откуда эта программа
    \end{itemize}
\end{frame}

\begin{frame}{Краткое введение (2)}
    \begin{itemize}
        \item Norvig-Chomsky debate
        \item Статистические подходы vs. символьные
        \item Возможен ли синтез, имеет ли он смысл?
    \end{itemize}
\end{frame}

% 2. Обоснование выбора статей, пара слов о Фреге и Бурбаки
\begin{frame}{Обоснование выбора статей}
    \begin{itemize}
        \item Первые несколько встреч -- об основных понятиях, базовых концепциях и т.п.
        \item Понятия и терминология -- объект или инструмент?
        \item Понятия для дальнейшего (пример: ``интенсиональная база данных'')
        \item Понятийный аппарат логики vs. понятийный аппарат лингвистики
    \end{itemize}
\end{frame}

\begin{frame}{Несколько слов о Готлобе Фреге ()}
    \begin{itemize}
        \item ``Begriffsschrift'' (``Исчисление понятий, или подражающий арифметике формальный язык чистого мышления''), 1879 г.
        \item Аксиоматическое построение логики предикатов. Кванторы.
        \item Логицизм
        \item ``Sinn und Bedeutung'' (``Смысл и значение'' или ``Смысл и денотат''), 1892 г.
        \item Парадокс Рассела и теория типов
    \end{itemize}
\end{frame}

\begin{frame}{Несколько слов о Николя Бурбаки}
    \begin{itemize}
        \item Родился в 1935 г. А. Картан, А. Вейль, К. Шевалле и пр.
        \item Официально умер в 1968 г. Серр, Гротендик, Эйленберг и пр.
        \item ``Элементы математики''
        \item Теория множеств, аксиоматизация, структуры
    \end{itemize}
\end{frame}

% 3. Разбор статьи Фреге "Смысл и денотат" 
\begin{frame}{<<Смысл и денотат>> Г. Фреге (1892 г.)}
    \begin{itemize}
        \item Треугольник Фреге
        \item Имя, денотат, концепт
        \item Номинативная теория предложения
        \item Значение предложения - его истинностное значение, смысл предложения - суждение
        \item Принцип композициональности (принцип Фреге)
    \end{itemize}
\end{frame}

% 4. Разбор статьи Бурбаки "Архитектура математики" 
\begin{frame}{<<Архитектура математики>> Н. Бурбаки (1948 г.)}
    \begin{itemize}
        \item Множество со структурой
        \item Иерархия структур
        \item Отношения между структурами (согласованность, гомоморфизм)
    \end{itemize}
\end{frame}

% 5. Рассказ про планирующиеся лабораторные
\begin{frame}{Лабораторные работы}
    \begin{itemize}
        \item ATP и соревнования пруверов (CASC)
        \item ASP (Potassco)
        \item Дедуктивные БД (DLV-DB)
        \item RDF + OWL Reasoners
        \item Coq
        \item GAP
        \item Stanford CompTop
        \item Deep Learning
    \end{itemize}
\end{frame}

% 6. Рассказ про SemEval'2014
\begin{frame}{SemEval 2014}
\end{frame}


\begin{frame}{}
    \thispagestyle{empty}
    \begin{center}
        {\large Спасибо!}
    \end{center}
\end{frame}


%%% слайд помещается сюда
%% \begin{frame}{Заголовок}
%% \end{frame}

\end{document}
