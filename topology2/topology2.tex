\documentclass{beamer}

\usepackage[T2A]{fontenc}
\usepackage[utf8]{inputenc}
\usepackage[english,russian]{babel}
\usepackage{amssymb,amsfonts,amsmath,mathtext}
\usepackage{cite,enumerate,float,indentfirst}

\usepackage{graphicx}
\usepackage{booktabs}
\usepackage{tabularx}

%% Gentzen style natural deduction proof trees
\usepackage{bussproofs}
\usepackage{latexsym}

\graphicspath{{images/}}

\usetheme{Pittsburgh}
\usecolortheme{whale}

\setbeamertemplate{footline}{\scriptsize{\hspace*{0.4cm}\insertframenumber}\vspace*{0.3cm}}
\beamertemplatenavigationsymbolsempty

\errorcontextlines 10000

\begin{document}
\title{\Large{Топологическая семантика для S4}}
\subtitle{\textit{часть вторая}}
\author{Константин Соколов}
\institute[]
{Mathlingvo, СПбГУ, i-Free\\ \bigskip  \url{http://nlu-rg.ru}}
\date{Санкт-Петербург, 2014} 
% Создание заглавной страницы
\begin{frame}
    \thispagestyle{empty}
    \titlepage
\end{frame}

\begin{frame}{План}
    \setcounter{framenumber}{1}
    \begin{itemize}
        \item Топологическая семантика для S4 (окончание)
        	\begin{itemize}
				\item семантика для логики первого порядка
				\item семантика для S4 с кванторами
        	\end{itemize}
		\item Категория предпучков $Sets^{\mathcal{C}^{op}}$
    \end{itemize}
\end{frame}

% 1. Конструкции и понятия
% 	   - топологическое пространство (повторение)
%      - предпучок и пучок (повторение)
%	   - гомоморфизм пучков
%      - накрытие, универсальное накрытие, сечение, поднятие
%      - немного алгебраической топологии: 
% 			- критерий существования поднятия через подгруппы фундаментальной группы
% 			- частичный порядок на множестве накрытий соответствует частичному порядку на множестве подгрупп фундаментальной группы
%      - пучок непрерывных сечений накрытия
%      - расслоенное произведение пучков

% 2. Семантика для логики первого порядка

% 3. Семантика для ``FOS4''

% 4. Теоретико-категорная формулировка понятия пучка, предпучок как функтор, категория предпучков $Sets^{\mathcal{C}^{op}}$


\begin{frame}{}
\begin{center}
	\textbf{Предварительные замечания}
\end{center}
\end{frame}

\begin{frame}{Предварительные замечания}
\begin{itemize}
	\item Задача -- расширить топологическую семантику Тарского-МакКинси на модальную логику первого порядка
	\medskip
	\item Нам понадобится обширный аппарат:
		\medskip
		\begin{itemize}
			\item общая топология
			\item теория пучков
			\item некоторые сведения из алгебраической топологии
		\end{itemize}
	\medskip
	\item Конструкция получается почти автоматически
\end{itemize}
\end{frame}



\begin{frame}{}
\begin{center}
	\textbf{Конструкции и понятия}
\end{center}
\end{frame}

\begin{frame}{Конструкции и понятия (1)}
\begin{itemize}
	\item топологическое пространство {\small \textit{(повторение)}}
	\item предпучок и пучок {\small \textit{(повторение)}}
	\item немного алгебраической топологии:
		\medskip
		\begin{itemize}
			\item накрытие
			\item сечение
			\item поднятие
		\end{itemize}
		\medskip
	\item пучок непрерывных сечений накрытия
	\item гомоморфизм пучков
	\item расслоенное произведение пучков
\end{itemize}
\end{frame}

%   топология
\begin{frame}{Конструкции и понятия (2)}
Дано множество $X$. Система подмножеств $\mathcal{O} \subseteq 2^X$ называется \textit{топологической структурой} (или \textit{топологией}) на $X$, если:\\
\bigskip
\begin{itemize}
	\item $\varnothing, X \in \mathcal{O}$
	\item объединение $\bigcup U_i$ \textit{произвольного} числа подмножеств $U_i \in \mathcal{O}$ принадлежит $\mathcal{O}$
	\item пересечение $\bigcap U_i$ \textit{конечного} числа подмножеств $U_i \in \mathcal{O}$ принадлежит $\mathcal{O}$
\end{itemize}
\bigskip
Пара $(X, \mathcal{O}(X))$ называется \textit{топологическим пространством}.
\end{frame}

%   предпучок
\begin{frame}{Конструкции и понятия (3)}
\begin{itemize}
	\item Топологическое пространство $(X, \mathcal{O}(X))$
	\item Каждому открытому множеству $U \subset \mathcal{O}(X)$ сопоставляется множество со структурой $\mathcal{F}(U)$ (\textit{сечение})
	\item Для любых двух открытых множеств $U, V \subset \mathcal{O}(X)$, т. ч. $V \subseteq U$, определим гомоморфизм $\rho^U_V : \mathcal{F}(U) \to \mathcal{F}(V)$ (\textit{ограничение})
	\item Если $U = V$, то $\rho^U_V = \rho^V_U = 1_U$
	\item Если $U \subset V \subset W$, то $\rho^V_W \circ \rho^U_V = \rho^U_W$
\end{itemize}
\bigskip
$\mathcal{F}$ - \textit{предпучок} над топологическим пространством $(X, \mathcal{O}(X))$.
\end{frame}

%   пучок
\begin{frame}{Конструкции и понятия (4)}
\begin{itemize}
	\item Аксиома локальности
		\medskip
		\begin{itemize}
			\item Если $U = \bigcup U_i$, то для любых $U_i, U_j \subset U$ определены морфизмы $\rho^U_{U_i}$ и $\rho^U_{U_j}$
			\medskip
			\item Тогда, если $\rho^U_{U_i}(x) = \rho^U_{U_j}(y)$, то $x = y$
		\end{itemize}
	\bigskip
	\item Аксиома склейки
		\medskip
		\begin{itemize}
			\item Если $U = \bigcup U_i$, то для любых $U_i, U_j \subset U$, т.ч. $U_i \cap U_j \neq \varnothing$, определены морфизмы $\rho^U_{U_i}$, $\rho^U_{U_j}$ и $\rho^U_{U_i \cap U_j}$
			\medskip
			\item Тогда, если $x \in \mathcal{F}(U)$, то $(\rho^{U_j}_{U_i \cap U_j} \circ \rho^U_{U_j})(x) = (\rho^{U_i}_{U_i \cap U_j} \circ \rho^U_{U_i})(x)$
		\end{itemize}
\end{itemize}
\bigskip
Если эти условия выполнены, то $\mathcal{F}$ называется \textit{пучком}.
\end{frame}

%   накрытие
\begin{frame}{Конструкции и понятия (5)}
Накрытие $p : \widetilde{X} \to X$:
\bigskip
\begin{itemize}
	\item покрытие пространства $X$ открытыми множествами $\{ U_\alpha \}$
	\item $p^{-1}(U_\alpha) = \bigcup \widetilde{U_i}$ -- объединение непересекающихся открытых множеств в $\widetilde{X}$
	\item $p$ гомеоморфно отображает $\widetilde{U_i}$ на $U_\alpha$
	\item возможно $p^{-1}(U_\alpha) = \varnothing$
\end{itemize}
\end{frame}

\begin{frame}{Конструкции и понятия (6)}
\begin{center}
	\begin{figure}[H]
		\includegraphics[scale=0.3]{covering1.png} 
	\end{figure}
\end{center}
\end{frame}


%   сечение и слой
\begin{frame}{Конструкции и понятия (7)}
\begin{itemize}
	\item сечение $s_U : U \to \widetilde{X}$, $U \in \mathcal{O}(X)$, $p \circ s_U = 1_U$
    \item слой $p^{-1}(U) = \{ s_U : U \to \widetilde{X} \; \vert \; p \circ s_U = 1_U \}$
    \item расслоение (локально тривиальное, касательное)
\end{itemize}
\end{frame}

%   поднятие
\begin{frame}{Конструкции и понятия (8)}
\begin{itemize}
	\item поднятие $\widetilde{f} : Y \to \widetilde{X}$ отображения $f : Y \to X$, $p \circ \widetilde{f} = f$, 
	\item критерий существования поднятия и подгруппы $\pi_1(X, x_0)$
\end{itemize}
\end{frame}

%   подгруппы фундаментальной группы
\begin{frame}{Конструкции и понятия (9)}
\begin{itemize}
	\item множество накрытий $\{ p_i : \widetilde{X_i} \to X \}$ и подгруппы $\pi_1(X, x_0)$
\end{itemize}
\end{frame}



% 3. Семантика для FOL
\begin{frame}{}
\begin{center}
	\textbf{Cемантика для FOL}
\end{center}
\end{frame}

\begin{frame}{FOL (1)}
\begin{itemize}
	\item Язык $\mathcal{L}$ с сигнатурой $\sigma = \{ R_i, f_j, c_k \}_{i \in I, j \in J, k \in K}$
		\medskip
		\begin{itemize}
			\item $R_i$ -- реляционные символы
			\item $f_j$ -- функциональные символы
			\item $c_k$ - константы
		\end{itemize}
	\bigskip
	\item Структура $M = \langle D, R_i^M, f_j^M, c_k^M \rangle_{i \in I, j \in J, k \in K}$
		\medskip
		\begin{itemize}
			\item домен $D$ 
			\item $R_i^M \subseteq D^n$ -- интерпретация реляционного символа $R_i$
			\item $f_j^M : D_n \to D$ -- интерпретация функционального символа $f_j$
			\item $c_k^M \in D$ -- интерпретация константы $c_k$
		\end{itemize}
	\bigskip
	\item Истинность формулы $\phi(x_1, \dots, x_n)$ в модели $M$
		\medskip
		\begin{itemize}
			\item $M \models \phi(a_1, \dots, a_n)$
		\end{itemize}
\end{itemize}
\end{frame}

\begin{frame}{FOL (2)}
$[ \! [ \, x \, \vert \, \phi \, ] \! ]$ -- интерпретация формулы $\phi$ со свободной переменной $x$\\
\bigskip
\begin{small}
\begin{itemize}
	\item $[ \! [ \, \sigma \, ] \! ] \subseteq D^0$ 
	\item $[ \! [ \, x_1, \ldots, x_n, y \, \vert \, \phi \, ] \! ] = [ \! [ \, x_1, \ldots, x_n \, \vert \, \phi \, ] \! ] \times D \; \subseteq \; D^{n+1}$ 
	\item $[ \! [ \, x, y \, \vert \, x = y \, ] \! ] = \{ \, (a, a) \in D \times D \, \vert \, a \in D \, \}$ 
	\item $[ \! [ \, x_1, \ldots, x_n \, \vert \, R(x_1, \ldots, x_n) \, ] \! ] = R^M \subseteq D^n$ 
	\item $[ \! [ \, x_1, \ldots, x_n \, \vert \, \phi \wedge \psi \, ] \! ] = [ \! [ \, x_1, \ldots, x_n \, \vert \, \phi \, ] \! ] \cap [ \! [ \, x_1, \ldots, x_n \, \vert \, \psi \, ] \! ]$ 
	\item $[ \! [ \, x_1, \ldots, x_n \, \vert \, \neg \phi \, ] \! ] = D^n \setminus [ \! [ \, x_1, \ldots, x_n \, \vert \, \phi \, ] \! ]$
	\item $[ \! [ \, x_1, \ldots, x_n \, \vert \, \exists y \phi \, ] \! ] = $\\ \smallskip $\{ \, (a_1, \ldots, a_n) \in D^n \, \vert \, (a_1, \ldots, a_n, b) \in [ \! [ \, x_1, \ldots, x_n, y \, \vert \, \phi \, ] \! ], b \in D \, \}$
\end{itemize}
\end{small}
\end{frame}

\begin{frame}{FOL (3)}
Геометрический смысл: \textit{проекция} $p_n : D^{n+1} \to D^n$\\
\medskip
\begin{itemize}
	\item если $D^0 = \{ a \}$ -- синглетон
		\begin{itemize}
			\item $p_0(\top) = a$
			\item $p_0(\bot) = \varnothing$
		\end{itemize}
	\item {\small $p_n(a_1, \ldots, a_n, b) = (a_1, \ldots, a_n)$}
	\item {\small $[ \! [ \, x_1, \ldots, x_n \, \vert \, \exists y \phi \, ] \! ] = p_n([ \! [ \, x_1, \ldots, x_n \, \vert \, \phi \, ] \! ])$}
	\item {\small $[ \! [ \, x_1, \ldots, x_n, y \, \vert \, \psi \, ] \! ] = p_n^{-1}([ \! [ \, x_1, \ldots, x_n \, \vert \, \psi \, ] \! ])$}
\end{itemize}
\end{frame}

\begin{frame}{FOL (4)}
\begin{center}
	\begin{figure}[H]
		\includegraphics[scale=0.45]{fol.png} 
	\end{figure}
\end{center}
\end{frame}



% 4. Топологическая семантика для FOS4
\begin{frame}{}
\begin{center}
	\textbf{Cемантика для FOS4}
\end{center}
\end{frame}


% Предпучок над предпорядком или $\mathcal{F} : \mathcal{C}^{op} \to \textbf{Sets}$
% Нас, конечно, интересует не отдельный предпучок, а все возможные. Можно определить отображение предпучков и построить категорию предпучков
%    $Set^{\mathcal{C}^{op}}$, объекты которых - функторы $\mathcal{F}$ (т.е. предпучки), а стрелки - естественные преобразования этих функторов,
%    т.е. отображения предпучков.
% Пример про росток функций
% Индуктивный предел сечений
% Конструкция топологического пространства $\mathcal{\bar{F}}$ на множестве $\mathcal{F}_x$
% Отображение $\pi : \mathcal{\bar{F}}(X) \to X$ - накрытие


% 4. Категория предпучков
\begin{frame}{}
\begin{center}
	\textbf{Категория предпучков $Sets^{\mathcal{C}^{op}}$}
\end{center}
\end{frame}



\begin{frame}{}
    \thispagestyle{empty}
    \begin{center}
        {\large Спасибо!}
    \end{center}
\end{frame}


%%% слайд помещается сюда
%% \begin{frame}{Заголовок}
%% \end{frame}

\end{document}
