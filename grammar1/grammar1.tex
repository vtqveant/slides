\documentclass{beamer}

\usepackage[T2A]{fontenc}
\usepackage[utf8]{inputenc}
\usepackage[english,russian]{babel}
\usepackage{amssymb,amsfonts,amsmath,mathtext}
\usepackage{cite,enumerate,float,indentfirst}

% *** Need to use xypic for the subgroup lattice
\usepackage{amscd, amsmath, xypic}

\graphicspath{{images/}}

\usetheme{Pittsburgh}
\usecolortheme{whale}

\setbeamertemplate{footline}{\scriptsize{\hspace*{0.4cm}\insertframenumber}\vspace*{0.3cm}}
\beamertemplatenavigationsymbolsempty

\errorcontextlines 10000

\begin{document}
\title{\huge{$\mathbb{NLU}/RG$, \textit{pt. 5}}}
\author{Константин Соколов}
\institute[]
{Mathlingvo, СПбГУ, i-Free\\ \bigskip  \url{http://nlu-rg.ru}
}
\date{Санкт-Петербург, 2013} 
% Создание заглавной страницы
\begin{frame}
    \thispagestyle{empty}
    \titlepage
\end{frame}

%%% 0. План
\begin{frame}{План}
    \setcounter{framenumber}{1}
    \begin{itemize}
        \item Домашнее задание (языки и сигнатуры)
        \item Плунгян, грамматические значения
        \item Айдукевич, ``О синтаксической связности''
        \item Естественный вывод для пропозициональной логики
    \end{itemize}
\end{frame}

\begin{frame}{Домашнее задание}
\end{frame}

\begin{frame}{Плунгян, грамматические значения}
\end{frame}

\begin{frame}{Айдукевич, ``О синтаксической связности''}
\end{frame}

\begin{frame}{Естественный вывод для пропозициональной логики}
\begin{itemize}
  \item Впервые предложен Я. Лукасевичем и С. Яськовским.\\
  \item Мы рассмотрим вариант Г. Генцена.\\
  \item Исчисления такого типа (``синтаксические'') - инструмент \textit{теории доказательств}.
\end{itemize}
\end{frame}



\begin{frame}{}
    \thispagestyle{empty}
    \begin{center}
        {\large Спасибо!}
    \end{center}
\end{frame}


%%% слайд помещается сюда
%% \begin{frame}{Заголовок}
%% \end{frame}

\end{document}
