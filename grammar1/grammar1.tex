\documentclass{beamer}

\usepackage[T2A]{fontenc}
\usepackage[utf8]{inputenc}
\usepackage[english,russian]{babel}
\usepackage{amssymb,amsfonts,amsmath,mathtext}
\usepackage{cite,enumerate,float,indentfirst}

%% Gentzen style natural deduction proof trees
\usepackage{bussproofs}
\usepackage{latexsym}

\graphicspath{{images/}}

\usetheme{Pittsburgh}
\usecolortheme{whale}

\setbeamertemplate{footline}{\scriptsize{\hspace*{0.4cm}\insertframenumber}\vspace*{0.3cm}}
\beamertemplatenavigationsymbolsempty

\errorcontextlines 10000

\begin{document}
\title{\huge{$\mathbb{NLU}/RG$, \textit{pt. 5}}}
\author{Константин Соколов}
\institute[]
{Mathlingvo, СПбГУ, i-Free\\ \bigskip  \url{http://nlu-rg.ru}
}
\date{Санкт-Петербург, 2013} 
% Создание заглавной страницы
\begin{frame}
    \thispagestyle{empty}
    \titlepage
\end{frame}

%%% 0. План
\begin{frame}{План}
    \setcounter{framenumber}{1}
    \begin{itemize}
        \item Домашнее задание (языки и сигнатуры)
        \item Плунгян, грамматические значения 
        \item Айдукевич, ``О синтаксической связности'' (1935)
        \item Естественный вывод для пропозициональной логики
    \end{itemize}
\end{frame}

\begin{frame}{Домашнее задание}
\end{frame}

\begin{frame}{Плунгян, грамматические значения (1)}
\begin{itemize}
  \item Критерий обязательности.
  \item Грамматическая категория как ``ярлык'' для отношений между грамматическими значениями и показателями.
  \item Зависимость от языка и универсальный набор.
\end{itemize}
\end{frame}

\begin{frame}{Плунгян, грамматические значения (2)}
\begin{itemize}
  \item Понятие части речи как семантическое с опорой на грамматическую сочетаемость. Класс сочетаемости. 
  \item Акциональная классификация предикатов. Состояния и ситуации. Ситуации как процессы и события. Предельные и непредельные процессы.
  \item Семантические категории и словоформы. Акциональная характеристика предиката как множество всех таксономических категорий для всех форм данного предиката.
\end{itemize}
\end{frame}

\begin{frame}{Плунгян, грамматические значения (3)}
Замечания о семантике в рамках ``уровневой теории''.\\
\bigskip
\begin{itemize}
  \item Формальная семантика как семантика предложения.
  \item Принцип композициональности и естественные языки (``значение частей'', как далеко можно и нужно спускаться при анализе ``значений частей сложного выражения и способа их композиции'').
  \item Грамматикализация и формальная семантика.
\end{itemize}
\end{frame}

\begin{frame}{Плунгян, грамматические значения (4)}
Замечания о частях речи и синтаксических категориях в формальной семантике.\\
\bigskip
\begin{itemize}
  \item Определяя синтаксические категории, Монтегю следует Айдукевичу. 
  \item Объект как носитель набора свойств, прилагательные как предикаты, экстенсионал как характеристическая функция.
  \item Трансляция синтаксических категорий в семантические классы.
  \item Возможно ли однозначно приписать слову синтаксическую категорию?
  \item Возможно ли ввести синтаксические категории без привязки к конкретному языку?
\end{itemize}
\end{frame}


\begin{frame}{Айдукевич, ``О синтаксической связности'' (1)}
\begin{itemize}
  \item Выяснение условий синтаксической связности.
  \item Развитие концепции С. Лесьневского о ``категориях значения'' (термин ввел Э. Гуссерль).
  \item Критерий принадлежности к одной категории - способность взаимно заменять в контексте.
  \item ``Лестница категорий значения'' - ``грамматическо-семантический эквивалент'' упрощенной иерархии логических типов.
\end{itemize}
\end{frame}

\begin{frame}{Айдукевич, ``О синтаксической связности'' (2)}
\begin{itemize}
  \item Два вида категорий значения: подстановочные и функторные.
  \item Две основные (подстановочные) категории: категории предложений ($s$) и имен ($n$).
  \item Неограниченная иерархия функторных категорий: $s/n$, $s/nn$, $s/n/s/n$  и т.п.
\end{itemize}
\bigskip
\textit{Замечание:} категории значений как типы: $s$, $n$, $n \to s$, $n \to (n \to s)$, $(n \to s) \to (n \to s)$
\end{frame}

\begin{frame}{Айдукевич, ``О синтаксической связности'' (3)}
$(\neg p \to p) \to p : s/s \;\; s \;\; s/ss \;\; s \;\; s/ss \;\; s$\\
\bigskip
``сирень пахнет очень сильно и роза цветет''\\$n \;\; s/n \;\; s/n/s/n/s/n/s/n \;\; s/n/s/n \;\; s/ss \;\; n \;\; s/n$\\
\bigskip
``очень'': $((n \to s) \to (n \to s)) \to ((n \to s) \to (n \to s))$
\end{frame}

\begin{frame}{Айдукевич, ``О синтаксической связности'' (4)}
\begin{itemize}
  \item Проблема неоднозначности. Как определить ``однозначность'' и каковы её условия.
  \item Предложение может быть правильно составленным (ср. п.п.ф.) и при этом быть неоднозначным.
  \item Понятия ``правильно составленное выражение'' и ``синтаксически связное выражение''.
  \item Синтаксически связное должно быть 1) правильно составленным и 2) ``типы должны быть согласованы'', т.е. тип всего выражения должен быть $s$ и должен иметься алгоритм проверки.
\end{itemize}
\end{frame}

\begin{frame}{Айдукевич, ``О синтаксической связности'' (5)}
Проблемы при определении категорий значений для операторов.\\
\bigskip
\begin{itemize}
  \item Оператор - то, что связывает переменные ($\forall$, $\exists$, $\Sigma$, $\Pi$, $\int$)
  \item Анализ операторов. Смешивание функций. 
  \item Оператор, функция которого состоит только в связывании.
\end{itemize}
\end{frame}

\begin{frame}{Айдукевич, ``О синтаксической связности'' (6)}
Знак ``\string^'' (введен Расселом и Уайтхедом)\\
\bigskip
\begin{itemize}
  \item Если ``fx'' - символ неопределенного значения функции (т.е. $f(x) \in Im(f)$), то ``fx\string^'' - это сама функция (т.е. $f$) 
  \item Выражение ``(x\string^).fx'' имеет денотатом то же, что ``f'', т.е. то же, что ``fx\string^''
  \item Выражения ``(x\string^y\string^).fxy'' и ``fx\string^y\string^'' эквавалентны \\(ср. $\lambda x.\lambda y.f(x, y)$)
\end{itemize}
\bigskip
\textit{Замечание:} ср. в PTQ: ``The expression $[\string^\alpha]$ is regarded as denoting (or having as its \textit{extension}) the \textit{intension} of the expression $\alpha$.'') и понимание эксенсионала как характеристической функции.
\end{frame}

\begin{frame}{Айдукевич, ``О синтаксической связности'' (7)}
Анализ квантора всеобщности\\
\bigskip
\begin{itemize}
  \item Две функции: связывание и квантификация
  \item Попытка ``распутать'' квантор $\forall$, введя ``универсальный функтор'' $U$ с индексом $s/s/n$ (или $(n \to s) \to s)$), чтобы $U(f) \approx \forall x.fx$, т.е. можно писать $U(\lambda x. f)$.
  \item Т.о. роль квантификатора всеобщности удалось бы заменить комбинацией ролей универсального функтора и оператора ``x\string^''.
  \item Очевидно, существует много универсальных функторов с различными категориями значения в зависимости от категории значения функтора, служащего для них аргументом.
\end{itemize}
\end{frame}


\begin{frame}{Естественный вывод для пропозициональной логики (1)}
\begin{itemize}
  \item Впервые предложен Я. Лукасевичем и С. Яськовским.
  \item Мы рассмотрим вариант Г. Генцена.
  \item Исчисления такого типа (``синтаксические'') - инструмент \textit{теории доказательств}.
\end{itemize}
\end{frame}

\begin{frame}{Естественный вывод для пропозициональной логики (2)}
\textit{Формальная система} или \textit{исчисление} задается с помощью
\begin{enumerate}
  \item формального языка
  \item множества аксиом
  \item правил вывода
  \item определения формального вывода
\end{enumerate}
\end{frame}

\begin{frame}{Естественный вывод для пропозициональной логики (3)}
\begin{itemize}
  \item Пропозициональная система естественного вывода основана на языке логики высказываний
  \item Множество аксиом пусто
  \item Правила вывода: правила введения и удаления связок (классическая и интуиционистсткая пропозициональная логика) + правило снятия двойного отрицания (для классической логики высказываний)
  \item Определениe формального вывода: \textit{чуть позже}
\end{itemize}
\end{frame}

\begin{frame}{Язык логики высказываний (Алфавит)}
\begin{itemize}
  \item $\land$, $\lor$, $\supset$, $\neg$, $($, $)$, $p$.
  \item $p_n = (p...p)$ (n раз) - переменные
\end{itemize}
\end{frame}

\begin{frame}{Язык логики высказываний (Формула)}
Формулы языка логики высказываний определяются \textit{рекурсивно}:
\begin{itemize}
  \item $p_n$ - формула
  \item если $A$ и $B$ - формулы, то $(A \land B)$, $(A \lor B)$, $(A \supset B)$ и $(\neg A)$ - формулы
  \item все прочие слова в том же алфавите - не формулы\\ (напр. $(ppppp) \neg (ppp) \supset$ - не формула)
\end{itemize}
\end{frame}

\begin{frame}{Правила вывода}

%% введение и удаление конъюнкции

\begin{prooftree}
  \AxiomC{$A$}
  \AxiomC{$B$}
  \RightLabel{\scriptsize $\land_I$}
  \BinaryInfC{$A \land B$}
\end{prooftree}

\begin{prooftree}
  \AxiomC{$A \land B$}
  \RightLabel{\scriptsize $\land_E$}
  \UnaryInfC{$A$}
\end{prooftree}

\begin{prooftree}
  \AxiomC{$A \land B$}
  \RightLabel{\scriptsize $\land_E$}
  \UnaryInfC{$B$}
\end{prooftree}

%% введение и удаление дизъюнкции

\begin{prooftree}
  \AxiomC{$A$}
  \RightLabel{\scriptsize $\lor_I$}
  \UnaryInfC{$A \lor B$}
\end{prooftree}

\begin{prooftree}
  \AxiomC{$B$}
  \RightLabel{\scriptsize $\lor_I$}
  \UnaryInfC{$A \lor B$}
\end{prooftree}

\begin{prooftree}
\AxiomC{$A \lor B$}
\AxiomC{[$A$]}
\noLine
\UnaryInfC{$C$}
\AxiomC{[$B$]}
\noLine
\UnaryInfC{$C$}
\RightLabel{\scriptsize $\lor_E$}
\TrinaryInfC{$C$}
\end{prooftree}


\end{frame}


\begin{frame}{Правила вывода}

%% введение и удаление импликации

%% введение и удаление отрицания

%% удаление двойного отрицания (для классической системы)

\end{frame}

\begin{frame}{Дерево вывода}

$(A \supset B) \supset (\neg B \supset \neg A)$\\

\begin{prooftree}
\AxiomC{$[A]^3$}
\AxiomC{$[A \supset B]^1$}
\RightLabel{\scriptsize $\supset_E$}
\BinaryInfC{$B$}

\AxiomC{$[\neg B]^2$}
\RightLabel{\scriptsize $\neg_I$ (3)}
\BinaryInfC{$\neg A$}

\RightLabel{\scriptsize $\supset_I$ (2)}
\UnaryInfC{$\neg B \supset \neg A$}

\RightLabel{\scriptsize $\supset_I$ (1)}
\UnaryInfC{$(A \supset B) \supset (\neg B \supset \neg A)$}
\end{prooftree}

\end{frame}


\begin{frame}{Выводимость}
Формула $F$ выводима из конечного множества формул $\Gamma$ ($\Gamma \vdash F$), если существует вывод для формулы $F$ такой, что его множество \textit{зеленых листьев} $\Delta$ есть подмножество $\Gamma$ (т.е. $\Delta \subseteq \Gamma$).\\
\bigskip
Если $\Delta = \varnothing$, пишут $\vdash F$.\\
\bigskip
\textit{Зеленый лист} - открытое (существенное) допущение.\\
\bigskip
\textit{Увядший лист} - закрытое (промежуточное) допущение.\\

\end{frame}


\begin{frame}{}
    \thispagestyle{empty}
    \begin{center}
        {\large Спасибо!}
    \end{center}
\end{frame}


%%% слайд помещается сюда
%% \begin{frame}{Заголовок}
%% \end{frame}

\end{document}
