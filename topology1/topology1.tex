\documentclass{beamer}

\usepackage[T2A]{fontenc}
\usepackage[utf8]{inputenc}
\usepackage[english,russian]{babel}
\usepackage{amssymb,amsfonts,amsmath,mathtext}
\usepackage{cite,enumerate,float,indentfirst}

\usepackage{graphicx}
\usepackage{booktabs}
\usepackage{tabularx}

\graphicspath{{images/}}

\usetheme{Pittsburgh}
\usecolortheme{whale}

\setbeamertemplate{footline}{\scriptsize{\hspace*{0.4cm}\insertframenumber}\vspace*{0.3cm}}
\beamertemplatenavigationsymbolsempty

\errorcontextlines 10000

\begin{document}
\title{\Large{Топологическая семантика для S4}}
\author{Константин Соколов}
\institute[]
{Mathlingvo, СПбГУ, i-Free\\ \bigskip  \url{http://nlu-rg.ru}}
\date{Санкт-Петербург, 2014} 
% Создание заглавной страницы
\begin{frame}
    \thispagestyle{empty}
    \titlepage
\end{frame}

\begin{frame}{План}
    \setcounter{framenumber}{1}
    \begin{itemize}
    	\item Предварительные замечания
        \item Начала общей топологии
        \item Семантика для S4 (Тарский и МакКинси, 1944)
        \item Пучки и расслоения
        \item Семантика для S4 с кванторами (Аводей и Кисида, 2008)
    \end{itemize}
\end{frame}

% 1. Предварительные замечания
\begin{frame}{}
\begin{center}
	\textbf{Предварительные замечания}
\end{center}
\end{frame}

\begin{frame}{Предварительные замечания (1)}
Модальная логика в основе лингвистических формализмов:\\
\bigskip
\begin{itemize}
	\item Typed Feature Structures (TFS)
		\begin{itemize}
			\item (Copestake, 2001) 
		\end{itemize}
	\item логика Каспера-Раундса $L^{KR}$
		\begin{itemize}
			\item (Kasper, Rounds, 1986)
		\end{itemize}
	\item гибридная логика $HL(@, \downarrow)$
		\begin{itemize}
			\item (Blackburn, 2000)
		\end{itemize}
	\item Hybrid Logic Dependency Semantics (HLDS)
		\begin{itemize}
			\item (Baldridge et al., 2007)
		\end{itemize}
\end{itemize}
\end{frame}

\begin{frame}{Предварительные замечания (2)}
Топологическая семантика в лингвистике:\\
\bigskip
\begin{itemize}
    \item Окрестностная грамматика
		\begin{itemize}
			\item Шрейдер, Борщёв, Хомяков, Лапшин
		\end{itemize}
    \item Формальная герменевтика
		\begin{itemize}
			\item Прозоров
		\end{itemize}
    \item Теория тропов
		\begin{itemize}
			\item (Mormann, 1995)
			\item F. Moltmann (разные работы)
		\end{itemize}
    \item Композициональные дистрибутивные семантические модели (CDSM)
		\begin{itemize}
			\item Oxford Quantum Group (Abramsky, Grefenstette et al.)
		\end{itemize}
\end{itemize}
\end{frame}


% 2. Начала общей топологии
\begin{frame}{}
\begin{center}
	\textbf{Начала общей топологии}
\end{center}
\end{frame}

\begin{frame}{Начала общей топологии (1)}
Разные топологии:\\
\bigskip
\begin{itemize}
	\item Топология как раздел математики
		\begin{itemize}
			\item общая топология (general topology, point-set topology)
			\item алгебраическая топология
			\item дифференциальная геометрия и топология
		\end{itemize}
	\bigskip
	\item Н. Бурбаки, ``Основания математики''
		\begin{itemize}
			\item топологические структуры
			\item алгебраические структуры
			\item структуры порядка
		\end{itemize}
\end{itemize}
\end{frame}

\begin{frame}{Начала общей топологии (2)}
Общая топология:\\
\bigskip
\begin{itemize}
	\item Получила развитие в первой половине XX в.
		\begin{itemize}
			\item обоснование анализа с помощью теоретико-множественных понятий
			\item определение непрерывности без метрических понятий
		\end{itemize}
	\bigskip
	\item Начальные главы (хороших) учебников анализа
		\begin{itemize}
			\item избавление от $\epsilon$-$\delta$ формализма
			\item упрощение перехода к многомерному анализу
		\end{itemize}
\end{itemize}
\end{frame}

\begin{frame}{Начала общей топологии (3)}
\begin{center}
	\begin{figure}[H]
		\includegraphics[scale=0.3]{keep_calm.png} 
	\end{figure}
\end{center}
\end{frame}

\begin{frame}{Начала общей топологии (4)}
Дано множество $X$. Система подмножеств $\mathcal{A} \subseteq 2^X$ называется \textit{топологической структурой} (или \textit{топологией}) на $X$, если:\\
\bigskip
\begin{itemize}
	\item $\varnothing \in \mathcal{A}$ и $X \in \mathcal{A}$
	\item если $U, V \in \mathcal{A}$, то $U \cap V \in \mathcal{A}$
	\item если 
	\item 
\end{itemize}
\end{frame}


% 3. Семантика для S4 (Тарский и МакКинси)
\begin{frame}{}
\begin{center}
	\textbf{Семантика для S4}
\end{center}
\end{frame}

% 4. Пучки и расслоения
\begin{frame}{}
\begin{center}
	\textbf{Пучки и расслоения}
\end{center}
\end{frame}

\begin{frame}{Предпучок (1)}
\begin{itemize}
	\item Топологическое пространство $(X, \mathcal{O}(X))$
	\item Каждому открытому множеству $U \subset \mathcal{O}(X)$ сопоставляется множество со структурой $\mathcal{F}(U)$
	\item Для любых двух открытых множеств $U, V \subset \mathcal{O}(X)$, т. ч. $V \subseteq U$, определим гомоморфизм $\rho^U_V : \mathcal{F}(U) \to \mathcal{F}(V)$
	\item Если $U = V$, то $\rho^U_V = \rho^V_U = id$
	\item Если $U \subset V, V \subset W$, то $\rho^V_W \circ \rho^U_V = \rho^U_W$
\end{itemize}
\bigskip
Последние два условия превращают $\mathcal{F}(\mathcal{O}(X))$ в категорию.
\end{frame}

\begin{frame}{Предпучок (2)}
\begin{itemize}
	\item Условие ``инъективности''
		\bigskip
		\begin{itemize}
			\item Если $U = \bigcup U_i$, то для любых $U_i, U_j \subset U$ определены морфизмы $\rho^U_{U_i}$ и $\rho^U_{U_j}$
			\medskip
			\item Тогда, если $\rho^U_{U_i}(x) = \rho^U_{U_j}(y)$, то $x = y$
		\end{itemize}
	\bigskip
	\item Условие ``существования pullback-а''
		\bigskip
		\begin{itemize}
			\item Если $U = \bigcup U_i$, то для любых $U_i, U_j \subset U$, т.ч. $U_i \cap U_j \neq \varnothing$, определены морфизмы $\rho^U_{U_i}$, $\rho^U_{U_j}$ и $\rho^U_{U_i \cap U_j}$
			\medskip
			\item Тогда, если $x \in \mathcal{F}(U)$, то $(\rho^{U_j}_{U_i \cap U_j} \circ \rho^U_{U_j})(x) = (\rho^{U_i}_{U_i \cap U_j} \circ \rho^U_{U_i})(x)$
		\end{itemize}
\end{itemize}
\bigskip
Если эти условия выполнены, то $\mathcal{F}(X)$ называется \textit{предпучком}.
\end{frame}

\begin{frame}{Предпучок (1)}
\begin{itemize}
	\item Если $U_i, U_j \subset U$, $a \in U_i, b \in U_j$, $\mathcal{F}(a) = x \in \mathcal{F}(U_i)$, $\mathcal{F}(b) = y \in \mathcal{F}(U_y)$, $f : \mathcal{F}(U) \to \mathcal{F}(U_i)$, $g : \mathcal{F}(U) \to \mathcal{F}(U_j)$, 
	\item
	\item
	\item
	\item
	\item
\end{itemize}
\end{frame}

\begin{frame}{Предпучок (1)}
\begin{itemize}
	\item
	\item
	\item
	\item
	\item
\end{itemize}
\end{frame}

% 5. Семантика для S4 с кванторами
\begin{frame}{}
\begin{center}
	\textbf{Семантика для S4 с кванторами}
\end{center}
\end{frame}






\begin{frame}{}
    \thispagestyle{empty}
    \begin{center}
        {\large Спасибо!}
    \end{center}
\end{frame}


%%% слайд помещается сюда
%% \begin{frame}{Заголовок}
%% \end{frame}

\end{document}
