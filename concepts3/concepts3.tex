\documentclass{beamer}

\usepackage[T2A]{fontenc}
\usepackage[utf8]{inputenc}
\usepackage[english,russian]{babel}
\usepackage{amssymb,amsfonts,amsmath,mathtext}
\usepackage{cite,enumerate,float,indentfirst}

% *** Need to use xypic for the subgroup lattice
\usepackage{amscd, amsmath, xypic}

\graphicspath{{images/}}

\usetheme{Pittsburgh}
\usecolortheme{whale}

\setbeamertemplate{footline}{\scriptsize{\hspace*{0.4cm}\insertframenumber}\vspace*{0.3cm}}
\beamertemplatenavigationsymbolsempty

\errorcontextlines 10000

\begin{document}
\title{\huge{$\mathbb{NLU}/RG$, \textit{pt. 4}}}
\author{Константин Соколов}
\institute[]
{Mathlingvo, СПбГУ, i-Free\\ \bigskip  \url{http://nlu-rg.ru}
}
\date{Санкт-Петербург, 2013} 
% Создание заглавной страницы
\begin{frame}
    \thispagestyle{empty}
    \titlepage
\end{frame}

%%% 0. План
\begin{frame}{План}
    \setcounter{framenumber}{1}
    \begin{itemize}
        \item Домашнее задание (отображения и стрелки)
        \item Рассел, пропозиции и дескрипции
        \item Карнап, контексты и интенсиональный изоморфизм
        \item Тарский, семантическая концепция истины
        \item Начала теории моделей
    \end{itemize}
\end{frame}

% 1. Домашнее задание
\begin{frame}{Домашнее задание (1)}
Сколько сюръекций\\
\bigskip
  \begin{itemize}
    \item $f : A \to B$ - сюръекция
    \item $\left\vert{A}\right\vert = n, \left\vert{B}\right\vert = m, n \geq m$
    \item Ответ: $m^{n-m} \cdot m!$
    \item Напр., $\left\vert{A}\right\vert = 10$, $\left\vert{B}\right\vert = 5$, различных сюръекций 375000
    \item $\left\vert{A}\right\vert = 10$, $\left\vert{B}\right\vert = 10$, различных сюръекций 10!
  \end{itemize}  
\end{frame}

\begin{frame}{Домашнее задание (2)}
Сколько инъекций\\
\bigskip
  \begin{itemize}
    \item $f : A \to B$ - инъекция
    \item $\left\vert{A}\right\vert = n, \left\vert{B}\right\vert = m, n \leq m$
    \item Ответ: ${m \choose n} \cdot n!$
    \item Напр., $\left\vert{A}\right\vert = 5$, $\left\vert{B}\right\vert = 10$, различных инъекций 30240 
    \item $\left\vert{A}\right\vert = 10$, $\left\vert{B}\right\vert = 10$, различных инъекций 10! 
  \end{itemize}  
\end{frame}

\begin{frame}{Домашнее задание (3)}
Сколько биекций\\
\bigskip
  \begin{itemize}
    \item $f : A \to B$ - биекция
    \item $\left\vert{A}\right\vert = n, \left\vert{B}\right\vert = m, n = m$
    \item Ответ: $n!$
    \item $\left\vert{A}\right\vert = \left\vert{B}\right\vert = 10$, различных биекций 10! 
  \end{itemize}  
\end{frame}

\begin{frame}{Домашнее задание (4)}
Если $f : A \to B$ - изоморфизм групп, то $\left\vert{A}\right\vert = \left\vert{B}\right\vert$\\
\bigskip 
  \begin{itemize}
    \item $f$ как отображение - биективен и сюръективен, т.е. все элементы из $A$ имеют образ в $B$ и все элементы из $B$ имеют ровно один прообраз в $A$ (при отображении $f$)
    \item т. е. $A$ содержит столько же элементов, сколько $B$.
  \end{itemize}  
\end{frame}

\begin{frame}{Домашнее задание (5)}
Если $f : A \rightarrowtail B$, то $Dom(f) \cong Im(f)$\\
\bigskip
  \begin{itemize}
    \item т. к. $f$ - мономорфизм, прообраз $f^{-1}(a), a \in Cod(f)$ единственен, значит $g : A \to Im(f)$ - инъекция
    \item по построению, $Cod(g) = Im(g)$, значит $g$ - сюръекция 
    \item т. к. $f$ - морфизм и сохраняет структуру, то $Im(f)$ снабжен необходимой структурой (не уточняем, какой именно), значит $g : A \to Im(f)$ - тоже морфизм
    \item $g$ - инъективный и сюръективный морфизм, т.е. изоморфизм
  \end{itemize}  
\end{frame}


%% Понятие контекста у Фреге, Рассела, Карнапа
%% Понятие интенсионального изоморфизма у Карнапа
%% Рассел о пропозициях и дескрипциях
%% Тарский о семантической концепции истины
%% Начала теории моделей
%% Формальная семантика

\begin{frame}{Понятие контекста у Фреге, Рассела, Карнапа}
  \begin{itemize}
    \item 
    \item 
    \item 
    \item 
  \end{itemize}
\end{frame}

\begin{frame}{Понятие интенсионального изоморфизма у Карнапа}
  \begin{itemize}
    \item 
    \item 
    \item 
  \end{itemize}
\end{frame}

\begin{frame}{Рассел о пропозициях и дескрипциях}
  \begin{itemize}
    \item 
    \item 
    \item 
  \end{itemize}
\end{frame}

\begin{frame}{Тарский о семантической концепции истины}
  \begin{itemize}
    \item 
    \item 
  \end{itemize}
\end{frame}

\begin{frame}{Начала теории моделей}
  \begin{itemize}
    \item 
    \item 
    \item 
    \item 
  \end{itemize}
\end{frame}


\begin{frame}{Формальная семантика}
``Formal semantics is model-theoretic and truth-conditional''\\
\bigskip
``I reject the contention that an important theoretical difference exists between formal and natural languages.'' (R. Montague) 
\end{frame}


\begin{frame}{}
    \thispagestyle{empty}
    \begin{center}
        {\large Спасибо!}
    \end{center}
\end{frame}


%%% слайд помещается сюда
%% \begin{frame}{Заголовок}
%% \end{frame}

\end{document}
