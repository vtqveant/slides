\documentclass{beamer}

\usepackage[T2A]{fontenc}
\usepackage[utf8]{inputenc}
\usepackage[english,russian]{babel}
\usepackage{amssymb,amsfonts,amsmath,mathtext}
\usepackage{cite,enumerate,float,indentfirst}

% *** Need to use xypic for the subgroup lattice
\usepackage{amscd, amsmath, xypic}

\graphicspath{{images/}}

\usetheme{Pittsburgh}
\usecolortheme{whale}

\setbeamertemplate{footline}{\scriptsize{\hspace*{0.4cm}\insertframenumber}\vspace*{0.3cm}}
\beamertemplatenavigationsymbolsempty

\errorcontextlines 10000

\begin{document}
\title{\huge{$\mathbb{NLU}/RG$, \textit{pt. 4}}}
\author{Константин Соколов}
\institute[]
{Mathlingvo, СПбГУ, i-Free\\ \bigskip  \url{http://nlu-rg.ru}
}
\date{Санкт-Петербург, 2013} 
% Создание заглавной страницы
\begin{frame}
    \thispagestyle{empty}
    \titlepage
\end{frame}

%%% 0. План
\begin{frame}{План}
    \setcounter{framenumber}{1}
    \begin{itemize}
        \item Домашнее задание (отображения и стрелки)
        \item Рассел, пропозиции и дескрипции
%%        \item Карнап, контексты и интенсиональный изоморфизм
        \item Тарский, семантическая концепция истины
        \item Начала теории моделей
    \end{itemize}
\end{frame}

% 1. Домашнее задание
\begin{frame}{Домашнее задание (1)}
Сколько сюръекций\\
\bigskip
  \begin{itemize}
    \item $f : A \to B$ - сюръекция
    \item $\left\vert{A}\right\vert = n, \left\vert{B}\right\vert = m, n \geq m$
    \item Ответ: $m^{n-m} \cdot m!$
    \item Напр., $\left\vert{A}\right\vert = 10$, $\left\vert{B}\right\vert = 5$, различных сюръекций 375000
    \item $\left\vert{A}\right\vert = 10$, $\left\vert{B}\right\vert = 10$, различных сюръекций 10!
  \end{itemize}  
\end{frame}

\begin{frame}{Домашнее задание (2)}
Сколько инъекций\\
\bigskip
  \begin{itemize}
    \item $f : A \to B$ - инъекция
    \item $\left\vert{A}\right\vert = n, \left\vert{B}\right\vert = m, n \leq m$
    \item Ответ: ${m \choose n} \cdot n!$
    \item Напр., $\left\vert{A}\right\vert = 5$, $\left\vert{B}\right\vert = 10$, различных инъекций 30240 
    \item $\left\vert{A}\right\vert = 10$, $\left\vert{B}\right\vert = 10$, различных инъекций 10! 
  \end{itemize}  
\end{frame}

\begin{frame}{Домашнее задание (3)}
Сколько биекций\\
\bigskip
  \begin{itemize}
    \item $f : A \to B$ - биекция
    \item $\left\vert{A}\right\vert = n, \left\vert{B}\right\vert = m, n = m$
    \item Ответ: $n!$
    \item $\left\vert{A}\right\vert = \left\vert{B}\right\vert = 10$, различных биекций 10! 
  \end{itemize}  
\end{frame}

\begin{frame}{Домашнее задание (4)}
Если $f : A \to B$ - изоморфизм групп, то $\left\vert{A}\right\vert = \left\vert{B}\right\vert$\\
\bigskip 
  \begin{itemize}
    \item $f$ как отображение - биективен и сюръективен, т.е. все элементы из $A$ имеют образ в $B$ и все элементы из $B$ имеют ровно один прообраз в $A$ (при отображении $f$)
    \item т. е. $A$ содержит столько же элементов, сколько $B$.
  \end{itemize}  
\end{frame}

\begin{frame}{Домашнее задание (5)}
Если $f : A \rightarrowtail B$, то $Dom(f) \cong Im(f)$\\
\bigskip
  \begin{itemize}
    \item т. к. $f$ - мономорфизм, прообраз $f^{-1}(a), a \in Cod(f)$ единственен, значит $g : A \to Im(f)$ - инъекция
    \item по построению, $Cod(g) = Im(g)$, значит $g$ - сюръекция 
    \item т. к. $f$ - морфизм и сохраняет структуру, то $Im(f)$ снабжен необходимой структурой (не уточняем, какой именно), значит $g : A \to Im(f)$ - тоже морфизм
    \item $g$ - инъективный и сюръективный морфизм, т.е. изоморфизм
  \end{itemize}  
\end{frame}


%% Понятие контекста у Фреге, Рассела, Карнапа
%% Понятие интенсионального изоморфизма у Карнапа
%% Рассел о пропозициях и дескрипциях
%% Тарский о семантической концепции истины
%% Начала теории моделей
%% Формальная семантика

\begin{frame}{Рассел о пропозициях и дескрипциях}
  \begin{itemize}
    \item Денотативная концепция значения
    \item Пропозиция как форма выражения
    \item Факт как объективный коррелят убеждения 
    \item Корреспондентская теория истины
    \item Отношение между формой пропозиции и формой её объективного коррелята
    \item Истинность и ложность - свойства пропозиций
    \item Одна реальность, один язык
    \item Дескрипции и несуществующие объекты
    \item Функции и переменные в формальном языке
  \end{itemize}
\end{frame}

\begin{frame}{Тарский о семантической концепции истины}
  \begin{itemize}
    \item Формализованный язык - язык с точно заданной структурой
    \item Объектный язык и метаязык
    \item Пропозициональные функции - выражения со свободными переменными
    \item Выполнимость - отношение между произвольными объектами и пропозициональными функциями
    \item Предложение истинно, если оно выполняется всеми объектами, и ложно в противном случае
    \item Понятие истины не совпадает с понятием доказуемости (непротиворечивость и полнота)
  \end{itemize}
\end{frame}

\begin{frame}{Начала теории моделей}
  \begin{itemize}
    \item Теория моделей и теория доказательств
    \item Логическое следование и выводимость
  \end{itemize}
\end{frame}

\begin{frame}{Начала теории моделей (сигнатура)}
  \textit{Сигнатура} или \textit{тип подобия} - это тройка $\sigma = (S_{func}, S_{rel}, ar)$\\
  \bigskip
  \begin{itemize}
    \item $S_{func}$ - набор функциональных символов (напр.: $+, \times, 0, 1$)
    \item $S_{rel}$ - множество реляционных символов (напр.: $\in, \leq$)
    \item $ar$ - функция арности: $S_{func} \cup S_{rel} \to \mathbb{N}$
  \end{itemize}
  \bigskip
  Сигнатура позволяет задать все нелогические символы формального языка
\end{frame}

\begin{frame}{Начала теории моделей (п.п.ф.)}
  \begin{itemize}
    \item \textit{Правильно построенная формула (п.п.ф.)} - формула, построенная ``по правилам'' из счетного множества символов переменных, символов сигнатуры и логических символов (напр., логики первого порядка)
    \item \textit{Язык, порожденный сигнатурой} - это все п.п.ф., задаваемые данной сигнатурой
    \item Например, группу можно описать в языке с сигнатурой $\sigma_{grp} = \{e, ^{-1}, \cdot\}$, поле в языке с сигнатурой $\sigma_{field} = \{0, 1, -, ^{-1}, +, \cdot\}$
  \end{itemize}
\end{frame}

\begin{frame}{Начала теории моделей (теория)}
\textit{Теория} - совместное множество аксиом и все выводимые из них п.п.ф.\\
\bigskip
Например, в языке с сигнатурой $\sigma_{grp} = \{e, ^{-1}, \cdot\}$ можно построить теорию групп с аксиомами\\
  \begin{itemize}
    \item $\forall \, x \; . \; x \cdot e = x \; \wedge \; e \cdot x = x$ 
    \item $\forall \, x \; . \;x \cdot x^{-1} = e \; \wedge \; x^{-1} \cdot x = e$ 
    \item $\forall \, x \; \forall \, y \, \forall \, z \; . \; (x \cdot y) \cdot z = x \cdot (y \cdot z)$
  \end{itemize}
Её можно превратить в теорию абелевых групп, добавив к набору аксиом $\forall x \, \forall y \; . \; x \cdot y = y \cdot x$
\end{frame}

\begin{frame}{Начала теории моделей (предложение)}
С помощью языка с сигнатурой можно формулировать некоторые ``утверждения'', напр. в языке с сигнатурой $\sigma_{grp}$ можно записать:
  \begin{itemize}
    \item $a \cdot a^{-1} = e$
    \item $\forall a \, \exists a^{-1} \; . \; a \cdot a^{-1} = e$
    \item $a \cdot b = c$
    \item $a \cdot b = b \cdot a$
  \end{itemize}
\bigskip  
\textit{Предложение} или \textit{замкнутая формула} - формула без свободных переменных.
\end{frame}

\begin{frame}{Начала теории моделей (структура)}
Структура - это тройка $\mathcal{A} = (A, \sigma, I)$, где $A$ - домен (носитель), $\sigma$ - сигнатура, $I$ - функция интерпретации сигнатуры в домене.\\
\bigskip
Например, группу $GL_n(\mathbb{R})$ можно задать как тройку $(A, \sigma_{grp}, I)$\\
  \begin{itemize}
    \item $A$ - множество матриц $n \times n$ с элементами из $\mathbb{R}$
    \item $\sigma_{grp} = \{e, ^{-1}, \cdot\}$
    \item $I$ сопоставляет символу $e$ единичную матрицу $n \times n$, символу $^{-1}$ операцию обращения матрицы, символу $\cdot$ операцию умножения матриц
  \end{itemize}
\end{frame}

%%% 
%%% \begin{frame}{Начала теории моделей (подструктура)}
%%% $\mathcal{A}$ индуцирована $\mathcal{B}$ ($\mathcal{A} \subseteq \mathcal{B}$), если\\
%%%   \begin{itemize}
%%%     \item $\sigma(\mathcal{A}) = \sigma(\mathcal{B})$
%%%     \item $\left\vert \mathcal{A} \right\vert \subseteq \left\vert \mathcal{B} \right\vert$
%%%     \item интерпретации согласованы
%%%   \end{itemize}
%%% \end{frame}

%%% \begin{frame}{Начала теории моделей ($\sigma$-гомоморфизм)}
%%% $\sigma$-гомоморфизм структур $\mathcal{A}$ и $\mathcal{B}$ - это отображение $h : \left\vert \mathcal{A} \right\vert \to \left\vert \mathcal{B} %%% \right\vert$, сохраняющее функции и отображения.
%%% \end{frame}

%%% \begin{frame}{Начала теории моделей (изоморфизм)}
%%%   \begin{itemize}
%%%     \item Две структуры $\mathcal{A}$ и $\mathcal{B}$ \textit{элементарно эквивалентны}, если в них выполнимы одни и те же предложения
%%%     \item изоморфизм структур
%%%     \item 
%%%     \item 
%%%   \end{itemize}
%%% \end{frame}


\begin{frame}{Формальная семантика (1)}
``Formal semantics is model-theoretic and truth-conditional''\\
\bigskip
  \begin{itemize}
    \item устранение идеальных сущностей из семантики
    \item приверженность принципу композициональности
    \item введение метаязыка
    \item понимание смысла как условий истинности
    \item интерпретация фрагментов языка в структурах, определенных с точностью до изоморфизма
    \item различные формализации для различных контекстов
  \end{itemize}
\end{frame}

\begin{frame}{Формальная семантика (2)}
В принципе, что естественный язык можно изучать таким способом - это вопрос установки.\\
\bigskip
\textit{``I reject the contention that an important theoretical difference exists between formal and natural languages.'' (R. Montague) }
\end{frame}


\begin{frame}{}
    \thispagestyle{empty}
    \begin{center}
        {\large Спасибо!}
    \end{center}
\end{frame}


%%% слайд помещается сюда
%% \begin{frame}{Заголовок}
%% \end{frame}

\end{document}
